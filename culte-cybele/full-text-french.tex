\documentclass[a4paper, 11pt, oneside, polutonikogreek, french]{article}
\usepackage[T1]{fontenc}
\usepackage[sfdefault]{atkinson}

% Load encoding definitions (after font package)

\usepackage{textalpha}

\usepackage{listings}
\lstset{basicstyle=\ttfamily}

% Babel package:
\usepackage[french]{babel}

% With XeTeX$\$LuaTeX, load fontspec after babel to use Unicode
% fonts for Latin script and LGR for Greek:
\ifdefined\luatexversion \usepackage{fontspec}\fi
\ifdefined\XeTeXrevision \usepackage{fontspec}\fi

% "Lipsiakos" italic font `cbleipzig`:
\newcommand*{\lishape}{\fontencoding{LGR}\fontfamily{cmr}%
		       \fontshape{li}\selectfont}
\DeclareTextFontCommand{\textli}{\lishape}

\usepackage{booktabs}
\setlength{\emergencystretch}{15pt}
\usepackage{fancyhdr}
\usepackage{microtype}
\begin{document}
\begin{titlepage} % Suppresses headers and footers on the title page
	\centering % Centre everything on the title page
	%\scshape % Use small caps for all text on the title page

	%------------------------------------------------
	%	Title
	%------------------------------------------------
	
	\rule{\textwidth}{1.6pt}\vspace*{-\baselineskip}\vspace*{2pt} % Thick horizontal rule
	\rule{\textwidth}{0.4pt} % Thin horizontal rule
	
	\vspace{1\baselineskip} % Whitespace above the title
	
	{\scshape\Huge Le Culte de Cybèle, Mère des Dieux, à Rome et dans l'Empire Romain}
	
	\vspace{1\baselineskip} % Whitespace above the title

	\rule{\textwidth}{0.4pt}\vspace*{-\baselineskip}\vspace{3.2pt} % Thin horizontal rule
	\rule{\textwidth}{1.6pt} % Thick horizontal rule
	
	\vspace{1\baselineskip} % Whitespace after the title block
	
	%------------------------------------------------
	%	Subtitle
	%------------------------------------------------
	
	{\scshape \Large Par Henri Graillot} % Subtitle or further description
	
	\vspace*{1\baselineskip} % Whitespace under the subtitle
	
        {\scshape\scriptsize } % Subtitle or further description
    
	%------------------------------------------------
	%	Editor(s)
	%------------------------------------------------
        \vspace*{\fill}

	\vspace{1\baselineskip}

	{\small\scshape Paris 1912}
	
	{\small\scshape{Fontemoing et Cie, Éditeurs \\Libraires des Écoles françaises d'Athènes et de Rome, de l'Institut français d'Archéologie orientale du Caire, du Collège de France et de l'École Normale Supérieure \\4, Rue le Goff, 4}}
	
	\vspace{0.5\baselineskip} % Whitespace after the title block

        \scshape Internet Archive Online Edition  % Publication year
	
	{\scshape\small Utilisation non commerciale --- Partage dans les mêmes conditions 4.0 International} % Publisher
\end{titlepage}
\setlength{\parskip}{1mm plus1mm minus1mm}
\clearpage
\section{Préliminaires.}
\begin{center}
Les Origines du Culte Métroaque.
\end{center}
Τοῖς γὰρ ὁρθῶς εἰδόσι  
τὰ θεῖα, μεῖζον Μητρὸς οὐχ ἔστιν ποτέ  
ὅθεν ὁ πρῶτος οὐχ ἀπαιδεύτως ἔχων  
ἱδρύσαθ᾽ ἱερὸν Μητρός.  
Alexis.*

*) Alexis de Thurium, poète comique (4e s.). Dans Stob. \emph{Floriteg.}, 79, 13, èd. Meineke 1866, 3, p. 83.

Antiquité du culte d'une Grande Mère chthonienne, déesse des montagnes et souveraine des fauves, dans le bassin de la mer Égée et en Asie Mineure. --- 1. La Rhéa crétoise et préhellénique. Sa survivance dans les cultes helléniques. --- 2. La Kybelè anatolienne. Hittites, Phrygiens, Lydiens. La ville sacrée de Pessinonte et ses Attis. Expansion du culte phrygien en pays grec.

\subsection{1.}

Une quinzaine de siècles avant notre ère, les Crétois adoraient une divinité féminine qui nous apparaît comme le prototype de Rhéa-Cybèle.* C'était la souveraine des montagnes. Lointaine et redoutable, elle réside sur les pics aigus, sur les cimes boisées, dans les profondeurs des grottes. Elle y est gardée par des lions. Quand elle parcourt ses domaines, un fauve l'accompagne. Déesse guerrière ou peut-être simplement chasseresse, elle brandit une arme, javeline, épieu ou double hache. On la figurait volontiers dans une attitude de combat. Mais elle est aussi la reine assise qui, une fleur à la main, reçoit les hommages et les actions de grâces de ses sujets. Car elle est bienveillante aux humains qui vivent sur son territoire et la révèrent selon les rites. Elle protège leurs foyers et leurs cités. On lui a réservé sa chapelle dans le palais de Knossos. Sur les portes des palais, sur les portes des villes, ses lions se dressent autour d'un pilier bétylique ou d'un autel, en signe de possession et de protection.

*) Sur la déesse crétoise, consulter en particulier : Evans, \emph{Mycenaean tree and pillar-cult}, dans \emph{Journal of Hell. St.}, 1901 ; \emph{Knossos excavations}, 1901-1903, dans \emph{British School Annual}, 7 (déesse aux lions, sur un pic, p. 29, fig. 9), 8, 9 (déesse en marche, accompagnée du lion, p. 59, fig. 37 ; dame aux serpents, p. 75, fig. 54) ; G. Karo, \emph{Altkretische Kultstaetten}, dans \emph{Archiv für Religionswissenschaft}, 7, 1904, p. 151 ss ; R. Dussaud, \emph{Questions mycéniennes} dans \emph{Rev. Hist. Religions}, 1905, et \emph{Fouilles récentes dans les Cyclades et en Crète}, dans \emph{Bult. et Mém. Soc. d'anthropologie de Paris}, 1906 ; Farnell, \emph{Cults of the greek States}, 3, 1907, p. 295 ss et pl. 33 ; H. Boyd Hawes, \emph{Gournia, Excavations 1901, 1903, 1904, publ. by the American Exploration Society}, 1908 (appendice A : Religion of the Minoans) ; Paribeni, \emph{Sarcofago dipinto di H. Triada} dans \emph{Monumenti antichi}, R. \emph{Accad. d. Lincei}, 29, 1908, p. 6-86 et 3 pl. ; Fr. von Dulin, \emph{Der Sarkophag aus H. Triada} dans \emph{Archiv f. Religionswiss.}, 1909 ; cf. Dussaud dans \emph{Rev. Hist. Religions}, 1908, 2, p. 365 ss, et A. J. Reinach, \emph{ibid.}, 1909, 2, p. 237 ss ; G. Karo dans \emph{Jahrbuch d. Inst.}, 1908, \emph{Arch. Anzeiger}, p. 122 s (fresque de Knossos avec scènes religieuses ; cf. Collignon dans \emph{Gaz. Beaux-Arts}, 1909, 2, p. 1-35). D'après la classification d'Evans, la troisième et dernière période du minoen, époque la plus brillante de la civilisation crétoise, correspondrait aux années 1800-1200, avec les subdivisions suivantes : 1, 1800-1600 ; 2 (époque des palais), 1600-1450 ; 3, 1450-1200. Après l'invasion achéenne (vers 1200), le minoen récent se continue par le mycénien ou achéo-égéen, qui dure jusqu'à l'invasion dorienne du 11e siècle.

Au moment où la dame aux lions entre dans l'histoire, en plein âge du bronze, elle est le produit d'une évolution déjà longue. Avant de s'entourer de fauves, elle fut une bête fauve. Avant de se fixer sur la pointe des rochers, elle fut le rocher sacré. Avant d'être le pilier de pierre ou de bois, elle avait été la pierre brute ou l'arbre feuillu. Longtemps même, à côté de l'image anthropomorphe, subsista ce type intermédiaire du pilier aniconique. Le totémisme primitif laissait de très nombreuses survivances. Ses traditions singulièrement tenaces vont se prolonger au-delà de l'ère mycénienne. Mais, dès la période finale du minoen, dans la religion naturaliste qui s'édifie sur les bases du polydémonisme, les antiques fétiches se groupent autour de quelques divinités supérieures. La plus importante paraît être la Terre, considérée dans ses forces productrices, principe féminin de vie, déesse de fécondité et de fertilité, mère et nourricière de tout ce qui naît et meurt ici-bas. « La terre fait sortir les fruits du sol, » chantaient d'après un vieil hymne les prêtresses de Dodone ; « appelons donc la Terre du nom de Mère.* » C'est elle que l'on adore sur les montagnes, qui sont ses trônes, dans les cavernes, qui sont ses premiers temples, dans les bois et près des sources, manifestations de sa mystérieuse et bienfaisante énergie. Pour la distinguer des mères topiques,* représentant les esprits d'un lieu déterminé, multiples génies de la roche, de l'antre, de l'arbre ou de l'eau, on dut lui attribuer le titre de Grande Mère très anciennement.* Elle portait aussi le nom de Rhéa, dont nous ignorons la signification primordiale et qui s'est conservé dans les cultes grecs.* C'est elle que l'on conçoit comme la dompteuse des fauves, terribles hôtes des montagnes et des forêts. Sans doute n'y eut-il à l'origine, dans cette conception du rôle divin, qu'une raison de pure magie. En lui soumettant les animaux dont on redoutait le pouvoir destructeur, l'homme pensait assurer contre le danger sa personne, ses troupeaux et sa maison. Mais c'est elle encore qui se révèle comme charmeuse de serpents. Dame aux lions et dame aux reptiles, rapprochées déjà dans les chapelles palatines de Knossos, tendent à confondre leur individualité, à réunir leurs attributs. Le serpent qui sort des crevasses du sol vient s'enrouler à la base des piliers-bétyles. Car il est le génie du monde souterrain ; et la déesse chthonienne étend à la fois son empire sur la terre et dessous. Celle-ci, d'autre part, est en relations avec le monde aérien. Sur le pilier d'abord, puis sur la tête ou sur les mains de l'idole, nous voyons se poser familièrement certains oiseaux sacrés. Ils sont devenus ses messagers célestes. La déesse de la terre est en effet associée à un dieu du ciel, dieu des phénomènes atmosphériques entre ciel et terre, dont les Grecs feront un Zeus. Maître de la foudre et des tempêtes, c'est un guerrier armé de la lance, le plus souvent de la bipenne, et du bouclier bilobé ; sa coiffure est un casque ou bonnet pointu. Quand il agite son bouclier de bronze et brandit sa double hache, il produit orages et pluies. Dans les sanctuaires minoens, le dieu mâle est encore représenté par ces fétiches skeuomorphes. Son culte reste lié, de même, à celui de certains animaux. Peu à peu s'absorbe en lui la personnalité de l'ancien dieu taureau, dont plusieurs légendes amoureuses caractérisent la puissance génésique* ; le taureau devient son attribut. Dieu qui féconde la terre, son culte est lié aussi à celui des arbres. On dresse la bipenne sur un pilier de bois, souvent garni de feuillages ; à Dodone, on l'enfonçait dans le chêne de Zeus. Plus tard, c'est le dieu lui-même que l'on figure assis sur un tronc d'arbre et entouré de plantes. Tel il nous apparaît sur les monnaies de Phaestos, dont le revers porte l'effigie taurine. Parmi les arbres sacrés, il y a plus spécialement ceux qui demeurent toujours verts : pins et cyprès, surtout oliviers et palmiers,* dont les fruits sont précieux à l'homme. Ils deviendront facilement des symboles d'immortalité et prendront une signification funéraire, s'ils ne l'ont déjà. Le dieu, fils de Rhéa qui est la Terre Mère, né dans une caverne comme beaucoup de dieux lumineux, meurt en pleine jeunesse, comme les dieux solaires et agraires* ; on montrait sa tombe à Knossos. Bien des éléments intervinrent dans la constitution progressive de ces deux personnalités divines. Il convient d'y faire une part, sans doute importante, aux influences égyptiennes et chaldéo-élamites ; ces dernières se seraient exercées sur le monde égéen par l'intermédiaire de la Syrie. Les divinités chthoniennes et célestes de l'Égypte et de la Chaldée, où se multiplient déesses lionnes et déesses serpents, dieux taureaux et dieux à la hache, ont pu faire sentir leur action sur les divinités similaires de Crète, en modifier ou en agrandir le concept. Beaucoup de ces déesses sont, comme Rhéa, les dames de la montagne : telles l'égyptienne Hathor, dont le serpent s'associe au lion solaire de Râ, et la babylonienne Ninhursag, toutes deux « mères et nourricières des rois* » ; telle Ishtar-Nana, qui debout sur un léopard ou un lion,* couronnée de tours et armée de l'arc, parcourt monts et vaux en criant : « je suis reine dans le combat » ; telles enfin la plupart des Beltis syriennes, debout aussi sur un fauve, les mains pleines de serpents et de fleurs. Certaines ne sont connues que sous leur vocable de Mères : la Moût léontocéphale de Thèbes, par exemple, et Mami, qui est une hypostase d'Ishtar, la Terre qui reverdit au printemps.* Il en est même qui sont déjà qualifiées de Mères des Dieux ; telles la Nana d'Ourouk et la Baou de Sirpourla, déification de la Terre maternelle, productrice de vie et protectrice des fruits de la vie, épouse d'un dieu solaire et guerrier. Mais ces divinités féminines ne sont que les égales des dieux parèdres, et souvent leur sont inférieures.* Ce qui distingue la Mère ou peut-être Vierge Mère* crétoise, c'est sa suprématie. Dans la religion égéenne prédomine l'aspect féminin de la divinité. Tradition héritée d'un ancien état social, où la toute-puissance du sang maternel était la loi ? Le matriarcat, régime de peuples arriérés, survivait encore en Lycie au temps d'Hérodote ; il n'avait pas encore complètement disparu de la civilisation étrusque, et l'on croit en retrouver le souvenir dans quelques coutumes des Ligures.* A la hiérarchie divine correspond la hiérarchie sacerdotale. Les femmes y occupent le premier rang ; c'est à elles que sont dévolues les principales fonctions de la liturgie. Le rôle des hommes paraît n'être qui accessoire.* Aussi bien, d'une façon générale, les hommes ne sont-ils pas admis dans l'enceinte du temenos.* Le culte est orgiastique. Il comporte des danses sacrées, au son de la lyre et de la double flûte qui accompagnent aussi les sacrifices. Les unes sont exécutées par des femmes. Les autres sont des danses d'hommes armés ; elles ont donné naissance au mythe des Curètes Idéens. Suivant un rite familier aux peuplades primitives, ces guerriers entrechoquent bruyamment leurs boucliers et leurs glaives* soit pour conjurer le tonnerre, soit pour évoquer la pluie et pour aider à la renaissance annuelle de la nature.*

*) Paus. 10, 12, 10. Sophocle, \emph{Philoct.} 391 ss, appelle Ga la déesse des montagnes, dame aux lions, qu'il identifie à Rhéa, mère de Zeus, et à la Kybêbè de Sardes. Dieterich, \emph{Mutter Erde} dans \emph{Archiv f. Relig.}, 1905, p. 1 ss.

*) Il y avait un culte très ancien des « Meteres » crétoises : Diod. Sic. 4, 80, ce qui prouve bien l'importance de cette idée de maternité dans la religion minoenne. Ces Mères sont appelées aussi Nymphes, cf. Timée, \emph{Frg.} 83.

*) Le vocable de Megalé Meter apparaît seul sur des dédicaces de Phaestos, en Crète (époque hellénistique), \emph{Museo Italiano}, 3, p. 736 ; cf. \emph{Ath. Mitth.}, 1893, p. 272 ; 1894, p. 290 ; de Delos, \emph{Bull. Corr. Hell.}, 1882, p. 500, n° 25 ; de Thespies, \emph{CIG Sept.} 1, 1811 ; de Sparte, Paus. 3, 12, 9 ; cf. Pind., \emph{Fragm.} 48 et 63 (Boeck).

*) Voici, à titre de curiosité, quelques-unes des étymologies proposées. Les anciens rapprochaient le mot Rhéa du verbe ῾ρεῖν, couler ; Gruppe, \emph{Griech. Mythol. und Religions Gesch.}, 1906, p. 1524 s, y voit le nom d'une pierre-fétiche, que l'on aurait adorée pour obtenir la pluie, cf. lapis manalis. D'autres voyaient dans Rhéa une métathèse d'Era = Terre : Eustath., \emph{Ad Iliad.}, 1, 55 ; \emph{Reinesius, Syntagma inscr. antiq.}, 1682, p. 73 ; Welcker, \emph{Griech. Goetterlehre}, 2, 216 ; Buchholz, \emph{Homer. Realien}, 3, 1, p. 11 ; Decharme dans Daremberg et Saglio, \emph{Dict. des Antiq.}, s. v. \emph{Cybelé}. Crusius, \emph{Beitr. z. griech. Mythol.}, Leipz. Progr., 1886, p. 26, n. 4, suppose Rhéa (Rheia dans Homère et Hésiode) = O]reia ; cf. Sonny dans \emph{Philologus}, 1889, p. 559, Dieterich, \emph{ibid.}, 1894, p. 5, et Immisch dans Roscher, \emph{Myth. Lex.}, 2, 1613. Les sémitisants rapprochent Rhéa de Raiah = amie, aimée, compagne : Assmann, \emph{Vorgeschichte von Krela}, dans \emph{Philologus}, 1908, p. 177, qui croit encore à l'hypothèse phénicienne, démentie par les fouilles récentes.

*) Légendes d'Europe, de Pasiphaé. Cf. Cook, \emph{Animal worship in the mycenaean age}, dans \emph{J. of Hell. St.}, 14, 1894, p. 120-132 : The cult of the bull.

*) Cf. Paribeni, l. c., p. 29 et 42. Rhéa possédait à Knossos son bois de cyprès : Diod., 10, S6. Von Duhn croit que, sur le sarcophage d'Haghia Triada, les piliers sont de petits obélisques en bois, recouverts de feuilles de cyprès. Il y avait à Gortyne un platane sacré dont les feuilles étaient persistantes (sans doute un sycomore) : Theophr., \emph{Hist. plant.} 1, 15.

*) Cf. Déchelette, \emph{Le culte du soleil aux temps préhistoriques dans Rev. archéol.}, 1909, 1, p. 357 : « comme les primitifs attribuaient une origine commune à l'éclair et aux rayons du soleil, le dieu de la foudre se trouve étroitement apparenté aux divinités du cycle solaire. »

*) Sur les frappantes analogies de ces deux divinités, d'après les textes et les monuments figurés, v. A. Boissier dans \emph{Oriental. Litteraturzeitung}, 11, col. 234-236. Influence du culte d'Hathor sur l'iconographie de la Meter dans l'art cypro-mycénien : Evans, \emph{Mycenaean tree and pillar-cult}, p. 69 ; l'auteur voit en particulier dans la couronne murale de la déesse anatolienne « un dérivé direct de la maison de Hor sur la tête d'Hathor » ; mais il y a plutôt ici une influence immédiate de l'art babylonien.

*) Cf. Maspero, \emph{Hist. anc. des peuples de l'Orient}, 1, p. 681. Le disque étoilé d'Ishtar se retrouve sur un anneau d'or de Mycènes.

*) Ces divinités ne sont pas plus en Égypte et en Chaldée qu'en Crète des personnes abstraites, qui président métaphysiquement aux forces de la nature. Chacune enferme en soi l'un des éléments de l'univers et est la matière même de cet élément avant d'en devenir l'esprit ; elle continue à y résider alors même que le progrès religieux l'en a isolée.

*) En Égypte règne l'égalité entre les dieux et les déesses, qui participent à l'exercice du pouvoir suprême : cf, Maspero, l. c., p. 101. 11 semble au contraire que les déesses babyloniennes aient en général un caractère passif et presque impersonnel.

*) Farnell, l. c., p. 305 s. ; cf. en Asie-Mineure, Ramsay, \emph{Cities and bishoprics of Phrygia}, 1895-97, 1, p. 136.

*) Références dans Paribeni, Z. c., p. 81. Pour les Ligures, cf. Jullian, \emph{Hist. de la Gaule}, 1, 1908, p. 178. Le mythe pessinontien d'Agdistis nous en offre peut-être un exemple en Phrygie : Agdistis porte un nom dérivé de celui de sa mère, la roche Agdos, fécondée par Zeus ; la filiation s'établit de la mère à l'enfant.

*) Citharède, flûtiste, porteurs, sur le sarcophage d'Haghia Triada.

*) Fresque de Knossos ; cf. A. J. Reinach, \emph{l. c.}, p. 243.

*) A l'origine, leur double hache ; cf. le Curète Labrandos (λάβρυς, hache) et les rapprochements proposés par Eisler, \emph{Kuba-Kybele} dans \emph{Philologus}, 68, 1909, p. 126, n. 27.

*) Les danses armées des Saliens, à Rome, avaient sans doute la même origine ; cf. Salmoneus, roi légendaire de Thessalie, imitant le tonnerre en faisant résonner une cuve de bronze. Sur le caractère prophylactique et apotropaïque du bronze, v. Cook, \emph{The gong of Dodona} dans \emph{J. of Hell. St.}, 22, 1902, p. 14 ss.

L'orgiasme conduit à l'extase. De plus, la mantique est un élément ordinaire des cultes chthoniens. La déesse crétoise semble avoir eu ses prophètes ou prophétesses. Mais, par l'orgiasme, c'est toute une tribu ou tout un peuple qui participe à la communion divine. Chaque année, pour bénéficier des vertus de l'arbre divin, on le met en pièces et l'on s'en partage les débris.* Certains sacrifices d'animaux sacrés, taureaux et bouquetins, s'accomplissaient probablement selon le même rite, que nous retrouvons chez les Thraces.* Peut-être usait-on du taurobole, qui est à l'origine un rite magique. Ne serait-ce pas aussi d'une pratique baptismale, faisant renaître à une vie nouvelle, que dérive la légende de Pélops, régénéré par les soins de Rhéa* ? Les Minoens enfin croient à une survie d'outre-tombe. La déesse mère et le dieu fils, qui a subi la mort et qui l'a vaincue, étendent sur cet autre monde leur suzeraineté. Ils continuent à y protéger leurs féaux d'ici-bas. Donc il faut les propitier pour le trépassé, parti sur une barque. Grâce aux prières et aux sacrifices, celui-ci achève son voyage dans un bige de griffons ailés, que conduit une déesse psychopompe et qui l'emporte au divin séjour.* Un lien étroit unit le rituel funéraire au culte des grands dieux. Le tombeau du mort héroïsé s'associe à l'autel, aux emblèmes de la divinité, aux arbres sacrés.

*) Des gemmes nous montrent l'olivier violemment secoué ou même arraché ; cf. la légende thessalienne d'Erysichtôn détruisant le bois sacré de Demeter, « mauvaise explication du rite primitif » : A. J. Reinach, \emph{l. c.}, p. 240.

*) On a cherché dans la légende des amours de Pasiphaè le souvenir dégradé d'une communion mystique avec le dieu taureau : Dieterich, \emph{Eine Mithras Liturgie}, p. 136 ; Farnell, \emph{l. c.}, p. 297.

*) Farnell, \emph{ibid.}

*) Paribeni, \emph{l. c.}, p. 24 s, 59 s, 77 ss. La barque rappelle l'Égypte. Les Babyloniens semblent avoir eu l'idée de morts voyageant en char : A. J. Reinach, \emph{l. c.}, p. 241.

Cette religion n'était point particulière aux Eteocrètes. Elle semble avoir été commune à toutes les populations de même race, ou de même civilisation, qui habitaient le continent grec, les îles égéennes et l'Asie Mineure. C'était la religion des Pélasges. Après l'invasion dorienne, elle se survit à elle-même dans le sanctuaire de Dodone ; mais à Delphes Apollon Pythios et Pythoktonos, dieu serpent devenu le tueur du serpent, se substitue à la pélasgique Gaia. Sous le vocable de Rhéa, la dame des montagnes entre dans le panthéon hellénique. Elle figure au vie siècle sur une inscription d'Ithaque,* au ive sur une inscription de Cos.* Athènes lui a dédié un sanctuaire dans le temenos de Zeus Olympios. Son culte se maintient surtout dans le Péloponèse, où l'influence crétoise fut très active, et particulièrement dans le massif d'Arcadie, où persistent les vieilles traditions.* A Phigalie, à Tégée, le mythe de Rhéa est populaire comme une tradition nationale.* Les Arcadiens localisent au mont Lykaion l'enfantement de Zeus,* adorent aussi Rhéa dans une grotte du mont Thaumasion, où seules ses prêtresses ont le droit de pénétrer,* et lui associent Zeus sur le mont Azan, « ritu Idaeo.* » Dans la vallée de l'Alphée elle conserva longtemps ses prophétesses.* A Olympie, elle occupait tout d'abord auprès de Zeus une situation éminente.* A Messène, à Epidaure, à Lycosura, comme à Olympie même, son origine préhellénique est attestée par la présence des Curètes crétois, des Dactyles Idéens et de l'Idéen Héraclès.* Enfin au Metrôon d'Akriai, dressé sur un promontoire de Laconie, cette origine est attestée à la fois par une légende, relative à la très haute antiquité du temple, et par les nombreux vestiges que la religion crétoise a laissés dans le pays 6. Mais, en s'hellénisant, la déesse peu à peu se dépouille de ses caractères orgiastiques. Par contre, à mesure que se constitue la théogonie nouvelle, la notion de sa maternité divine s'amplifie. A l'époque homérique, la mère de Zeus est devenue également celle de Poséidon, Hadès et Héra. L'auteur de la \emph{Théogonie} dite hésiodique, au 8e siècle, ajoute Hestia et Demeter. C'est entre le 8e et 6e vie siècles, dans un fragment d'hymne, que nous voyons apparaître une « Mère de tous les Dieux,* » principe de toute vie divine et humaine. Pour la première fois, ce titre de Mère des Dieux est présenté comme un vocable personnel, qui se suffît à lui-même. La divinité qu'il désigne passe pour habiter « les montagnes aux échos sonores et les ravins boisés, » en compagnie des loups et des lions. Elle est donc l'un des aspects de Rhéa. La littérature du Ve siècle n'a pas tort de confondre les deux déesses.* La religion toutefois les distingue. Avec deux appellations différentes, une même divinité se dédouble et exige deux autels ; les mots peuvent créer des dieux. A vrai dire, dans la Grèce classique, Rhéa est une divinité déchue, comme son époux et parèdre Chronos. Elle fut supplantée par la Mère des Dieux, dont le culte restait en communication plus intime avec la vie de la terre, des sources et des arbres.* Dans cette expansion d'un culte au détriment d'un autre, quelle est la part des influences extérieures ? D'après une tradition athénienne, la Mêter serait une importation de l'Anatolie ; et son culte officiel en Attique ne daterait que du 5e siècle.* La tradition se manifeste, il est vrai, sous les apparences d'une légende. Signalée seulement à la fin du paganisme, elle est sans valeur historique ; c'est une forme tardive du mirage oriental. Mais on ne saurait mer toute influence de l'Asie Mineure, où la Grande Mère avait conservé sa suprématie.

*) Entre Héra et Athéna. Roehl, \emph{Inscr. graec. ant.}, 336.

*) Paton et Hicks, \emph{Inscr. of Cos}, 1891, n° 38.

*) Cf. Farnell, \emph{l. c.}, p. 294, n. \emph{e.} Sur le culte de Rhéa en Arcadie, cf. Immerwahr, \emph{Rhea Sage und Rhea Kult in Arkadien} dans \emph{Bonner Studien für Kekulé}, 1890, p. 188-193, et \emph{Die Kulte und Mythen Arkad.}, 1, p. 213 ss.

*) Paus. 8, 41, 2 ; 47, 3.

*) Callim., \emph{Hymn. in Jov.}, 10.

*) Paus. 8, 36, 3.

*) Lact. Plac., ad Stat., \emph{Theb.} 4, 292.

*) Dio. Chrys., \emph{Or.}, p. 59 s.

*) Cymbales de bronze, consacrées dans le temple de Zeus, mais qui étaient associées sans doute au rituel de la Meter : \emph{Bronzen v. Olympia}, texte, p. 70.

*) Paus. 4, 36, 6 (Messenia, statue de la mère des dieux) et 9 (megaron des Curètes, temple d'Eileithyia) ; 5, 8, 1 (Dactyles Idéens et Curètes à Olympie), 8, 37, 2 (naos de Demeter, autel de la Megale Meter à Lykosura) et 3 (Curètes, Corybantes) ; Cavvadias, \emph{Fouilles d'Epidaure}, n°s 40 (autel des Curètes), 64 (dédicace à Megale Meter Theôn) ; cf. Demeter et Héraclès, considéré comme l'un des Dactyles Idéens, à Megalopolis et à Mycalessos : Paus. 8, 31, 3 ; 9, 19, 5.

*) Paus. 3, 22, 4. Cultes de Britomartis, de Pasiphaé, d'Apollon Delphinios dans le sud de la Laconie. Farnell, \emph{l. c.}, p. 294, n. \emph{a.}

*) \emph{Hymn. hom.} 14, 1 ss. Sur l'importance de ce titre de Mère des dieux, qui implique un dogme et tout un système religieux, cf. Farnell, \emph{l. c.}, p. 291.

*) « The primary question must be first discussed whether this identification of Rhea with the Θεῶν Μήτηρ of the Greek mainland is an original fact explaining the religious dogma expressed by the title, or whether it is one of those later syncretisms so common in all polytheistic religions. » Farnell, \emph{l. c.}, p. 293. Rapp, dans Roscher, \emph{Myth. Lex.}, 2, p. 1660, s. v. Kybele, tend à distinguer la Mère des Dieux hellénique et la Rhéa Cybèle crétophrygienne. Farnell, qui tire parti des découvertes crétoises, défend la première hypothèse.

*) Sur les monuments figurés, la M. d. D. est souvent unie aux Nymphes, à un dieu fleuve et à Pan.

*) Julian., \emph{Or.} 5, p. 159 AB ; Suid., s. v. Metragyrtes ; Phot., s. v. Metrôon ; Schol. in Aristoph., \emph{Plut.} 431 ; cf. Hepding, \emph{Attis}, 1903, p. 135. Par contre, sur l'hypothèse qui rattache le nom même de l'Attique à un culte préhistorique d'Attis, cf. Eisler, \emph{l. c.}, p. 166, n. 149.

\subsection{2.}

Vers l'an 1500, selon la chronique de Paros, aurait eu lieu sur les monts Kybèles, dans le pays qui devint beaucoup plus tard la Phrygie, une miraculeuse apparition de la Mère des Dieux.* En Asie Mineure comme en Crète, la Grande Mère, Mâ, était au premier rang des divinités indigènes. Elle régnait sur les montagnes, dont les forêts de pins lui sont consacrées, sur les plateaux aux sources rares et aux maigres pâturages, sur les plaines fertiles. Les plus anciens sanctuaires que nous lui connaissons se dressent au bord de fontaines et d'étangs.* Ses plus anciennes images sont taillées dans la roche vive. Le lion est près d'elle ou sous ses pieds ; à Iasili-Kaïa, en Ptérie, son effigie humaine n'est même pas encore complètement dégagée du type léontomorphe.* La déesse mère est associée à un dieu jeune, imberbe,* qui est un dieu lumineux, et que l'on nomme Attis ou Atès.* Ce dieu est son fils. C'est pourquoi, dans la période d'expansion hellénique, Meter Lêto et Apollon pourront si facilement se substituer au couple primitif.* Nombreux et variés sont les souvenirs d'anciens cultes, qui se rattachent à la même religion que ceux de la Crète minoenne et de la Grèce mycénienne. En Lydie, un mythe situe sur le Tmolos la retraite de Rhéa et la naissance d'un Zeus Hyetios, seigneur de la pluie, gardé par les Curètes.* Dans la région du Sipyle, on montrait l'emplacement de deux villes disparues, qui furent Sipylos Idaia et la forteresse des Curètes.* A Éphèse, le culte des Curètes est associé à celui d'Artémis-Lêto. En Mysie, la Meter de Cyzique a pour acolytes les Dactyles Idéens crétois.* Et l'Ida troyen rappelle les liens antiques de l'Asie Mineure avec la Crète, qui remontent au temps de l'hégémonie crétoise sur le bassin de l'Égée.*

*) \emph{CIG.} 2374, l. 19.

*) A Boghaz-Koï (Cappadoce), Ibriz et Efflatum-Bunar (Lycaonie). Dans ces deux dernières localités, l'image regarde la source, qui lui baigne le pied : cf. Sarre, \emph{Reise in Phrygien} dans \emph{Arch. ep. Mitt.}, 1896, 19, p. 40 ss ; Reber, \emph{Die Phryg. Felsendenkm.}, 1897, p. 4 s. Au mont Sipyle, près de l'image de la Meter Plasténè, taillée aussi dans le roc, petit lac « fed by countless springs that gush from the rocky foot of the mountain » : Frazer, à Paus. 5, 13, 7, \emph{Comment.}, 3, p. 554 (bibliogr., p. 555 s) ; c'est probablement l'ancien lac Saloe.

*) Perrot et Chipiez, \emph{Hist. de l'art dans l'antiq.}, 4, fig. 320, p. 647. Sur le lion en Asie Min., cf. Otto Keller, \emph{Die antike Tierwelt}, 1, 1909, p. 24-61.

*) Perrot et Ch., \emph{ibid.}, p. 650 s ; 5, p. 216 (Iasili-Kaïa) ; Ramsay dans \emph{Ath. Mitth.}, 14, 1889, p. 172 (Fassiler).

*) Ramsay, \emph{Cities}, 1, p. 132 et 169. Le nom apparaît pour la première fois sous la forme \emph{Ates} (monument de Midas, cf. Perrot et Ch., \emph{op. l.}, 5, p. 82-89 et fig. 48). Démosthène, \emph{Pro Cor.}, 260, donne la forme Ἄττης, qui est aussi dans Hermesianax (cf. Paus. 7, 17) et Nicander, \emph{Alexipharm.}, 8 et schol. ; même forme rituelle du nom à Pessinonte, Paus. 1, 4, 5, à Dymé et Patras, au 2e s. de notre ère, Paus. 7, 17, 9 et 20, 3. Ἄτυς nom théophore dans les dynasties royales de Lydie, Hérodot. 1, 34 ; nom d'un prêtre de Cybèle dans Dioscorides, \emph{Anth. Pal.}, 6, 220 (Ἄτις potius quam Ἄτυς cod. ante rasuram). Ἄττις, dans le poète comique Theopompe, contemporain d'Aristophane : \emph{Fragm. Comic. Gr.}, éd. Meineke, 2, p. 801 ; \emph{Com. Att. fragm.}, éd. Kock, 1, p. 740. La consonne est redoublée dans les noms de plusieurs villes d'Anatolie, qui paraissent dérivés de celui du dieu, Attaia, Attouda, etc. Parmi les différentes étymologies que l'on a cherchées, cf. Arnob. 5, 6 : Attis = scitulus, mignon, en Lydie. Attis = Adôn, seigneur : Haakh dans \emph{Stuttgart. Philol. Versamml.}, 1857, p. 176 s ; Tomaschek dans \emph{Ak. Wien. Sitzungsber.}, 130, 1894, p. 42 ; cf. Eisler, \emph{l. c.}, p. 178, n. 176. Attis, contraction d'Agdistis = fils de la montagne : Ramsay dans \emph{J. of. Hell. St.}, 3, 1882, p. 56. Attis = père : Vanicek, \emph{Fremdwoerter im gr. und lat.}, 1878, s. v. Ἄττης ; Kretschmer, \emph{Einleitung in die Gesch. d. gr. Sprache}, 1896, p. 350 ss (avec les diverses formes du nom) ; Gruppe, \emph{Gr. Mythol.}, 1906, p. 1548.

*) Ramsay, \emph{op. l.}, p. 6 s, 130 s, 146, 192, 264, 305.

*) \emph{Anthol. Pal.}, 4, p. 511 ; Lyd., \emph{De mens.} 10, 71 (48) ; cf. Hyes Attes dans Demosth., \emph{l. c.}, et Jupiter \emph{Pluvialis}, \emph{CIL.} 9, 324 ou \emph{Pluvius}, Tib. 1, 7, 26 ; Usener, \emph{Goetternamen}, 1896, p. 46 s ; Farnell, \emph{op. l.}, 5, p. 124 s.

*) Strab. 1, 3, 17 et 12, 8, 21 ; Plin., \emph{H. n.} 5, 29.

*) Strab. 14, 1, 20, p. 640 ; cf. en Carie le culte de Zeus Cretagénès et des Curètes : Lebas-Waddington, \emph{Inscr.}, 5, 338, 394 et 496.

*) Apoll. Rh., \emph{Argon.} 1, 1, 1125-29 et Schol., qui signale la même association de Rhéa et des Δάκτυλοι Ἰδαῖοι Κρηταιέες.

*) Sur les rapports primitifs de la Crète et de la Phrygie, cf. Ramsay, \emph{Cities}, 1, pp. 94 et 358 ; Gruppe, \emph{op. l.}, p. 301 : Farnell, \emph{op. l.}, 4, p. 165 s. Une tradition faisait remonter à l'Arcadien ou Crétois Dardanos et à son fils Idaios l'introduction du culte métroaque dans la région de l'Ida troyen : Dion. Hal. 1, 61 ; Diod. Sic., 5, 49.

Plusieurs des monuments où figure soit le couple divin, soit la seule déesse, sont d'origine hétéenne, par conséquent antérieurs au 11e siècle. A l'époque indiquée par les marbres pariens, les Hittites, descendus probablement du massif du Taurus,* étendaient leur domination de l'Euphrate à la Méditerranée et à la mer Noire. Leurs princes traitaient d'égaux à égaux avec les rois de Babylone et les Pharaons d'Égypte, maîtres de la Phénicie et de la Syrie méridionale. Ce peuple a subi, dans ses conceptions religieuses, le double ascendant de l'Égypte et de la Babylonie* ; et les éléments de la civilisation qu'il apporte aux grossières tribus de l'Anatolie occidentale, c'est surtout aux Chaldéo-Babyloniens qu'il les emprunte. La pénétration profonde du sémitisme est donc antérieure aux deux invasions assyriennes, qui datent des 11e et 9e siècles. Au moment même où le monde mycénien s'écroule sous la poussée dorienne, nous voyons se consommer la ruine de l'empire hétéen. Mais l'influence sémitique est entretenue, à l'intérieur, par l'établissement définitif des Syriens Blancs sur la rive droite de l'Halys et, dans les régions côtières, par l'extension du commerce phénicien ; elle se renouvelle au 3e siècle par la fondation de colonies séleucides en Lydie et en Phrygie. Elle s'était exercée sur les rites et sur le mythe. Sans doute il faut lui attribuer l'introduction ou, du moins, le développement de l'eunuchisme.* Il est possible que l'interdiction du porc, dans la ville sainte de Pessinonte, soit un emprunt fait aux Sémites.* C'est au contact du mythe chaldéo-babylonien d'Ishtar et d'Adôn-Doumouzi, légué aux religions assyrienne et phénicienne, que le dieu fils s'est transformé en dieu favori.* L'Attis lydien est même tué, comme Adonis, par un sanglier.* Agdistis, le monstre androgyne, dérive à la fois de la chaldéenne Thalatth et de la phénicienne Astoreth.* Jusque sur les rives du Sangarios, on retrouve les vestiges de Nana, la dame d'Ourouk.* C'est enfin sous l'inspiration de la sémitique Reine des cieux que prit naissance le mythe de Mêter Basileia, sœur de Rhéa, identifiée elle-même à Rhéa-Kybelè, et mère d'Hélios et de Séléné.*

*) Cf. de Morgan, \emph{Les premières civilisations}, 1909, p. 281.

*) Ramsay, dans \emph{Ath. Mitth.}, 14, 1889, p. 175 ss, croit retrouver des influences assyriennes sur les monuments d'Ibriz, égyptiennes sur ceux de Ptérie ; cf. Sarre, \emph{l. c.}, et Reber, \emph{op. l.}, p. 3 ss, 18-24 (infl. mésopotamiennes sur la tombe dite des lions, Arslantasch, en Phrygie, 8e siècle ? ).

*) Farnell, \emph{op. l.}, 3, p. 297, n. \emph{a.} : « The orgiastic dances in Crete and Phrygia were officially performed by men or eunuchs. » Mais rien n'autorise à supposer la présence d'eunuques dans le culte de la Mère Cretoise. Par contre, il y a des eunuques dans le culte de l'Istar-Nana d'Ourouk et dans celui de la déesse syrienne Atergatis, qui est une forme tout à fait sémitisée de la dame aux lions et dont le mythe conserve le souvenir d'influences babyloniennes. Sur les eunuques dans le culte phrygien, cf. infra, le chapitre consacré aux Galles.

*) Sur l'interdiction du porc à Pessinonte, cf. Paus. 7, 17 et Frazer, \emph{Comment.}, 4, p. 137 ; dans les mystères d'Attis, Julian., \emph{Or.} 5, p. 177 BC ; Hepding, \emph{Attis}, p. 157 ; cf. Keller, \emph{op. l.}, p. 401. Ramsay, \emph{Historical Geogr. of Asia Minor}, p. 32, voit dans cette proscription rituelle une trace de la domination exercée par les Syriens de la Ptérie. Même exclusion dans les cultes de la déesse syrienne : Lucian., \emph{Dea syr.} 54 ; de Mâ, à Comana du Pont : Strab. 12, 8, 9, p. 575 ; de Mên Turannos : Foucart, \emph{Assoc. ret. en Grèce}, p. 219, n° 38, Dittenberger, \emph{Syll. inscr. gr.}, 379 ; d'Alectro à Ialysos (Rhodes) : \emph{ibid.}, 357, l. 25 ss ; d'Hemithea à Castabos (Chersonèse de Thrace) : Diod. Sic. 5, 62. Mais elle se retrouve aussi en Crète, où elle est rattachée au mythe de Zeus enfant, nourri par une truie : Agathocles de Babylone dans Athen., 9 p. 375 F-376 A ; cf. Cook, \emph{Animal worship, etc., l. c.}, p. 152-154. Il est donc possible aussi qu'elle soit très antérieure à l'influence syrienne dans le culte de la Meter. D'autre part Ramsay, en établissant une distinction trop nette entre mangeurs et contempteurs de porc (pig eaters, pig haters), ne tient pas compte de ce fait que l'horreur religieuse d'un animal est une autre forme de vénération ; cf. Robertson Smith, \emph{Religion of the Semites}, 2° éd., p. 152 ss, 448 ss ; Frazer, \emph{Golden Bough}, 2° éd., 2, p. 304 ss. L'exemple de la Crète montre nettement l'identité de l'animal sacré et de l'animal impur. Dans un tumulus de Bos-Euiuk, en Phrygie, on a découvert une mâchoire de porc : \emph{Ath. Mitth.}, 24, p. 1-45 ; MM. Koerte, \emph{Gordion}, 1904, p. 8 s, en concluent que les constructeurs du tumulus étaient aryens ; ils ne tiennent pas davantage compte du principe mis en valeur par Robertson Smith.

*) Cf. Vellay, \emph{Culte et fêtes d'Adonis-Thammouz}, 1904, chap. 1: \emph{La légende d'Adonis}, et 2 : \emph{L'exode du culte} ; Frazer, \emph{Golden Bough}, 3e éd., 4, 1907 : \emph{Adonis, Attis, Osiris, Studies in the history of oriental religion}. D'autre part, la nature bisexuée d'Agdistis rappelle des croyances analogues au sujet d'Astarté.

*) Paus., \emph{l.c.}

*) Cf. Gruppe, \emph{Die griech. Kulte und Mythen}, p. 515.

*) A moins que ce nom de Nana ne soit un vieux mot indigène signifiant « mère » ; cf. Gruppe, \emph{Gr. Mythol. und Religions Gesch.}, p. 1525, n. 4, et p. 1527.

*) Diod. Sic. 3, 57.

L'écroulement de l'empire hétéen avait d'autre part ouvert les voies à l'invasion phrygienne. Les Phrygiens ou Briges, tribus aryennes venues de Thrace, occupaient depuis le 15e siècle le littoral asiatique de l'Hellespont et de la Propontide.* Ils s'y trouvaient encore vers le milieu du 9e siècle, au temps d'Homère, qui les situe sur la frontière orientale de l'état troyen. Ce fut la pression d'autres envahisseurs thraces qui les refoula vers le plateau central. De longue date, par conséquent, ils avaient appris à connaître et à vénérer la Grande Mère d'Anatolie, redoutable comme ses fauves quand ils ravagent les troupeaux, bienfaisante comme les sources qu'elle empêche de tarir pendant les ardeurs excessives des étés. Tant qu'ils furent groupés en royautés cantonales, chaque tribu sans doute assimila sa propre déesse mère à la puissante divinité indigène. Si le nom d'Adrastos est thraco-phrygien,* l'Adrasteia mysienne, tantôt considérée comme une nymphe de l'Ida, suivante de la Mère Idéenne, tantôt identifiée à la Mêter, serait l'un des plus anciens exemples de cette assimilation. De même Ippa est une nymphe du Tmolos, la nourrice de Dionysos ; mais son nom est aussi l'un des vocables de la Grande Mère.* Mêter Mida ou Misa,* ancêtre mythique d'une dynastie royale, se confondit avec l'androgyne Agdistis de Pessinunte, qui est l'une des formes les plus barbares de la Mère des Dieux, issue du rocher Agdos. La Bérécynthienne est la Mère de la tribu des Bérécynthes.* D'autres tribus auraient importé le terme de Dindymène, qui désigne la déesse comme Dame du Dindymos, c'est-à-dire encore de la montagne.* A l'organisation d'une forte monarchie, réunissant les petits royaumes phrygiens sous le sceptre des Midas, correspond une nouvelle phase d'activité du culte métroaque. La Grande Mère préphrygienne va devenir la divinité nationale, par excellence, du peuple phrygien : MEGALÈ METER PHRYGIA.* C'était au fondateur même de ce nouvel empire que les Pessinuntiens attribuaient la construction de leur Metrôon.* Certains dynastes ont ajouté, ce semble, et finalement substitué à leur nom théophore celui d'Atès.* On peut supposer qu'ils s'étaient emparés, comme feront plus tard les rois galates, du sacerdoce majeur de la Mêter. Ainsi la royauté bénéficiait de riches trésors et d'une protection puissante. Mais elle reconnaissait en même temps et imposait la suprématie de la déesse autochtone. Des mythes peu à peu se constituèrent pour fondre ensemble les deux religions, qui n'étaient pas sans analogie. Les dieux thraco-phrygiens de la végétation, tel que Sabazios-Dionysos, ceux qui président aux sources, tel que Midas, entrèrent dans le cycle de la Mêter anatolienne. On les lui donna même comme fils.* Elle reçut pour époux le Zeus Tonnant des Thraces.* Attis fut qualifié de Pappas,* ce qui est aussi le nom du Zeus bithynien et du dieu céleste des Scythes. Pour consacrer la divine alliance, un mythe montre Dionysos admis aux orgiasmes de Rhéa, sur les monts Kybèles.* Ce qui est certain, c'est que les nouveaux maîtres du sol pratiquaient eux-mêmes une religion orgiastique, et qu'elle dut singulièrement leur faciliter l'adoption des dieux et des rites indigènes. Paysans fanatiques, au vieux culte naturaliste, qu'ailleurs épuraient et civilisaient les Hellènes, ils conservèrent toute son originelle barbarie.

*) Cf. Perrot et Chipiez, \emph{op. l.}, 5, chap. 1 : \emph{La nation phrygienne} ; Tomaschek, \emph{Die alten Thraker}, 1, dans \emph{Ak. Wien. Sitzungsber.}, 128, 1893, p. 1 ss ; G. und A. Koerte, \emph{Gordion}, p. 1-27 : \emph{Geschichte Phrygiens}.

*) Hérodot. 1, 35. Pour les étymologies proposées, cf. Hepding, \emph{Attis}, p. 101, n. 6. Le nom d'Adrasteia était interprété par les Grecs comme une épithète de Némésis, l'inévitable, mais dut avoir originairement une signification différente. Neustadt, \emph{De Iove Cretico}, inaug. dissert. Berlin, 1906, cherche à prouver qu'Adrasteia est une divinité d'origine phrygienne, hypostase de la Meter Idaia. Sur le culte, cf. Roscher, \emph{Myth. Lex.}, s. v. Adrasteia ; Ramsay, \emph{Cities}, 1, p. 169.

*) Cf. Dieterich dans \emph{Philologus}, 52, 1894, p. 5 ; Farnell, \emph{op. l.}, 3, p. 306.

*) Dieterich, \emph{ibid.}, p. 1-12 : identité de Mida Meter, mère de Midas, et de l'androgyne Mise, que célèbre l'hymne orphique 42. De môme, Adrasteia est considérée comme androgyne dans \emph{Orph. frg.} 36, Abel.

*) Strab. 14, 5, 29, d'après Xanthos de Lydie (5e s.), nous apprend qu'il existait sur la rive européenne du Pont Euxin un pays des Bérécyntes, point de départ de l'émigration phrygienne. En Asie Mineure, les B. constituaient une tribu à part, 10, 3, 12, mais qui avait disparu sans laisser de traces, 12, 8, 21. Cette tribu s'était peut-être implantée dans le haut bassin du Méandre, sur les rives du Marsyas ; c'est là que s'élevait, d'après Ps. Plut., \emph{De flum.}, 10 « le mont Bérékynte, ainsi nommé de Berekynthos, premier prêtre de la Mère des dieux. » Servius place cette montagne, avec un château du même nom, près du Sagaris. Mais il y eut sans doute plusieurs monts B., comme il y avait plusieurs Dindymes. D'autre part, d'après Diod. Sic. 5, 54, 5, il aurait existé en Crète un mont B., habité par les Dactyles ; cf. Gruppe, \emph{Gr. Mythol. u. Relig.}, p. 311 et 1528. Ce mot est généralement considéré, toutefois, comme synonyme de Phrygiens. Certains étymologistes croient y reconnaître un titre honorifique, = les Brillants, les Illustres, que se seraient donne les Phrygiens : cf. Fick, \emph{Spracheinheit d. Indogerm. Europas}, p. 412 ; Vanicek, \emph{Fremdwoerter im gr. u. lat.}, 1878, s. v. ; Kretschmer, \emph{Einleitung in d. Gesch. d. gr. Sprache}, 1896, p. 193.

*) Dindymè, épouse du roi de Phrygie et de Lydie Maiôn, mère de Kybélè : Diod. Sic. 3, 58. Un mont Dindymos à Pessinunte, Strab. 12, 5, 3 ; à Cyzique, 12, 8, 11 ; aux sources de l'Hermos, Hérodot. 1, 80 et Strab. 13, 4, 5. Ptolem. 5, 2, qualifie de Dindyme toute la chaîne qui s'étend du Sangarios à la Propontide. Tomaschek, dans \emph{Ak. Wien. Sitzungsber.}, 131, 1894, p. 33 et 72, considère le mot Dindyme comme une forme plus récente de Dindryme, qu'il rapproche du sanscrit « druma » = arbre, bois ; cf. la même réduplication dans le grec δένδρον. Kretschmer, \emph{op. l.}, p. 194, suppose un mot phrygien « dindu » = hauteur, tandis que J. et Th. Baunack, \emph{Stud.} 1, p. 198, interprètent Dindyménè comme un équivalent de Chtonia.

*) Strab. 10, 3, 12 : Φρυγίαν θεὸν μεγάλην. \emph{CIG.} 2, 2107 b : Μητρὶ Φρυγίαι. \emph{Orph. Hymn.} 42, 6 ; Julian., \emph{Ep.} 21.

*) Arnob., \emph{Adv. gent.} 2, 73 ; cf. Diod. Sic. 3, 59, qui dit simplement que Midas contribua beaucoup à la magnificence du temple et des fêtes de Pessinunte. Sur le Midas mythique et les Midas historiques, cf. Koerte, \emph{Gordion}, p. 14 ss.

*) Le nom d'Atès figure sur le tombeau d'un Midas ; cf. supra, p. 10, n. 3. Ces noms théophores supposent des états à constitution sacerdotale.

*) Strab. 10, 3, 15 : Sabazios = fils de la Bonne Mère ; cf. Ramsay, \emph{Cities}, 1, p. 272. Hygin., Fab. 191 : « Midas filius Matris deae » ; 274 : « Cybèles filius Phryx » ; cf. Suidas, s. v. ἔλεγος.

*) L'aigle de Zeus Brontôn figure très souvent sur les tombeaux phrygiens et apparaît sur certains monuments métroaques de l'époque gréco-romaine. Sur l'aigle et Midas, cf. Koerte, \emph{l. c.}, p. 12 s.

*) Diod. Sic. 3, 58 ; Hippolyt., \emph{Refut. omn. haeres.}, 5, 9 ; Arrian., \emph{Bithyn.} = \emph{Fragm. Hist. Gr.}, 3, p. 592, 30 ; Ramsay dans \emph{J. of Hell. St.}, 5, 1884, p. 257, n° 8 (inscr. de Nacoleia, 2e s. de notre ère) ; A. Koerte dans \emph{Ath. Mitth.}, 22, 1897, p. 32 n° 8 (inscr. de Bejad, Phrygie) ; \emph{CIL.} 5, 766 (inscr. d'Aquilée : Atte Papa). Sur l'identité d'Attis Papas et de Zeus, cf. Hepding, \emph{Attis}, p. 193 et 213.

*) Apollod., \emph{Biblioth.}, 3, 5, 1 = Mythogr. gr., ed. Wagner, 1894, 1, p. 116. D'autre part, Diod. Sic. 3, 59, nous montre Cybèle se rendant, après la mort d'Attis, chez Dionysos à Nysa.

Dans le bassin supérieur du Sangarios ou Sagaris,* centre delà monarchie phrygienne, de nombreux monuments attestent encore la souveraineté de la Mêter aux 8e et 7e siècles.* Son culte restait lié à celui des pierres ; et l'icône était inhérente à la roche sacrée, demeure de la divinité. Grossièrement sculptée sur des rocs isolés, en des niches qui continuent la tradition des grottes cultuelles, tantôt assise et tantôt debout, vêtue d'une longue robe et coiffée d'une haute tiare, seule ou accompagnée de ses lions, qui sur ses épaules posent leurs pattes de devant, elle régnait sur les vivants et sur les morts. Maîtresse de toute vie, elle est entourée d'autels où elle recevait les hommages de ses féaux, et qui sont taillés aussi dans le rocher. Maîtresse de la mort, elle se dresse au milieu des nécropoles. Ainsi que dans la Crète minoenne, son culte est associé au rituel de la tombe.* Sur les roches où se creusent les sépultures des défunts héroïsés, figurent les lions ; elles sont de véritables sanctuaires, placés sous l'invocation de la Mère des Dieux et des hommes. Le nom rituel de la Dame, celui que les populations primitives transmirent aux Phrygiens, est gravé sur l'un des autels : \emph{Matar Kubile}.* Il reparaît dans la littérature ionienne du 6e siècle sous les formes de Kubêlis, Kubêkè, Kubêbè.* C'est ce dernier nom qu'elle porte en Lydie,* au temps des Mermnades (687-546). Kubêbè est aussi la grande déesse nationale des Lydiens,* la mère de leurs montagnes, où déjà sans doute pousse la vigne, la mère de leurs cours d'eau, qui fécondent les riches plaines et roulent de l'or. Elle est la mère divine de leurs rois, qui de dynastie en dynastie prennent les appellations théophores d'Atys, Sadyattès, Alyattès, Attalès. Mais chez ce peuple de commerçants, qui entretient de constantes relations avec l'Asie sémitique et dont la capitale était devenue celle de l'Orient, dans le voisinage immédiat des cités ioniennes, où se renouvelle l'art d'Anatolie, le type de Cybèle s'est quelque peu transformé.* Debout, tenant de chaque main un lionceau, parfois un bouquetin, ou l'un de ces oiseaux qui pullulaient dans les marais de la côte, la déesse est ailée. Si ce type lydo-ionien dérive, à n'en pas douter, du prototype minoen de la Rhéa chasseresse et dompteuse de fauves, il y ajoute l'attribut des ailes recoquillées, dont l'origine orientale n'est pas moins certaine. On le retrouve sur une stèle de Dorylée (vers 560), sur des bijoux de Rhodes et de Chypre (VIIe-VIe s.), sur une tablette votive de Corinthe. L'art industriel des céramistes et des bronziers le propage jusque dans la Grande Grèce et en Étrurie. Après les conquêtes de Cyrus, qui transforment l'empire lydien en satrapies du royaume de Perse, sa vogue diminue. Elle cesse au début des guerres médiques. Dès le commencement des hostilités, Ioniens et alliés, en incendiant Sardes (500), avaient détruit le principal temple de la Kybêbè lydienne. Il semble que les Perses l'aient identifiée à leur déesse orgiastique de la nature, Anaïtis, dame des eaux, dont on fît plus généralement un Artémis. A l'Artémis persique, fort honorée en Lydie,* fut sans doute réservée cette figuration particulière de la reine des fauves. Aussi bien l'art grec d'Asie Mineure, de plus en plus libéré des influences de l'Orient, avait-il déjà produit un autre type, plus conforme au caractère d'une divinité matronale et à la tradition même. Certaines images, de fabrication ionienne et paraissant dater du 6e siècle, représentent la déesse assise, comme la statue colossale du mont Sipylos ; et sur ses genoux est couché un lionceau.* Mais les effigies anthropomorphes n'avaient pas remplacé partout les pierres brutes. La litholâtrie des âges préhistoriques, où la Mère était adorée dans ses sanctuaires naturels, se perpétuait dans les temples des villes. Parfois l'idole était un aérolithe, considéré comme un éclat de foudre,* parfois un fragment de roche volcanique, affectant une forme grossière de pyramide ou de cône, d'autres fois une pierre roulée par les eaux, qui offrait de vagues ressemblances avec un visage.* Le Metrôon de Pessinonte, en Phrygie, renfermait un ou plusieurs de ces bétyles.

*) La forme Sangarios, Σαγγάριος, apparaît dans l'Iliade et dans Hésiode (en latin Sangarius, dans Tite Live) ; elle est également employée par Strabon. La forme Sagaris est donnée par Ps. Plut., \emph{De flum.}, 12, 1 (cf. un personnage nommé Sagaris, IGSI, 688). Celle-ci serait la forme archaïque : Koerte dans \emph{Ath. Mitth.}, 25, 1900, p. 441. Aujourd'hui : Sakaria.

*) Cf., en particulier, dans la vallée de Doganlu (cité de Midas, Iasili-Kaïa) : Ramsay dans \emph{J. of Hell. St.}, 3, 1882, p. 41 s. et fig. 9 ; Perrot et Chipiez, \emph{op. l.}, 5, p. 151 et fig. 107 ; Reber, \emph{op. l.}, p. 55 s. et phot. rectifiant le dessin de Ramsay ; nombreux sanctuaires et autels en plein air ; figure de la Meter, assise de face, tenant dans la main gauche une phiale ; la tête, qui était peut-être peinte, ressemble simplement à un disque. --- A une vingtaine de km. au sud-sud-ouest, près d'Arslan-Tasch : Ramsay, \emph{ibid.}, 9, 1888, p. 372 s. ; Perrot et Ch., \emph{l. c.}, p. 158, fig. 111 ; Reber, \emph{op. l.}, p. 57 ; dans une niche à fronton, haute d'env. 2 m., figure fruste de la Meter, debout, avec coiffure très élevée ; autel à quatre faces, constitué par un rocher isolé, auquel on accédait de tous côtés par des degrés. --- Plateau d'Arslan-Kaia (rocher des lions), près de Liyen, à 7 km. env. au sud de Düwer : Ramsay, \emph{ibid.}, 5, 1884, p. 245 ; Perrot et Ch., \emph{l. c.}, p. 156 s., fig. 110 ; Reber, p. 31-35 (description très précise, avec bonne héliogravure) ; au fond d'une niche rectangulaire, creusée dans un rocher isolé, haute de 2 m. 35, large de 2 m. 20, profonde de 2 mètres, deux lionnes, affrontées et dressées sur leurs pattes de derrière, appuient leurs pattes de devant sur la tête de la déesse, coiffée d'une haute tiare ; le contour subsiste seul, mais il semble que les mains étaient appliquées sur le buste ; dans le fronton triangulaire, deux sphinx ailés ; sur les faces latérales, à g. lion ( ? ) marchant, à dr. lion debout. Sur le même plateau : Reber, p. 58 et fig. 10 ; dans une niche à fronton triangulaire, à laquelle on accède par quatre marches, la déesse debout. Voir aussi Koerte, \emph{Gordion}, p. 219-226 : excursus 1, \emph{Die phrygischen Felsendenkmaeler}.

*) Ramsay, \emph{A study of Phrygian art} dans \emph{J. of Hell. St.}, 9, 1888, p. 352 : « The tomb which in the earliest time took the form of a shrine of the Goddess. »

*) Près d'Arslan-Tasch, nécropole d'Ayazinn. Perrot et Ch., \emph{op. l.}, 5, p. 32, fig. 3, d'après Ramsay. L'ingéniosité des étymologistes anciens et modernes s'est exercée aussi sur le nom de Kybelè. Festus, s. v. \emph{Cybebe} : « mater quam dicebant magnam, ita appellabatur quod ageret homines in furorem, quod Graeci κύβηβον dicunt » ; les Cybêbes sont les Galles, et ce sont eux qui ont pris le nom de leur déesse. D'autres l'ont rapproché du verbe grec κυβιστᾶν, qui exprime les violents mouvements de tête des Galles dans leurs danses orgiastiques ; cf. Serv., \emph{ad Aen.} 3, 111 et 10, 220. Mordtmann, \emph{Die altphryg. Sprache} dans \emph{Ak. Muench. Sitzungsber.},1862, p. 32, le rapproche d'un mot kobjel = lat. « polire, » à cause de l'aérolithe de Pessinonte ( ? ). Mais voici qui est, du moins en apparence, plus sérieux. En raison d'un passage d'Hésychius (Κύβελα ὄρη Φρυγίας καὶ ἄντρα καὶ θάλαμοι), ce nom a été rattaché tantôt à l'hébreu Gebel, cf. Gabal, Djebel en arabe = montagne, tantôt à l'hébr. et ar. Qubba = voûte, salle voûtée, par suite grotte : Lewy, \emph{Die semit. Fremdwoerter im griech.}, 1895, p. 249. Déjà dans l'antiquité on faisait dériver Kubelè de κύβος = eubus, ἀπὸ τοῦ κυβικοῦ σχήματος (cf. Pessinonte dérivé de πεσσός, pierre cubique, dé : Kretzschmer, \emph{op. l.}, p. 402). Eisler, \emph{Kuba-Kybele}, dans \emph{Philologus}, 68, 1909, p. 118-151 et 161-205, essaie d'appliquer la méthode de la mystique verbale, ou onomatomancie, pour rendre compte des divers vocables de la Meter anatolienne ; il y retrouve constamment des influences proto-sémitiques ; Kybélè ne serait qu'un diminutif de Kuba (cf. Cuba et Cubola, constructions arabes, à Palerme) ; intéressantes comparaisons entre la pierre noire de Pessinonte et la Ka'aba de la Mecque.

*) Les deux premières formes dans Hipponax d'Éphèse, \emph{Fr.} 120 et 121 ; la dernière dans Charon de Lampsaque, puis dans Hérodote au 5e s. Celle-ci se retrouve dans les \emph{Anacreontea}, 12 (11), 1 et 53 (51), 5. Elle redevient à la mode dans la religion archaïsante des 3e et 4e s. de notre ère : \emph{CIL.} 6, 513, « deam Cybeben » (dédicace d'une statue par un clarissime) ; cf. Festus. \emph{l. c.} ; P. Optat. Porphyr., \emph{Carm.} 27, 9 (ed. L. Mueller) ; Prudent., \emph{Peristeph.} 10, 196 ; Claudian. 15 (\emph{de bello Gildon.}, 1), 120, 18 (\emph{in Eutrop.}, 1), 277 ; Hesych., \emph{Lex.}, s. v. Κυβήβη ; \emph{Aegrit. Perdicae}, 29, dans \emph{P. Lat. Min.}, éd. Baehrens, 5, p. 112. La forme Κυβέλη, qui est plus généralement adoptée, apparaît dans Pindare, \emph{Frg.} 80, Bergk.

*) Homère ne semble pas connaître le nom des Lydiens, qui sont étroitement liés alors aux Phrygiens. Maiôn, père de Cybèle, est dit roi de Phrygie et de Lydie ; cf. le rôle du Pactole dans le mythe de Midas.

*) Hérodot. 5, 102 (à Sardes), ἱρὸν ἐπιχωρίης Θεοῦ Κυβήβης. \emph{Soph., Philoct.} 391 ss, invocation à la Potnia Meter, dame aux lions, qui habite les rives du Pactole. Les deux colonnes ioniques, encore debout sur la rive dr., et les débris gisant sur le sol sont-ils les ruines du Metrôon ? En tout cas, ils ont appartenu à un édifice qui n'est pas antérieur aux Séleucides ; bibliogr. dans Radet, \emph{Lydie}, p. 295.

*) V. l'étude de Radet, \emph{La déesse Cybèbé d'après une brique de terre cuite récemment découverte à Sardes}, dans \emph{Rev. Etud. anc.}, 1908, p. 109-160 et pl. 11.

*) Les rois de Perse avaient établi des colons le long de la grande route royale de Suse à Sardes.

*) Cf. les statuettes en pierre calcaire trouvées à Marseille, « apportées sans doute de Phocée ou de quelque autre ville de l'Asie Mineure » (Heuzey, \emph{Catal. des figur. antiques du Louvre}, 1882, p. 239) ; elles ont été rapprochées à juste titre des statues de Kymé et d'une statue provenant de la nécropole de Milet (S. Reinach dans \emph{Bull. Corr. Hell.}, 13, 1889, p. 548). Une terre cuite qui paraît dater du 6e s., représentant la déesse avec un lionceau sur les genoux et provenant d'Athènes, doit avoir la même origine ionienne : \emph{Arch. Anzeiger}, 1895, p. 129 (au musée de Berlin). Statuette de même type, provenant de Myrina ( ? ) : \emph{ibid.}, 1892, p. 106. Autre, provenant de Smyrne : Froehner, \emph{Catal. de la coll. Gréau}, in-12, p. 199, n° 1326.

*) Les Grecs nommaient kéraunies ces pierres tombées du ciel. C'est un de ces aérolithes que Pindare aurait vu tomber à ses pieds avec fracas et éclair : Schol. ad \emph{Pyth.} 3, 137.

*) Sur le caractère sacré de certaines pierres trouvées dans l'eau, cf. Ps. Plut., \emph{De flumin} 9 (le Méandre), 10 (le Marsyas), 12 (le Sangarios), 13 (le Scamandre). A toutes ces pierres se rattache une superstition relative à la Mère des dieux. Dans le Sangarios, une pierre nommée autoglyphe porte, naturellement empreinte, l'image de la déesse.

Depuis l'invasion cimmérienne de 696, la Phrygie n'existait plus en tant que puissance politique. Elle fît partie de l'empire lydien, qui peut-être lui laissa quelque indépendance. Sous la domination des Perses, elle disparut à peu près de l'histoire. Aux yeux des Grecs, elle ne sera plus désormais qu'une pourvoyeuse d'esclaves. Mais elle restait le domaine privilégié de la Mère des Dieux. A tous les régimes, jusqu'au début de l'empire romain, survit un petit royaume dont elle est vraiment la souveraine. Pessinonte, qui a grandi autour d'un temple d'Agdistis et où se localise la légende mythique de Cybèle et Attis, en est la capitale.* Les archiprêtres du Metrôon, qui prennent le nom du dieu Attis, en sont les rois.* L'enceinte sacrée abrite tout un peuple de prêtres, de prêtresses, d'hiérodules et de Galles, qui sont des ministres eunuques. Si le territoire de cet état théocratique ne dépassa jamais de beaucoup, sans doute, les hauteurs voisines, sa renommée fut considérable. Située sur l'une des grandes routes de caravanes,* Pessinonte était à la fois une ville sainte et un gros emporium. Foires importantes et marchés y coïncidaient avec les pèlerinages. C'est à l'équinoxe du printemps qu'avaient lieu les principales fêtes. Après les jours de deuil, où l'on déplorait le trépas d'Attis, une explosion de réjouissances commémorait sa résurrection, c'est-à-dire la renaissance de la nature et le rajeunissement du soleil. « Le climat du plateau d'Anatolie est extrême. L'hiver y est rude, long, glacé ; les pluies du printemps développent soudain une floraison vigoureuse, que grillent les ardeurs de l'été. Les brusques contrastes de cette nature, tour à tour généreuse et stérile, éclatante et morose, provoquaient des excès de tristesse et de joie inconnus dans les régions tempérées. Les Phrygiens pleuraient désespérément la longue agonie et la mort de la végétation ; puis, lorsqu'en mars la verdure reparaissait, ils s'abandonnaient à toute l'exaltation d'une joie tumultueuse. Des rites sauvages exprimaient la véhémence de ces sentiments opposés.* » Au milieu des clameurs, dans le vacarme des tambourins et des cymbales, que domine le bruit strident des flûtes, ce sont courses folles sur la montagne, danses échevelées autour des autels, flagellations, mutilations, sacrifices de sang humain, baptêmes de sang. Le plus barbare de ces rites est le sacrifice volontaire de la virilité, qui fait entrer le dévot dans le saint ordre des Galles et lui confère le don de prophétie. Le mythe engendré par ces orgiasmes en reflète le caractère d'ardente passion et de cruelle sauvagerie.* Cybèle s'était éprise d'un amour insensé pour Attis, que les populations pastorales des plateaux conçoivent comme un jeune berger. Mais la jalouse déesse exige de ceux qu'elle aime un attachement exclusif ; et qui lui résiste devient dément. Dans une crise de fureur mystique, le beau pâtre s'émascule au pied d'un pin et meurt de sa blessure. Autour de cette fable, qui a fini par constituer le noyau central du mythe, s'accumulent, se juxtaposent, se combinent une foule d'éléments hétérogènes. Persistance de traditions locales, renouvellement des couches ethniques, influences de religions voisines en expliquent assez la rare complexité. Nous possédons un état de la légende au début du me siècle.* La roche Agdos y a conservé son rôle, comme génératrice de la Grande Mère qui se dédouble en Agdistis ; elle restait de même l'objet d'un culte. On y voit figurer aussi la montagne, la grotte, le fleuve Sagaris, le torrent Gallos, les sources, les forêts, certains arbres sacrés, pin, grenadier, amandier. Nana la babylonienne y devient une nymphe, fille du roi Sagaris. Le thrace Dionysos y introduit son vin qui enivre ; et le phrygien Midas y veut marier sa fille, autre nymphe, avec le bel Attis. Celui-ci est fils de l'arbre issu du principe mâle de l'androgyne Agdistis. Ses origines remontent donc à la roche Agdos, identifiée à la Grande Mère. Le mythe du dieu favori n'avait point fait perdre complètement la primitive notion du dieu fils.

*) Cf. Hepding, \emph{Attis}, p. 125 s. On y montrait aux pèlerins le tombeau d'Attis : Paus. 1, 4, 5.

*) Ces petites principautés théocratiques, constituées autour d'un sanctuaire, étaient fort nombreuses en Asie Mineure ; cf. les domaines sacrés de Comana, Venasa, Tyana en Cappadoce ; de Comana, d'Ameria (à Mên Pharnakou) dans le Pont ; de Zizima (Dindyma) en Lycaonie ; d'Antioche (à Mên-Askaenos) en Pisidie ; les villages sacrés d'Atyokhorion, Attoukomé, Menokomé, \emph{etc.}

*) P. était sur la route royale de Ptérie.

*) Cumont, \emph{Les religions orientales dans l'empire romain}, 1907, p. 62.

*) Sur les diverses formes du mythe, v. Hepding, \emph{op. l.}, chap. 2.

*) Timolh. dans Arnob. 5, 5-7, « ex reconditis antiquitatum libris et ex intimis mysteriis, quemadmodum ipse (Timotheus) scribit insinuatque. » L'Eumolpide Timothée fut appelé à Alexandrie par Ptolémée 1er.

Orgiasmes et extatisme, aberrations sexuelles dont témoignent les légendes d'Agdistis et d'Attis, tendance même de cette religion au sacerdotalisme, tout en elle répugnait à l'esprit hellénique.* Moins exposés que les Grecs d'Asie Mineure à l'influence du génie oriental, ceux d'Europe éprouvaient pour le culte phrygien une réelle antipathie. Ménandre et Démosthène expriment, à cet égard, le sentiment national des vrais Hellènes. La pythagoricienne Phintys avait inscrit dans ses enseignements que toute femme honnête doit fuir les orgies métroaques.* Jusque dans les îles les plus proches de la côte levantine, la loi de certains temples interdit aux Galles l'accès du Metrôon ; et les femmes ne peuvent y accomplir aucun rite orgiastique.* Avant la fin de la République romaine, nous ne connaissons pas une seule ville de la Grèce propre qui ait accordé à Cybèle et à ses rites le droit de cité.* Mais, en dépit de cette aversion naturelle, qui lui ferme les religions d'état, ni la pensée ni l'imagination grecques n'ont pu se soustraire entièrement à son attirance. Ses zélateurs pratiquent des mystères qui préoccupent les théologiens ; car on y révélait une sagesse divine. C'est l'Eumolpide Timothée, l'un des fondateurs du culte alexandrin de Serapis, sous Ptolémée Ier, qui nous a transmis les détails du mythe pessinontien. Pour ses essais de réforme religieuse il avait voulu s'en instruire avec soin. Aussi bien les prêtres phrygiens surent-ils mettre à profit la savante curiosité de ces penseurs, qui réagit heureusement sur les mystères de la Mère des Dieux. Ils s'assimilèrent, au cours des siècles, beaucoup de spéculations empruntées à la Grèce* ; ainsi se préparait la prodigieuse fortune de leur religion. Celle-ci, d'autre part, en raison même de son mysticisme enthousiaste et passionné, avait depuis longtemps séduit certaines âmes de poètes. Elle avait inspiré avec succès le lyrisme de Pindare et d'Euripide. Pindare, qui écrit entre 501 et 446, confond dans ses chants la Mère des Dieux hellénique et la Kybélè de Phrygie.* Il les confond aussi dans sa dévotion. Avec l'approbation de l'oracle de Delphes, il avait tenté d'introduire à Thèbes le culte privé de la Dindymène ; il hellénisait toutefois l'étrangère en l'unissant au dieu Pan. Euripide, dans sa tragédie des Bacchantes (en 405), la rapproche au contraire du thraco-lydien Dionysos et lui conserve mieux son aspect exotique. « Heureux, » proclame le chœur de ses Bacchantes lydiennes, « heureux le mortel qui, possédant la science sacrée, s'abandonne sur les montagnes à de pieux transports ! qui, pour purifier sa vie et se sanctifier, célèbre selon le rite les orgies de la Grande Mère Kybélè* ! » Passagère exaltation d'artiste ; le poète s'est laissé gagner par l'émotion qu'il veut traduire. Mais il n'est pas surprenant que la déesse ait conquis dans le peuple quelques adeptes. C'est après les guerres médiques et précisément du vivant de Pindare, vers le milieu du Ve siècle, que se manifeste en Grèce une première invasion de dieux barbares. « Athènes avait porté sur toutes les côtes de la mer Égée ses flottes et ses colonies ; les marins et les soldats rapportèrent dans leur patrie les religions de la Thrace, de la Phrygie, de Cypre ; les nombreux étrangers que le commerce attirait au Pirée en furent aussi les propagateurs.* » Cybèle comptait de plus à son service les Métragyrtes ou mendiants de la Mère, véritables missionnaires de sa foi. Ils débarquaient avec les denrées du Levant et, de ville en ville, colportaient l'icône de leur Dame, leurs rites et leurs magies. Vers 430, d'après une tradition plus ou moins légendaire, les Athéniens mirent à mort l'un d'entre eux ; il exerçait sa propagande avec trop de fanatisme et avait initié déjà plusieurs femmes.* Une quinzaine d'années après, en pleine agora, un homme sautait sur l'autel des Douze Dieux et s'y coupait les organes génitaux avec une pierre* ; c'était un fanatique de la Grande Mère, et il accomplissait sur lui le sanglant sacrifice des Galles. Vers le même temps, Aristophane nous apprend que de jeunes débauches se faisaient admettre aux mystères de la Despoina Kybélè ; dans une autre comédie, un personnage a recours aux sortilèges des Corybantes.* L'orateur Eschine était un myste de Sabazios et d'Hyès Attès ; sa mère occupait dans la hiérarchie mystique la fonction d'initiatrice.* On racontait que Denys le jeune, chassé de Syracuse en 343, avait misérablement terminé ses jours à Corinthe comme métragyrte.* A la tin du 4e siècle, les prêtres mendiants pénètrent sans vergogne dans les maisons, avec leur chapelle portative* ; ils savent bien que les femmes vont accourir au tintamarre de leurs cymbales. Quand les fidèles sont assez nombreux, ils s'organisent en confréries.

*) Cf. Foucart, \emph{Assoc. relig. en Grèce}, 1873, 2e partie, chap. 9 et 11 ; 3e partie, ch. 16, \emph{Jugement des anciens sur les thiases} ; Dieterich, dans \emph{Rhein. Mus.}, 48, 1893, p. 275 ss.

*) Stob., \emph{Florileg.}, éd. Meineke, vol. 3, p. 63.

*) A Eresos (Lesbos) : \emph{Classical Rev.}, 1902, p. 290 ; \emph{Arch. Iahresh.}, 5, 141.

*) Au 2e siècle de notre ère, Pausanias cite deux temples de la Dindyménè et d'Attis, l'un à Dymae, l'autre à Patras, 7, 17, 9 et 20, 3. Mais, à cette époque, le culte phrygien est officiellement reconnu dans l'empire romain ; peut-être le culte fut-il introduit dans ces villes achéennes par les pirates que Pompée établit comme colons à Dymae : Lobeck, \emph{Aglaoph}, p. 1152 ; Foucart, \emph{op. l.}, p. 90 ; Gruppe, \emph{Gr. Mythol.}, p. 1551.

*) Il semble, en tout cas, que la religion phrygienne ait subi une influence très ancienne de l'orphisme ; sur les rapports mythiques de Midas et d'Orphée, cf. Koerte, \emph{Gordion}, p. 12 et 15. Relations des rois phrygiens avec les cités ioniennes : un Midas épousant la grecque Demodikè, fille d'un roi de Kymé : un autre envoyant dédier un trône à Delphes. Influences de l'art hellénique sur la décoration des tombeaux phrygiens au 6e s. : Reber, \emph{op. l.}, p. 49.

*) Pind., \emph{Fragm.} 48 et 63 ; \emph{Pyth.} 3, 77 = 137 et schol. ; Paus. 9, 25, 3.

*) Eurip., \emph{Bacch.} 77 ss, 125 ss ; cf. \emph{Hel.} 1301 ss.

*) Foucart, \emph{op. l.}, p. 57. Figurine provenant d'Athènes, v. supra, p. 18, n. 2.

*) V. p. 9, n. 2.

*) Plut., \emph{Nicias}, 13.

*) \emph{Av.} 876 s et schol. ; \emph{Vesp.} 115.

*) Demosth., \emph{Pro cor.}, 259-260.

*) Aelian., \emph{Var. Hist.} 9, 8 : τελευταῖον δὲ μητραγυρτῶν καὶ κρούων τύυπανα καὶ καταυλούμενος.

*) V. les textes de Ménandre groupés et commentés par Foucart, \emph{op. l.}, p. 173 ss.

Ces thiases prospèrent surtout dans les ports marchands, tels que le Pirée et Corinthe, de population cosmopolite.* Phrygiens et Lydiens, Mysiens et Bithyniens retenus par leur négoce loin du pays, se sont d'abord groupés ensemble sous la protection de leur bonne Mère. Mais ils ont fait eux-mêmes, dans leur clientèle grecque, œuvre de prosélytisme. Nous en pouvons juger par l'association métroaque du Pirée. Son décret le plus ancien à notre connaissance date de la seconde moitié du 4e siècle ; plusieurs datent du début du me siècle. Elle avait obtenu du conseil et du peuple l'autorisation de construire un Metrôon, qui servait aussi de sanctuaire à d'autres thiases. Elle possédait son personnel sacerdotal, un prêtre, une prêtresse, une zacore, des hiéropes ou sacrificateurs. Les personnes étrangères à la communauté pouvaient offrir un sacrifice à la déesse, moyennant redevance. Tous ses membres, les Orgéons, qui payaient cotisation, avaient droit au temple et à l'autel. Ils y pratiquaient le culte de rite phrygien. Chaque printemps ils y célébraient les fêtes des « Attideia » ; les jours d'assemblée, ils s'y retrouvaient pour les offices des saints mystères. Or les confrères dont les noms nous sont parvenus appartiennent tous à divers dèmes de l'Attique. Nous sommes, il est vrai, dans la période macédonienne ; et, depuis l'établissement des Macédoniens en Asie, l'orientalisme gagne du terrain. Néanmoins, malgré ces inévitables défaillances, on peut dire que l'esprit public demeure indemne. Pour le Grec, la religion phrygienne reste chose méprisable. « En condamnant les adeptes de la Mère des Dieux ou de Sabazios, les anciens ne se plaçaient nullement à un point de vue dogmatique ; ils ne mettaient en doute ni leur existence ni leur puissance. Ce qu'ils réprouvaient, c'étaient les pratiques honteuses, le charlatanisme et les désordres de leurs fidèles.* » Voilà pourquoi la comédie attique, se faisant l'interprète du bon sens et de la morale, s'acharne contre les métragyrtes : « c'est de beaucoup l'engeance la plus détestable, » s'écrie un personnage d'Antiphane.* Le culte officiel de la Mère des Dieux, dans les cités grecques, ne se laissa point entamer par l'influence néfaste des cultes privés ; il sut conserver sa dignité calme. Sous l'Empire romain, ce sont des Grecs qui protestent avec le plus d'énergie contre l'intrusion du métèque Attis dans l'Olympe.* La Grèce sera la dernière à résister contre la mode barbare des tauroboles.

*) Τελετὴ Μητρός à Corinthe, Paus. 2, 3, 4. Τελεστῆρες τᾶς Μεγάλας Ματρός à Troezène, dans la seconde moitié du 3e s. avant notre ère : Collitz-Betchel, 3364, 12 ; Ziebarth, \emph{Griechische Vereinswesen}, 1896, p. 41. Orgéons du Pirée : Foucart, \emph{op. l.}, p. 85-101 ; Ziebarth, \emph{op. l.}, p. 36 ; Hepding, \emph{Attis}, p. 136 s.

*) Foucart, \emph{op. l.}, p. 155.

*) \emph{Fragm.} 95.

*) Plut., \emph{Amator.}, 13, p. 756 C ; Lucian., \emph{Icaromenipp.} 27. \emph{Juppiter trag.} 8, \emph{Deor. concil.} 9.

Ce ne fut pas en Grèce que les Romains allèrent chercher la Grande Mère des Dieux. Ils la firent venir d'Asie Mineure, sous la forme d'une pierre noire.
\clearpage
\section{Chapitre 1}
\begin{center}
L'Introduction du Culte à Rome.
\end{center}
\paragraph{}
1. Désastres de la seconde guerre punique et consultation des Livres sibyllins en 549/205. Interprétation de l'oracle. Idée religieuse. La Grande Mère invoquée contre l'ennemi de race étrangère. Gomment les Romains connaissaient déjà la déesse. --- 2. Idée politique. La diplomatie sénatoriale en Orient. La république romaine et le royaume de Pergame. Le culte de l'Idéenne et la légende des origines troyennes. L'Idéenne est venue de Pergame, non de Pessinonte. L'ambassade auprès d'Attale. --- 3. Arrivée de la déesse en 550/204. Le cérémonial. Désignation d'un citoyen et d'une matrone, chargés de recevoir la Dame noire. Les fêtes d'Ostie et de Rome. Rôle des Optimates. --- 4. Déformation légendaire de l'événement. P. Cornelius Scipio Nasica, « hôte de l'Idéenne. » Miracle d'Ostie et réhabilitation de Claudia Quinta. Ex-voto « Matri Deum et Navi Salviae. »

\subsection{1.}

En l'année 549 de Rome, 205 avant notre ère, le sacré collège des Décemvirs demanda l'introduction officielle du culte de la Grande Mère. Le 4 avril de l'année suivante, 550/204, la Dame noire, venue d'Asie Mineure, faisait son entrée dans les murs de Rome.*

*) Liv.  29, 10 (consulat de P. Licinius Crassus et P. Cornelius Scipio, le futur Africain) et 14 (consulat de P. Sempronius Tuditanus et M. Cornelius Cethegus). Cf. pour le 4 avril le calendrier de Praeneste, \emph{CIL.} 1. p. 390.

C'était après examen des Livres sibyllins que les Décemvirs avaient formulé leurs instructions. Sur la sentence même de l'oracle nous avons conservé les quatre versions différentes de Tite Live, Diodore de Sicile, Ovide et Appien.*

*) Le récit de Polybe a disparu. Ammien Marcellin nous apprend, 22, 9, 5, qu'il avait écrit \emph{per excessum}, dans les \emph{Acta Commodi} (sans doute à propos de l'attentat de Maternus, qui eut lieu pendant les fêtes métroaques de mars), \emph{super adventu. (deae) pauca cum aliis huic materiae congruentibus} ; mais cette partie de son œuvre ne nous est pas parvenue.

Le collège décemviral, déclare Tite Live,* trouva dans les recueils cette prophétie (\emph{carmen}) : « quand l'ennemi de race étrangère aura porté la guerre sur le sol italien, pour le chasser et le vaincre il faudra d'abord amener de Pessinonte à Rome la Mère Idéenne. » Le poète Silius Italicus s'inspire du texte de l'historien latin ou suit la même autorité b Voici, d'après Diodore,* \emph{ce qui était écrit} : « il faut que les Romains se construisent un temple de la Grande Mère des Dieux et fassent venir chez eux les \emph{hiera} de Pessinonte, ville d'Asie. » Selon Ovide,* les prêtres auraient simplement découvert cet inintelligible oracle : « la Mère n'est pas ici ; c'est la Mère que tu dois quérir, ô Romain ; tel est mon ordre. « On ne savait qu'elle était cette Mère, ni où elle était. Apollon, que l'on fit interroger à Delphes, expliqua l'énigme : « il s'agit de la Mère des Dieux ; vous la rencontrerez sur le mont Ida. » Le Grec Appien* rapporte ainsi la déclaration décemvirale : « ces jours-ci, à Pessinonte, là où les Phrygiens adorent la Mère des Dieux, tombera du ciel quelque chose qu'il faudra transporter à Rome. » Et il ajoute : « peu après, on fit savoir que ce quelque chose était tombé ; c'était une grossière idole.* »

*) Liv. \emph{l. c.} 10 : « invento carmine in libris Sibyllinis,... quandoque hostis alienigena terrae Italiae bellum intulisset, eum pelli Italia vincique posse, si Mater Idaea a Pessinunte Romam advecta foret. »

*) \emph{Punic.} 17, 1-4: « hostis ut Ausoniis discederet advena terris | fatidieae fuerant oraeula prisea Sibyllae, | Cœlicelum Phrygia genetricem sede petitam | Laomedonteae sacrandam moenibus urbis. »

*) Diod. Sic. 34, 33 : ἐν τοῖς τῆς Σιβύλλης χρησμοῖς εὑρέθη γεγραμμένον ὅτι δεῖ τοὺς ῾Ρωμαίους ἱδρύσασθαι νεὼν τῆς μεγάλης Μητρὸς τῶν θεῶν καὶ τῶν μὲν ἱερῶν τὴν καταγωγὴν (sans doute pour indiquer le transport par mer) ἐκ Πεσσινοῦντος τῆς ᾿Ασίας ποιήσασθαι.

*) \emph{Fast.} 4, 259 : « Mater abest, matrem iubeo, Romane, requiras ; 263 : Divum arcessite Matrem ; | ...in Idaeo est invenienda iugo. »

*) \emph{Hannib.} 56 : ἐξ οὐρανοῦ τι ἐς Πεσσινοῦντα τῆς Φρυγίας, ἔνθα σέβουσιν οἱ Φούγες θεῶν μητέρα, πεσεῖσθαι τῶνδε τῶν ἡμερῶν καὶ δεῖν αὐτὸ ἐς τὴν Ῥώμην ἐνεχθῆναι. Cf. Herodian. 1, 11.

*) Il emploie le mot βρέτας.

Appien reproduit sans doute une légende populaire. D'après une tradition, en effet, la pierre était un aérolithe.* Mais la formule attribuée aux Décemvirs est vraiment trop puérile. Le récit d'Ovide a le mérite d'être artistement présenté. Il est même très séduisant. Repose-t-il sur un fondement historique ? Confiant dans le témoignage du poète, voici l'ingénieux commentaire dont on a cru pouvoir l'illustrer. Les Livres auraient eu pour unique rôle de faire accepter par la communauté patricienne et plébéienne, par suite de faire entrer dans la religion nationale certains cultes purement plébéiens, qui étaient des cultes italiques.* Ils auraient donc réclamé, en cette circonstance, l'adoption de quelque antique déesse mère du pays latin.* Le choix de la Mère des Dieux fut imposé par la Pythie de Delphes, priée d'éclaircir l'obscur langage de la Sibylle. Pour la prophétesse grecque de la fin du 3e siècle, le nom de Mère restait en effet attaché à la divinité de Phrygie ou à la Megalè Mêter des cultes helléniques. C'est donc en conséquence d'un véritable malentendu que Cybèle vint à Rome ; « et les indications de la Pythie durent paraître assez étranges au Sénat romain.* » Mais cette thèse spécieuse s'appuie tout entière sur un paradoxe, à savoir que les recueils sibyllins sont d'origine latine, et non pas orientalo-grecque. Elle ne tient compte ni de la tradition cuméenne, ni de la langue de la prophétesse, ni de la compétence des interprètes. De plus, l'idée seule de rechercher dans Ovide la forme première de la consultation est déjà paradoxale. Que de poétiques invraisemblances ! Peu importe à quel moment précis de l'évolution religieuse de Rome le culte métroaque pénètre dans la cité. Il suffît au poète de dire que « Rome, âgée de cinq siècles déjà, levait fièrement la tête au-dessus de l'univers dompté.* » Ovide parle ici en contemporain d'Auguste, qui jouit de la paix romaine. Aussi bien n'a-t-il pas la prétention d'être un historien. La prédiction des Livres lui paraît trop claire. Il faut, pour l'effet dramatique, une formule vague et mystérieuse qui jette le trouble dans les esprits : « cherchez la Mère.* » Si Attale, désigné comme roi de Phrygie, refuse de céder la déesse, c'est pour permettre au poète d'imaginer un tremblement de terre ; c'est aussi pour lui donner l'occasion de flatter ses compatriotes. La Cybèle d'Ovide redit aux Romains le compliment jadis attribué à la Junon Véienne : « j'ai souhaité moi-même d'être enlevée ; Rome seule est le digne rendez-vous de tous les dieux.* » Et « mille mains » de construire un vaisseau à la déesse, à seule tin que l'on voie tomber les pins de l'Ida « sous les coups innombrables des haches, » comme dans l'Énéide.* Ovide s'attarde longuement à retracer l'itinéraire de la Dame, des côtes de la Troade jusqu'à celles du Latium, pour satisfaire aux exigences de la poésie descriptive.* Bientôt on oublie avec l'auteur qu'il s'agit de la réception faite à la Grande Mère en 204, et l'on se retrouve aux cérémonies du Bain rituel sous le principat d'Auguste. Il faut d'autant moins croire Ovide sur parole qu'il est facile d'expliquer par une confusion le rôle attribué à la Pythie. Jamais ailleurs nous ne voyons Apollon Pythien consulté sur la signification d'un oracle sibyllin.* C'est au collège décemviral qu'est réservée cette fonction d'interprète. Mais, en cette même année 205, deux anciens édiles furent envoyés à Delphes pour offrir au dieu sa part du butin conquis sur Hasdrubal.* Ils firent connaître, à leur retour, qu'Apollon avait agréé leurs ex-voto et leurs sacrifices, et qu'il promettait à Rome une victoire prochaine, encore plus grande. Le vague de cette prophétie était conforme aux habitudes d'Apollon. On crut y voir l'annonce d'un succès définitif sur Carthage. Quelque temps après, la députation chargée d'aller chercher la Mère en Orient visita Delphes, chemin faisant. Elle s'y renseigna sur l'issue qu'elle pouvait espérer de sa mission diplomatique* ; la réponse favorable du dieu pouvait faciliter la tâche des ambassadeurs. Ces relations amicales, que la République entretenait à dessein avec le clergé de Delphes, font comprendre l'erreur du poète ou excusent la licence qu'il a cru pouvoir s'octroyer.*

*) Cette tradition permettait aussi de trouver une étymologie au nom de Pessinonte ; on le dérivait du grec πέσειν, et il signifiait que la pierre était tombée en ce lieu.

*) C'est la thèse d'Hoffmann dans son étude sur les livres sibyllins, \emph{Rhein. Museum}, 1895, p. 90-113. Des idées toutà fait contraires sur le rôle des oracles sibyllins sont exposées par Diels, \emph{Sibyll. Blätter}, 1890.

*) Les prodiges dont parle T. Live auraient été primitivement considérés comme les manifestations de la colère d'une ancienne divinité locale, dépossédée par les dieux du Capitole.

*) Hoffmann, \emph{op. l.}, p. 94.

*) Ovid., \emph{Fast.} 4, 255-256. Hérodien, qui paraît oublier aussi les angoisses de la guerre punique, nous donne mieux, 1, 11, le dernier état de la tradition que les temps ont simplifiée : « l'empire romain ayant grandi, l'oracle avait annoncé qu'il serait durable et le plus grand de tous si les R. amenaient chez eux la déesse de Pessinonte. » La pierre noire est considérée, au même titre que les fétiches de la Regia, comme un \emph{pignus imperii}.

*) Il est intéressant de retrouver dans Virgile le même oracle en termes identiques. Apollon Pythien, dans Ovide, s'adresse aux Romains, descendants d'Énée, et les engage à faire chercher la déesse mère du pays troyen sur la terre-mère des Aeneades : \emph{matrem iubeo... requiras} ; Apollon Délien, dans l'Énéide, 3, 96, s'adresse à Énée et l'invite à gagner la campagne de Rome, terre-mère de l'ancêtre troyen Dardanos : \emph{antiquam exquirite matrem}. Il est vraisemblable qu'il n'y a pas là rencontre fortuite. On pourrait croire que les deux poètes puisent à une source commune, par exemple dans l'œuvre d'un poète ou d'un historiographe qui aurait vécu à la cour de Pergame. Mais il est beaucoup plus probable qu'Ovide s'inspire de Virgile.

*) Ovid., \emph{Fast.} 4, 269-270. Cf. pour la Junon de Voies, Liv. 5, 22.

*) Sur ce passage d'Ovide (274 ss) et le rapprochement avec Virgile, v. Kuiper dans \emph{Mnemosyne}, 1902, p. 292-295 ; il semble que les deux poètes s'inspirent ici d'écrivains de l'école hellénistique de Pergame.

*) Voir tout le passage 272-291. Rapp, dans Roscher, \emph{Myth. Lex.}, 2, p. 1666, signale ici l'imitation des poètes alexandrins ; il est possible que tout ce récit procède d'un poème de l'école pergaméenne et qu'Ovide n'ait pas même inventé le tremblement de terre.

*) Après Cannes, il y avait eu des prodiges. On consulte les Livres et \emph{simultanément} on envoie Fabius Pictor à Delphes ; il ne s'agit pas de faire interpréter l'oracle de la Sibylle par celui d'Apollon : Liv. 22, 57 ; 23, 11.

*) Liv. 28, 45 ; 29, 10. Les présents, offerts au nom de Scipion, étaient une couronne d'or et des trophées en argent.

*) Liv. 29, 11.

*) Julien suit une tradition analogue dans son traité sur la Mère des Dieux, 1 (\emph{Or.} 5, p. 159). Il attribue à l'oracle de Delphes l'introduction de Mater Magna.

La précision de Diodore contraste avec l'obscurité voulue d'Ovide, qui propose une énigme. L'historien grec écarte tout élément légendaire. Mais, d'autre part, comme il ne cite l'oracle qu'en passant et à propos des Scipions, il ne se préoccupe guère d'en reproduire ou d'en reconstituer la teneur. Le texte qu'il en donne n'a rien de l'allure sibylline. Ne croirait-on pas lire la conclusion même, en deux points, du rapport des Décemvirs au Sénat ? Tite Live, au contraire, dans le récit détaillé des événements de l'année 349/205, prétend nous transmettre le « carmen » traduit des livres grecs. Mais ne soyons pas dupes des mots. Les Décemvirs ne dévoilaient jamais la sentence originale de l'oracle* ; pour être efficace, elle devait rester mystérieuse. Leur rapport monitoire n'en formulait que le commentaire pratique, adapté au cas particulier qui avait déterminé leur intervention.* Le rôle du sacré collège comportait deux phases distinctes. Il y avait d'abord examen, ce qui suppose une sélection des textes appropriés aux circonstances. Il y avait ensuite interprétation du texte choisi. Le « carmen » que Tite Live incorpore à son récit est autre chose qu'une pure traduction ; c'est une prophétie clarifiée, qui a subi tout un travail d'exégèse. Quel que soit son degré d'authenticité, un tel document offre du moins une certaine vraisemblance historique. Il nous révèle la principale raison d'état qui eut pour effet l'introduction du culte de la Grande Mère. Sur ce point il est d'accord avec la tradition la plus ancienne qui nous soit parvenue.* Mais il laisse place encore à bien des hypothèses. Plus d'une considération d'ordre politique dut influer sur le choix des Décemvirs.

*) Les anciens oracles sibyllins n'étaient pas, à proprement parler, des \emph{vaticinia} ; on les appelait \emph{remedia sibyllina}. C'étaient des instructions pour l'expiation des mauvais présages et le détournement des maux ; cf. Hoffmann, \emph{op l.},p. 92.

*) V. les textes réunis par Marquardt, \emph{Culte chez les R.}, tr. Brissaud, 2, p. 50 à 52.

*) Cic., \emph{De har. Resp.}, 13, 27 : « defessa Italia punico bello atque ab Hannibale vexata, sacra ista nostri maiores adscita ex Phrygia Romae collocarunt. »

Ceux-ci n'examinaient les Livres sibyllins que sur réquisition du Sénat, et le Sénat ne s'adressait aux Décemvirs qu'en des circonstances exceptionnelles. Il faut que les prodiges soient d'une nature, d'une intensité ou d'une fréquence telle que la science théologique des pontifes et l'art divinatoire des haruspices deviennent insuffisants pour obtenir « la paix des Dieux.* » Or, en 205, les pluies de pierres avaient été, nous dit-on, particulièrement nombreuses.* On y remédiait en général, \emph{more patrio}, par une neuvaine.* Comme le fléau persistait, on eut recours à l'oracle. Mais cette pluie de pierres intervient avec un tel à-propos, pour expliquer le culte de la pierre tombée du ciel, que le prodige prend toutes les apparences d'une légende. La réponse tirée des Livres contient une allusion directe à de tout autres préoccupations.* Rome était dans la quatorzième année de la seconde guerre punique. Après les désastres de Trasimène et de Cannes, et à la suite de l'héroïque effort qui fit son salut, les alternatives de succès et de revers, d'espoir exalté et de découragement, avaient aggravé son état de surexcitation morbide. Cette même année, à peu de semaines d'intervalle, Rome tressaille d'épouvante en voyant Hannibal sous ses murs, et d'allégresse patriotique en apprenant la capitulation de Capoue. Il fallait combattre en Italie, en Sicile, en Espagne, en Grèce. En 347/207, Hasdrubal avait franchi les Alpes, traversé les plaines du Pô, soulevé les Gaulois. Il s'avançait pour rejoindre son frère. Le danger était aussi grand qu'aux jours de Cannes. Il y eut tumulte dans Rome affolée. « On se demandait avec angoisse quels dieux seraient assez propices pour assurer le triomphe de la République à la fois sur deux ennemis.* » Il faut lire dans Tite Live l'anxiété de Rome quand elle sait que la bataille est imminente aux bords du Métaure. Les sénateurs se tiennent en permanence dans la Curie, le peuple sur le forum, les femmes dans les temples.* Quand le péril semble écarté, Rome souffre encore de tous les sacrifices que depuis si longtemps elle s'impose, de la misère qu'entraîne la cherté des vivres, de l'épuisement des alliés qui commencent à refuser contributions et contingent. Elle est lasse de cette lutte sans fin. En 549/205 la situation ne s'est guère améliorée. Hannibal est toujours en Italie. Il a fallu reconquérir une fois de plus l'Espagne. Un jour on apprend que Magon a surpris Gênes, qu'il a des intelligences en Étrurie et qu'il va peut-être renouveler la tentative d'Hasdrubal. Un autre jour, le consul Licinius écrit qu'une épidémie meurtrière sévit au camp. P. Scipion, l'autre consul, organise en Sicile une expédition d'Afrique. Son audace effraie le Sénat, mais enthousiasme la foule ; car Scipion était prophète en son pays, et le peuple croyait volontiers que les dieux inspiraient les desseins du jeune héros.

*) Cf. Liv. 42, 2.

*) Liv. 29, 10 : « propter crebrius eo anno de coelo lapidatum. »

*) Liv. 21, 62, 25, 7, 27, 37,  29, 14, 30, 38 (id prodigium more patrio novemdiali sacro... expiatum), 34, 45, 35, 9, 38, 13.

*) Cf. des exemples analogues : Liv. 3, 10 : prodiges, tremblements de terre, \emph{etc.} ; les livrés sont consultés : « pericula a conventu \emph{alienigenarum} praedicta. » ; et 5, 16 : l'oracle delphique, interrogé sur une crue du lac d'Albe, répond ennemis, guerre et victoire.

*) Liv. 27, 40 ; cf. 44.

*) Liv. 27, 50.

Cette longue période de crise avait exalté les imaginations, développé dans la masse une religiosité d'un caractère maladif et assombri. Rome s'agitait dans une fièvre de superstition. « On attribuait aux Dieux tous les événements heureux ou malheureux ; on annonçait partout des prodiges. » Tite Live, qui les énumère chaque fois avec une piété si confiante, ne peut s'empêcher d'en trouver quelques-uns bien ridicules Mais le plus sinistre de tous ces prodiges, c'était la présence de « l'ennemi de race étrangère » sur le sol d'Italie depuis bientôt trois lustres. Il y avait là, au sens le plus religieux du mot, une souillure dont on cherchait en vain l'expiation. C'est sous l'empire de tels sentiments que Rome adressa ses premiers vœux à la Grande Mère Idéenne.

*) Liv. 26, 19 ; 27, 23 ; 28, 11.

Aussi bien n'était-ce point la première fois que l'invasion carthaginoise participait, comme élément actif, à l'évolution de la religion romaine. Durant toute la seconde guerre punique, l'Étranger, \emph{Alienigena}, est un personnage qui reparaît souvent dans les oracles. Il est l'unique objet des fameuses prédictions de Marcius qui, dans sa langue imagée de prophète, l'appelle tumeur de l'Italie.* Après Cannes, le Sénat envoie une ambassade à Delphes pour demander au dieu quelle sera l'issue de la guerre ; et c'est ce désastre de Cannes qui entraîne la fondation des Jeux Apollinaires. Tous les dieux de la Grèce sont intéressés au triomphe de Rome. Car le Punique est vraiment, pour le Grec comme pour le Romain, l'homme d'une autre race. De plus Hannibal s'est créé dans la Grèce et en Orient de sérieux appuis ; il ne faut pas laisser aux alliés d'Hannibal une divinité que l'on sait être puissante. Or la Mère des Dieux est la dernière divinité du monde orientalo-grec dont Rome n'ait pas encore imploré la protection. Il faut sauver la République, et cette Mère est par excellence une déesse de salut, \emph{Mater Salutaris}. Enfin elle seule peut balancer la puissance de la grande déesse mère de Carthage.* Elle sera pour les Romains la Nicéphore. La tradition qui établit un rapport de cause à effet entre les origines de son culte à Rome et la libération du territoire italien est constante, de Cicéron à Julien.*

*) Liv. 25, 12 : « amnem Troiugena Cannam, Romane, fuge, ne te \emph{alienigenae} cogant consercre manus... Hostem, Romani, si expellere vultis \emph{vomicamque} quae gentium venit longe, etc. »

*) Cf. Clermont Ganneau, \emph{Études d'arch. orientale}, 1, p. 155 ; \emph{Recueil d'arch. or.}, 4, p. 229. La grande déesse de Carthage invoquée dans le traité de paix qui fut conclu entre Hannibal et Philippe de Macédoine : Polyb. 7, frg. 5.

*) Cf. p. 29, n. 5 et Sil. Ital. \emph{Pun.} 17, 1 ; Arnob. 7, 40, fait une allusion très claire à l'Idéenne : « iussis et monitis vatum transmarinis ex gentibus quosdam deos accitos, ...viribus hostium fractis frequentissime triumphatum, etc. » Julien, \emph{l. c.}, représente la déesse comme une \emph{alliée} de Rome dans la guerre contre Carthage.

Mais doit-on supposer qu'avant de recevoir un caractère officiel ce culte s'était répandu dans la ville, et qu'il avait déjà pris possession de la cité quand il reçut droit de cité* ? Rome n'a pas encore sa tourbe d'esclaves asiatiques. Nous ne savons rien sur les pérégrins d'Ostie, de l'Aventin et du Trastevere, qui pouvaient adorer chez eux leurs divinités nationales. Il est cependant certain que Rome, à la recherche de stimulants pieux, avait accueilli plusieurs rites étrangers avec empressement. Le Sénat inquiet fut obligé de sévir en 541/213. « Toute une religion exotique, dit Tite Live, avait envahi la ville qui semblait avoir tout d'un coup changé d'hommes et de dieux. Déjà ce n'est plus en secret et dans l'intérieur des maisons que les anciens cultes sont abolis. Mais en public même, sur le Forum, au Capitole, on voit une multitude de femmes qui n'offrent plus de prières et de sacrifices aux dieux selon la coutume de la patrie. Des sacrificateurs et des prophètes se sont emparés des esprits. La masse des paysans, que la misère ou la peur a jetés dans Rome, permet à ces exploiteurs d'exercer avec plus de profit leur charlatanisme. Les triumvirs faillirent être maltraités quand ils voulurent chasser du Forum cette foule et disperser l'appareil du nouveau culte. Un édit fut nécessaire pour empêcher de sacrifier en public selon des rites étrangers et inconnus.* » Il est fâcheux que ce texte soit aussi peu précis. Ne s'agit-il pas ici d'une première apparition des cultes orientaux ou alexandrins ? Ces prophètes charlatans, avec leur attirail liturgique, font penser à des Galles de Cybèle. Ne seraient-ils pas des métragyrtes, que la guerre aurait refoulés de la Grande Grèce et de Campanie vers l'Italie centrale ? D'oppidum en oppidum, de cité en cité passaient les Galles, avec leur âne ou leur chariot, leur petite chapelle portative, leurs icônes, leurs médailles de sainteté et leurs tambourins. C'est peut-être par eux que pour la première fois la population de Rome connut la Dame phrygienne et ressentit l'attrait mystérieux des religions d'Asie.

*) Cf. ce qui se passe en Grèce dès la fin du 4e siècle, p. 23.

*) Liv. 25, 1 ; cf. 39, 16.

Mais de toute façon Rome ne pouvait pas ignorer la Mère des Dieux. Depuis sa victoire sur Pyrrhus, elle a étendu ses relations avec les états grecs. Depuis qu elle est en guerre avec Philippe, sa flotte croise entre Brindes et Corinthe, entre le Péloponèse et l'Eubée.* Ses alliés delà ligue étolienne adorent la déesse. Selon toute vraisemblance, elle n'est pas sans rapports de commerce avec les grandes cités marchandes qui constituent alors les Échelles du Levant, Rhodes, Mytilène, Smyrne, Cyzique,* centres prospères du culte métroaque. Marseille reste fidèle à la Dame de Phocée,* et son Metrôon est aussi vieux que la colonie phocéenne. Or Marseille trafique avec Rome, dont elle est depuis deux siècles l'amie et l'alliée.* Un établissement de Massaliotes s'était même fixé sur l'Aventin ou au bord du Tibre ; il y importa le culte de l'Artémis d'Éphèse.* La Mère avait dû suivre aussi les colons de Kymè qui s'installèrent sur les côtes de la Campanie. N'était-il pas naturel de lui consacrer, comme en Asie Mineure, les grottes chargées de vapeurs que les Grecs appelaient Charonia, Plutonia,* et où ses Galles allaient volontiers chercher l'inspiration prophétique ? Il est fort possible que le culte de la Mater Baiana,* protectrice des eaux thermales de Baies, soit antérieur à celui de la Palatine. Bien avant l'Empire, dans la région du Vulturne, la déesse régnait sur les sommets.*

*) En 540/214, le préteur M. Valerius commande la flotte de Brindes (Liv. 24, 40) ; au printemps de 544/210, Laevinus part de Corcyre, double Leucate et se rend à Naupacte et Anticyre (Liv. 26, 26) ; en 545/209, P. Sulpicius aborde entre Sicyone et Corinthe (Liv. 27, 31) ; il passe l'hiver à Égine (\emph{ibid.}, 33). D'autre part, les députés de la ligue étolienne font un long séjour à Rome. Sur le culte de la déesse à Corinthe, cf. supra p. 23. Vestiges du culte métroaque à Délos au 3e siècle avant notre ère : \emph{Bull. Corr. Hell.}, 1882, p. 500, inscr. 22 et 25.

*) Il ne s'agit pas ici des relations diplomatiques et officielles avec ces différentes cités ; elles ne font que commencer ou sont postérieures. Holleaux, dans \emph{Mélanges Perrot}, pp. 183-190, abaisse de cent ans la date donnée par Polybe surle premier traité entre Rhodes et Rome. C'est seulement en 201 ou 200 que les deux républiques nouent des relations d'amitié. C'est après la défaite d'Antiochos aux Thermopyles que Smyrne et Lampsaque sollicitent le secours de Rome : députés à Flamininus en 558/196, au Sénat ? en 560/194.

*) Quarante et une stèles votives, exhumées en 1863, quand on créa la rue Impériale, aujourd'hui rue de la République. A une seule exception près, elles représentent une déesse voilée, assise dans un naiskos ionique ; sur quelques exemplaires, on voit un lionceau couché sur les genoux de la Mère. Ces monuments, que l'on a rapprochés à juste titre des statues de Kymè, paraissent remonter aux 6e et 5e siècles et ne peuvent être que d'importation phocéenne (sur la possibilité d'une colonisation préphocéenne par les Cretois, cf. l'existence d'une rivière Massalia en Crète). Musée Borely ; salle des sculptures archaïques, n° 23-63. Froehner, \emph{Catal. des antiq. gr. et rom. du musée de Marseille}, pp. 11-14, avec la bibliographie (v. en particulier Heuzey, \emph{Catal. des figurines du Louvre}, p. 239 ; S. Reinach, dans \emph{Bull. Corr. Hell.} l889, pp. 552-535 et \emph{Chroniques d'Orient} 1, p. 650). Ajouter Benndorf dans \emph{Jahresh. d. Oest. Inst.}, 1899, 2, p. 33, fig. 35, et Clerc, \emph{Les stèles de Marseille}, communication au \emph{Congrès intern. d'archéologie}, Athènes, 1905. Pour l'illustration, ajouter Dar. et Saglio, \emph{Dict. des antiq.} 1, p. 86, fig. 135 ; \emph{Assoc. Fr. Avanc. des Sc.}, Marseille, 1891, p. 603. Le temple de la déesse était, d'après l'emplacement des stèles, au pied de l'acropole ou sur la butte des Carmes.

*) Strab. 4, 1, 5. Traité de paix entre Rome et Marseille à l'époque des Tarquins, Diod. Sic. 14, 93. Ce sont les Marseillais qui, au début de la seconde guerre punique, tiennent les Romains au courant des projets d'Hannibal : Liv. 21, 20. Après Cannes, ils offrent à Rome tous les secours dont ils peuvent disposer.

*) Strab. \emph{l. c.} ; ailleurs (13, 1, 41) il constate la ressemblance des images phocéennes, massaliotes et romaines d'Athéna ; cf. Merlin, \emph{L'Aventin dans l'antiquité}, 1906, pp. 222-225.

*) Les Grecs avaient précisémcnl donné ce nom à l'Avcrne et sans doute aussi aux grottes du voisinage, en particulier à la fameuse grotte du chien : Strab. 5, 4, 5.

*) \emph{CIL.} 10, 3698. La dédicace grecque à la Dindyménè, \emph{CIG.} 5856, est fausse.

*) Le \emph{Liber Coloniarum} dit qu'Auguste, en fondant la colonie Iulia Augusta Venafrum, réserva les cimes des montagnes à la Mère des Dieux, \emph{iure templi} : cf. \emph{CIL.} 10, p. 477.

Dans la Grande Grèce, en Sicile, les Romains l'avaient rencontrée plus d'une fois. Son sanctuaire de Locres et celui qui dominait la pointe occidentale de l'Italie,* près de Reggio, dataient probablement de l'époque hellénique. Peut-être existait-il, dans le voisinage d'Hipponium, un temple de Cybèle.* Sur les feuilles d'or que l'on déposait dans les tombes et qui constituaient comme le « livre des morts » des sectes orphiques, le nom de Kubelè figure avec ceux de Demeter, de Korè, de Tychè, de Gê Pammêtôr et de Prôtogonos.* A Gela et Agrigente, Rhodiens et Crétois avaient sans doute importé le culte de Rhéa.* Au pied des monts Nébrodes, sur le territoire d'Engyium, le temple très ancien des Mères passait pour être de fondation crétoise.* Il était dédié à la fois, ce semble, aux déesses Mères, divinités des sources locales, et à la Grande Mère. Cicéron l'appelle \emph{fanum Matris Magnae}* et le qualifie de « très auguste et très saint. » C'était un lieu de pèlerinage,* que les siècles avaient enrichi d'ex-voto. Les chefs vainqueurs avaient coutume d'y déposer des trophées d'armes, lances, casques et boucliers.* Bientôt P. Scipion y consacrera des casques et des cuirasses en bronze ciselé, chefs-d'œuvre de la toreutique corinthienne et « monuments de sa victoire.* » Syracuse paraît avoir eu son Metrôon sur l'Achradine. Il nous en reste une stèle votive, qui rappelle les « naiskoi » métroaques de l'Attique au 4e siècle* : on y voit, autour de la déesse assise, le sceptre en main, une jeune fille qui tient la torche et un éphèbe qui porte l'œnochoë. Une grotte sacrée, près de la porte dite Hexapyla, était dédiée à la « souveraine des fauves » Artémis,* associée ou identifiée à Cybèle, dominatrice des lions ; on y a retrouvé une « stips » votive qui date du 4e siècle.* Dans la colonie syracusaine d'Akrai, on adorait aussi la Mère des Dieux. Sur une colline qui se dresse à l'est de la citadelle et fait face à la nécropole, on avait sculpté dans la roche vive, près d'une source sacrée, plusieurs images colossales de la déesse. Elle tient d'une main le tambourin et de l'autre le sceptre, qui la désigne ici comme reine des morts.* Des dieux secondaires l'accompagnent ; des adorants sont à ses pieds. Or les relations de Syracuse avec Rome remontaient aux temps déjà lointains de la lutte commune contre les Étrusques (premier quart du 4e siècle) ; elles prirent une importance exceptionnelle à partir du jour où Rome s'approvisionna de blé sur le marché sicilien.* Elles se continuèrent sans interruption jusqu'à la défection fameuse qui aboutit au pillage de 542/212. Cybèle a pu figurer dans ce peuple de statues que Marcellus fît porter à Rome ; les Romains auraient possédé son image avant de lui adresser leurs prières. Déjà leur étaient venus de Sicile certains éléments du mythe de Rhéa, que le travail inconscient de l'imagination populaire et la complaisance des historiographes transformaient en légende nationale. Cette légende se rattachait aux origines mêmes de la cité. Car Rhéa Silvia, mère de Romulus et Remus,* paraît n'être qu'une transposition de Rhéa Oreia ou Idaea,* Mère divine, protectrice des montagnes et des forêts. Aussi, plus tard, sembla-t-il invraisemblable d'admettre que Rome ait attendu plus de cinq siècles pour adorer la déesse. On reculait volontiers la fondation de son premier sanctuaire jusqu'aux âges fabuleux des rois. Une tradition l'attribuait aux Sabins de Tatius.* En important le culte de Rhéa, ils auraient préparé les voies à l'Idéenne.

*) \emph{CIL.} 10, 24 et 8339 \emph{d}, à Locres ; 7, au promo toire de Leucopetra (en 79 après J.-C.). Zoega, \emph{Bassirilievi antichi}, 1808, 1. p. 92, n. 55, avait déjà dit : « e da supporsi che nella Magna Grecia il culto frigio piu antico fosse ch' in Roma. » Il y a des garnisons romaines à Crotone, Locres, Rhegium dès 472/282.

*) D'après un texte apocryphe, que le P. Marafioti, à la fin du 16e siècle, rapporte en latin et prétend avoir traduit d'un traité de Proclus sur les oracles : « adest in Italia ab Hippone non longe Cybelis castrum,... juxta quod et ipsius deae fanum constructum apparet, Hipponensium opus, etc. » ; cf. Mars. Ficin, \emph{Lib. de sacrif. daemonum}, et Bisogni de Gatti, \emph{Ipponii seu Vibonis V. vet Montisleonis historia}, Napoli, 1710, p. 42. Sur ces ruines, qui datent du Haut Empire, et sur les raisons de douter de l'authenticité du texte (malgré l'autorité de Gottfried Hermann), v. Fr. Lenormant, \emph{Grande Grèce}, 3, pp. 236-239.

*) Fiorelli dans \emph{Not. Scavi}, 1879, p. 157 : Comparetti dans \emph{J. of Hell. St.}, 3, p. 114 ; Dieterich, \emph{De hymnis orphicis}, p. 36.

*) Cf. Ciaceri, \emph{Contributo alla storia dei culti dell' antica Sicilia}, Pisa, 1894, p. 53.

*) Diod. Sic. 4, 79, 7 : hieron des Mêteres ; Plut., \emph{Marcellus}, 20 : hieron des Materes. Rhéa confia Zeus aux Curètes, qui le confièrent aux Nymphes ; les Mères sont les Nymphes qui élevèrent Zeus. Cf. le Sicilien Timée dans \emph{Fragm. Hist. Gr.} 1, fr. 83 : νύμφας εἶτα Μητέρας καλουμένας. D'autre part Diodore dit qu'Engyium tire son nom delà source qui coule dans la ville.

*) \emph{In Verr.} 4, 44 ; 5, 72. Faut-il admettre une erreur de la part de Cicéron (comme le prétend, p. ex., Farnell, \emph{Cults of the greek States}, 3, p. 295, note \emph{e}) ? Les armes votives consacrées par P. Scipio paraissent s'adresser à la Grande Mère qui a donné la victoire à Rome, et ses hydries aux Mères topiques, déesses des eaux d'Engyium. De plus, nous venons de signaler les rapports étroits qui existaient entre Rhéa et les Meteres nourrices. Enfin sur l'association de la Meter et des Nymphes ou des dieux fluviaux dans la religion hellénique, cf. la grotte des Nymphes, au sud de l'Hymette (la déesse assise parait bien être la M. d. D. ; Curtius et Kaupert, \emph{Atlas v. Athen}, pl. 8, 1 ; Milchhoefer dans \emph{Ath. Mitt.} 1880, p. 217 ; Furtwaengler, \emph{Cott. Sabouroff}, texte des planches 17, 18) ; le relief de Tanagra au musée d'Athènes (d'après Milchhoefer, \emph{l. c.}, p. 216 ss : la M. d. D., les Nymphes, Pan) ; le relief d'Andros (Cybèle avec le tympanon ; Nymphes, Pan, Acheloos : Conze dans \emph{Arch. Zeitung}, 1880, p. 5) ; le relief de Paros (consacré aux Nymphes, avec images de Pan, Acheloos, Cybèle et Attis ( ? ), \emph{etc.} : Conze, \emph{l. c.} et 1888, p. 205 ; Loewy dans \emph{Arch. ep. Mitt.} 11, p. 168 ; Furtwaengler, \emph{op. c.}, texte de la planche 137).

*) Sans doute à cause de l'Apparition dont parle Plutarque, \emph{Marcellus}, 20.

*) D'après Plut., \emph{l. c.}, on y voyait des casques en bronze et des fers de lances sur lesquels étaient gravés les noms du crétois Mêrion (compagnon d'armes d'Idoménée au siège de Troie, enseveli à Knossos) et d'Oulixès (forme crétoise du nom d'Ulysse).

*) Cic., \emph{In Verr.} 4, 44 : « in hoc fano loricas galeasque aeneas coelatas opere corinthio hydriasque grandes... idem ille P. Scipio posuerat et suum nomen inscripserat » ; 5, 72, à propos des ex-voto de l'Africain : « monumenta victoriae. » Sur la Meter considérée comme Nicéphore, cf. précisément des balles de plomb portant son nom et trouvées en Sicile : Kaibel, \emph{IGSI.} 2407, 7 a. Dans l'hymne orphique 14, 7 : πολεμόκλονε.

*) Stèle en marbre blanc, trouvée sur l'Achradine en 1880. Musée de Syracuse, salle des marbres, n° 46. Ilaut. 0 m. 69, larg. 0 m. 41, ép. 0 m. 25. Signalée par Koldewey u. Puchstein, \emph{Die gr. Temp. in Unteritalien u. Sicilien}, Berlin, 1899, p. 57. La photographie ci-jointe, pl. 1, a été prise par moi en 1895.

*) On conduisait une lionne aux processions d'Artémis à Syracuse : Theocr. 2, 68. Cf. à Thèbes les lions de l'hieron d'Artémis Eukleia.

*) Orsi dans \emph{Not. Scavi}, 1900, p. 363 ss. Fragments de figurines en terre cuite. Artémis appuie la main droite sur la tête d'un lion ou d'une lionne. Seize têtes coiffées d'un bonnet phrygien (p. 365, fig. 8. n°s 1 et 5) ; type en général féminin (cf. Kekulé, \emph{Terrac. v. Sicil.} pl. 11, 6), « ma non escludo che qualcuna rappresenti Attis, dato il carattere pastorale e le forme androginiche di codesto elemento frigio. » Joueuses de tympanon (fig. 8, n° 6) et de double flûte (p. 383, fig. 29). --- Sur l'association d'Artémis et de Cybèle, cf. en Arcadie, Pans. 8, 37, 2 ; à Plakia, près de Cyzique, \emph{Ath. Mitt.} 7, 1882, p. 155 ; cf. aussi Gruppe, \emph{Gr. Mythol.}, pp. 1535-1540.

*) Au lieu-dit les \emph{Santoni}, sur le côté sud de la colline : Houel, \emph{Voyage de Sicile}, 3, p. 112, pl. 196-198 ; Serradifalco, \emph{Antich. di Sicilia}, 4, p. 165 et pl. 35, 2 : « in quasi tutte queste sculture si vede la figura di una donna sedente o dritta, col modio sul capo, armata di asta (il s'agit du sceptre) o di scudo (tambourin) » ; Conze dans \emph{Arch. Zeitung}, 1880, p. 5 ; cf. Schubing, \emph{Akrae-Palazzolo} dans \emph{N. Jahrb. f. Phil.}, suppl. 4, p. 671. --- Dans l'une des niches, la M. des D. est très nettement caractérisée par le polos, le manteau ramené sur les genoux, le sceptre dans la main g. et le lion qui est à ses pieds. La fontaine se nomme encore Aqua Santa.

*) Cf. E. Païs, \emph{Gli elementi sicelioti nella piu antic a storia di Roma}, dans \emph{Studi siorici}, 2, 1893, p. 147.

*) Plut., \emph{Romul.} 3, 4, rappelle l'identité d'Ilia, Rhéa et Silvia. Déjà Ennius (soldat en 204) confond la mère des jumeaux et Ilia, fille d'Énée. Le nom de Silvia se retrouve dans le mythe troyen de la Sicile occidentale. Le caractère de Rhéa, déesse des sources, reparaît dans le mythe de Rhéa Silvia, jetée au fleuve dont elle devient l'épouse, et dans le mythe des jumeaux exposés au bord du fleuve. Dernières controverses au sujet de l'origine italique de cette déesse : Costa, \emph{Rea Silvia e} Ῥέα Ἰδαία ; Costanzi, \emph{Ancora l'ilalicita di Rea Silvia} dans \emph{Riv. di storia ant.} 12, 1908, p. 49-52 (divinité latine qui explique le nom de la ville de Reate). Faut-il chercher un rapport entre les lions qui gardaient la tombe de Romulus et les lions de Rhéa-Kybele ? Cf. Cecil Smith, dans \emph{Classical Rev.} 1899, p. 87.

*) Idaea = Oreia ; Ἴδη = gorge boisée, dans les montagnes.

*) Dion. Hal. 2, 50, 3. Ils auraient importé le culte de Cronos et Rhéa.

[Planche 1. --- 1. \emph{Cybèle dans un naiskos}. --- Marbre. Au Musée de Syracuse.](https://cdn.solaranamnesis.com/HenriGraillot/1-1.jpeg)

\subsection{2.}

Mais en 549/205, Rome regarde par-delà la Grèce, vers l'Orient. Sa politique d'outre-mer l'oblige à s'intéresser aux affaires d'Asie Mineure. « Au moment, dit Tite Live, où Philippe de Macédoine s'était mal à propos déclaré son ennemi, elle s'était attachée par une alliance Attale, roi d'Asie.* » Le petit royaume de Pergame, encore tout neuf, pouvait devenir en effet un précieux auxiliaire de la République. Attale est le glorieux vainqueur des Galates Tolistoages et Tectosages, d'Antiochos l'Épervier et de Séleucos le Foudre. Un temps, après sa campagne contre Antiochos, il avait possédé presque toute l'Asie Mineure des Séleucides ; il avait exercé sa souveraineté sur toutes les cités grecques de la côte, sauf sur celles qui dépendaient de la Macédoine et de l'Égypte.* Depuis 536/218, il est vrai, après la revanche d'Achæos,* il avait dû ramener sa frontière méridionale au golfe Élæatique ; et l'étendue de ses domaines, du sud au nord ou de l'ouest à l'est, ne mesurait pas cent kilomètres à vol d'oiseau. Mais il gardait encore comme villes sujettes Myrina, Aegæ, Kymè, Phocée, Temnos, Teos et Colophon, qui lui payaient tribut. Il comptait comme alliées fidèles les villes libres de Smyrne, Ilion, Alexandrie de Troade et Lampsaque. Par elles et par ses ports d'Attæa et d'Élæa, Pergame était une puissance maritime. Elle tenait sous son protectorat presque tout le commerce du Levant. Sa flotte de guerre était imposante et comportait trente-cinq galères à cinq rangs de rames.* Home, préoccupée d'en finir avec Hannibal et Carthage, ne songe guère à dominer l'Asie. Elle ne considère donc pas le petit royaume comme une future base d'opérations pour ses propres conquêtes. Mais elle voit en Attale le meilleur instrument pour déjouer les intrigues d'Hannibal dans le monde grec.

*) Liv. 26, 37. --- Bibliographie. Politique générale d'Attale : Thraemer, \emph{Pergamos}, Leipzig, 1888 ; Pedroli, \emph{Regno di Pergamo}, Turin, 1896 ; Cardinali, \emph{Regno di Pergamo} dans \emph{Studi di storia antica} de G. Beloch, fasc. 5, 1906 --- Attale et les Galates : van Gelder, \emph{Galatarum res in Graecia et Asia gestae,} diss., Amsterdam, 1888 ; Staehelin, \emph{Geschichte der Kleinasiatischen Galater}, diss., Bâle, 1897 (chap. 3 : \emph{Der Zusammenstoss mit dem pergam. Reiche}). --- Rap- prochement d'Attale et de Rome : Perrot. \emph{Galatie}, p. 176 ss ; Clementi, \emph{La guerra annibalica in Oriente}, dans \emph{Studi di st. antica}, fasc. 1, p. 51 ss ; Niese, \emph{Geschichte der griech. und makedon. St.} 2, p. 480 ss : Cardinali, \emph{op. l.}, pp. 49-57.

*) Cf. Cardinali, \emph{op. l.}, tout le chapitre 2 : \emph{L'effimera conquista attalica dell' Asia Minore}, pp. 17-48. Sur les possessions égyptiennes en Asie Mineure, bibliogr. à la page 19, cf. p. 85 ; Bouché-Leclercq, \emph{Hist. des Lagides}, 1, p. 261.

*) Les campagnes d'Achaeos, qui avait pris le commandement des troupes syriennes après le meurtre de Séleucos, se prolongent de 223 à 217 ; cf. Radet, \emph{La camp. d'Attale 1 contre Achaeus} dans \emph{Rev. des Universités du Midi}, 1896, pp. 1-18. Attale s'allie à Antiochos en 216. Sur les villes tributaires ou alliées dans la période comprise entre 216 et 196, cf. textes dans Cardinali, p. 95 ss. Entre 207 et 205, après une guerre heureuse contre Prusias, le royaume s'agrandit du côté de la Mysie.

*) Liv. 28, 5.1.

Il se trouva que, d'autre part, Attale avait grand besoin de Rome. Ses frontières n'étaient pas sûres. Les Galates, refoulés en Phrygie ou établis sur l'Hellespont, ne demeuraient point en paix ; les villes de la cote troyenne et les contins de la Mysie restaient sous la perpétuelle menace de leurs incursions.* Le roi de Bithynie, Prusias, était hostile à celui de Pergame ; et, de fait, la guerre éclatait en 547/207 entre les deux princes.* Après l'assassinat de Séleucos, les rapides succès d'Achæos avaient prouvé la faiblesse de Pergame au sud. Bien qu'Attale eût fait alliance en 538/216 avec le successeur de Séleucos, il redoutait la puissance syrienne. Le roi de Macédoine enfin était un adversaire toujours prêt à envahir ses états. Supérieur sur mer à Philippe,* Attale ne réussissait pas, avec ses seules armées de terre, à l'expulser de Carie. Sa politique le tournait vers l'Occident. Aussi, depuis quelques années, s'était-il rapproché de la ligue étolienne. En 543/211, il a fait un pas décisif en entrant dans l'alliance des Étoliens et de Rome contre Philippe. En 544/210, il achète aux Étoliens l'île d'Égine, que venait de prendre P. Sulpicius Galba ; ce port d'attache facilite ses communications avec la flotte romaine. L'année suivante, il est nommé président de la ligue. Roi et proconsul passent ensemble à Égine l'hiver de 546/208. Au printemps qui suit, nous les trouvons avec leurs soixante quinquérèmes à Lemnos, presque dans les eaux levantines. « Il semblait, déclare Tite Live,* que déjà la fortune promettait aux Romains l'empire d'Orient. »

*) En Mysie, φθόραι des Galates : Paus. 18, 41. Siège d'Ilion en 216 par les Aegosages, qu'Attale avait établis sur l'Hellespont en 218 : Pol. 5, 111 ; cf. Niese, \emph{op. l.} 2, p. 392 et Cardinali, p. 88. En 196, doléances des gens de Lampsaque, pillés par des bandes de Tolistoages : Dittenberger, \emph{Syll.} 200 ; Lolling, dans \emph{Ath. Mitt.} 6, 1881, p. 96 ss ; Staehelin, \emph{op. l.}, p. 58.

*) Bataille au lieu-dit Βοὸς κεφαλαί : Ératosthenes dans Steph. Bvz., s. v. ; cf. Liv. 38, 39: parmi les territoires que Rome annexe au royaume de Pergame, la Mysie « que le roi Prusias avait perdue. »

*) Ce sera dans un combat naval, près de Chios, qu'il remportera sa principale victoire sur Philippe, en 200.

*) Liv. 26, 37 : « iam velut despondente fortuna Romanis imperium Orientis. »

Or c'est dans la région soumise à l'hégémonie de Pergame que se dresse la montagne sacrée de l'Ida,* trône et autel de la Mère des Dieux. La déesse est l'une des divinités poliades de Pergame même. On y vénère comme sainte patronne la Mère des Dieux Pergamène.* Sous le vocable de Mêter Basileia, Mère et Reine, elle a son temple dans la ville, ses prêtresses, ses confréries de mystes.* Sous le vocable de Grande Mère, elle est Notre-Dame la Noire d'un Megalesion hors les murs 5. Sous le vocable topique d'Aspordène, elle habite la cime d'une montagne désolée qui domine la cité.* Les noms théophores de Mêtrodôros, Mêtrophantès, Mêtris, fréquents à Pergame,* y témoignent de la popularité de son culte. La famille d'Attale manifeste aussi pour elle une particulière dévotion* ; car la reine Apollonis est une Cyzicène.* N'était-il pas d'une habile diplomatie, départ et d'autre, d'attacher le royaume à la république par les liens sacrés d'un culte commun ? Rome se ménage des sympathies en Anatolie ; Attale se crée des droits à la protection de Rome.*

*) L'Ida forme la limite nord-est du royaume.

*) \emph{CIG.} 6835 ; relief à l'image de Cybèle.

*) Fraenkel, \emph{Inschr. v. Pergamon}, 334 (mystes) et 481 (prêtresse). Diod. Sic. 3, 55, raconte le mythe de Basilea, « nommée aussi Grande Mère, » fille d'Ouranos et de Titea-Gê, sœur des Titans et de Rhéa-Pandora, mère d'Hélios et Sélénè. C'est un essai d'adaptation du vieux mythe anatolien de la Meter à la théogonie hellénique. Le dieu fils, Hélios, est noyé dans l'Eridan ; la déesse mère tombe dans une sorte de folie et disparaît pendant une grande pluie, au fracas du tonnerre. Le titre de Basilea révèle-t-il une influence sémitique ? Cf. la « reine du ciel, » chez les Syriens, qui est aussi mère des astres, et l'Anassa Dionè de Chypre.

*) Varr., \emph{De ling. lat.} 6, 15 : « prope murum, Megalesion. »

*) Strab. 13, 2, 6.

*) Ajouter à Fraenkel une liste de noms, époque des Attalides, dans \emph{Ath. Mitt.}, 1900, p. 171.

*) La principale divinité de la famille est Dionysos Kathegemôn, ἀρχηγὸς τοῦ γένους ; cf. Fraenkel, \emph{op. l.} 221, 236, 248 ; Prott, \emph{Kult der} Θ. Σωτῆρες, dans \emph{Rhein. Mus.}, 1898, p. 164. On trouve réunis sous l'Empire les deux cultes de la M. d. D. et de Dionysos Kath., \emph{CIG.} 6206 = \emph{IGSI.} 1449.

*) Sur les rapports de Pergame et de Cyzique : \emph{J. of Hell. St.} 22, 1902, p. 195, et Cardinali, \emph{op. l.}, p. 11. Remarquer d'autre part que le nom d'Attale parait être un nom théophore, dérivé d'Attis.

*) Diels, \emph{Sibyll. Blaetter}, pp. 77-103 et particulièrement p. 101, exagère le rôle de la politique orientale de Rome dans l'introduction de ce culte, lorsqu'il considère la pierre noire comme « le talisman de l'Asie Mineure » ; Rome n'a pas encore le projet de conquérir l'Orient. D'autre part, Kuiper, \emph{De matre magna Pergamenorum}, dans \emph{Mnemosyne}, 1902, p. 277 ss, a tort de ne tenir compte que des intérêts d'Attale. C'est Attale qui aurait imaginé de faire adopter par Rome la déesse anatolienne ; T. Live aurait interverti l'ordre des facteurs en attribuant à Rome, par gloriole nationale, une idée du roi. L'hypothèse est ingénieuse ; en tout cas, il est difficile de l'appuyer sur le texte d'Ovide, auquel M. K. attache une si grande importance.

Aussi bien, la communauté de culte confirmait la communauté d'origine. Déjà Rome se croit fille de Troie. La légende avait été préparée par les épopées du cycle troyen, par les hymnes héroïques du Sicilien Stésichore, qui chantait la navigation d'Énée en Hespérie, par des traditions lointaines, connues de certains logographes et dont on retrouve l'écho dans Aristote. Elle avait été fixée dans ce 3e siècle par l'historien Callias, qui présente la troyenne Roma comme l'épouse de Latinus, et par Timée de Taormine, qui conduit Énée à Lanuvium et à Rome. I1 y avait un siècle au moins qu elle était venue de Campanie ou de la Sicile occidentale en pays latin.* Depuis un demi-siècle déjà elle constituait un élément officiel de la vie nationale. En 504/250, le Sénat, pour justifier son intervention diplomatique en faveur des Acarnaniens, rappelle que « seuls jadis ils n'ont pas envoyé contre les Troyens, auteurs de la race romaine, » des secours à l'armée grecque.* Quelques années plus tard, c'est Ilion la Neuve qui détermine l'un des premiers contacts politiques de Rome avec l'Asie. Les Iliens sont les « parents du peuple romain » ; et le Sénat écrit à Séleucos de Syrie pour lui promettre amitié, s'il renonce à son projet d'annexer leur ville.* L'année même où Rome demande à l'Orient la Grande Mère, elle fait figurer à ses côtés les Iliens dans le traité de paix qu elle accorde à la Macédoine.* Cette fois elle payait une dette de reconnaissance ; car Ilion, jouant sérieusement son rôle de métropole, avait pris le parti de Rome, avec Athènes et Pergame, contre Philippe et Hannibal. Mais Rome ne mettait dans ses rapports avec Ilion aucune sentimentalité. Le gouvernement, qui d'abord avait favorisé la légende troyenne pour resserrer les liens de la République avec la Sicile, s'était ensuite servi du mythe d'Énée pour s'immiscer aux questions d'Orient. Il ne protégeait Ilion, médiocre oppidum, que pour se créer des droits en Anatolie. Les oracles grecs donnent raison aux prétentions des Romains. Pour la Pythie, comme pour Marcius, ceux-ci sont les Trojugènes.* Enfin les érudits pergaméens de cette fin du 3e siècle, grands fabricants de titres de noblesse pour villes et aristocraties d'origine obscure, ne demandent qu'à flatter la vanité de la puissante république alliée. Et Attale, par intérêt dynastique, les y encourage.* Ils complètent donc l'histoire légendaire de Rome. Ils recherchent des ancêtres troyens pour les représentants du patriciat romain.* L'analyste Fabius Pictor, leur contemporain, puisa sans doute abondamment dans leurs compilations.

*) Textes réunis et commentés en dernier lieu par Ciaceri, \emph{Come e quando la tradizione trojana sia entrata in Roma}, dans \emph{Studi storici} de Païs, 4, 1895, pp. 503-529 ; cf. Païs, \emph{Storia di Roma}, 1. p. 157. La légende paraît être venue delà région de l'Éryx (c'est elle qui aurait attiré à Rome le culte de Vénus Érycine). D'autre part, au début de la première guerre punique, les Campano-Mamertins qui appellent l'aide des Romains se prévalent de leur origine commune (Pol. 1, 10) ; cf. la tradition qui fait du troyen Kapys le fondateur de Capoue. En leur double qualité de commerçants et de mercenaires, les Campaniens durent avoir une part prépondérante dans la diffusion de la légende troyenne.

*) Justin. 28, 1 (probablement d'après Timée) : « auctores originis suae. »

*) Sueton., \emph{Claud.} 25 : « consanguineos suos Ilienses. »

*) Liv.  29, 12 ; cf. 37, 37 : joie des soldats romains de voir le berceau de leur nation, au cours de la campagne contre Antiochos ; 38, 39 : privilèges accordés à Ilion, \emph{originum memoria}. Continuation des privilèges d'Ilion sous l'Empire : textes dans Chapot, \emph{Province rom. d'Asie}, 1904, p. 433.

*) En 557/197, un peu avant la bataille de Cynocéphales, la Pythie prédit la victoire des γένεα Τρώων : Plut., \emph{De pyth. or.} 2 ; cf. en 558/196, T. Flamininus inscrivant le titre d'Aeneades sur ses ex-voto de Delphes. --- Le nom de Trojugènes, dans les \emph{carmina Marciana}, montre combien les Grecs d'Italie, vendeurs de prophéties, se préoccupaient de flatter la vanité romaine. Sur ces oracles de Marcius (ou des frères Marcii), considérés comme pastiches de vers sibyllins, v. Bouché-Leclercq, \emph{Hist. de la divination}, 4, pp. 128 et 289.

*) Cf. Attale 2 faisant reproduire le mythe de Rhéa Silvia et des jumeaux sur les reliefs du temple de sa mère Apollonis, à Cyzique : \emph{Anthol. pal.}, ed. Stadtmueller, 3, 19, p. 66. C'est le plus ancien mythe latin qui ait été représenté en Asie : cf. Lebas-Waddington., \emph{Inscr. d'Asie M.}, 88.

*) Sur le rôle des historiens de Pergame dans la composition des légendes d'Éneades : Wilamowitz, \emph{Antig. v. Karyst.}, p. 161. Pour les Servilii, cf. Païs. \emph{Storia di Roma}, 1, p. 207.

Nous avons quelques fragments d'Agathoclès de Cyzique, l'un de ces doctes.* Il y est dit que Romus, fils d'Énée et fondateur de Rome, vint de la Bérécynthie.* Mais la Bérécynthie est la terre de Cybèle. « Les Bérécynthes, qui sont une tribu phrygienne, et ceux des Troyens qui habitent les alentours de l'Ida, » écrit Strabon, « pratiquent le culte orgiastique de Rhéa » Ce culte, d'après Démétrios de Skepsis, appartient à la Troade autant qu'à la Phrygie.* La légende d'Énée devait nécessairement attirer vers lui l'attention des Romains. Ne disait-on pas que Dardanos, l'ancêtre de la race troyenne, avait enseigné le premier les mystères de la Mère des Dieux* et qu'un de ses fils avait construit le Metrôon du mont Ida* ? C'est sur l'Ida qu'Énée, après la prise de Troie, a cherché refuge ; et la Mère Idéenne fut sa grande protectrice.* Ce sont Aphrodite et Cybèle qui enlèvent Créuse, l'épouse d'Énée, pour la soustraire à l'esclavage achéen.* Ainsi se révélait, entre le culte de la Dame et les nouvelles traditions de Rome, une secrète affinité. Les historiens anciens ont parfois su la dégager et la mettre en valeur 6. Voici comment l'exprimait Ovide* : « le Dindyme et le Cybèle, les jolies sources de l'Ida et la puissance d'Ilion, toujours la Mère les aima ; lorsqu'Énée transportait Troie sur la terre d'Italie, peu s'en fallut qu'elle ne suivît ses vaisseaux ; mais les n destinse réclamaient pas encore sa présence divine dans le Latium ; elle comprit et resta aux lieux accoutumés. » On dira de même qu'elle chérit les Romains en mémoire des Troyens.* Le rôle que Virgile attribue à l'Idéenne, patronne d'Énée, il ne le trouve pas tout entier dans son imagination de poète ; son génie épique vivifie et immortalise une des traditions populaires de Rome.

*) Dans Festus, s. v. \emph{Roma}, p. 269 M. D'après une hypothèse de Mueller dans \emph{Fr. Hist. Gr.}, 4, p. 288, acceptée par Susemihl, \emph{Gesch. d. gr. Litter. in d. alexandr. Zeit}, 1, p. 345, et 2, p. 383, Agathoclès aurait été l'élève de Zénodote d'Éphèse (né vers 325 av. J.-C. ) ; mais les raisons ne sont pas probantes, et peut-être vivait-il précisément vers 205

*) V. supra, p. 13, n. 5, et Gruppe, \emph{op. l.}, p. 1528, index p. 1800, col. 1.

*) Dans Strab. 10, 3, 20.

*) Clem. Al., \emph{Cohort. ad gentes}, 2 : ὁ Δάρδανος, ὁ Μητρὸς Θεῶν καταδείξας τὰ μνστήρια ; cf. Diod. Sic. 5, 49.

*) Dion. Hal., \emph{Ant. rom.} 1, 61. La confusion était au surplus facile entre Troyens et Phrygiens, et Denys d'Halicarnasse, 1, 29, déclare que beaucoup l'ont faite. L'art dut y contribuer pour une grande part ; sur les vases peints, les héros troyens portent le costume phrygien.

*) Cf. l'histoire de son vaisseau, construit avec les pins sacrés de l'Ida (Dion. Hal. 1, 47 ; Virg., \emph{Aen.} 9, 85 ss, 10, 156-158), et longtemps conservé dans Rome, au dire de Procope, \emph{B. G.} 4, 22.

*) Paus. 10, 26, 1.

*) Hérodien, 1, 11, rapporte la tradition d'après laquelle les Romains, pour obtenir la déesse, se réclamèrent de leur parenté avec les Phrygiens et rappelèrent les souvenirs d'Énée le Phrygien.

*) Ovid., \emph{Fast.} 4, 250-254. Il résume ainsi tout ce passage, au vers 272 : « in Phrygios Roma refertur avos. »

*) Tertull., \emph{Apol.} 25 : « urbem romanam ut memoriam Troiani generis adamavit, vernaculi sui scilicet adversus Achivorum arma protecti. »

Une triple influence avait donc pu s'exercer sur la détermination des Décemvirs, chargés d'inspecter les livres. Mus par une idée religieuse, ils recherchaient pour les armes romaines le concours d'une divinité puissante. Mus par une idée politique, ils considéraient la grande déesse d'Anatolie comme l'auxiliaire indispensable de la diplomatie sénatoriale. Enfin une arrière-pensée de vanité nobiliaire devait les attirer vers l'Idéenne. Mais les prétentions de l'aristocratie gouvernante se confondaient, en cette occasion, avec les intérêts mêmes du peuple romain. Elles devenaient raison d'état, puisqu'elles permettaient à Rome de se poser bientôt comme héritière naturelle de l'Asie Mineure.

Le sacré collège, interprète de la pensée des gouvernants beaucoup plus que de la volonté des dieux, pouvait-il trouver dans le texte même des Livres capitolins le nom précis de la Grande Mère ? L'origine des recueils sibyllins est ici mise en question. Proviendraient-ils simplement de l'Italie méridionale ou de la Sicile, la présence de ce vocable n'offrirait rien d'anormal, puisque la déesse est une divinité hellénique. Et plus on rapproche du 3e siècle la date de leur importation à Rome, plus il devient vraisemblable que la déesse y joue son rôle ; car de plus en plus, depuis le 5e siècle, le culte métroaque se propage dans les cités grecques. Mais, d'après la tradition antique, ils étaient originaires de la Troade. C'était le recueil de Gergis, ville située au pied de l'Ida, qui avait passé d'Erythræ d'Ionie à Cumes de Campanie, par l'intermédiaire de la métropole éolienne ou peut-être de la colonie samienne de Pouzzoles.* Lorsqu'après l'incendie du Capitole, en 1)71/83, on voulut reconstituer les Livres, on s'adressa tout spécialement à Samos, Erythræ, Ilion.* Si l'on admet la provenance anatolienne de la rédaction primitive, la Grande Mère d'Anatolie possédait tous les droits pour y figurer en bonne place. Aussi bien, entre son culte et celui d'Apollon, auquel se rattachent les Sibylles, on constate d'étroites relations. Le prophétisme est un élément commun aux deux cultes. On adorait la déesse dans les grottes ; et le mystère des cavernes n'est pas moins indispensable à l'inspiration sibylline. D'autre part, Attis est ou tend à devenir un dieu solaire. C'est pourquoi Lêto et Apollon, déesse mère et dieu fils, ont pu se substituer aisément, en Phrygie même, à Cybèle et Attis. On s'explique mieux ainsi qui Apollon intervienne dans le mythe phrygio-lydien de Cybèle.* Il avait fait, disait-on, la connaissance de la déesse à Nysa, chez Dionysos. Le voici bientôt amoureux de l'inconsolable amante d'Attis ; pour ne la point laisser seule dans ses courses affolées, il la suit jusque chez les Hyperboréens. D'après un mythe ionien, et peut-être mysien, Apollon-Hélios était le fils de la Grande Mère Basilea.* Sur la côte mysienne de la Propontide, des thiases associaient dans un même sanctuaire Apollon et la Mêter Kybelè.*

*) Il y avait un recueil samien qui était le même que celui de Gergis : Paus. 10, 12, 5. D'une façon générale, v. sur les oracles sibyllins les textes réunis dans Marquardt, \emph{Culte}, tr. Brissaud, 2, pp. 45 et 46. Diels, \emph{op. l.}, combat la tradition et prétend que les Livres de Rome ne sont ni de date ancienne, ni d'origine orientalo-grecque ; contredit par Reitzenstein dans \emph{Ined. Poet. Gr. fragm.}, 2, 9 et Mommsen dans \emph{Eph. epigr.}, 8, p. 234.

*) Tac., \emph{Ann.} 6, 12.

*) Diod. Sic. 3, 59, 8.

*) Cf. supra, p. 40, n. 4.

*) Relief votif au musée d'Athènes (Sybel, n° 570) : à g., Cybèle assise et Apollon Musagète debout ; à dr., autel, arbre, serviteur poussant un mouton, joueuse de flûte, orante. Sur un registre inférieur, dix thiasotes attablés. Au-dessous, dédicace. Daté de l'année 178 (d'après l'ère bithynienne, 119 av. J. -C.). Bibliographie dans \emph{Bull. Corr. Hell.}, 23, 1899, p. 593. La ressemblance de cet ex-voto avec d'autres permet de le considérer comme originaire de Triglia, l'antique Bryllium-Caesarea. Le thiase adorait aussi Hypsistos. --- Rapprocher de ce relief une plaque de bronze d'origine anatolienne, avec bustes d'Hélios et de Cybèle : \emph{Gaz. arch.}, 1879, pp. 92-94 ; et une inscr. d'Eumeneia, où sont réunis Zeus Sôter, Apollon, Artémis, Asklepios et Agdistis : \emph{CIG.} 3886. A Dionysopolis, Leto et Apollon se sont substitués à Cybèle et Attis : cf. supra, p. 10, n. 4.

Dans l'oracle que rapporte Tite Live, la Grande Mère est aussi caractérisée par l'épithète topique d'Idéenne. C'est sous ce vocable qu'elle sera vénérée dans Rome et, plus tard, dans tout le monde romain. Le mot n'est pas usité dans les cultes helléniques. Avant l'époque où Rome lui fît une telle fortune, il ne paraît pas avoir franchi les frontières de la Mysie troyenne. Mais la région de l'Ida est la patrie des Sibylles. Pausanias qualifie d'Idéenne la Sibylle d'Erythræ.* Celle-ci passait pour être fille d'une des nymphes de l'Ida, compagnes de Cybèle. Que signifie cette filiation mythique, sinon que des émigrants troyens avaient importé dans la ville d'Erythræ leurs oracles et leurs cultes ? Μητρόθεν ᾿Ιδογενής, avait dit d'elle-même l'Érythréenne Hérophile, dans un vers qui suscita d'ardentes querelles de patriotisme local entre les archéologues de l'Ionie et ceux de la Troade.* Or tous ces cahiers sibyllins, qui de la mer Égée à la mer Tyrrhénienne ont suivi le même trajet que la légende d'Énée, devaient trahir par des noms identiques de dieux et de lieux leur commune provenance. La mention de l'Ida s'y rencontrait probablement plus d'une fois. Pour l'expliquer dans le \emph{carmen} capitolin, il n'est donc pas nécessaire de supposer une interpolation des Décemvirs.

*) Paus. 10, 12, 8.

*) Paus. 10, 12, 3. Les archéologues d'Erythrae, pour sauvegarder les droits de leur patrie sur la Sibylle, prétendaient qu'ldogénès n'était pas une épithète géographique et désignait simplement une nymphe des bois. Les autres, en particulier Dcmetrios de Skepsis, voyaient dans la fin du vers (πατρὶς δὲ μοί ἐστιν ἐρυθρὴ) une allusion à Marpessos la Rouge, située au pied de l'Ida. Sur cette question, cf. Diels, \emph{op. l.}, p. 94 ; Gruppe, \emph{Gr. Kulte und Mythen}, 1, p. 686.

La mention de Pessinonte, en revanche, n'a pas sa raison d'être. Elle est en contradiction avec le titre d'Idéenne. A Pessinonte, la Mère des Dieux est la Pessinontide, la Phrygienne, la Dindymène, Agdistis* ; jamais elle n'y est l'Idéenne. Toute la Mysie orientale, toute la Phrygie helles-pontiaque et le nord de la Grande Phrygie séparent l'Ida de Pessinonte. Ce n'est donc pas de la ville des Attis que les Romains ont pu rapporter ce nom avec l'idole noire. Il n'est pas moins étrange qu'ils soient allés chercher sur les rives du Sangarios, au pied du Dindymos, la Dame du mont Ida. D'autre part, Attale n'a jamais pu enfermer Pessinonte dans les limites de son domaine, ni seulement dans les liens d'une confédération.* Il ne possède même pas le pays intermédiaire. Le traité de 538/216 ne lui a pas concédé la petite Phrygie.* Ses états s'arrêtent, vers l'est, aux montagnes qui limitent le modeste bassin du Caïcos,* à plus de trois cents kilomètres de Pessinonte. Le territoire qu'il a conquis sur Prusias, après la campagne de 547/207, ne peut s'étendre qu'au nord-ouest des plateaux phrygiens et le rapproche du bas Rhyndacos, non du haut Sangarios. Même après l'annexion des deux Phrygies, qui fut l'une des conséquences du traité de 565/189, Pergame n'obtint aucun droit sur la cité sainte. État indépendant, qui garde ses domaines sacrés et son gouvernement théocratique, Pessinonte est enclavée dans la Galatie et, par suite, échappe à l'influence directe des Attalides. Pour qu'elle passe sous leur protectorat, il faut une longue guerre d'Eumène 2 contre les tribus celtiques et l'incorporation de la Galatie au royaume (571/183). Il est donc invraisemblable qu'Attale ait pu, en 550/201, livrer aux Romains la divinité d'une ville dont il n'est pas le maître. A plus forte raison n'a-t-il pas conduit là-bas les membres de la mission romaine, comme le prétend Tite Live.* Il n'a pu leur faire franchir plusieurs centaines de kilomètres en pays étranger et sans doute hostile. Au surplus, le récit de Tite Live est d'une brièveté qui le rend suspect. « Les délégués du Sénat arrivent à Pergame ; le roi les accueille de manière affable, les emmène à Pessinonte, en Phrygie, leur remet une pierre sacrée qui passait pour être la Mère des Dieux et les prie d'emporter cette pierre à Rome. » Attale les aurait-il tout simplement accompagnés aux portes de Pergame, la narration ne serait pas plus sommaire !

*) Strab. 10, 3, 12.

*) Pessinonte est enclavée dans la Grande Phrygie, avant la création de la Galatie. Or la Grande Phrygie est restée aux Séleucides jusqu'à Seleucos Callinicos. Elle fut cédée à Mithridate 4 de Pont, quand ce prince épousa la sœur de Seleueos.

*) Meyer, dans Pauly-Wissowa, \emph{R. Encycl.}, 3, 1, 518, et Niese, \emph{op. l.}, 3, 70, suivis par Staehelin, \emph{op. l.}, p. 48, n'ont aucune raison d'identifier à la Phrygie Epictète le territoire mysien qu'Attale a conquis sur Prusias de Bithynie (Liv. 38, 39) et que les Romains annexent définitivement au royaume de Pergame. Pedroli, \emph{op. l.} 30, suppose que la Phrygie Epictète fut donnée à Attale par Antiochos, comme récompense des services rendus après 216 ; cette hypothèse est également inadmissible et contraire au texte de Strabon, 12, 4, 3 (p. 563).

*) Nacrasa, sur le Caïcos, et Attaleia, sur le Lycos, en face de Thyatira, paraissent avoir été ses extrêmes places fortes vers l'est. Thyatira était encore en 191 une place syrienne, en pays syrien (Liv. 37, 8). Cf. Schuchhardt dans \emph{Ath. Mitt.} 13, 1888, p. 1 ss : Thraemer, \emph{Pergamos}, p. 192.

*) Liv.  29, 11.

En réalité, ils ne sont pas allés plus loin. Le plus ancien témoignage qui nous soit parvenu sur la translation de l'idole ne contient aucune allusion à Pessinonte. « D'après le conseil des Livres sibyllins, on s'était adressé au roi de Pergame, Attale. Près de l'enceinte de cette ville est le Megalesion, temple de la déesse ; c'est de ce temple qu elle fut amenée à Rome.* » Ce texte, déjà postérieur d'un bon siècle et demi à l'événement, est de Varron, qui fut un archéologue fort érudit. A la même époque, Cicéron ne paraît pas avoir connu d'autre tradition. Dans le discours où il accuse Clodius d'avoir souillé, pillé, vendu la ville sainte et son temple, il n'aurait pas manqué d'évoquer un souvenir qui eût rendu plus odieuse encore l'impiété de son adversaire. Il consacre tout un développement oratoire à l'universelle renommée du sanctuaire que « l'antiquité tout entière, Perses, Syriens, rois d'Europe et d'Asie, Romains d'autrefois, » n'ont cessé de vénérer comme le séjour le plus sacré de la grande déesse. S'agit-il de prouver le respect religieux de Rome elle-même et de rappeler les pieuses relations de la République avec Pessinonte, voici le seul exemple qu'il trouve à citer : plusieurs généraux, pendant les plus terribles guerres, firent le vœu de s'y rendre en pèlerinage et sont allés, après la victoire, y sacrifier sur l'autel majeur.* Le silence de Cicéron prend donc la valeur d'un véritable argument. Dans la génération suivante, c'est encore la tradition pergaméenne qui semble inspirer le récit d'Ovide.* Les exigences de la versification empêchent le poète, il est vrai, d'introduire le nom même de Pessinonte dans ses distiques ; mais cette ville reste complètement étrangère au fait historique dont on entrevoit le souvenir, au travers des épisodes romanesques ou merveilleux. Enfin ne pourrait-on trouver dans Polybe et Tite Live* un argument contre Tite Live lui-même ? Lorsqu'ils nous montrent les Galles accourus de Pessinonte à la rencontre de Manlius, en 565/189, n'est-il pas surprenant que, dans les vœux intéressés de l'Attis, n'apparaisse aucune allusion aux liens sacrés des deux cités ? On a supposé qu'il y eût eu deux translations successives* : celle de la Mère Pergaméenne à la fin du 3e siècle, celle de la Mère Pessinontide au cours du 2e siècle. L'hypothèse n'est pas vraisemblable. Toute translation de divinité, chez les anciens, comme de reliques au moyen âge, comporte une fête commémorative. Or l'on n'a jamais célébré, dans les Mégalésies romaines, que l'anniversaire d'une seule réception. La Dame noire que l'on adorait au Palatin sous l'Empire est celle-là même qui fit son entrée dans Rome le 4 avril 204. A quand remonte la légende de la Pessinontide ? Sans doute à la fin du 2e siècle, au temps où Rome, héritière des Attalides, assume le patronage du fameux temple dont ces rois furent les bien-faiteurs. Elle put d'autant mieux s'accréditer que le sanctuaire suburbain de Pergame ne jouissait d'aucune renommée. En Asie, les prêtres de Pessinonte avaient tout intérêt à la propager. Diodore* et Strabon,* au début de l'Empire, l'ont-ils recueillie sur place dans leurs voyages d'études ? Le témoignage de Tite Live prouve qu'elle était aussi très répandue en Italie. Comment n'aurait-elle pas fait fortune dans la Rome d'Auguste, maîtresse de presque tout l'Orient, éprise d'orientalisme, envahie déjà par les Orientaux ? Alors que Mâ-Bellone était arrivée tout droit de Cappadoce, la Mère Phrygienne pouvait-elle avouer quelle provenait simplement du voisinage de la côte ? \emph{Major e longinquo reverentia}. Pouvait-on croire que le fétiche de pierre brute avait pour patrie l'une de ces villes grecques où trônait une Mère des Dieux idéalisée par l'art ? On signalait d'autres pierres « autoglyphes, » qui portaient l'empreinte de l'image divine ; mais elles ne se rencontraient que sur les bords ou dans les eaux du Sangarios.* Enfin le culte qui avait passé de Pergame à Rome n'était pas un culte hellénique. Le Megalesion de Pergame était desservi, non par un clergé de race grecque, mais par des Phrygiens et selon le rite phrygien. Pessinonte fit oublier Pergame déchue. Après Tite Live, cette légende pessinontienne prend définitivement place dans l'histoire nationale.* Pourtant, vers la fin de l'Empire, Julien se borne à déclarer que « l'on demanda la déesse aux rois de Pergame, alors maîtres de la Phrygie » ; et deux poètes rappellent encore, à la suite d'Ovide et sous la même forme dénaturée, le souvenir de la tradition véridique.*

*) Varr., \emph{De ling. lat.} 6, 15 : « Megalesia dicta a Graecis quod ex libris Sibyllinis arcessita ab Attalo rege Pergamo : ibi prol e murum Megalesion templum eius deae, unde adrecta Romain. » Il ne s'agit pas seulement d'une halte de la Dame, comme le dit Staehelin, \emph{op. l.}, p. 48, n. 2.

*) \emph{De har. resp.} 13, 28. El cependant, dit-il, « Rome et l'Italie sont remplies de temples. » N'avait-il pas là l'occasion de parler de la Dame noire, si vraiment c'était l'idole même de Pessinonte ? Précédemment (13, 27), il parle des \emph{sacra adscita ex Phrygia} ; mais ce mot désigne aussi la Troade et peut rappeler simplement l'Ida Phrygien ; cf. la note suivante.

*) \emph{Fast.} 4, 265 ss. Les Romains, dit en substance Ovide, vont chercher la déesse dans le massif de l'Ida ; et comme Attale est roi de Phrvgie (cf. la Phrygie de l'Ida dans Strab 10, 3, 23), c'est à lui que l'on s'adresse ; il commence même par refuser, pour bien établir ses droits.

*) Pol. 22, 20 (18), 4 : γάλλοι παρὰ Ἄττιδος \emph{etc.} ; Liv. 38, 18 : « Galli Matris Magnae a Pessinunte occurrere, etc. »

*) Kuiper, \emph{l. c.}, p. 304. On pourrait aussi, pour concilier les deux traditions, émettre une autre hypothèse. L'idole aurait été apportée, en des temps plus anciens, de Pessinonte à Pergame. Serait-elle venue avec les marchands phrygiens qui fréquentaient l'emporium de Pergame ou avec les négociants de Pergame qui se rendaient aux foires de Pessinonte ? On pourrait expliquer ainsi la localisation du sanctuaire en dehors de l'enceinte (Varron, il est vrai, dit simplement \emph{prope}). Serait-elle venue avec Philétaere l'eunuque, ou le premier Eumène, ou Attale lui-même, à la suite d'une expédition dans le bassin du Sangarios ? Était-elle un cadeau des Attis, offert en un temps où ils recherchaient l'amitié du Galatonique, sauveur de l'Asie ? Il a pu exister entre Attale et les maîtres de Pessinonte un traité remontant à l'alliance conclue en 216 entre le roi de Pergame et Antiochos 1er, ou bien un traité datant de 207, quand Attale fut rappelé de ses campagnes de Grèce par une invasion de Prusias 1er de Bithynie (peut-être les Galates furent-ils mêlés à cette dernière guerre, s'il y a lieu d'y rattacher la bataille de Booskephalai) : cf. A. J. Reinach, \emph{Les mercenaires et les colonies militaires de Pergame} dans \emph{Rev. archéol.}, 1909, 1, p. 104. Mais il subsiste toujours cette difficulté de savoir comment la déesse importée de Pessinonte aurait pu être adorée sous le vocable d'Idéenne.

*) Diod. Sic. 34, 33 ; cf. p. 26, n. 2. Mais peut-être y eut-il interpolation du nom topique.

*) Strab. 12, 5, 2. Il parle de la translation après avoir signalé la reconstruction du naos de Pess. par les Attalides, postérieure à 183. Mais il s'agit bien de l'événement de 204, puisqu'il est question de l'oracle sibyllin.

*) Ps. Plut., \emph{De flumin.}, 12, 2.

*) Au 1er siècle : Val. Max. 8, 15, 3 : Pessinunte arcessitam ; cf. 7, 5, 2 : à Phrygiis sedibus... migrantem ; Sil. Ital., \emph{Pun.} 17, 3 : Phrygia sede petitam. --- Au 2e s. : Arrian., \emph{Tact.} 33, 4 : ἐκ Πεσσινοῦντος ἐλθοῦσα. App., \emph{Hannib.} 56 (cf. p 26, n. 4). --- Au 3e s. : Dio Cass. 57, 61 (éd. Dindorff, 1, p. 104) : ἐκ Πεσσινοῦντος κομιζομένην. Herodian. 1, 11 : Πεσσινουντίαν θεὸν. --- Au 4e s. : Arnob. 7, 49: aecita ex Phrygio Pessinunte ; Augustin., \emph{Civ. Dei} 3, 12 : à Pessinunte ; Amm. Marcell. 22, 9, 5 : a quo oppido (Pessinunte) simulacrum translatum.

*) \emph{De M. D.} (\emph{Or.} 5) init. ; Claudian., \emph{De bello gildonico}, 117 : « Palatinis mutasti collibus Idam » ; Prudent., \emph{Contra Symmachum} 1, 187 : « utque Deum Mater Phrygia veheretur ab Ida. »

Pour obtenir et ramener la déesse, Rome délègue une ambassade de cinq membres.* Trois d'entre eux appartiennent à la vieille aristocratie patricienne ; les deux autres sont de la noblesse d'origine plébéienne. Toute la hiérarchie des magistratures se trouve représentée ; il y a un consulaire, un ancien préteur, un ancien édile curule, deux anciens questeurs. Le chef de la mission est M. Valerius Laevinus, qui a commandé plusieurs années de suite la division navale de Grèce, négocié la première alliance avec la ligue étolienne, lutté contre Philippe avec succès sur terre et sur mer, combiné l'expédition d'Eubée avec Attale lui-même, bref l'homme qui connaît le mieux la question d'Orient. L'un des attachés, Ser. Sulpicius Galba, porte un gentilice qui n'est pas moins cher au roi de Pergame.* On met à leur disposition cinq quinquérèmes, c'est-à-dire des galères du type le plus important de la marine nationale. A coup sûr, si l'on entoure cette mission de tout le prestige qui pouvait rehausser la majesté du nom romain, ce n'est point seulement pour honorer la Grande Mère. Par le choix de l'ambassadeur et de ses collaborateurs, par ces préparatifs de démonstration navale dans les eaux du Levant, le Sénat prouve une fois de plus qu'en cette affaire la religion secondait admirablement la politique. L'adoption de l'Idéenne nous apparaît mieux encore comme un épisode de l'alliance entre Rome et Pergame. Mais un lien étroit la rattache quand même à la présence de l'étranger sur le sol italien et aux nécessités d'expiation d'une telle souillure ; car cette alliance n'a d'autre origine que la seconde guerre punique et les menées d'Hannibal en Orient.

*) Liv.  29, 11. M. Valerius Laevinus, préteur pérégrin en 215, propréteur en Grèce de 214 à 211 (egregie adversus Philippum regem terra marique gessit : Liv. 26, 22), consul en 210 et, d'après T. Live, une seconde fois ; M. Caecilius Metellus, édile plébéien en 208, préteur urbain en 206 ; Ser. Sulpicius Galba, édile curule en 209 ; Cn. Tremellius Flaccus et M. Valerius Falto.

*) P. Sulpicius Galba, consul en 211, proconsul de Macédoine, sans cesse renouvelé, venait d'hiverner avec Attale et d'opérer avec lui une croisière.

\subsection{3.}

Avant de prendre le chemin de l'Asie, l'escadre avait gagné le golfe de Corinthe et le port de Delphes.* Les membres de la mission étaient montés au temple pour consulter Apollon. Ils avaient demandé à la Pythie de compléter les prescriptions de la Sibylle. D'après Tite Live, il semble qu'ils l'aient interrogée sur deux points : l'opportunité de la démarche auprès d'Attale, et la nature des rites spéciaux qu'exigeait la réception de la déesse. La première question témoignait d'une habile diplomatie. Attale et Rome étaient deux clients fidèles de la Pythie. On faisait appel à son autorité, si puissante dans le monde grec, pour consacrer leur union définitive ; et la réponse en effet ne pouvait être douteuse. La seconde question avait sa raison d'être dans le formalisme de la religion romaine. Pour introduire dans l'enceinte de la cité un dieu nouveau, il fallait un cérémonial particulier, conforme à la fois au caractère de la divinité et aux instructions des Livres. En général, les Décemvirs faisaient connaître eux-mêmes le rituel à observer. C'était une de leurs fonctions d'interprètes. Quand les Livres imposèrent le culte de Venus Erucina, ils désignèrent en même temps, pour vouer le sanctuaire, « le citoyen qui exerçait la suprême autorité dans la république.* » Douze ans plus tard, la Pythie de Delphes ordonnait de choisir « le meilleur des bons dans tout l'ensemble des citoyens » pour rendre à la Mère Idéenne les devoirs d'hospitalité. L'analogie entre les deux formules est telle que l'on est tenté de leur supposer une même origine. Cette hypothèse est confirmée par Diodore de Sicile, Silius Italicus, Appien,* qui attribuent également la dernière prescription aux Livres, c'est-à-dire aux Décemvirs. La Pythie, si vraiment elle fut consultée à ce sujet, n'aurait donc eu qu'à ratifier la décision du sacré collège ; elle confirmait au parti noble ses droits sur le nouveau culte. Mais il y a plus, et Tite Live est incomplet. C'est un véritable sacerdoce qu'est chargé d'exercer « le meilleur citoyen. » Or le sacerdoce de la Mère des Dieux comporte à la fois un prêtre et une prêtresse. Durant le long voyage sur mer, un prêtre et une prêtresse du Megalesion de Pergame accompagneront la Dame noire. Il faut, pour la recevoir quand elle abordera sur le sol latin, un Romain et une Romaine. Les Décemvirs ou la Pythie demandèrent que l'on élût la plus vertueuse des Romaines en même temps que le meilleur des Romains. La tradition authentique s'était conservée au temps de Cicéron et a été connue de Diodore.*

*) La mission envoyée par un Ptolémée à Sinope pour obtenir la statue de Pluton commence aussi par le pèlerinage de Delphes : Tac., \emph{Hist.} 4, 84.

*) Liv. 22, 10 : « is cuius maximum imperium in civitate esset. »

*) Diod. Sic. 34, 33 ; Sil. Ital., \emph{Pun.} 17, 5-7 ; Appian., \emph{Hannib.} 56 ; cf. \emph{De vir. ill.} 46.

*) Cic., \emph{Pro Coelio} 14, 34 et \emph{De harusp. resp.} 13, 26 : « (sacra) vir is accepit qui est optimus P. R. iudicatus, P. Scipio, femina autem quae matronarum casissima putabatur, Q. Claudia » ; cf. Diod. Sic., \emph{l. c.}

Ce fut le Sénat qui choisit. Au retour de l'escadre, et comme elle était déjà dans les eaux tyrrhéniennes, M. Valerius Falto fut dépêché à Rome pour annoncer que la Dame était proche et préciser sans doute quels rites étaient nécessaires. Il y eut consultation immédiate de la Curie, afin de nommer les deux personnages et d'établir le programme définitif des fêtes. L'élu, le « glorieux vainqueur,* » fut P. Cornelius Scipio Nasica. Il était fils de ce Cn. Scipio Calvus qui était tombé sur un champ de bataille de l'Espagne, et cousin germain du général qui était alors l'homme le plus populaire de la république. Il n'avait, pas vingt-huit ans. Il ne s'était pas encore mêlé à la vie politique ; et son jeune passé n'avait ni ennemis ni envieux. [L'élue, que les matrones avaient elles-mêmes, ce semble, désignée,* fut Claudia Quinta.* Elle était fille de P. Claudius Pulcher, un consul de l'année 305/249, et sœur d'Appius Claudius, consul en 542/212. Peut-être était-elle mariée à un Valerius.*

*) Liv.  29, 14 : veram victoriam. « Nomen melius maiusque triumphis, » dit Silius Italicus, qui suit de très près ici le récit de Tite Live. Plin., \emph{H. n.} 7, 34 : « vir optimus semel a condito aevo iudicatus est Scipio Nasica a iurato senatu. »

*) Cf. Plin., 2. n. 7, 35 : « pudicissima femina, matronarum sententia, » à propos de Sulpitia, qui fut choisie parmi les cent principales matrones pour dédier une statue de Vénus ; au sujet de Claudia, Pline suit la tradition du miracle et croit qu'elle fut jugée la plus honnête religionis experimento ; cf. infra.

*) Pour sa généalogie, Ljunggren, \emph{De gente patricia Claudiorum}, Upsal, 1898, p. 53 ; cf. de Vit, \emph{Onomastic}, p. 301, et Pauly-Wissowa, \emph{R. Encycl.}, 3, p. 2899.

*) Diod., \emph{l. c.}, la nomme Valeria.

Sur le programme de la réception officielle, nous ne sommes qu'imparfaitement et contradictoirement renseignés. Ce qui paraît certain, c'est qu'il comprenait deux parties : une fête à Ostie et une autre à Rome. S'il faut en croire Tite Live, les matrones avaient accompagné au port d'Ostie Claudia et Scipio. Dès que le navire chargé du divin fardeau parvient en face de l'embouchure du Tibre, Scipio monte sur une embarcation et va chercher la déesse. Il la reçoit des mains des prêtres anatoliens. Dans les bras du jeune homme elle gagne la terre ferme, et celui-ci la remet aux femmes du plus haut rang, qui h tour de rôle la portent jusqu'à Rome.* « Toute la cité était accourue à leur rencontre. Aux portes des maisons, sur le chemin de la procession, l'encens brûlait dans les turibules. On priait sur le passage de la déesse. On la suppliait d'entrer de son plein gré dans la ville et de lui être propice. On la déposa dans le sanctuaire de la Victoire, sur le Palatin, la veille des nones d'avril, et ce jour devint désormais un jour de fête. Le peuple se rendit en foule au Palatin avec des offrandes. Il y eut un lectisterne et des jeux que l'on appela Mégalésies.* » Pourquoi choisit-on le temple de la Victoire pour y loger la Dame, en attendant que le sien fût construit ? Sans doute parce qu'il était le plus voisin de l'emplacement que Ton destinait à l'Idéenne.* Mais comme le séjour de la bonne et puissante Mère dans Rome était un gage de victoire, ne devons-nous pas supposer qu'il y eut aussi une intention symbolique, ou simplement superstitieuse* ? Diodore se borne à dire que la population tout entière fut convoquée, par ordre de l'oracle, pour aller au-devant de la Dame et lui faire escorte à son entrée. Scipio marchait en tête des hommes, et Claudia conduisait la procession des matrones ; ce furent eux deux qui reçurent les « hiera. » Ovide nous montre Rome tout entière à Ostie : sénateurs, chevaliers, le peuple, « les matrones, leurs filles et leurs brus, » et la sainte congrégation des Vestales. C'est par voie fluviale que l'on amène la déesse jusqu'aux portes de la cité. Le vaisseau s'enlise dans les sables. Il faut un miracle de la Mère des Dieux pour le renflouer. L'histoire se transforme chez le poète en légende dorée. Mais le miracle ne s'est produit que sur le tard. « Déjà il était nuit ; on attache à un tronc de chêne le câble qui tient le navire ; on dîne et tout le monde va se coucher. » Le lendemain, on immole une génisse devant la poupe enguirlandée ; on brûle de l'encens, puis on détache l'amarre. Au confluent de l'Almo, un vieux prêtre phrygien, vêtu de rouge, baigne l'idole. Portée sur un char que traînent des génisses fleuries, elle entre par la porte Capena. « Scipio la reçut dans Rome. » La fête aurait donc duré deux jours. La déesse et son cortège auraient remonté en bateau le cours du Tibre jusqu'à l'Almo. Avant de franchir l'enceinte, on aurait accompli le rite de la Lavation. Ce détail, à vrai dire, paraît être inventé tout exprès pour rappeler et expliquer une cérémonie qui se renouvelait tous les ans. Il est invraisemblable, d'autre part, qu'en venant du Tibre on ait adopté l'itinéraire indiqué par Ovide. C'est par la porte Capena que passait en effet la solennelle procession du Bain ; mais on ne pouvait arriver d'Ostie que par la porte Trigemina, entre le fleuve et l'Aventin, ou par la porte Raudusculana, entre l'Aventin proprement dit et la Remuria. Aussi bien, avons-nous déjà constaté qu'Ovide n'a pas la prétention d'être un scrupuleux historien. Silius Italicus, de son côté, paraît surtout préoccupé de rechercher l'effet pittoresque. Il représente Scipio sur la grève, attendant l'apparition de la flotte. Dès que l'on aperçoit à l'horizon les vaisseaux, Scipio prend l'attitude de la supplication et tend vers la déesse les paumes de ses mains. Quand le navire de l'Idéenne a franchi la passe du Tibre, les matrones elles-mêmes s'emparent du câble de halage. La population est en prières et clame ses vœux. Cependant, sur le pont, autour de la Dame, les Galles chantent des cantiques et font résonner cymbales et tambourins. D'après saint Augustin,* qui a puisé tant de renseignements dans Varron, Scipio garda le fétiche dans ses mains et l'apporta lui-même à Rome. Quant à l'empereur Julien, il est aussi banal que sobre. « Le peuple sort de la ville avec le sénat ; en tête marche le clergé au grand complet, revêtu des costumes de cérémonie. Lorsque le navire pénètre dans le port, chacun se prosterne sur le rivage. » Comme Ovide, Julien a hâte d'en venir au miracle.

*) Si le trajet se fait par voie de terre, c'est évidemment en voiture qu'elles franchissent cette vingtaine de kilomètres. La loi Oppia autorisait les femmes à se servir \emph{iuncto vehiculo sacrorum publicorum causa} : Liv. 34, 1.

*) Liv.  29, 14 (les mss. indiquent la veille des ides au lieu de celle des nones, qui est le jour férié) ; cf. Diod. Sic., \emph{l. c.} ; Ovid., \emph{Fast.} 4, 290-348 ; Sil. Ital., \emph{l. c.} 10 ss.

*) Peut-être aussi parce qu'une tradition rattachait les origines de ce temple à la légende d'Évandre et à celle d'Énée : Païs, \emph{Storia di Roma}, 1, p. 206.

*) Cf. supra, p. 32 et 36, n. 2.

*) \emph{Civ. Dei} 2, 5.

Cette réception de la Mère des Dieux est caractérisée à la fois par le rôle confié aux matrones et par celui qu'y assume la noblesse. La participation officielle des femmes était une nécessité religieuse, puisqu'il s'agissait d'une divinité féminine. Elle était conforme aux traditions orientalo-grecques du culte métroaque, où les femmes détiennent la majorité des services rituels. Elle était conforme aux traditions romaines : Pontifes et Décemvirs attribuent volontiers aux femmes une place importante dans les cérémonies extraordinaires.* Leur présence enfin s'imposait plus particulièrement cette fois, comme l'hommage des mères à la Mère de Salut, après les longues terreurs d'une guerre qui avait endeuillé tous les foyers. Il ne manqua pas une matrone à Ostie ; elles étaient fières de pouvoir rappeler ce titre à la reconnaissance publique, lorsque l'on proposa, moins de dix ans après, d'abroger une loi qui limitait leur droit au luxe.* Leur mission fut d'accueillir l'Idéenne au moment où elle débarquait sur la terre latine et d'assurer son voyage jusqu'au Palatin. Mais on avait respecté la hiérarchie des castes. Les dames nobles étaient aux premiers rangs ; et les patriciennes seules* eurent le privilège de porter dans leurs mains la pierre sainte. Claudia Quinta, qui avait été l'objet de la plus insigne distinction, appartenait à la plus haute aristocratie ; elle était d'une famille réputée entre toutes pour sa morgue nobiliaire.

*) Cf. Liv. 21, 62, 22, 1, 27, 37.

*) Liv. 34, 3 et 5.

*) Matronae primores civitatis. Liv.  29, 14.

De même, sous ce régime aristocratique, le \emph{vir optimus} ne pouvait être qu'un patricien. Cornelius Scipio et son cousin, le consul sortant, représentent les générations nouvelles de l'antique patriciat. Loin de rien abdiquer de la fierté des ancêtres, la caste est d'autant plus jalouse de ses privilèges et de sa force, en cette fin du 3e siècle, qu'elle a conscience d'avoir bien servi l'État pendant la guerre et quelle a reconquis, par son mérite, les hautes charges de la République. Les Cornelii vont être, durant plus d'un siècle, les chefs des optimates, les défenseurs ardents du gouvernement des meilleurs. L'élection de P. Cornelius, qui dans cinq ans sera commissaire colonial et dans quatorze ans consul, offre donc une signification politique nettement accusée. Ainsi le parti aristocratique, après avoir fait venir d'Orient la Mère Idéenne, l'accapare dès son arrivée. C'est lui qui donne à la déesse l'hospitalité dans Rome et contraint la religion nationale à cette rupture avec ses plus rigoureuses traditions. Il était interdit, en effet, aux divinités pérégrines de franchir l'enceinte sacrée du pomoerium ; elles s'étaient accumulées au Champ de Mars, sur l'Aventin, dans l'île du Tibre. L'Idéenne pénétrait au cœur même de la cité, s'installait sur la colline où Rome était née, où les Romains montraient encore avec orgueil la cabane de Romulus, où ils possédaient leurs plus anciens sanctuaires.* Et c'était le parti conservateur qui avait opéré cette révolution. A vrai dire, le principe d'exclusion des dieux étrangers ne se vérifie plus, comme loi exacte, à partir de 537/217 ; cette année-là, des temples furent voués sur le Capitole à deux divinités grecques, ou du moins hellénisées, Venus Erucina et Mens. Après Trasimène, en raison même de la nature et du nombre des prodiges à conjurer, les portes de Rome s'ouvrirent plus largement aux divinités helléniques. Le trouble des esprits achevait de désorganiser la religion primitive. Toutefois, si nous considérons la liste des trente temples d'État construits entre 216 et 50 avant notre ère, nous n'en trouvons que huit seulement à l'intérieur du pomoerium ; et encore, sur ce nombre restreint, en voyons-nous cinq qui sont dédiés à des divinités indigètes ou italiques. Puisque Mens, Venus Erucina et Mater Idaea restent les seuls \emph{novensides} d'origine hellénique ou orientalo-grecque, ce sont des raisons du même ordre qui sans doute leur ont fait accorder ce même avantage. Or l'adoption de la Vénus du mont Eryx se rattache aussi à des considérations d'urgente politique : on redoutait un soulèvement de la Sicile occidentale. L'introduction du culte de Mens, lequel paraît venir de Campanie ou de Grande Grèce, doit avoir une cause analogue : peut-être voulait-on éviter une défection d'alliés. Le privilège sans précédent qui, à douze ans d'intervalle, mettait ces trois déesses de pair avec les divinités du vieux sol latin, témoigne d'une même diplomatie. Le salut de la République est jugé d'un intérêt supérieur au maintien des traditions. Aussi bien les liens de l'Idéenne et de l'Erycine avec la légende d'Énée pouvaient-ils servir de prétexte à la noblesse et au Sénat pour justifier ce régime d'exception. Est-ce par un pur hasard que le temple de la déesse phrygienne fut bâti près du Lupercal et que Rhéa Cybèle vint s'établir au milieu des reliques les plus saintes des fils de Rhéa Silvia ? Quoi qu'il en soit, le Palatin n'avait pas cessé d'être habité par les vieilles familles de l'aristocratie. La déesse allait demeurer sur la colline patricienne, de même que la plébéienne Cérès avait son temple sur la colline plébéienne de l'Aventin. Enfin, ce sont les patriciens et les nobles qui se chargent de fêter comme il convient la Grande Mère. Le droit de célébrer par des repas de corps l'anniversaire de la réception est réservé aux « princes » romains ; et ces banquets offrent le même caractère, portent le même nom que ceux de la plèbe aux fêtes de Cérès. Il y a donc là, ce semble, le témoignage d'un véritable antagonisme entre les nobles et les non nobles (\emph{ignobiles}). Au début du 2e siècle, les deux cultes de la Mère Idéenne et de Gérés représentent les deux éléments de la cité, comme leurs deux temples se font face. Mais en pleine guerre punique, cet antagonisme, qu'accentue chaque année l'affaiblissement de la classe moyenne et l'accroissement du prolétariat urbain, ne peut pas être une cause de désordre et de révolution. Rome est encore à plus de soixante ans des Gracques. Le peuple se borne à murmurer contre le gouvernement, qui l'accable de réquisitions, et contre les chefs de la guerre. On forme parfois des attroupements séditieux sur le Forum. On y reproche aux patriciens et aux nobles plébéiens de pactiser contre la plèbe ; on les accuse presque de complicité avec Hannibal. « Les consuls prennent plaisir à ruiner la plèbe, à la réduire en lambeaux.* » Puisqu'on lui reproche d'être la cause de tout le mal, l'aristocratie sénatoriale, qui s'identifie avec l'État, présente le remède. Mais il doit paraître naturel aussi qu'elle-même offre l'hospitalité à la déesse, qu'elle-même assure la perpétuité de la confrérie cultuelle.

*) D'après Agathoclès de Cyzique (cf. supra, p. 42, n. 4), les compagnons d'Énée auraient construit sur le Palatin un temple de Fides. Confusion possible avec le temple de la Victoire : cf. Païs, \emph{l. c.}, p. 171.

*) Liv. 22, 34 (en 216) : « ab hominibus nobilibus, per multos annos bellum quaerentibus, Annibalem in Italiam adductum ; ab iisdem... fraude id bellum trahi ; ...id foedus inter omnes nobiles ictum ; ...plebeios nobiles contemnere plebem... » ; 26, 35 (en 210) : « plebem romanam perdendam lacerandamque sibi consules sumpsisse... ; haec non in occulto, sed propalam in foro ingens turba circumfusi fremebant. »

La plèbe ne songea point à protester. Elle ne se demanda pas non plus si la pierre noire était un symbole d'union avec d'autres peuples, un gage diplomatique ; elle y vit un talisman. A ses yeux, la sainte alliance était avec les dieux, non avec les hommes. La déesse apportait avec elle, non pas des promesses lointaines de conquêtes, mais des promesses de paix prochaine, de prospérité nouvelle et de riches moissons que ne saccagerait plus l'ennemi. Avec elle, aussi, l'Idéenne amenait son clergé oriental, vêtu de robes éclatantes, chargé d'icônes, diseur de bonne aventure, sa musique de tambourins, de cymbales, de flûtes et de chants étrangement rythmés, tout un exotisme qui charme la foule et l'entraînera vers le Palatin, où doit reposer la Dame.

\subsection{4.}

Le souvenir de la réception faite à la Mère Idéenne, ce 4 avril, ne pouvait s'effacer des traditions populaires ; car l'anniversaire de cette cérémonie demeura, jusqu'à la fin du paganisme, une fête du calendrier romain. Mais la légende vint rapidement embellir l'histoire. 

Elle ne pouvait guère s'exercer sur la personnalité de P. Cornelius Scipio. Sa vie d'homme politique, fixée dans les écrits des analystes, laissait trop peu de place au mystère. Le titre d'Optime entre les Optimes lui mérita le respect de la postérité, qui lui décerna celui de Très Saint,* par reconnaissance ; il ne fit point travailler l'imagination des foules.* Mais les légendes, à Rome, n'ont pas toutes une origine populaire. L'orgueil des grandes familles en créa beaucoup, qu'il substituait sciemment à l'histoire. Il semble que les Scipions aient tenté d'en propager une, au sujet du rôle de l'Optime. On disait de Nasica, pour définir son pieux ministère, qu'il avait « reçu » la déesse* ; on disait aussi qu'il fut « l'hôte de la Mère Idéenne. » Tite Live attribue à la Pythie cette dernière formule, rapportée par les délégués. Le terme même d'hôte offre une signification très précise, dont on ne manqua point de tirer parti. Cornelius n'aurait pas seulement reçu la déesse dans ses mains, pour la transporter du navire à terre ; il l'aurait ensuite reçue chez lui, dans sa maison de Rome. Il l'aurait momentanément hébergée auprès des pénates de son foyer. Quelle autre maison pouvait se glorifier d'un pareil privilège ? Cette tradition est enregistrée par Valère Maxime,* au temps d'Auguste et de Tibère. A vrai dire, le fait d'une hospitalisation n'était pas rituellement impossible. La théoxénie est un rite grec, et l'oracle delphique avait bien pu en suggérer l'idée. Dans beaucoup de villes, à certaines fêtes, certaines divinités recevaient ainsi l'hospitalité chez les citoyens : tel, à Delphes même, Apollon ; telle, à Mégare, la Mère des Dieux.* Cet usage commémorait-il, en général, l'introduction d'un culte étranger ? On le croirait volontiers en voyant, par exemple, la théoxénie d'Isis à Kios. Pratiqué tout spécialement à Rome pour la réception de la Grande Mère, le rite religieux pouvait s'être imposé surtout comme une nécessité juridique. La Dame noire est venue avec ses \emph{sacra}, c'est-à-dire avec son culte national et les ministres de son culte.* Le prêtre et la prêtresse de Phrygie, qui l'ont accompagnée sur le navire, la suivent dans Rome. Ils restent toutefois étrangers à la cité romaine. Hôte de leur déesse, Nasica leur assurait les droits d'hospitalité ; il constituait leur personnalité légale. Ainsi comprise, l'hypothèse d'une théoxénie ne serait pas incompatible avec le récit de Tite Live. Pour que la formalité rituelle fût accomplie, ne suffisait-il pas d'un simulacre d'hospitalisation, d'un court arrêt du cortège dans la maison de l'Optime, avant de monter au temple de la Victoire* ? Mais ce ne sont là que des conjectures. Peut-être est-ce attacher trop d'importance au texte de Valère Maxime, anecdotier plus qu'historien, et qui n'a jamais eu la réputation de préférer la vérité aux flatteries. L'absence de toute commémoration théoxénique, dans les Mégalésies romaines d'avril, est un sérieux argument contre la Métroxénie de Nasica. Ce qui ne la rend pas moins suspecte, c'est que nous connaissons d'autres traditions de même ordre qui sont purement légendaires. L'introduction du culte d'Asclepios dans Athènes, après la peste de 431, nous en fournit un exemple typique.* Quand Asclepios arriva d'Epidaure, le 17 boedromion 421, il reçut l'hospitalité chez les Déesses d'Eleusis. Or Sophocle écrivit en son honneur un péan, que l'on chantait encore au 3e siècle de notre ère ; dévot du dieu guérisseur, il lui éleva même un autel. Il n'en fallait pas davantage pour créer une légende. Dès la génération suivante, le poète passait pour avoir logé chez lui le dieu. Bientôt les Athéniens reconnaissants faisaient de Sophocle un héros sous le nom de Dexiôn, l'Accueillant. Les Romains se contentèrent de qualifier Cornelius de Saint. Mais la légende romaine semble être une réplique de la légende grecque.

*) Val. Max. 7, 3, 2 : « sanctissimis manibus » ; 8, 15, 3 : « sanctissimo viro » ; Juv. 3, 137 s : « da testem Romae tam sanctum quam fuit hospes Numinis Idaei. »

*) Il est à noter toutefois que l'on finit par lui attribuer le voyage à Pessinonte : Amm. Marc. 22, 9, 5.

*) Cic., \emph{l. c.} et \emph{Brut.} 20 : (is) qui sacra accepit.

*) Val. Max. 8, 15, 3 : « eius manibus et \emph{penatibus senatus}, Pythii Apollinis monitu, deam excipi voluit ; » cf. Schol. ad Juv., \emph{l. c.}, éd. Iahn, 1851, p. 205 : « hospes... id est ut simulacrum domi suae haberet, dum ei templum fieret. »

*) Dioscures à Paros et Agrigente, Apollon à Pallène d'Achaie, \emph{etc.} ; ref. Daremberg et Saglio, \emph{Dict. des Antiq.}, s. v. \emph{hospitium}.

*) Cf. le chapitre suivant.

*) T. Live ne se contredisait donc pas en déclarant, 36, 36, que Cornelius « apporta la déesse de la mer au Palatin » ; sa mission avait continué dans Rome. Ovide, \emph{Fast.} 4, 347, paraît faire allusion à un rite qui aurait eu lieu à l'intérieur de l'enceinte : \emph{Nasica accepit}.

*) Foucart, \emph{Grands mystères d'Éleusis}, 1900, p. 116 s. La version qui faisait honneur à Sophocle de l'hospitalité offerte au dieu provoqua les protestations d'un contemporain de l'événement, Telemachos d'Acharnae. Celui-ci fît graver sur un marbre le véridique récit de l'arrivée du dieu ; l'inscription a été retrouvée dans les fouilles de l'Asclepieion.

Le rôle de Claudia Quinta fut peut-être plus purement honorifique. Mais c'est Claudia qui devient l'héroïne légendaire de la fête. Sa renommée éclipsera celle de Scipio. Car Scipio reste dans l'histoire l'élu du Sénat ; Claudia devient, dans la tradition, l'élue de la Mère Idéenne. D'après les témoignages les plus anciens, elle était considérée de son temps comme la plus vertueuse des matrones.* On lui confia pour cette raison l'honneur de recevoir la Dame. La légende se forma par un phénomène de renversement de la cause et de l'effet. Ce fut l'intervention de Claudia dans la cérémonie d'Ostie qui, aux yeux de tous, manifesta sa vertu ; cette vertu était donc suspecte ; cette intervention fut donc fortuite. Quelque incident, grossi par l'imagination populaire, vint déplacer l'intérêt au détriment du rôle officiel, qui finit par être oublié. Peut-être avait-on crié au miracle.* L'élément merveilleux passa au premier plan dans le souvenir de la postérité. Mais aucun texte ne signale ce miracle avant les premiers temps de l'Empire. Cicéron, qui rappelle par deux fois à Clodius l'antique vertu de Claudia, n'en fait pas mention. En revanche, Properce* en parle comme d'une histoire bien connue. Ovide contribua sans doute beaucoup à la répandre. « Les hommes, malgré leur zèle, fatiguent en vain leurs bras à tirer la corde tendue. A grand peine le navire qui porte la déesse remonte le courant. Sa carène s'enfonce dans la vase et y demeure immobile, telle une île en mer. Ce prodige est comme un coup de foudre ; les hommes restent sur place, épouvantés. Claudia Quinta était de la race de l'antique Clausus ; elle était belle... Vertueuse, elle passait pour ne point l'être. Sa coquetterie, la variété élégante des coiffures dont elle se parait en public, lui avaient fait tort* ; et, de l'avis des vieillards sévères, elle parlait trop. Claudia sort du groupe des honnêtes matrones, trois fois répand sur sa tête l'eau pure du fleuve, et trois fois lève au ciel les paumes de ses mains. Elle tombe à genoux, fixe les yeux sur l'image de la déesse et, les cheveux épars, prie : « Je suis la suppliante, Mère des Dieux, toi qui les as enfantés et nourris. On dit que je ne suis point chaste. Si tu me condamnes, je m'avouerai coupable et vaincue ; la mort sera mon châtiment : car c'est une Déesse qui me juge. Mais si le crime est imaginaire, tu donneras un gage réel et visible de la pureté de ma vie. Chaste divinité, tu t'abandonneras à de chastes mains. » Elle dit, et presque sans effort tire le câble. La déesse s'émeut, suit son guide, pour la gloire de Claudia. Un cri de joie monte jusqu'aux astres. »

*) Cic., \emph{De har. resp.} 13 ; cf. Diod. Sic., \emph{l. c.}

*) Il est à remarquer qu'un miracle accompagne toujours l'arrivée d'une divinité étrangère ; il a pour but non seulement de manifester la puissance du dieu, mais aussi de prouver que le dieu pénètre de son plein gré dans sa nouvelle résidence. J'ai déjà signalé l'exemple de la Junonde Véies. Cf. le Pluton de Sinope que sont allés chercher les ambassadeurs d'un Ptolémée : le roi du pays hésite trois ans à livrer la statue ; celle-ci finit par quitter son temple et monter d'elle-même à bord de la flotte (Tac., \emph{Hist.} 4, 84). Plutarque, \emph{De Is. et Osir.}, déclare simplement que, las d'attendre, les envoyés enlevèrent l'idole. Cf. Bouché-Leclercq dans \emph{Rev. des religions}, 1902, 2, p. 14 ss ; Dattari dans \emph{Rev. arch.}, 1906, 2, p. 323 : la statue serait arrivée en 213, sous Ptolémée 4.

*) Propert. 4, 11, 52 ; cf. Plin., \emph{H. n.} 7, 34.

*) Pour les vers 309 et 317, cf. Lactant., \emph{Inst.} 2, 7, 12. Auguste, considérant sa fille comme une honnête femme en dépit des apparences, la comparait à Claudia : Macrob., \emph{Saturn.} 2, 5, 4.

Cette légende était si récente que Tite Live a craint de compromettre sa dignité d'historien en la rapportant. Il se borne à une allusion des plus obscures et paraît vouloir expliquer en même temps l'origine de cette tradition merveilleuse. Claudia, si elle était vraiment coupable, ne pouvait exercer « un ministère aussi sacré » sans s'attirer le courroux de la divinité. Celle-ci manifesta l'innocence de la matrone en ne protestant pas. C'était là une forme très rudimentaire de miracle ; mais ce pouvait être le point de départ d'une légende. Tite Live en affaiblit du reste le merveilleux, en spécifiant que le navire qui portait l'idole n'entra point dans le Tibre et s'arrêta au large. Si l'accident n'arriva qu'à l'embarcation qui fit le service entre la galère et la rive, le voici déjà ramené à des proportions plus modestes !

Ce fut sans doute le culte particulier des Jules pour la déesse qui raviva la popularité de la Mère Idéenne et fit s'épanouir autour du temple du Palatin cette floraison de légendes locales. Mais l'avènement des Claudes à l'Empire, avec Tibère, donne un singulier relief au rôle de Claudia, la grande Sainte de la nouvelle famille impériale. Aussi la légende va-t-elle s'amplifier jusqu'à la fin du paganisme, et nous pouvons suivre les différents états de son évolution. Dans les derniers temps de la dynastie Claudienne, Silius Italicus* s'inspire encore beaucoup des \emph{Fastes} d'Ovide ; mais déjà il ajoute un élément nouveau de merveilleux : dès que Claudia saisit le câble, on entend les lions de Cybèle rugir, les tambourins sacrés résonner d'eux-mêmes, et l'on voit le vaisseau s'avancer tout seul, comme poussé par les vents. A partir du 2e siècle, le câble a disparu ; c'est sa propre ceinture que l'héroïne attache à la proue.* Un peu plus tard, on contera que la ceinture, lancée du rivage sur le vaisseau, s'était nouée d'elle-même à l'avant.* Pour rendre plus éclatant le triomphe miraculeux de la vertu, ce ne sont plus seulement les hommes qui sont impuissants à mouvoir le navire ensablé. On attelle des bœufs, on apporte des machines* ; tous les moyens dont dispose l'humanité sont inutiles. Au 4e siècle, on ira jusqu'à remplacer la ceinture de Claudia par ses cheveux aux longues tresses* ! Il est vrai qu'ils avaient leur rôle dans la première légende, puisque la matrone leur devait en partie sa mauvaise réputation. Ce détail nous aide à comprendre quel travail le temps opéra sur le fonds primitif de la tradition.

*) Sil. Ital., \emph{Pun.} 17, 20-47.

*) Déjà peut-être dans Sénèque, v. infra ; Appian., \emph{l. c.} ; Tertull., \emph{Apolog.} 22 ; Lactant., \emph{l. c.} ; Augustin., \emph{Civ. Dei}, 10, 16 ; Julian., \emph{l. c.}

*) Herodian., 1, 11.

*) Augustin., \emph{l. c.} ; Jul., \emph{l. c.}

*) Claudian., \emph{Laus Serenae}, 17 :« ducens Claudia virgineo cunctantem \emph{crine} Cybeben. »

Peu à peu, la condition même de Claudia s'était aussi modifiée. La matrone était devenue Vestale. Les accusations pesaient donc plus graves sur elle, et les proportions de l'événement se trouvaient grandies. Le péché de la femme adultère ne retombait que sur elle et sur les siens ; le crime de la Vestale compromettait la République tout entière. L'accident du navire devenait un signe manifeste de la colère des Dieux contre Rome.

Cette autre forme de la légende n'est pas antérieure à la dynastie Claudienne. Elle se rencontre pour la première fois dans un fragment de Sénèque cité par saint Jérôme,* si toutefois le titre de « Virgo Vestalis » n'est pas dans un commentaire plutôt que dans une citation. Sous les Flaviens, Stace* donne à Claudia l'épithète de vierge, sans spécifier toutefois qu'elle fût Vestale. Nous sommes à la période de transition. La transformation n'apparaît vraiment définitive qu'au me et au 4e siècles, avec Hérodien, Julien et l'auteur du \emph{De viris illustribus}.* Mais la tradition ancienne, celle de Cicéron, d'Ovide et de Tite Live, n'est jamais oubliée. Pour Pline l'Ancien, Claudia est toujours la matrone. Tacite et Suétone* nous parlent d'elle sans la qualifier. Le fait ne s'expliquerait guère s'il s'agissait d'une Vestale. Car l'usage voulait qu'on honorât ces prêtresses de leur titre. Pourquoi Suétone le réserve-t-il à une autre Claudia, lorsqu'il énumère toutes les femmes illustres de la famille impériale ? Au 2e siècle, Appien, qui connaît déjà la légende de la ceinture, dit encore formellement que l'on accusait Claudia Quinta d'adultère et que la Déesse enlisée appelait à son secours une matrone ayant gardé la foi conjugale. Même aux siècles suivants, nous retrouvons parfois la tradition historique.* Il est, en outre, facile de déterminer sous quelles influences elle s'est altérée. Il y eut dans la gens Claudia, vers l'an 600 de Rome, un demi-siècle après l'arrivée de la Mère Idéenne, une Vestale demeurée célèbre pour un de ces traits d'audace qui n'étaient point rares dans la famille. Comme son père, d'après Cicéron, ou son frère, d'après Suétone, triomphait sans l'ordre du peuple, elle monta sur le char triomphal jusqu'au Capitole, afin d'empêcher toute intervention des tribuns. Avec le temps il s'établit une confusion entre les deux héroïnes, qui portaient le même nom et dont les glorieux souvenirs restaient inséparables dans l'histoire de la maison des Claudes.* On racontait d'autre part que certaines Vestales avaient manifesté leur innocence par des faits miraculeux. Tucia avait transporté de l'eau du Tibre dans un crible : Æmilia avait rallumé le feu sacré en jetant sur l'autel sa tunique de lin. C'était un miracle analogue qu'avait accompli la Mère des Dieux en faveur de Claudia Quinta : le rapprochement était naturel, et Properce l'indique. La transformation de la matrone en Vestale ne se trouvait-elle pas ainsi toute préparée ? On pourrait ajouter d'autres raisons encore. L'esprit de syncrétisme, qui fit identifier la Grande Mère à Vesta et à cette Bonne Déesse honorée d'un culte particulier par les Vestales, facilita peut-être le travail de l'imagination populaire. Je crois surtout qu'il importe d'attribuer une large part aux jeux scéniques dans la diffusion des détails légendaires qui ont défiguré l'événement d'Ostie. Nous savons en effet que l'on représentait ce miracle aux Jeux des Mégalésies.*

*) Hieron., \emph{Adv. Iovin.} 1, 25 = Sen. \emph{Fragm.} 80 (éd. Haase, 1886, 3, p. 433) : « Claudia virgo Vestalis quum in suspicionem venisset stupri, ...ad comprobandam pudicitiam suam fertur cingulo duxisse navem, etc. » Mais la véritable citation de Sénèque ne commence qu'à la phrase suivante.

*) \emph{Silv.}, 1, 2, 245 ss.

*) Herodian., \emph{l. c.} : τὴν ἱέρειαν τῆς Ἑστίας ; Julian., \emph{l. c.} ; cf. \emph{De vir. ill.}, 46: \emph{Claudia Virgo Vestalis} ; Claudian., \emph{l. c.}

*) Plin., \emph{H. n.} 7, 35 ; Tac., \emph{Ann.} 4, 64 ; Suet., \emph{Tiber.} 2. Encore peut-on dire que le terme de \emph{pudicitia}, employé par Suétone, s'applique surtout aux matrones ; c'est précisément le vocable dont se servent T. Live et Pline l'ancien.

*) Min. Fel., \emph{Octav.} Lactant., \emph{l. c.} ; sans doute aussi Augustin., \emph{l. c.}

*) On les réunissait très souvent dans la même formule, comme exemples de \emph{domestica laus in gloria muliebri} : Cic., \emph{Pro Coel.} 14, 34.

*) Ovid., \emph{Fast.} 4, 328 : « scena testificata loquar. »

L'art n'avait pas moins contribué que la poésie et le théâtre à perpétuer le souvenir de Claudia. Une statue de la matrone se dressait dans le pronaos du temple de l'Idéenne, au Palatin. D'après Valère Maxime,* elle s'y trouvait déjà lors du premier incendie de cet édifice, à la fin du 2e siècle (643/111), quatre-vingts ans seulement après la dédicace du sanctuaire. Comme si la déesse continuait à protéger celle qui jadis avait sauvé son navire, la statue échappa miraculeusement à cet incendie et à celui qui ravagea le Palatin sous le règne d'Auguste (756/3). Mais Tacite* déclare au contraire qu elle ne fut consacrée dans le temple qu'après avoir été respectée deux fois par les flammes. Le témoignage de Valère Maxime est suspect. Car l'auteur montre peu de scrupules quand il s'agit de flatter Tibère et sa famille. Cependant, comme il est contemporain de l'incendie de l'an 3, on peut admettre que la consécration est antérieure à cette date et la faire remonter au dernier siècle de la République. Borghesi et M. Babelon* croient reconnaître la statue sur un « aureus » et sur un denier de C. Clodius C. f. Pulcher, qui furent frappés après la mort de César (711/43). Ces médailles portent au revers une figure de femme assise et voilée, avec l'inscription « Vestalis » ; et les historiens modernes qui comme Borghesi et Mommsen, admettent que Claudia fut une Vestale, y voient une confirmation des documents littéraires. Les numismates ont en effet raison de chercher dans cette effigie un type de statue et d'y reconnaître le portrait d'une ancêtre illustre de la famille du monétaire. Mais il ne peut être ici question que de la fameuse Vestale dont Cicéron et Suétone ont raconté le séditieux exploit. Aussi bien ne comprendrait-on guère l'attitude assise de Claudia Quinta.

*) Val. Max. 1, 8, 11 : « Claudiae statua in vestibulo templi Matris Deum posita, bis ea aede incendio consumpta, prius P. Nasica Scipione et L. Bestia, item M. Servilio et L. Lamia cos., in sua basi flammis intacta stetit. »

*) \emph{Ann.} 4, 61 : Claudiae Quintae statuam. vim ignium bis elapsam maiores apud aedem Matris deum consecravisse. Tacite compare ce miracle à celui qui sauva de l'incendie une statue de Tibère, laquelle se trouvait dans la maison d'un particulier.

*) Borghesi, \emph{Numism.}, 2, p. 178-183 ; Babelon, \emph{Monn. de la rép. rom.}, 1, 352-354 (avec reproduction) : cf. Cohen, \emph{Monn. cons.}, pl. 12.

La sculpture avait reproduit non seulement les traits de l'héroïne, mais encore la scène même du miracle. Julien nous dit que « des images en bronze » perpétuaient dans Rome cette histoire. Il nous reste le souvenir de ces représentations figurées sur un médaillon de Faustine l'aînée et sur un autel votif qui est au musée du Capitole. Le médaillon* porte au revers, sans légende, l'effigie de la déesse assise à l'avant d'une trirème. La Dame n'a ni couronne murale ni, ce semble, les attributs coutumiers ; mais à droite du trône est un lion. Debout sur la rive, Claudia tire le vaisseau. Près de Claudia, trois femmes symbolisent le groupe des matrones ; elles tiennent des torches* ; l'une d'elles tend les mains vers la divinité. Sur l'autel capitolin* la Mère et Claudia sont seules. Celle-ci tire au moyen d'une chaîne le navire, le long duquel pend la rame devenue inutile. La déesse est assise sur le pont ; et, derrière elle, à la poupe, se dresse le reposoir qui l'abrita durant le voyage. Est-ce une chapelle au toit arrondi ? N'est-ce pas plutôt, comme sur le vaisseau d'Énée que décrit Virgile,* une réduction symbolique de l'Ida, une petite montagne factice, avec la grotte sacrée ? Cette fois encore, l'Idéenne est sans attributs ni couronne ; elle n'a pas même à ses pieds le lion fidèle. Rien ne distingue la déesse de la mortelle. Toutes deux sont vêtues de la robe talaire et du manteau ; et leur tète se couvre d'un voile. L'image de Claudia est posée sur une base très apparente, comme si l'artiste avait copié quelque statue de la matrone, tenant en main le câble, ou s'était inspiré d'un groupe. Cette conjecture est d'autant plus vraisemblable que, malgré le mauvais travail du relief, le costume et l'attitude du personnage semblent révéler l'influence d'un modèle antérieur et de bon style.

*) Cohen, 2e éd., 2. p. 439-440. n° 307 (avec figure). Cf. Visconti dans \emph{Annali}, 1867. p. 296 ss et pl. S : Froehner, \emph{Médaillons rom.}, p. 78 : Stevenson, \emph{Diction. of roman coins}, p. 211.

*) Serait-ce une illustration du « nox aderat » d'Ovide ?

*) Voir la bibliogr., \emph{CIL.} 6, 492 et dans Helbig, \emph{Guide}, 1, n° 436. Le relief est reproduit dans Bottari, \emph{Museo Capit.}, 4, pl. à la p. 67 et dans Lorenzo Re et Nibby. Ces dessins ont été souvent répétés ; cf. Dar. et Saglio, \emph{Dict. d. Ant.}, 1, p. 1684, fig. 2243. Mais ils ne sont pas exacts. Ils interprètent particulièrement mal la ceinture de Claudia, dont on a fait une sorte de grosse et lourde chaîne. Sur l'original ces anneaux n'existent pas.

*) \emph{Aen.} 10, 158 : « imminet Ida super. »

« A la Mère des Dieux et à \emph{Navisalvia} ou \emph{Navis Salvia}, » lisons-nous sur la dédicace de l'autel. Habitués à voir l'Idéenne et Claudia réunies ainsi par l'iconographie religieuse, les gens de Rome en arrivèrent-ils parfois à ne plus les séparer dans leurs prières ? Leur pieuse reconnaissance envers la matrone se changea-t-elle un jour en un véritable culte ? L'hypothèse n'est guère admissible.* D'abord elle nous oblige à supposer un terme nouveau : le mot Navisalvia ne se retrouve pas ailleurs. Mais surtout elle constitue un cas unique, anormal : il est sans exemple qu'un personnage de la République ait été divinisé, même longtemps après sa mort. Ce n'est donc ni par erreur ni par ignorance que le lapicide a séparé les deux mots \emph{Navis} et \emph{Salvia}. Les hommages s'adressent au navire sacré. D'ailleurs ce vocable de Salvia ne nous est pas inconnu. C'est le nom d'une ville. Ce qui nous intéresse davantage, c'est celui de plusieurs trirèmes de la flotte prétorienne de Misène.* Il fut donné à des liburnes ou bateaux marchands. Le vaisseau de l'Idéenne est certainement le vénérable ancêtre dont les autres sont fiers de porter le nom. Il existait beaucoup de navires qui s'appelaient Salus, Ops, Fortuna, Isis.* Salvia est de même une dénomination de bon augure, qui les met sous la protection immédiate de la Mère des Dieux. Mais comment expliquer l'origine première de cette expression ? Voici une ingénieuse hypothèse, dont le récit d'Ovide a encore fourni le point de départ.* Les pins sacrés de l'Ida servirent, dit le poète, à construire l'embarcation sur laquelle devait monter la déesse phrygienne. En d'autres termes, ce vaisseau fut, comme la pierre noire, un présent d'Attale. Puisqu'il devait amener avec lui la délivrance et la victoire, il reçut du roi le nom de Salut, ΣΩΤΗΡΙΑ.* Du mot grec, dont on conserva la désinence, les Romains firent Salvia.* L'explication étymologique est trop subtile pour entraîner la conviction ; mais on ne peut refuser au fait initial un certain caractère d'authenticité. Il est fort possible, en effet, que la Dame n'ait pas accompli son voyage sur l'une des quinquérèmes de Rome 6. De même qu'elle se faisait accompagner par des prêtres phrygiens, elle devait rester sur le sol d'Orient jusqu'à son arrivée dans le Latium. Ce n'était pas dans les eaux du Levant, c'était sur les bords du Tibre qu'elle devait recevoir l'hospitalité de la République. Laevinus n'avait point qualité pour la lui offrir. Après le succès de son ambassade, il n'eut plus d'autre mission que de fournir à la Mère Idéenne l'escorte de ses galères. Nous ne le voyons pas figurer aux cérémonies d'Ostie, et la pierre fut remise directement à Scipio par le clergé oriental. Mais, en même temps que la déesse, le navire étranger devenait l'hôte de Rome, \emph{hospita navis}. Ovide conte qu'après le miracle on couronna de guirlandes fleuries la poupe et que devant elle on immola une génisse vierge et sans tache. Or c'est quand on va débarquer, ou quand le navire est au repos, que la poupe est tournée vers le rivage. Le poète décrit-il un sacrifice qui précéda l'instant où la déesse descendit à terre ? Ou bien fait-il allusion à un rite annuel qui complétait les fêtes commémoratives de la réception ? Et doit-on supposer, en ce cas, que la piété des Romains conserva longtemps le vaisseau dans le port d'Ostie ou dans celui de Rome, comme une relique sainte* ? Par la provenance idéenne de ses matériaux, par son rôle de sanctuaire flottant, par le miracle de la Dame noire, il était en effet trois fois sacré. Avec les siècles, on aurait oublié le caractère simplement commémoratif de la Fête du Navire. Les hommages ne s'adressaient tout d'abord qu'à la Mère des Dieux sur son vaisseau Salvia* ; celui-ci finit par recevoir une part des hommages. Parce qu'il avait été comme le génie tutélaire de la Dame sur les flots, il devint le bon génie de ceux qui traversent les mers. On lui attribua un \emph{numen}, une force divine de salut. Mais il n'est pas nécessaire qu'il ait survécu à sa fonction. Il importe plutôt qu'il ait disparu pour que l'on ait pu croire à sa divinité. Quand Virgile transforme en déesses les navires d'Énée,* ils subissent une véritable métamorphose et ne gardent plus rien de leur primitif aspect.

*) Elle est soutenue en particulier par Borghesi, \emph{l. c.}, p. 183, Mommsen dans \emph{CIL.} 6, 492, Decharme dans Dar. et Saglio, s. v. \emph{Cybelé}, 1, p. 1684.

*) \emph{CIL.} 6, 3094 ; 10, 3532, 3580, 3600. Cf. Aschbach, \emph{Die latin. Inschr. mit den Namen roem. Schiffe} dans \emph{Sitzungsber. Wien. Akad.}, 79, 1875, p. 200. Ajouter une \emph{l(iburna) Sal(via) Augusta}, sur une inscr. de Sardaigne : Haverfield dans \emph{Classical Rev.}, 1889, p. 228.

*) Cf. Ferrero, \emph{L'ordinamento delle armate rom.}, p. 29.

*) Bloch dans \emph{Philologus}, 52, 1894, p. 581. Jordan, dans Preller, \emph{Roem. Myth.}, 3e éd., p. 58, avait émis déjà l'hypothèse \emph{navisalvia}, = \emph{navis salvans}, nom d'une trirème.

*) Attale portait lui-même le titre de Sôter.

*) D'après l'hypothèse de Bloch, l'affranchie Claudia Syntyche, qui sait mal le latin, considère Salvia comme la traduction du substantif grec Sôteira. Par suite, la répétition de \emph{Salviae} ne serait pas une erreur du lapicide ; Claudia voulait dire : à la Grande Mère et au navire Salvia, après le vœu formé pour le salut, \emph{etc.} Cette explication est vraiment bien compliquée.

*) Cependant, Sil. Ital. 17, 8 : « \emph{Latia} portante Cybebe puppe. »

*) Cancellieri, \emph{Le sette cose}, p. 28 ; Visconti dans \emph{Annali}, 1867, p. 301. Procope, \emph{B. G.} 4, 22, déclare que l'on conserva longtemps à Rome le navire d'Énée ; confondrait-il avec celui de l'Idéenne ?

*) Comme protectrice des marins ; cf. Conze, Hauser et Neumann, \emph{Arch. Untersuch. auf Samothrake}, 2, p. 108.

*) Construits également en pins de l'Ida et placés sous la tutelle de la Mère des Dieux, \emph{Aen.} 10, 156.

Le culte de Mater Deum et Navis Salvia est du reste tout à fait localisé à Rome, dans le quartier du port ; et c'est un culte de confrérie. La chapelle s'élevait en plein emporium, aux bords du Tibre et au pied de l'Aventin. Elle était surtout fréquentée par la population de bateliers, de marchands, de portefaix, de petits employés du service maritime ou du service de l'annone, qui grouillaient sur les quais et autour des immenses magasins. Un certain nombre de fidèles s'étaient groupés en un collège de \emph{Cultores}, qui devait avoir en même temps une destination funéraire. La confrérie ne nous est signalée qu'au temps des Sévères.* Elle est sans doute plus ancienne. Sa fondation paraît être antérieure à l'époque tardive où l'on adora le navire. Car les confrères du 3e siècle, qui invoquent la Mère et son vaisseau, s'intitulent encore \emph{Cultores ejus}, comme s'ils ne s'adressaient qu'à une seule puissance divine. Mais ce singulier n'est-il pas mis à dessein pour bien marquer le caractère indissoluble des deux divinités, aussi inséparables que leurs deux images au fond du sanctuaire et sur les naïfs ex-voto suspendus aux murailles ?

*) A en juger par le caractère de l'épigraphie : Huebner, \emph{Exempla script. epigr.}, 1885, aux nos 519 et 520.

C'est ainsi qu'à une époque où le temple du Palatin était devenu le centre d'une religion orientale, où les fêtes nationales d'avril, anniversaires de l'arrivée de la Dame et de la dédicace de son temple, étaient éclipsées par les fêtes universelles de mars, commémorant la Passion et la Résurrection d'Attis, ce modeste sanctuaire de Mater Deum et Navis Salvia, dans le quartier le plus plébéien de Rome et le plus cosmopolite, perpétuait le souvenir lointain de cette réception du 4 avril 204, qui s'était manifestée avec un caractère si particulièrement aristocratique et si purement romain.
\clearpage
\section{Chapitre 2}
\begin{center}
Le Culte à Rome sous la République.
\end{center}
\paragraph{}
\emph{Veteres Romani in constituendis religionibus castissimi cautissimique.} Aul. Gell. 2, 28, 2.

1. Adaptation du culte de la Grande Mère Idéenne à la religion romaine. Impossibilité religieuse d'éliminer entièrement le culte phrygien. Restrictions apportées à l'exercice de ce culte. Le clergé phrygien à Rome. --- 2. Substitution du rite gréco-romain au rite phrygien dans la religion officielle. Sacrifices : génisse, moretum, prémices des moissons ; rôle du prêteur urbain dans les sacrifices accomplis au nom de l'État. Jeux : représentations scéniques au Palatin, jeux du Cirque ; présidence des édiles curules. Sodalités de nobles et repas sacrés, dits mutitations. Caractères du culte officiel sous la République. --- 3. Popularité de la déesse. Les Romains en Phrygie et à Pessinonte au 2e siècle. Un grand-prêtre de Pessinonte à Rome en 651/103. Un affranchi qui se fait Galle. Galles et Bellonaires. Anarchie religieuse et extension des religions orientales au temps des guerres civiles. --- 4. Cybèle et Attis dans la littérature latine à la fin de la République. Un poète dilettante : Catulle. Un théologien : Varron. Un philosophe: Lucrèce.

\subsection{1.}

« Il faut éviter, dit le jurisconsulte Paul, les cultes qui troublent les âmes des hommes.* » Le culte orgiastique* de la Grande Mère Phrygienne, avec son mysticisme inquiétant, les séductions énervantes de ses rites et le fanatisme délirant de ses prêtres, était bien de ceux-là. Le Sénat romain ne pouvait l'accepter qu'en le transformant. Mais ce serait une erreur de croire que, pour romaniser la déesse étrangère, on se hâta de l'assimiler aux divinités telluriques de la religion indigète. De telles assimilations ont une origine populaire et sont l'œuvre naïve et lente, de la masse obscure des fidèles ; ou bien elles ont une origine savante et sont l'œuvre tardive des mythologues. Si Maia est invoquée dans certaines oraisons liturgiques sous le nom de Magna Mater,* on ne pensa point à la confondre avec Cybèle, comme on avait identifié Cérès avec Demeter. La personnalité plus fortement accusée de la déesse phrygienne n'absorba point celle de la déesse latine. D'autre part, l'étrangère ne perdit pas son nom, comme Demeter, le jour où Rome adopta son culte ; elle resta l'Idéenne en devenant la Palatine. Si poètes et savants identifient volontiers Cybèle, Ops et Tellus,* chacune de ces déesses continue, durant tout le paganisme, à recevoir un culte particulier ; on n'immole pas à la Mère des Dieux la truie consacrée à Maia, à Ops, à Cérès. Le syncrétisme n'a donc rien à voir avec les préoccupations du Sénat pour soumettre le culte métroaque aux exigences de la loi et l'approprier à l'organisme de la cité romaine. Le souci de maintenir dans son intégrité la tradition nationale se manifeste : 1° par l'élimination des éléments orgiastiques du culte ; 2° en conséquence, par la substitution du rite gréco-romain au rite oriental dans la religion officielle ; 3° par la création de sodalités recrutées uniquement dans la classe dirigeante, qui reste ainsi maîtresse du culte nouveau.

*) \emph{Sent.} 5, 21, 2.

*) Le culte importé à Rome est bien un culte orgiastique ; Dion. Hal. 1, 61, parle des orgies et mystères de la Mère Idéenne en pays troyen.

*) Macrob. 1, 12, 20 : « (Maia) et mater magna in sacris vocatur. » Henzen dans \emph{Annali}, 1856, pp. 110-112, et Marquardt, \emph{Culte}, 2, p. 68, en concluent à tort qu'il y eut assimilation ; \emph{mater magna} n'est ici qu'une épithète rituelle.

*) Varr. dans Augustin., \emph{Civ. Dei}, 27, 24 ; Tibull. 1, 4, 68 : « Idaeæ Opis » ; Ovid., \emph{Trist.} 2, 24 : « Turrigera Ops. » Sur un \emph{ara augusta} de l'an 754/1, Ops est rapprochée de Mater Magna : \emph{Not. Scavi}, 1890, p. 388. Les mythographes romains, confondant Mater Deum avec Rhéa, l'unissaient à Saturne, cf. Arnob. 3, 32 : « Mater Deum quam Nigidius (à la fin de la République) autumat matrimonium tenuisse Saturni » ; or Saturne est également uni à Ops : Wissowa, \emph{Rel. u. Kultus d. R.}, p. 168. D'autre part, sur l'association tardive des cultes de Mater Deum et de Bona Dea, cf. une inscr. de Rome, \emph{CIG.} 6206 = \emph{IGSI.} 1449 : Aurelius Antonius ἱερεὺς πρῶτον Βοναδίης εἶτα Μητρὸς θεῶν. Parallélisme du mythe de Faunus, père de Bona Dea, qui « transfigurasse se in serpentera creditur et coisse cum filia » (Macr. 1, 12, 24), et de la tradition orphique de Zeus s'unissant sous la forme d'un serpent à Rhéa et à Proserpine.

C'est la légende d'Attis, le berger aimé de Cybèle, qui donne au culte de la déesse son caractère orgiastique. Le culte d'Attis faut-il nécessairement introduit à Rome avec celui de la Grande Mère ? D'une part, le témoignage de Denys d'Halicarnasse* est formel. Denys est surpris de ne retrouver dans la Rome d'Auguste, « malgré la dégénérescence des mœurs, » aucune de ces fêtes de deuil qui symbolisent la Passion et la Disparition (Aphanie) de certaines divinités : d Perséphone, Dionysos et tant d'autres. » Les autres ne peuvent être qu'Attis, Adonis, Osiris, les dieux qui meurent et qui ressuscitent. L'allusion au culte d'Attis n'est pas douteuse. Car l'historien admire précisément Rome d'avoir adopté l'Idéenne sans se laisser envahir par toutes les superstitions phrygiennes, qu'il traite avec mépris de charlataneries mythiques. Cette attestation d'un Grec d'Asie est d'une indéniable valeur. Elle nous explique en même temps pourquoi Varron, qui dans ses Antiquités divines analyse longuement les attributs de la Mère des Dieux, n'y fait point mention d'Attis. Son intention, toute patriotique, est de mieux faire connaître aux Romains leur religion nationale. Or Attis n'est pas une divinité d'État. Attis n'a pas reçu la consécration du Sénat. Les citoyens ne sont même pas légalement libres de l'adorer dans l'intimité de leurs sanctuaires domestiques.

*) Dion. Hal. 2, 19.

Mais d'autre part Attis ne reste pas inconnu des Romains.* Un denier de la République reproduit son image. Catulle lui consacre un poème. Enfin et surtout, un clergé phrygien, sous la sauvegarde de l'État, célèbre à Rome même le culte phrygien.

*) Bibliographie récente : Rapp dans Roscher, \emph{Myth. Lexik.}, 1, 1, p. 724, s. v. \emph{Attis} ; F. Cumont dans Ruggiero, \emph{Dizion. epigr.}, 1, et dans Pauly-Wissowa, \emph{Real-Encycl.}, 2, s. v. \emph{Attis} ; Showerman, \emph{Was Attis at Rome under the Republic?} dans \emph{Transactions of the American Philol. Assoc.}, 1900, pp. 46-59 ; Id., \emph{The great mother}, 1901, p. 263 ; Hepding, \emph{Attis}, 1903, p. 142 ss.

Le denier* est de P. Cornelius Cethegus, qui fut monétaire vers l'an 650/104, c'est-à-dire un siècle exactement après l'introduction de Mater Magna. Il présente, au revers, un personnage encore enfant, coiffé du bonnet phrygien, qui porte une branche sur l'épaule et chevauche un bouc. A défaut d'autre explication possible, il faut bien admettre que ce groupe figure l'enfance d'Attis,* lequel fut nourri du lait d'une chèvre* et folâtrait parmi les troupeaux. De même, certaines monnaies de la gens Cestia, de la gens Norbana, de la gens Volteia,* montrent Cybèle trônant sur un bige de lions, parfois aussi une tête jeune de Corybante.* Toutes ces effigies ont la même signification commémorative. Elles rappellent une des principales fêtes agonistiques de l'année romaine, les Jeux de la Grande Mère donnés par les monétaires en leur qualité d'édiles curules. Mais si le motif de l'enfant au bouc prouve que Rome n'ignorait pas alors la légende phrygienne, il ne révèle pas l'existence d'un culte officiel d'Attis.* La singularité même du sujet semble témoigner, au contraire, que le divin pasteur n'est pas encore un dieu de Rome. On devine une fantaisie de l'art hellénistique, on ne reconnaît pas un type iconographique de l'art religieux.

*) Exemplaire unique au Cabinet de France. Ce type est à rapprocher de l'image d'Attis tenant le pedum et la syrinx, assis sur un bélier bondissant vers la droite ; bas-relief en marbre à la Bibliothèque du Vatican, « Museo Profano » : Buonarotti, \emph{Osserv. istor. sopra alcuni medaglioni antichi}, Roma, 1694, 1, p. 375 ; Pistolesi, \emph{Vaticano descritto e illustrato}, Roma, 1829, 3, pl. 106, 3 ; pierre gravée signalée par Passeri, \emph{Lucernae fict.}, 1739, 1, p. 25 : « Atys insidet arieti, itemque in marmoreo simulacre, quod vidi dudum Romae in hortis Pamphiliis in Janiculo. »

*) Hypothèse soutenue par Ch. Lenormant, dans \emph{Rev. numism.}, 1842, p. 245 ; Cavedoni, dans \emph{Bull. Inst. archéol.} 1844, p. 23 ; Mommsen, \emph{Hist. de la monnaie rom.}, 2, p. 371 ; Babelon, \emph{Monnaies de la Rép. rom.}, 1, p. 395. Lenormant rapproche le nom de Cethegus du mot phrygien Attêgus qui signifie bouc ; un jeu de mots aurait déterminé le choix de ce type monétaire. Cavedoni croit que l'image d'Attis rappelle la translation du culte métroaque, parce qu'Attis passait pour avoir propagé en Lydie la religion de Cybèle. --- Ne pas oublier qu'un Cornelius Cethegus était consul l'année même où Rome adopta le culte de Magna Mater.

*) Pausan. 7, 17, 9 ; Arnob. 5, 6. Sur le rôle important de la chèvre sauvage et du bouc dans les cultes primitifs de Crète, v. Evans, \emph{Mycen. tree}, p. 83 ; Cook, \emph{Animal worship in the mycen. age}, dans \emph{J. of Hell. St.}, 14, 1894, pp. 133-138 et 150-152. Le bouc paraît être l'animal consacré au Zeus de la grotte de Dicté. La légende du bouc d'Andira, au sanctuaire de Cybèle Andirène, rappelle sans doute un rite cultuel. Boucs et chèvres comme animaux sacrés dans le culte phrygien, hieron de Dionysopolis : Ramsay, \emph{Cities}, 1, p. 138 s.

*) Babelon, \emph{Monnaies de la Rép. rom.}, 1, pp. 339, 340 (L. Cestius) ; 2, p. 261 (C. Norbanus Flaccus), p. 566 (M. Volteius) ; cf. Mommsen, \emph{Hist. de la monnaie rom.}, 2, p. 468, note. Ne serait-ce pas aussi l'image de Cybèle que représente la tête tourelée de femme sur les monnaies de Q. Caecilius Metellus (Babelon, 1, p. 280), C. Fabius Buteo (p. 486, 487), P. Furius Crassipes (p. 526), P. Licinius Crassus Junianus (2, p. 135), M. Plaetorius Cestianus (p. 310), A. Plautius (p. 324) ?

*) Au droit de la monnaie de M. Volteius : tête jeune de Corybas, avec casque sans cimier, orné d'une couronne de laurier. Mommsen pense que la tête juvénile est celle d'Attis ; mais Babelon, après Cavedoni (\emph{Nuovi studj}, pp. 27 et 28), fait observer qu'Attis porterait le bonnet phrygien.

*) « Si Attis n'avait pas reçu un culte à Rome, jamais on ne l'aurait figuré sur une monnaie frappée dans la cité, » dit Cumont, dans Ruggiero, \emph{l. c.}, p. 764. Nous verrons plus loin que la présence des prêtres phrygiens à Rome suppose en effet l'existence d'un culte d'Attis ; mais ce culte, toléré dans des conditions très déterminées, n'est pas un culte reconnu par la loi.

Moins de cinquante ans après, Catulle écrivait son poème galliambique d'Attis.* En une centaine de vers dont le rythme se brise et s'effémine, il exprimait les transports d'amour mystique de celui qui consacre à Cybèle sa virilité. Cet Attis n'est pas le dieu même de Pessinonte ; c'est l'un de ses nombreux prêtres théophores, un Galle. Il n'est pas d'origine phrygienne ; c'est un jeune Grec, en proie à la divinité impérieuse et barbare qui habite sur les cimes de l'Ida. L'œuvre aussi est d'origine hellénique. Elle est une importation de la littérature alexandrine. Elle ne s'inspire pas directement d'une tradition religieuse. Elle reproduit un pur motif d'art : telles les images d'Attis dont on décorait les supports des tables de marbre et les anses des vases de bronze. Pour l'histoire du culte d'Attis à Rome, elle n'offre donc, à proprement parler, aucune valeur documentaire.* Tout au plus peut-on dire quelle témoigne de l'attrait qu'exerçait sur l'imagination romaine, au dernier siècle de la République, le culte oriental de la Mère des Dieux.

*) Catull. 63. Pour de plus amples détails, v. la fin du chapitre.

*) Cette idée a été très heureusement mise en valeur par Wilamowitz-Mollendorf, \emph{Die Galliamben des Kallimachos und Catullus}, dans \emph{Hermes}, 14, 1879, p. 194 ss, en particulier p. 197 : « Seine Attis ist kein Document fur den religiösen Sinn ihrer Zeit, \emph{etc.} »

La Dame était venue d'Orient avec son personnel asiatique* ; et dès l'origine un prêtre et une prêtresse de Phrygie furent attachés au temple du Palatin. Leur présence n'est point, malgré les apparences, un fait anormal. Elle est conforme aux habitudes religieuses de la cité. C'est, du reste, une tradition constante dans l'antiquité que les dieux étrangers arrivent avec leur propre clergé.* Pour le formalisme romain surtout, le culte de chaque divinité est une véritable science ; il exige un personnel compétent, qui ait une connaissance impeccable du rituel liturgique. Il avait fallu des spécialistes pour recevoir le serpent d'Esculape. Quand la déesse Mâ de Comana deviendra divinité romaine, sous le vocable de Bellone, son culte sera confié à un clergé cappadocien. Les \emph{sacra} de Gérés, pris à la Grande Grèce, sont desservis à Rome par des prêtresses grecques, de même que le sanctuaire de Demeter et Korè, dans la Carthage punique, est desservi par des Grecs. « Mais, dit Cicéron,* si l'on faisait venir des pays helléniques la prêtresse chargée d'enseigner et d'accomplir les cérémonies de ce culte hellénique, du moins voulut-on, pour les accomplir au nom de la cité, une citoyenne ; pour prier les dieux immortels, il lui fallait, avec sa science étrangère, l'âme nationale et le civisme d'une Romaine. » Ce passage de Cicéron est d'une importance capitale ; car il nous fait comprendre dans quel esprit et sur quelles bases sont organisés les cultes exotiques, lorsqu'ils deviennent cultes d'État. Il s'agit de concilier deux éléments : \emph{scientia peregrina et externa, mens domestica et civilis}. Pour Cérès la combinaison est facile. On choisit les prêtresses dans les villes fédérées de la Grande Grèce, soit à Naples, soit à Velia. Quand ces villes n'avaient pas encore le droit de cité romaine, le préteur urbain proposait au peuple d'accorder personnellement à la nouvelle prêtresse, lors de son entrée en fonctions, le titre de \emph{civis romana}. Pour la Mère Idéenne, les éléments sont incompatibles. Plus particulièrement encore dans ce culte oriental, les observances rituelles supposent l'existence d'un clergé spécial. Elles comportent en effet la célébration de véritables offices et une liturgie sans doute quotidienne. Les prêtres phrygiens sont seuls à posséder cette science divine. Ils sont seuls aussi à pouvoir la pratiquer. Car, à l'exemple d'Attis, ils doivent être eunuques.* La participation au sacerdoce, par le seul fait qu'elle exige la suppression avilissante de la virilité, est interdite à tout Romain. Se rendre eunuque, c'est commettre un attentat contre la patrie ; déjà peut-être la loi prévoit et punit ce crime.* Un castrat ne peut devenir citoyen romain. Il est du reste impossible d'élever à la dignité de citoyen un barbare, qui appartient à une race méprisable d'esclaves. Il faut donc subir la nécessité d'un double culte.

*) Cic., \emph{De har. resp.} 13, 26 ; Liv.  29, 14 : Diod. 34, 33 ; Sueton. \emph{Tib.} 2, parlent tous de sacra (hiera), ce qui signifie autre chose que l'idole. Tite Live est encore plus formel et nomme les prêtres.

*) Cf. un Ptolémée qui, voulant établir à Alexandrie le culte Eleusinien, fait venir comme prêtre un Athénien de la famille des Eumolpides : Tac., \emph{Hist.} 4, 83.

*) Cic., \emph{Pro Balbo} 24, 55.

*) Il semble bien que le \emph{famulus} de la Dame noire, à Rome, soit un Galle, c'est-à-dire un eunuque : cf. Varr., \emph{Eumenides}, 34, éd. Riese, p. 132. La question est cependant discutable.

*) Lex Moenia : cf. \emph{ibid.}, p. 153. Mais, à vrai dire, nous ne connaissons aucun édit contre l'eunuchisme qui soit antérieur à Domitien ; cf. \emph{Digest.} 48, 8, 4, 2 (rescrit d'Hadrien) : « Nemo liberum servumve invitum sinentemve castrare debet, neve quis se sponte castrandum praebere debet. » Hitzig, dans Pauly-Wissowa, \emph{Real-Encycl.}, s. v. \emph{Castratio}.

Mais il importait de réduire au minimum le corps sacerdotal et son rôle dans la vie de la cité. Une série de dispositions administratives, élaborées sans doute par les Décemvirs et sanctionnées par le Sénat,* fixa la constitution de ce clergé, détermina ses rapports avec le peuple romain et réglementa les manifestations extérieures du culte phrygien. Voici celles que nous connaissons. 1° En principe, la dualité du sacerdoce oriental est respectée, mais un seul prêtre et une seule prêtresse d'origine phrygienne sont tolérés par l'État.* 2° Un citoyen romain ne peut pas être prêtre ou galle, une citoyenne ne peut pas être prêtresse de la Magna Mater. 3° La même interdiction s'applique aux esclaves. 4° Le couple phrygien ne peut accomplir ses rites et sacrifices dans un lieu public, en dehors de l'enceinte du temple auquel il est attaché. 5° Une procession annuelle est autorisée pour le bain de la Dame Noire ; le cortège se rend du Palatin au ruisseau de l'Almo par la porte Capena. Il est vraisemblable que cette cérémonie, qui nous est signalée pour la première fois sous Auguste, remonte aux origines. Mais elle ne coïncidait pas avec les fêtes d'avril ; elle avait déjà lieu à la fin de mars.* 6° Une quête est permise pour l'entretien du culte et du clergé.* Elle était à la fois conforme aux habitudes des métragyrtes et aux traditions de la religion romaine. La \emph{stips} ou cotisation volontaire faisait vivre déjà plusieurs cultes de Rome* ; et l'on y avait souvent recours lorsque les dieux réclamaient des cérémonies extraordinaires.* Mais cette quête est rigoureusement limitée par la loi romaine à certains jours de l'année (\emph{justi dies stipis}), qui sont peu nombreux ; car, dit Cicéron, « elle remplit les âmes de superstition et épuise les maisons. » Aux jours permis, une fois par mois peut-être, les prêtres de l'Idéenne ont le droit de circuler dans les rues de Rome avec leur costume liturgique, leurs images religieuses et leur musique orientale, de chanter leurs cantiques en public, de tendre la main aux passants et de mendier aux portes des maisons. 7° Enfin défense est faite à tout Romain de sacrifier à la déesse selon le rite de Pessinonte. Car « rien n'est plus propre à détruire la religion que l'introduction des pratiques étrangères et contraires au culte national.* » L'affaire des Bacchanales, qui, dix-huit ans seulement après l'arrivée de la pierre noire, révélait de si effrayante façon le danger des cultes orgiastiques, nous fait mieux comprendre la nécessité de toute cette réglementation. Elle contribua sans doute à rendre plus sévère encore la surveillance exercée sur le personnel du temple et sur les Galles ambulants que la Dame pouvait attirer dans Rome.

*) Dion. Hal. 2, 19.

*) Ces deux desservants avaient-ils avec eux et sous leur propre dépendance un personnel inférieur ? Il ressort du texte de Denys d'Hal. qu'il existait au moins un flûtiste, la double flûte phrygienne étant un instrument rituel, mais que les prêtres faisaient eux-mêmes fonction de tympanistes.

*) Ovide laisserait croire qu'elle avait lieu dans la matinée du 4 avril. Mais un Menologium rusticum, \emph{CIL.} 6, 2305, cf. 2306, antérieur à l'adoption dès l'êtes phrygiennes, la place au mois de mars ; elle y occupe la dernière place.

*) Cic., \emph{De leg.} 2, 9, 22 : « Praeter Idaeae Matris famulos eosque justis diebus ne quis stipem cogito » ; 16, 40 : « Stipem sustulimus nisi eam quam ad paucos dies propriam Idaeae Matris excepimus ; implet enim superstitione animos et exhaurit domus. » Dion. Hal. 2, 19 ; Ovid., \emph{Fast.} 4, 350.

*) Sur la \emph{stips} exigée du peuple pour le culte d'Apollon, Liv. 25, 12, 14 ; Fest., \emph{Ep.}, p. 23 ; Apul., \emph{De magia} 42 ; d'une façon générale, pour les cultes étrangers, Marquardt, \emph{Culte}, 1, p. 171 ; Wissowa, \emph{Rel. u. Kult. d. R.}, p. 363.

*) Par exemple, Macrob. 1, 6, 13 : « Lectisterniumque ex collata stipe faciendum. »

*) Liv. 39, 16, à propos des Bacchanales.

Quand les Romains passaient devant l'enclos sacré du Palatin, ils entendaient parfois un chœur de voix criardes et le bruit assourdissant des tambourins.* C'était le clergé phrygien qui célébrait de mystérieux offices. On ne voit pas qu'un citoyen de la République n'ait jamais eu la tentation de se faire initier. D'autre part le culte n'avait pas encore à Rome, du moins pendant les cent premières années, cette clientèle considérable d'esclaves et d'affranchis orientaux,* dont le zèle fanatique propagera la dévotion à Cybèle et Attis. Le dieu Attis, dont les images resplendissaient sur la poitrine des prêtres, resta longtemps sans faire parler de lui. Il n'eut pas l'avantage d'être persécuté, comme fut le Bacchus des orgiastes. Les deux divinités n'étaient pourtant pas sans analogie. Leur assimilation remontait à l'époque même où les Phrygiens, venus de Thrace, s'étaient introduits au milieu des populations anatoliennes. Mais Dionysos-Bacchus était un dieu que la Grèce avait paré de séductions. Attis a conservé son costume infamant de barbare et sa tare d'eunuque. La répugnance des vrais Romains à concevoir un tel dieu dut faciliter, dans les premiers temps, la tâche des autorités. Le clergé mena la même existence obscure que son culte. Il n'a pas d'histoire. Aussi bien ne reconnaissait-on même pas à ces Phrygiens le titre de prêtres. Ils sont les serviteurs de la Dame, ses \emph{famuli}.* Jusqu'au temps indéterminé où le recrutement du personnel se fit parmi les affranchis, nous ignorons quelle fut leur condition sociale. Les premiers durent être des étrangers hospitalisés.* Pour que le service de la Dame ne restât pas aux mains d'étrangers, prit-on ensuite des esclaves publics de race phrygienne ? Un personnage de Plaute,* environ quinze ans après l'événement religieux de 204, résumait ainsi l'opinion romaine à leur égard : « ils ne valent pas une coquille de noix, tous ces frotteurs de tambourins. »

*) Varr., \emph{Eumenides}, 33 et 34, éd. Riese.

*) Ce sont les guerres contre Antiochus, et plus tard contre Mithridate, qui provoqueront le transport d'une foule d'esclaves orientaux en Italie ; cf. les esclaves introduisant en Sicile le culte de Dea Syria.

*) Cic., \emph{De leg.} 2, 9, 22 : « Idaeae matris famulos » ; cf. Varr., \emph{l. c.}

*) Cf. supra, p. 59.

*) Plaut., \emph{Trucul.} 2, 7, 48. Cette pièce a été composée vers 565/189. C'était aussi l'opinion des Grecs à l'égard des métragyrtes ; et Plaute ne fait peut-être que traduire ici l'un de ses modèles attiques.

\subsection{2.}

Comment l'État romain accomplit-il ses devoirs religieux envers la Mater Magna ? Le Sénat est maître en matière de culte, et il exerce ses droits par l'intermédiaire des magistrats. C'est lui qui a naturalisé la déesse ; c'est lui qui vote la construction du temple, donne aux censeurs de 550/204, M. Livius et C. Claudius, l'ordre d'en adjuger l'entreprise et délègue le préteur M. Iunius Brutus, chargé de la juridiction urbaine et pérégrines-en 563/191, pour dédier le monument.* Il décrète aussi la célébration des fêtes nouvelles et en détermine, suivant certains principes, la date et le caractère. La journée du 4 avril, à partir de l'an 551/203, fut consacrée aux fêtes anniversaires de la réception de la Dame noire. Celles-ci reçurent le nom de \emph{Megalesia} ou \emph{Megalensia}, « parce que la Déesse est appelée (en grec) Megalè* » ; le vocable de \emph{Matralia}, qui aurait pu leur convenir, était déjà réservé au 11 juin, fête de Mater Matuta. Treize ans plus tard, l'\emph{aedes Matris Magnae in Palatio} était livrée au culte. La déesse eut désormais une seconde fête, le 10 avril, jour anniversaire de la dédicace de son temple* ; ce fut son \emph{dies natalis}. Pour d'autres sanctuaires, celui de l'\emph{Ara Pacis Augustae}, par exemple, nous retrouvons deux jours fériés ; et l'un d'eux, purement commémoratif, perpétue de même le souvenir heureux de l'événement qui donna lieu à l'érection de cet autel.

*) Liv. 36, 36.

*) \emph{Fast. Praen.} (entre les années 752 et 763), ad 4 Apr., \emph{CIL.} 1, 2, p. 316 ; Varr., \emph{De ling. lat.} 6, 15, cf. supra, p. 47, n. 2 ; Cic., \emph{De har. resp.} 12, 24.

*) \emph{Fast. Praen.}, ad 10 Apr. : « M(atri) D(eum) M(agnae) I(daeae) in Pal[atio] quod eo die aedis ei dedicata est. »

Les fêtes nationales de la Grande Mère comprenaient trois éléments : des sacrifices au temple du Palatin, des jeux et des banquets.

Les sacrifices étaient accomplis au nom de la République et pour son salut par le préteur urbain.* En général, les magistrats n'offrent que les sacrifices exceptionnels, prescrits par le Sénat pour une cause tout à fait spéciale ; ce sont les prêtres compétents qui sont chargés de ce rite, lorsqu'il est au nombre des cérémonies annuelles. Mais, comme les prêtres phrygiens de la Grande Mère ne peuvent sacrifier \emph{more romano} et que la Déesse n'a point à son service de prêtre romain, cet office rentre dans les attributions des magistrats. Ceux qui jouissent de l'\emph{imperium}, c'est-à-dire les consuls et les préteurs, ont seuls le droit de faire des vœux au nom de l'État. Or les consuls, surtout en ces temps de guerres, étaient souvent hors de Rome ; la présence continue du préteur urbain dans la Ville le désignait tout naturellement pour la célébration des fêtes à jours fixes. Nous avons d'autres témoignages de sa participation aux actes religieux de la Cité, mais seulement à des actes d'observance grecque* : il organise les Jeux Apollinaires et sacrifie tous les ans, \emph{graeco ritu}, sur l'Ara Maxima. Le sacrifice à la Grande Mère était certainement aussi une cérémonie de rite grec ; l'officiant avait la tête découverte et portait une couronne de laurier. Il faut admettre qu'il y avait deux sacrifices, l'un le 4, l'autre le 10 avril. La victime est une jeune génisse qui ne doit avoir connu ni le joug ni l'accouplement.* A ces grandes solennités du culte officiel, de même qu'aux jeux, le peuple a toujours un rôle passif de spectateur. Mais le temple lui était ouvert aux deux jours de fête,* pour qu'il pût accomplir ses dévotions et présenter ses offrandes. Les particuliers apportaient à la déesse des plats de \emph{moretum} ; c'était leur mets national, composé de fromage blanc et de légumes hachés ou pilés. Le choix de ce sacrifice avait sans doute été spécifié dès l'origine par la \emph{lex templi}. Avait-il pour but de favoriser l'identification de Cybèle avec quelque divinité rustique de l'ancienne Rome ? Il est beaucoup plus probable que les organisateurs du culte avaient simplement remplacé par le \emph{moretum} un plat analogue, consacré à la Mère des Dieux dans la religion orientale ou hellénique. C'est ainsi que dans Athènes, le jour des Galaxia,* on donnait à la Mêtêr une sorte de bouillie d'orge au lait ; et cet aliment même pourrait bien être l'un des mets nationaux des paysans phrygiens.* Le \emph{moretum}, contenu dans des écuelles en bois ou en terre cuite, était déposé sur les \emph{mensae sacrae} ou tables liturgiques, devant la Dame. Celle-ci, durant l'offerte, était placée sur le \emph{lectus}. Comme au 4 avril 204, ce lectisterne précédait immédiatement les Jeux donnés au Palatin en l'honneur de la déesse. Elle restait sur son \emph{pulvinar} pour les contempler. Aussi bien ce rite du lectisterne, devenu tout à fait romain, était-il d'importation grecque. Nous le retrouvons en Orient, dans le culte d'Aphrodite et d'Adonis.* Il pouvait être un élément ancien du culte métroaque. Au Pirée, la \emph{Klinè} du Metrôon* n'avait peut-être pas d'autre destination.

*) Dion. Hal. 2, 19, 4, dit que les sacrifices annuels à la déesse sont célébrés par les στρατηγοί, et dans le rite national.

*) Cf. Wissowa dans Pauly-W., \emph{Real-Encycl.}, 2, 698.

*) \emph{Iuvenca} : Ovid., \emph{Fast.} 4, 346. Le taureau des tauroboles n'interviendra que plus tard et avec un caractère tout à fait spécial. Il est de règle de réserver aux déesses des animaux femelles.

*) Le 4 avril 204, le temple de la Victoire avait été ouvert au peuple pour qu'il pût offrir ses dons à la Dame Noire. Liv.  29, 14 : « Populus frequens dona Deae in Palatium tulit. »

*) Ovid., \emph{l. c.}

*) Hesych. 1, p. 794, s. v. \emph{Galaxia} ; Bekker, \emph{Anead.}, p. 229 ; cf. Dieterich, \emph{Eine Mithrasliturgie}, p. 145, n. 2.

*) Les Phrygiens boivent de la bière d'orge et du lait. C'étaient des « mangeurs de beurre, » comme les Thraces ; cf. Koerte, \emph{Gordion}, 1904, pp. 86 et 237. Sur le rôle du lait dans la nourriture des mystes de Cybèle : Sallust., \emph{De diis et mundo}, 4 (Orelli. p. 16) : cf. Hepding, \emph{Attis}, p. 197.

*) Theocr. 15, 127.

*) Considérée comme un lit funèbre par Foucart, \emph{Assoc. relig. chez les Grecs}, p. 93. Cf. la Klinè sacrée d'Athena, à Tégée, Pausan. 8, 47.

Existait-il une troisième forme de sacrifice ? On offrait, ce semble, à la Grande Mère les prémices de la moisson ou la dernière gerbe, tressée en couronne. Un tel rite s'accordait très bien avec le caractère agraire de Cybèle,* Mère du blé, Dame aux épis.* Il manifeste un symbolisme clair et universel. Les Romains eux-mêmes le pratiquaient à l'égard d'autres divinités : c'est ainsi qu'au jour du Florifertum on portait à Flora les premiers épis. Dans le culte de Magna Mater, c'est un édile curule qui aurait eu mission d'apporter l'offrande au temple. En présence du clergé phrygien, au bruit des instruments rituels scandant les cantiques, il l'aurait déposée sur les genoux de la déesse assise. Mais cette hypothèse séduisante n'a pour base qu'un texte à peu près inintelligible de Varron.*

*) Artemid., \emph{Oneir.} 2, 39, lui donne l'épithète de γεωργοῖς ἀγαθή.

*) Cybèle tenant des épis dans la main droite, sur des monnaies de l'époque impériale, en Phrygie : Mionnet, 4, p. 280, n° 495 ; p. 291, n° 556 : Weber dans \emph{Numism. Chronicle}, 1892, p. 208, n° 38 et pl. 16, 18.

*) Les mss. portent : « Video gallorum frequentiam in templo, qui deum essena hora nam adlatam imponeret aedilis signo siae et deam gallantes, etc. » Conjectures de Lachmann : « ...qui dum messem hornam adlatam imponunt Attidis signo, synodiam gallantes » ; de Riese, \emph{Varronis Sat. Menipp. rel.}, p. 132 : « dum messem hornam adlatam imponeret aedilis signo Cybelae, deam gallantes » ; de Buecheler (cf. Hepding, \emph{Attis}, p. 12): « cum e scaena coronam adlatam imponeret aedilis signo, synodiam gallantes. »

Le second élément des Fériés, ce sont les Jeux. On en célébra, nous l'avons vu, le jour même de la réception ; par eux se clôtura la fête. Ils furent renouvelés tous les ans, au 4 avril. Le 10 avril 563/191, jour de la dédicace du temple, on donna aussi des Jeux,* qui devinrent annuels. Ce fut sans doute dès cette époque, ou peu après, qu'entre les deux anniversaires du 4 et du 10 on intercala des représentations quotidiennes. Les réjouissances se prolongèrent ainsi pendant sept jours consécutifs. N'aurait-on pas reculé jusqu'au 10 la dédicace à seule fin d'obtenir cette prolongation ? Car c'est précisément au début du 2e siècle que le goût croissant du théâtre fit augmenter, de façon générale, le nombre des jours de spectacle. C'est à partir de cette même année 563/191 que les Jeux Romains durent dix jours ; déjà peut-être les Jeux Apollinaires duraient huit jours.

*) Liv. 36, 36.

Le nom de \emph{Megalesia},* donné dès l'origine à la fête du 4, fut appliqué à l'ensemble des Jeux de la Megalè, devenue Magna Mater. On les appelait aussi \emph{Megalensia}, en latinisant le vocable grec, ou bien encore \emph{Ludi Megaleses}, \emph{Megalenses},* et même, avec une autre désinence grecque, \emph{Ludi Megalesiaci}.* Ce sont les seuls Jeux de Rome, fait observer Cicéron, qui portent un nom exotique. Il n'en faudrait pas conclure qu'ils sont les seuls à manifester l'influence prépondérante de l'hellénisme. Nous ne pouvons toutefois préciser si cette influence ici fut directe ; car nous ignorons si le rite existait dans les cultes métroaques d'Anatolie et de Grèce.* Mais ce que nous savons fort bien, c'est le caractère essentiellement national des Jeux dans la religion romaine. Vouer les Jeux à une divinité, en temps de guerre, était une très ancienne coutume de Rome ; on lui doit l'institution des Jeux Romains par excellence, \emph{Ludi Romani}. Quatre ans après Cannes, on fondait les Jeux Apollinaires « pour obtenir la victoire. » Jeux Apollinaires et Jeux Mégalésiaques, dont les mêmes circonstances politiques déterminèrent la création, ont eu la même signification religieuse. Ils ne sont pas seulement des hommages rendus à la divinité ; il faut y voir des satisfactions qui lui sont offertes pour détourner l'effet des prodiges.* Dans le langage de la théologie romaine, ce sont des procurations. Si les Jeux sont perpétuels, ils conservent à perpétuité leur vertu expiatoire. Ceux des Mégalésies, qui constituent la partie la plus populaire et en apparence la moins religieuse des fêtes, gardent le titre de « Jeux très saints et très purs.* » On n'oublie pas qu'ils se rattachent à un principe de purification nationale. Vers la fin de la République, alors que Carthage était détruite depuis un siècle, on les célébrait toujours « pour apaiser la Grande Mère en faveur du peuple romain » ; et cette formule sacrée se transmit certainement, avec le rite même, jusqu'à la fin du paganisme.*

*) Cic., \emph{De har. resp.} 11 et 12 ; Varr., \emph{De ling. lat.} 6, 15 ; Liv. 34, 54 ; 36, 36 ; Ovid., \emph{Fast.} 4, 357 ; Juven. 6, 69 ; Prudent., \emph{Adv. Symm.} 1, 628 ; Megalensia dans \emph{Arnob.} 7, 33.

*) Cic., \emph{l. c.}, 12, 24 ; Tac., \emph{Ann.} 3, 6 ; Mart. 10, 41, 5 ; Gell. 2, 24, 2 ; Tertull., \emph{De spect.} 6 ; Anonyme de 394 dans Baehrens, \emph{Poet. Lat. Min.}, 3, p. 286, vers 77 et 107. --- Cf. en tête de certaines pièces de Térence, la mention : « \emph{Acta ludis Megalensibus}. »

*) Juven. 11, 193 ; Cal. Philoe., ad 10 apr. ; cf. \emph{Megalesiaca Mater}, dans Auson., \emph{Eclog. de fer. rom.} 2.

*) Nous savons cependant qu'à Attouda les jeux de la cité sont appelés Olympia Heracleia Adrastèa, en l'honneur d'Héraclès et de la Meter Adrasteia : Ramsay, \emph{Cities}, 1, p. 169 ; les jeux dits Chrysanthina, à Sardes, étaient peut-être aussi consacrés à la Kybêbè : \emph{ibid.}, p. 108. D'autre pari, Cic., \emph{l. c.}, semble dire que les jeux des Mégalésies font partie de la \emph{religio} importée d'Asie Mineure.

*) Aussi voyons-nous qu'en 191 les Jeux furent célébrés « cum maiore religione, quod novum cum Antiocho instabat bellum, » Liv. 36, 36.

*) Cic., \emph{l. c.}, 12, 21 : « maxime casti, religiosi... ; quorum religio tanta est... » ; 13, 27 : « castissimi. »

*) Cf. Arnob. 7, 33 : « (ludis) lenior Mater Magna efficitur. »

Le caractère civique et patriotique de telles cérémonies en exclut, par définition, l'ennemi, c'est-à-dire l'étranger, et pour la même raison l'esclave, qui est de race étrangère.* Car, pour que ces rites soient efficaces, il ne faut pas que l'étranger puisse en pénétrer le mystère, en paralyser la vertu. La paix des Dieux et le salut de l'État sont à ce prix. Les \emph{Ludi Megalenses} sont donc, en principe, des spectacles à huis clos. Seuls les citoyens, avec les matrones et leurs enfants, peuvent y assister ; et ils y ont leurs entrées libres. Avant chaque séance, le président des Jeux, par la voix du héraut, commande aux esclaves de sortir.* L'an 698/56, on vit aux Mégalésies un scandale inouï. Le fauteur d'anarchie, P. Clodius, profita de ses fonctions présidentielles pour faire envahir le théâtre par une bande de serfs, au cours d'une représentation. Il fallut le sang-froid et l'énergie du consul Lentulus pour empêcher un massacré. Mais les Jeux n'en étaient pas moins souillés, profanés, sacrilèges.* Bientôt, dans la campagne romaine, on put entendre d'étranges grondements ; et le sol frémit. C'était la déesse irritée qui, au galop de ses lions, parcourait champs et bois en manifestant sa colère.*

*) Cf. Diels, \emph{Sibyll. Blaetter}, p. 96.

*) Cic., \emph{De har. resp.} 2 et 12 ; cf. pour les obsécrations, Sueton., \emph{Claud.} 22 ; pour les Jeux séculaires, Zozim. 2, 5, 1.

*) Cic., \emph{l. cit.}, 13, 27 : « Nihil te... permovit quominus... ludos omni flagitio pollucres, dedecore maculares, \emph{scelere obligares} » ; cf. 11 : (s'il y a eu le moindre manquement au rituel des Jeux) « mentes deorum... ludorum instauratione placantur. »

*) \emph{Ibid.}, 11, 24 : « Hanc Matrem Magnam cuius ludis violati, polluti... sunt, hanc, inquam, accepimus agros et nemora cum quodam strepitu fremituque peragrare. » On avait consulté les haruspices précisément parce que « in agro latiniensi auditus est strepitus cum fremitu. »

Considérés comme l'accomplissement d'un vœu national, les Jeux ne peuvent être offerts que par les magistrats avec \emph{imperium} ; car ces magistrats seuls ont le droit de vouer des Jeux au nom de l'État. On ne nous dit pas qui donna ceux du 4 avril 204. Ceux du 10 avril 191 furent confiés au préteur urbain et pérégrin, M. Junius Brutus, qui avait en même temps la mission de dédier le temple. Mais seuls les Jeux Apollinaires, à ce qu'il semble, restèrent dans les attributions du préteur urbain. Il paraît constant que, du jour où les Jeux deviennent une fête perpétuelle et stative, les magistrats impériaux y délèguent les édiles. Les origines plébéiennes du culte de Cérès avaient fait attribuer aux édiles de la plèbe les \emph{Ludi Ceriales}. Dans le culte aristocratique de Magna Mater, il n'y avait de place que pour les édiles curules. Ce sont eux qui furent chargés d'assurer le service des \emph{Ludi Megalenses}.* Ils conservèrent cette fonction jusqu'en l'an 732/22 ; à cette date, Auguste imposa la présidence de tous les Jeux aux préteurs.

*) Liv. 34, 54, veut-il dire que les édiles curules furent chargés pour la première fois de ce service en 561/193 ?

Institués au nom de l'État et pour le salut de l'État, ces Jeux sont célébrés, en principe, aux frais de l'État. Nous ne savons pas quelle était la somme allouée par le Sénat pour les Mégalésies,* ni si les édiles eurent recours, ici comme ailleurs, à des souscriptions publiques, à des prélèvements sur la caisse des amendes, voire à des contributions imposées aux provinces et aux alliés. Mais s'ils jugeaient insuffisante la subvention du Trésor, ils avaient la faculté de coopérer de leurs propres deniers à ces fêtes nationales. Cette coutume est antérieure à la fondation des Jeux Mégaliens.* P. Cornelius Scipio, le consul de 205, avait rendu son édilité célèbre par ses largesses. En 558/196, P. Scipio Nasica, le \emph{vir optimus} qui avait reçu la Dame noire, suivit dignement l'exemple de son cousin.* De plus en plus, progrès du luxe et puissance de l'argent, exigences du public et souci d'une popularité qui pouvait conduire aux honneurs suprêmes firent, de cette participation bénévole aux dépenses des Jeux, une obligation fort onéreuse, mais fort enviée. « A la richesse des fêtes l'électeur mesurait la capacité du candidat.* » L'édile Jules César fut de ceux qui célébrèrent avec le plus d'éclat les Jeux Mégaliens, pour se concilier les bonnes grâces du peuple.*

*) Les chiffres que nous connaissons sont très variables, suivant les jeux et les époques. Pour les Ludi Romani, on avait d'abord fixé 200000 sesterces ; en 217 on éleva la somme à un peu plus de 330000. Pour les Jeux Apollinaires la somme était, en 212, de 12000 as. Cf. Wissowa, \emph{Rel. u. Kult. d. R.}, p. 383.

*) Elle paraît s'être développée surtout à partir de 213.

*) Liv. 33, 25, à propos des Ludi Romani. Sur la « splendor aedilitatum » recherchée par les \emph{Optimi Viri}, cf. Cic., \emph{De off.} 2, 16.

*) Mommsen, \emph{Hist. rom.}, trad. Alexandre, 4, p. 87.

*) Dio Cass. 37, 8.

Si la somptuosité de ces fêtes était subordonnée à l'ambition des édiles, ceux-ci n'avaient pas le pouvoir d'en modifier à leur gré les éléments. Il y avait un cérémonial des Jeux qui en déterminait la nature, le nombre et l'ordre. Les fêtes du 4 avril 204 paraissent n'avoir comporté que des Jeux du Cirque. Mais, en 560/194, on profita du dixième anniversaire pour ajouter des jeux scéniques.* C'est alors que la durée des Mégalésies fut portée à deux jours au moins ; car les deux espèces de Jeux n'avaient jamais lieu dans la même journée. Le 10 avril 563/191, ce furent des représentations théâtrales qui suivirent la cérémonie de la dédicace ; on y joua peut-être le \emph{Pseudolus} de Plaute.* A partir de l'année où les Mégalésies durèrent sans interruption du i au 10, les spectacles du Cirque furent remis au dernier jour des fêtes. Les divertissements offerts dans le Cirque étaient, au 2e siècle : des courses de chevaux, avec exercices de voltige (\emph{desultores}), des courses de chars, des manœuvres militaires qu'exécutaient, comme au Champ de Mars, de jeunes citoyens, enfin des exercices de gymnastique. Malgré leur variété, ces Jeux n'absorbaient pas encore toute une journée ; même sous les premiers Césars il n'y avait pas plus de dix à douze courses par spectacle, ce qui fait un maximum de quatre heures. Quant aux Jeux scéniques, on y donnait des tragédies, des comédies, avec intermèdes musicaux, des farces, des pantomimes avec chœurs et des ballets. Il ne se jouait qu'une pièce dans la même journée, entre le déjeuner (\emph{prandium}) et la cène, c'est-à-dire environ entre midi et trois heures. Plusieurs comédies de Térence furent représentées pour la première fois aux Mégalésies : en 588/166, l'\emph{Andria} ; en 589/165, l'\emph{Hecyra} ; en 591/163, l'\emph{Heautontimoroumenos}.* L'\emph{Eunuchus} fut produit aux Jeux de 593/161, mais c'était une reprise.* Le \emph{Phormio} parut sur le programme de 613/141. Les sujets des ballets et des pantomimes, en général empruntés à la mythologie, devaient rappeler plus spécialement ici les principales légendes du mythe métroaque. Nous savons que, dans les derniers siècles de l'Empire, on y étalait sans vergogne sur la scène les amours de Cybèle et d'Attis.* Mais la déesse avait aussi sa légende romaine. Du temps d'Auguste, on aimait à représenter au théâtre le miracle de Claudia Quinta.*

*) Liv. 34, 54. D'après Valerius d'Antium (Liv. 36, 36, 4), ces jeux scéniques n'auraient eu lieu pour la première fois qu'en 191 ; mais il est contredit par d'autres témoignages. Ce fut Scipion l'Africain qui, pendant l'année de son deuxième consulat, en 194, sépara au théâtre les places du Sénat et celles du peuple ; et il accomplit cette innovation, d'après Cicéron, \emph{De har. resp.} 12, 24, et Val. Max. 2, 4, 3, aux jeux Mégalésiens.

*) Cf. Ritschl, \emph{Parerga Plautina et Terent.}, 1845, p. 294. C'est encore à la noblesse, éprise de culture hellénique, qu'il faut attribuer cette prépondérance des représentations de la scène.

*) Cf. les didascalies, dont voici le type : « Andria P. Terenti... acta ludis Megalensibus, M. Fulvio M'. Glabrione aed. cur... ; tota facta prima M. Marcello C. Sulpicio coss.. » En 165, les édiles curules qui donnèrent l'Hécyre étaient Sex. Iulius Caesar et Cn. Cornelius Dolabella ; les édiles étaient, en 163, L. Cornelius Lentulus et L. Valerius Flaccus ; en 161, L. Postumius Albinus et L. Cornelius Merula. Sur ces didascalies, v. Dziatzko, dans \emph{Rhein. Mus.}, 20, 1865, pp. 570-598, et 21, 1866, pp. 64-92.

*) Didasc. : \emph{facta secunda}.

*) Arnob. 4, 35 ; cf. Augustin., \emph{Civ. Dei} 2, 4, et une série de médaillons contorniates publiés par Ch. Robert : \emph{Les phases du mythe de Cybèle et d'Attis rappelées par les méd. cont.} (extr. de la \emph{Rev. Numism.}, 1885), par Cohen, 2e éd., 8, \emph{Contorniates}, pp. 277-318, n°s 26, 63, 94, 103-105, 107, 197, 212-216, 302, 304, 307, 362, 364, et par Gnecchi dans \emph{Rivista numism. ital.}, 8, 1895, p. 291 ss, n°s 6, 33, 50, 54, 95, 98. Sur la représentation scénique du miracle d'Ostie, cf. Ovid., \emph{Fast.} 4, 326.

*) Cf. p. 64, n. 3.

Le rituel veut que la divinité assiste aux Jeux donnés en son honneur. Chaque année, quand venait le temps des Mégalésies, on édifiait un théâtre devant le temple du Palatin* ; lorsque les fêtes étaient terminées, on le démolissait. Le spectacle se déroulait ainsi sous les yeux mêmes de la Mère des Dieux. Peut-être n'avait-elle pas besoin de sortir du sanctuaire ; sinon, il suffisait de la transporter sous le portique du pronaos. L'escalier monumental de la façade aboutit à une esplanade à peu près triangulaire ; et l'on pouvait utiliser la pente supérieure de la colline, là où sont les ruines d'une maison de l'époque impériale. Du côté sud, le théâtre s'appuyait probablement sur la muraille en \emph{opus quadratum} qui flanquait un vieux chemin, dit « escalier de Cacus. » On avait pu creuser quelques gradins dans le tuf même du Palatin. En profitant aussi des degrés du temple, on disposait d'un assez grand espace. Les constructions, en bois, étaient fort simples. Elles ne comportaient que deux éléments essentiels : une estrade pour la scène, et une barrière qui fermait l'enclos. Primitivement les spectateurs s'entassaient debout et pêle-mêle ; seuls les magistrats et les prêtres avaient des places réservées et assises. Une importante innovation signala les premiers jeux scéniques donnés aux Mégalésies (560/194). Pour la première fois dans un théâtre de Rome, on établit cette année-là deux catégories de places : il y eut désormais celles du Sénat, aux premiers rangs, et celles du peuple, par derrière. Ce fut le grand Scipion, le vainqueur de Zama, chef de la noblesse, consul pour la seconde fois, choisi par les censeurs comme prince du Sénat, qui prit l'initiative de cette mesure aristocratique.* Ce furent ces mêmes censeurs, Sex. Aelius Paetus et C. Cornelius Cethegus, qui se chargèrent officiellement de la faire appliquer par les édiles curules ; la censure était alors l'instrument tout puissant de l'oligarchie nobiliaire. Un peu plus tard, mais encore dans cette première moitié du 2e siècle, la masse du public prend à son tour l'habitude de s'asseoir au spectacle. Après la conquête de la Grèce, on perfectionne l'agencement des théâtres romains. Dans les dernières années de la République, le théâtre des Mégalésies comprenait, outre la scène et la barrière traditionnelle, un véritable édifice, avec gradins superposés, couloirs et galeries en arcades.* Lorsque Rome eut ses théâtres permanents et en pierre (théâtres de Pompée, 700/54, de Marcellus, 741/13), s'empressa-t-on d'y transférer les Jeux scéniques de la Grande Mère ? C'est peu vraisemblable ; car la coutume d'improviser des théâtres en bois auprès des temples persista longtemps sous l'Empire. Pour ce qui est des courses, elles exigeaient un espace beaucoup plus vaste et tout spécialement aménagé. Aussi, dès la période royale, les Romains avaient-ils leur Cirque. Ils en possédaient même deux en 204. Mais celui de Flaminius, tout récent, en dehors des murs, était destiné surtout aux Jeux Plébéiens. \emph{Les Ludi circenses} des Mégalésies durent avoir lieu toujours dans le Cirque Maxime, que dominait le temple de l'Idéenne. La déesse s'y rendait selon les rites de la \emph{Pompa circensis} et y prenait place au \emph{pulvinar} des dieux.

*) Cic., \emph{l. c.} : (ludi) « quos in Palatio nostri maiores ante templum in ipso Matris Magnae conspectu Megalensibus fieri celebrarique voluerunt » ; cf. Huelsen dans \emph{Roem. Mitt.}, 10, p. 28. Autres exemples cités par Wissowa, \emph{op. l.}, p. 395, n. 2 et 3.

*) Cic., \emph{l. c.} : « quibus ludis primum ante populi consessum senatui locum P. Africanus iterum consul ille maior dedit. » Val. Max. 2, 4, 3.

*) Cic., \emph{l. c.}, 11, montre les esclaves se précipitant, sur l'incitation de Clodius, « fornicibus ostiisque omnibus in scenam. »

Comme si la célébration de sacrifices et de Jeux publics, offerts par les magistrats, ne suffisait point pour assurer le culte national de la déesse, il se créa dans Rome, dès l'année 204, des confréries delà Grande Mère.

Ces confréries portent le nom de sodalités.* La sodalité est un collège exclusivement religieux, dont les membres sont liés entre eux par l'exercice en commun de certains devoirs de piété. Elle est à l'origine une \emph{gens} fictive, dépositaire de l'un des nouveaux cultes adoptés par l'État et que l'État ne veut point pourvoir d'un sacerdoce public ; elle accomplit un ministère sacré dans un temple déterminé. S'il était vrai que l'Idéenne fut hospitalisée d'abord chez les Scipions, voici comment pourrait s'expliquer l'origine du collège. Le culte aurait été confié tout d'abord à la famille des Scipions, c'est-à-dire à une branche de la gens Cornelia. Quand la déesse eut quitté leur maison pour habiter son propre sanctuaire, la confrérie gentilice se serait transformée en sodalité. Mais nous avons vu que la Grande Mère fut sans doute installée le jour même de son arrivée dans le temple de la Victoire. D'autre part, nous savons par Cicéron que ces sodalités sont contemporaines de l'introduction du culte. En troisième lieu, il serait difficile d'expliquer ainsi la raison d'être de plusieurs sodalités ; un seul collège de marchands avait suffi pour desservir le culte de Mercure, de même qu'à la fondation du temple de Venus Genitrix se liera l'institution d'un collège unique. Enfin les confréries de la Grande Mère ne nous apparaissent point comme des sacerdoces ; et l'on ne peut dire que l'État leur transmet, à proprement parler, l'exercice d'un culte public. Elles représentent un type d'association religieuse très différent de celui des sodalités dont nous venons de parler, plus récent dans la religion romaine, et emprunté sans doute à la religion hellénique. Il semble en effet qu'elles aient été conçues à l'image des éranes et des thiases métroaques. Leur principale, sinon leur unique fonction était l'organisation de banquets, qui rappellent les festins éraniques et les symposies des thiasotes. Le jugement que porte sur elles Cicéron est identique à celui d'Aristote sur les thiases de son temps.* « C'est surtout une agréable occasion de se réunir et de converser entre amis, » fait dire Cicéron à Caton l'ancien, questeur en 204, qui avait été l'un des membres fondateurs de ces sodalités. Mais il ne faudrait point conclure de cette appréciation d'un Romain qu'elles sont avant tout, et dès l'origine, des cercles ou, suivant l'expression de Mommsen, de véritables clubs. Leurs banquets sont des repas sacrés ; ils sont encore un élément du culte ; ils constituent en quelque sorte l'achèvement des sacrifices et des jeux. Il n'y a pas de jeux sacrés chez les Romains sans un festin sacré, \emph{epulum} ; quand ce festin n'est pas public, il est célébré par un collège spécial, fondé à cette intention.* La communion de culte (\emph{communio sacrorum}), qui lie entre eux les compagnons d'une sodalité, consiste précisément ici dans la participation au banquet. En raison même de leur nature religieuse, les repas sacrés des Mégalésies sont inscrits dans les calendriers liturgiques au nombre des fêtes romaines, à la suite des Jeux.* On ne saurait donc prétendre que les associations chargées de les organiser sont dues à l'initiative privée et dépourvues de tout caractère officiel. L'État ne se contente pas de leur reconnaître une existence légale. Il ne pouvait rester indifférent à cette fondation pieuse. Celle-ci est l'œuvre des Décemvirs. Ce sont eux qui en ont fixé la composition et le fonctionnement.

*) Cic., \emph{Cato maj.} 13, 45 : « \emph{sodalitates autem me} (c'est Caton qui parle) \emph{quaestore constitutae sunt, sacris Idaeis Magnae Matris acceptis. Epulabar igitur cum sodalibus etc.} »

*) \emph{Eth. Nicom.} 8, 9 : « On y honore les dieux et l'on s'y procure à soi-même à la fois repos et plaisir. »

*) Aul. Gell. 2, 24, 2, rattache nettement ces banquets de \emph{sodales} aux ludi \emph{Megalenses}.

*) Cal. Praen., au 4 avril.

Leur composition était exclusivement aristocratique. Seuls pouvaient y être affiliés les patriciens, les nobles, ceux que l'on qualifiait de princes romains.* A cet égard elles différaient tout à fait des thiases de la Grèce propre, qui se recrutaient parmi les esclaves, les étrangers et les citoyens de petite condition. Mais elles ressemblaient beaucoup aux saintes confréries d'Asie Mineure, dont les mystes, Bouviers de Dionysos ou Corybantes de la Mère des Dieux,* appartenaient aux meilleures familles et étaient les premiers de leur cité. Pergame possédait des confréries de ce genre. Elles ont pu servir de modèles. Voulut-on que la Dame retrouvât au Palatin, en même temps que ses serviteurs phrygiens, l'équivalent de ses nobles compagnons ? Il serait intéressant de voir Rome préoccupée de faire place, dans l'organisation du culte national de l'Idéenne, à certains éléments du culte orientalo-grec. Mais le principal souci de la classe dirigeante, c'est plutôt d'éviter toute confusion entre le culte romain et le culte phrygien, entre dévotion et orgie ; c'est de conserver à la religion romaine sa dignité calme. Les sodalités nobles ne risquaient pas de dégénérer en orgéons.

*) \emph{Nobiles}, dit le calendrier de Præneste ; \emph{principes civitatis}, dans Aul. Gell. 2, 24, 2 ; cf. 18, 2, 11 : \emph{patricii}.

*) Ziebarth, \emph{Griech. Vereinswesen}, 1896 : p. 50, boukoloi de Pergame ; 56, 149, archiboucoloi ; 52, Corybantes d'Erythrae.

L'existence de plusieurs sodalités nous laisse supposer que toute la noblesse y était représentée. Peut-être se groupait-on par quartiers. Dans chaque sodalité la liste des membres ne pouvait atteindre un chiffre bien considérable : sans quoi l'application du principe adopté pour les banquets n'eût pas été possible. Ces confréries sont en effet des sociétés de repas mutuels. Les compagnons d'une même sodalité s'invitent à tour de rôle.* On appelait ces repas \emph{dominia}, pour indiquer que les convives étaient reçus chez un maître de maison.* Mais leur nom officiel était celui de mutitations,* ce qui signifie que l'on changeait chaque fois de logis. Plus tard, on se plut à y retrouver un symbole du voyage accompli par la déesse, qui avait voulu changer de séjour.* Le rite de la mutitation n'était pourtant pas nouveau dans Rome. On l'avait introduit déjà dans le culte de Cérès, où il est confié à la plèbe. Il parait être de provenance hellénique.*

*) Ovid., \emph{Fast.} 4, 353 : « vicibus factis. »

*) Gell. 2, 24. 2 : « mutua inter se dominia. »

*) Cal. Praen., \emph{l. c.} : « nobilium mutitationes cenarum » ; Gell. \emph{ll. cc.}

*) C'est l'explication donnée par le Calendrier de Praeneste et par Ovide.

*) Kios, \emph{CIG.} 3727 (restitution douteuse ; époque romaine) ; Panormos (Panderma), dans Ziebarth, \emph{op. l.}, p. 66 (cf. \emph{Rev. Et. Grecques} 1894, p. 391). A Éphèse, pendant les fêtes annuelles de la Lêto d'Ortygie, les jeunes gens rivalisent entre eux à qui donnera les repas les plus somptueux : Strab. 14, 1, 20. Dans l'\emph{Eumenides} de Varron, un cynique parle d'un cercle de philosophes qui s'est rassemblé chez lui, parce que c'était à son tour de traiter la confrérie (éd. Riese, p. 126, fr. 6 ; cf. Ribbeck, \emph{Hist. de la poésie lat.}, p. 311). Antoine et ses compagnons forment en Égypte des sociétés où l'on se traite à tour de rôle: Plut., \emph{Ant.} 28 et 79.

Les mutitations se prolongent pendant tout le temps des Jeux, du 4 au 10 avril. Elles ont lieu après le spectacle, à l'heure de la cène, c'est-à-dire vers trois heures de l'après-midi. Furent-elles d'abord aussi modestes que Cicéron le fait dire à Caton l'Ancien ? En tout cas, très vite on y rivalisa de somptuosité ; c'était pour la noblesse une occasion trop belle de manifester ses goûts de luxe. On y servait les plats les plus raffinés de la cuisine orientale ; on y consommait les vins de Grèce les plus rares ; on y exposait des trésors d'argenterie, chefs-d'œuvre de l'art alexandrin que se disputaient les riches Romains. Moins d'un demi-siècle après leur fondation, en 593/161, le Sénat fut obligé de voter un décret contre elles.* Les membres des sodalités durent s'engager par serment, devant les consuls, à ne point dépenser pour chaque banquet plus de 120 as par tête, sans compter les légumes, la farine et le vin ; à n'y boire que des vins du pays ; à ne pas consacrer de vaisselle d'argent au service de table pour un poids de plus de cent livres. Ce sénatus-consulte était, comme la loi cibaire du consul Fannius qui fut promulguée la même année,* un triomphe du parti de réaction contre la dégénérescence morale de la cité. Aussi bien le parti conservateur, que dirigeait toujours le vieux Caton, avait-il conquis à ses idées beaucoup de nobles de la jeune génération. Mais toutes les lois somptuaires et cibaires, sans cesse aggravées, abrogées et reprises, restaient impuissantes à contenir les excès. Les sodalités de la Grande Mère, comme tant d'autres confréries, devinrent-elles des foyers d'agitation politique, pendant la longue période des guerres civiles ? Épargnées par les lois restrictives, elles survécurent à la République. Au début de l'Empire, les mutitations des nobles figurent toujours sur le calendrier.* On dut alors les célébrer avec d'autant plus de magnificence que l'Idéenne était particulièrement chère à la dynastie impériale. Juvénal y fait peut-être allusion* et semble dire que les femmes y sont admises.* Il est probable qu'au 4e siècle les chefs de la noblesse, fervents dévots de Cybèle, eurent à cœur de ne point laisser perdre la tradition.

*) Aul. Gell. 2, 24, 2, d'après C. Ateius Capito.

*) Il y avait eu déjà en 181 la loi cibaire Orchia, limitant le nombre de ceux qui pouvaient prendre part à un repas ; en 143, la loi Didia, également cibaire.

*) Cf. Cal. Praen. et Ovide.

*) Juven. 11, 192, invitant un ami à dîner le 10 avril, dernier jour des Mégalésies : « ingratos... pone sodales. »

*) Aux vers précédents, 186-189, généralement mal interprétés ; v. Jessen dans \emph{Philologus}, 59, 1900, p. 519.

En résumé, le culte officiel de la Grande Mère Idéenne, tel qu'il fut organisé à l'origine, conserve dans toutes ses manifestations un caractère national et aristocratique. Il est national par le principe même qui en a déterminé l'institution. Car il est un acte expiatoire, qui n'intéresse que le peuple romain et dont le peuple romain doit se réserver à lui seul l'accomplissement. Il est national par ses rites, qui sont ceux de la religion gréco-romaine du 3e siècle. Nous n'en avons pas signalé qui ne fût pratiqués déjà dans d'autres cultes de Rome. Les rites proprement orientaux n'ont pas plus acquis droit de cité que le clergé phrygien de la déesse. Exception fut faite pour la Lavation, à laquelle assistaient les Décemvirs. Mais on ne pouvait laisser la Dame sortir sans escorte ; la participation de ces personnages au cortège n'était peut-être à l'origine qu'une mesure de police. D'autre part, rien ne prouve que cette cérémonie ait pris place, dès le début, parmi les fêtes du calendrier officiel ; elle reste étrangère aux véritables Mégalésies romaines. Enfin ce rite du bain n'a pas en lui-même un caractère d'orgiasme et n'appartient pas en propre aux religions orientales.* Le culte est aristocratique parce qu'il est, à tous égards, l'œuvre propre de l'aristocratie. Ce sont les optimates, maîtres du gouvernement, qui ont introduit la déesse anatolienne, pour des raisons de politique extérieure ; ce sont eux qui, pour des raisons de politique intérieure, assurent la célébration de son culte. Car ils ont eu conscience des responsabilités qu'ils assumaient en laissant pénétrer dans Rome une divinité orgiastique. La présence du clergé phrygien était devenue un mal nécessaire ; mais il importait d'en prévenir les conséquences dangereuses. Or c'est sur la plèbe que pouvait s'exercer la contagion ; c'est par elle que pouvait se produire la contamination du culte nouveau. En éliminant de cet organisme tout élément plébéien, ils espéraient éviter l'infiltration lente des superstitions phrygiennes. En imposant à la plèbe un rôle purement passif, ils voulaient sauvegarder l'intégrité de la religion nationale. Ils y réussirent en partie, puisque, sous la République, l'histoire n'enregistre aucune altération apparente du culte de Mater Magna. Et l'on comprend que la gravité persistante de ce culte métroaque, au début de l'Empire, excite la vive et sincère admiration du Grec d'Asie Denys pour le génie romain.

*) En Grèce : bain de Héra Argeia dans la source Kanathos près de Nauplie, après les fêtes du \emph{hieros gamos} avec Zeus ; également à Argos, bain de Pallas dans l'Inachos ; bain de Demeter Lusia dans le Ladon (Arcadie), après son \emph{hieros gamos} avec Poséidon ; cf. Hepding, \emph{Attis}, p. 175, n. 7. A Rome, lavation de Venus Verticordia (= Aphroditè Apostrophia) dont le culte fut importé de Grèce en 640/114 ; Ovid., \emph{Fast.} 4, 136 : « tota lavanda dea est. »

\subsection{3.}

La Dame Noire fut tout de suite populaire à Rome. Car elle ne tarda pas à donner des témoignages publics de sa protection. Le jour même de son arrivée, on avait peut-être crié au miracle. En cette année 204, on obtint dans la campagne romaine une magnifique récolte ; c'était la plus belle depuis dix ans. On ne manqua point de l'attribuer à la déesse, qui en Orient faisait mûrir les épis.* L'année suivante, Hannibal quittait pour toujours l'Italie. La prédiction de l'oracle s'était réalisée. La Grande Mère de Phrygie avait été plus puissante que la Grande Mère de Carthage. Le Sénat décréta cinq jours de supplications aux dieux et le sacrifice de cent vingt victimes majeures.* On dut se porter en foule vers le temple de la Victoire Palatine, provisoire demeure de celle qui venait d'être « le salut du peuple romain et la cause de tant de joies.* » Les poètes qui désormais la placent aux côtés de Mars et des dieux Indigètes,* divinités tutélaires de la patrie, ne font qu'exprimer le sentiment de la reconnaissance générale. La déesse à qui Rome devait son triomphe sur Carthage était maintenant intéressée au succès de toutes les expéditions de la République. Elle avait été propice aux armes romaines ; elle devint une déesse des armées.* Les généraux l'invoquent dans les guerres difficiles, dans les batailles décisives. D'aucuns font même vœu de se rendre à Pessinonte et de sacrifier sur le grand autel du fameux temple.* Ce pèlerinage votif, qui n'était point rare, put servir de prétexte plausible à certain voyage de Marius, désireux à la fois de quitter Rome en un temps de réaction et d'aller se ménager un rôle dans les affaires d'Orient.*

*) Plin., \emph{H. n.} 18, 4, 16.

*) Liv. 30, 21, 10.

*) Arnob. 7, 49 : « Salutaris populo et magnarum causa laetitiarum ; nam et diu potens hostis ab Italiae possessione detrusus est et gloriosis inlustribusque victoriis decus urbi restitutum est pristinum » ; cf. 7, 40.

*) Silius Italicus, \emph{Pun.} 9, 290, la cite à côté des grands dieux protecteurs de Home, des dieux Indigètes et de Quirinus.

*) Sur ce rôle des divinités telluriques à Rome, cf. Tellus, à qui l'on vouait les légions ennemies, et Lua Mater, à qui l'on brûlait une part du butin de guerre. Pour la Mère des Dieux, cf. supra, pp. 32 et 36, et Arnob. 7, 51 : « bellicas res amet (Mater Magna). »

*) Cic., \emph{De har. resp.} 13, 28 : « Nostri imperatores maximis et periculosissimis bellis huic deae vota facerent eoque in ipso Pessinunte ad illam ipsam principem aram et in illo loco fanoque persolverent. » Val. Max. 1, 1 : « Matri Deum saepenumero imperatores nostri, compotes victoriarum, suscepta vota Pessinuntem profecti solverunt. »

*) Plut., \emph{Marius} 33 : « Marius s'embarqua pour la Cappadoce et la Galatie, sous prétexte d'aller accomplir les sacrifices qu'il avait voués à la Mère des Dieux ; mais ce voyage avait un autre motif. »

Il était naturel que la Dame favorisât particulièrement dans son pays d'origine l'œuvre des légions. Elle eut bientôt à les y défendre contre Hannibal, que Rome retrouvait, toujours menaçant, auprès du roi de Syrie. Elle fut la divine auxiliaire de la diplomatie sénatoriale. Non seulement le culte de l'Idéenne resserre l'alliance de la République avec le royaume de Pergame ; mais, un peu plus tard, il détermine aussi d'utiles relations entre Rome et Pessinonte. Il propage l'influence de Rome jusqu'au cœur même de l'Anatolie ; et des milliers de pèlerins orientaux, accourus aux panégyries de la ville sainte, pouvaient y apprendre à respecter et à redouter la majesté du nom romain. Les prêtres de Cybèle, interprètes de la volonté divine, ne négligent point les occasions de manifester leur dévouement intéressé à la cause de Rome. En 565/189, après la défaite d'Antiochos à Magnésie, le consul M. Manlius faisait campagne en Gallo-Grèce contre les Galates, alliés ou mercenaires du roi. Il campait sur les rives du Sangarios quand il vit venir à lui, du côté de Pessinonte, une caravane de prêtres, revêtus des plus riches insignes de leur culte. C'étaient des Galles, envoyés par l'Attis, avec mission de saluer le général et de lui transmettre une prophétie. La déesse avait prédit aux Romains bonne route, victoire assurée et l'empire du pays.* L'Attis exprimait ainsi tous ses vœux pour le succès d'une expédition qui pouvait le débarrasser de dangereux voisins. On exploitait volontiers la vénération des Romains pour la puissante déesse ; et ceux-ci, qui à Rome méprisaient les Galles, leur témoignaient en Orient quelque déférence, par politique. M. Manlius leur fît très bon accueil. L'année précédente (564/190), les soldats de C. Livius, préfet de la flotte, étaient devant Sestos. La ville allait être prise d'assaut et livrée au pillage. Tout à coup surgit une bande de ces « fanatiques, » en costumes sacerdotaux ; ils étaient dépêchés à Livius par les Sestiens, à bout de ressources. « Nous sommes, criaient-ils, les serviteurs de la Mère des Dieux ; c'est par son ordre que nous venons vous conjurer d'épargner la place. » On se garda bien de commettre aucun sacrilège sur leurs personnes. On obéit à la déesse. Et Sestos se rendit aussitôt.* Le traité de 566/188 affirma la puissance de Rome en Orient. Quelques lettres des Attalides aux Attis, heureusement retrouvées à Pessinonte,* en sont pour nous un intéressant témoignage. Sur les rois et les prêtres-rois pèse lourdement la crainte du Romain. « Perdre le secours de Rome, écrit Attale 2, c'est perdre le secours des dieux. »

*) Polyb. 22, 20 (18), 4, cité par Suidas ; Liv. 38, 18. Le texte de Polybe paraît plus complet. Variante dans T. Live : Manlius longe la rive au lieu de camper. Commentaire topographique dans Koerte, \emph{Gordion}, p. 31.

*) Polyb. 21, 6, 7 ; Liv. 37. 9, 8.

*) Dans le cimetière arménien de Sivri-Hissar. Publiées d'abord par Mordtmann dans les \emph{Muenchner Sitzungsber.}, 1860, p. 180 ss ; puis, après révision du texte, par Domaszewski dans \emph{Arch. ep. Mitt. a us Oesterr.}, 8, 1884, p. 95-101 ; Michel, \emph{Recueil d'inscr. gr.}, 45 ; Dittenberger, \emph{Syll.}, 315. Commentées par Perrot, \emph{Galatie}, p. 184-185, Mommsen, \emph{Roem. Gesch.}, 2, 8e éd., p. 52 ; \emph{Hist. Rom.}, tr. Alexandre, 4, p. 354 (cf. appendice, p. 379) ; Wilamovitz dans \emph{Lectiones epigr.}, Goettingen, 1885-86, p. 16 ss ; Thraemer, \emph{Progr.}, p. 15-17 ; Hennig, \emph{Symbolae ad Asiae Min. reges sacerdotes}, Leipzig, 1893, p. 49-53 ; Staehelin, \emph{l. c.}, p. 91 ; Wilhelm, \emph{Goetting. gel. Anz.}, 1898, 3, p. 211 ; Niese, \emph{op. l.}, 3, p. 69 et 360. --- La première et seule datée de ces lettres est de 164 ou 163.

L'année 631/103 marque une date fort importante dans les annales du culte de l'Idéenne. Un grand-prêtre de Pessinonte se risque à faire le voyage de Rome. Le récit de cette visite, qui nous a été transmis par Diodore et Plutarque,* contient de curieuses révélations sur l'attitude du gouvernement et sur celle de la population à l'égard du prêtre-roi. C'est un précieux document sur l'état moral de la cité, cent ans après l'installation de la Dame au Palatin.

*) Diod. 36, 13 ; Plut., \emph{Marius} 17.

Le grand prêtre qui entreprit ce voyage était le Battakès, lequel partageait alors avec l'Attis le sacerdoce suprême.* Pour justifier sa venue, il invoquait la volonté de la Mère des Dieux, ses devoirs de prêtre souverain et les droits de Pessinonte à la protection de Rome, qui était l'héritière des Attalides. Il avait à défendre les intérêts sacrés de son temple, récemment profané, nous ignorons dans quelles circonstances. La déesse de Phrygie, par la bouche de son prophète, réclamait aux Romains des expiations publiques. Mais, quoique irritée, elle tenait h se montrer toujours bienveillante ; elle leur promettait victoire et puissance.* On était précisément au moment critique de la guerre contre les Cimbres et les Teutons. Rome se trouvait encore secouée par l'émotion du désastre d'Orange, qui rappelait et dépassait la journée de Cannes. Marius, le vainqueur de Jugurtha, campait sur les bords du Rhône, attendant les Barbares ; mais il y avait de sinistres présages. Les prédictions rassurantes de Cybèle venaient à point. De nouveau l'Italie avait connu les terreurs de l'invasion menaçante, et voici que de nouveau la Grande Mère la protégeait contre l'envahisseur de race étrangère. Par scrupule religieux autant que par politique, le Sénat ne pouvait confondre Battakès avec les nombreux prophètes et devins qui demandaient à vaticiner en sa présence. La Curie avait pour principe de leur fermer ses portes. Elle les ouvrit au Pessinontien. D'après le protocole adopté, celui-ci eut d'abord des entretiens secrets avec les premiers magistrats ; puis il fut admis à s'expliquer devant les Pères Conscrits, qui vouèrent un autre sanctuaire à la Dame si les armes romaines étaient victorieuses ; enfin il se présenta au peuple et parla du haut des rostres. On fit plus : il obtint tous les honneurs de l'hospitalité publique, telle que la recevaient les représentants des puissances amies. Il fut logé aux frais de l'État, put assister aux Jeux et aux sacrifices, et fut gratifié des cadeaux d'usage. Le Sénat s'empressa-t-il de lui accorder ce privilège dès son arrivée ? Ou céda-t-il, comme semblerait le dire Diodore, à la pression du sentiment populaire, exalté par le verbe du prophète ? Ce qui n'est pas douteux, c'est l'engouement du public. Si les honneurs officiels s'adressaient au principicule oriental, c'est vers le prêtre suprême de la religion phrygienne que se portaient les hommages de la foule. Les visions inspirées du Battakès et ses prédications fanatiques troublaient les âmes ; elles développaient une dévotion superstitieuse et passionnée. Ce qui frappait encore les imaginations et contribuait singulièrement au succès du prêtre-roi, c'est qu'il se montrait avec tout l'appareil de sa dignité sacerdotale et royale. Il portait une longue robe, brodée de fleurs d'or, chamarrée d'amulettes, de symboles, de médaillons ; et sur sa tête brillait une énorme couronne d'or. Il bravait audacieusement les antiques traditions de la République. Mais ces traditions faiblissaient. Rome s'était éprise d'exotisme. Ne voyait-on pas un consul* mener partout à sa suite, avec de réels témoignages de respect, Marthe la Syrienne, connue pour son esprit prophétique, sa robe de pourpre et sa javeline enrubannée et fleurie ? Toutefois il se trouvait encore des Romains pour s'émouvoir de ces spectacles inusités. Un tribun du peuple, A. Pompejus, usant de son droit de coercition, voulut interdire au Phrygien d'exhiber sa couronne et de monter aux rostres. Un autre tribun prend la défense du Battakès. Pompejus et ses gens s'emparent des rostres, en chassent le prêtre, qui s'entend traiter de mendiant,* comme le dernier des Galles. Ils le poursuivent d'insultes jusqu'à son logis. Cette intervention maladroite et brutale eut le seul effet qu'on en pouvait attendre. Elle accrut la popularité du persécuté. On ne le vit plus, mais il fît savoir qu'en sa propre personne on avait « couvert de boue » la déesse et qu'elle allait châtier les impies. En rentrant chez lui, Pompejus se sentit indisposé. Une mauvaise fièvre se déclara, puis une angine ; quelques jours après,* il mourait. Naturellement le peuple vit dans cette fin rapide l'effet d'une vengeance divine.* Le prophète reparut, triomphant. Personne ne l'empêcha plus de porter ses insignes. On le combla de présents. Et, le jour de son départ, une foule imposante lui fit escorte.

*) Polyb. 22, 20 (18), 4 : Ἄττιδος καὶ Βαττάκου τῶν ἐκ Πεσσινοῦντος ἱερέων τῆς Mητρὸς τῶν Θεῶν. Diod., \emph{l. c.} : Βαττάκης τις ὄνομα, ἐκ Πισινοῦντος, ἱερεὺς ὑπάρχων τῆς Μεγάλης τῶν Θεῶν Μητρός. Plut., \emph{l. c.} : Βατάκης ἐκ Πεσσινοῦντος ὁ τῆς Μεγάλης Μητρὸς ἱερεύς. A en juger par la désinence, ce nom de B. paraît être d'origine perse. On le retrouve sur une épitaphe d'Ilghin, en Phrygie (Anderson, \emph{A summer in Phrygia}, dans \emph{J. of Hell. St.}, 18, 1898, p. 123) : Βατάκης Mαιφάτει τῇ μητρί, \emph{etc.} ; or, sur une inscription du Pont galatique, le nom de Maiphatès est accompagné du nom Zaraêtos (forme génitive), qui est perse : Hesych., s. v. Ζαρῆτις : Ἄρτεμις, Πέρσαι. Le Battakès de Pessinonte pourrait être le représentant d'une famille sacerdotale que les Perses, à la suite de leurs conquêtes (6e siècle), auraient imposée auprès de l'Attis phrygien et qui prélevait sa part des gros revenus du sanctuaire. Il se serait maintenu après la prise de possession du sacerdoce par les Galates ; cf. Thraemer dans Pauly-Wissowa, \emph{Real-Encycl.}, 3, p. 146, s. v. : « Beachtenswert ist die antigalatische Haltung der pessinuntischen Priesterschaft. » Il ne figure pas à côté de l'Attis « hiereus » dans la correspondance échangée entre ce prêtre et les Attalides ; on doit donc supposer que l'Attis lui était hiérarchiquement supérieur. On ne rencontre plus de traces de son existence au dernier siècle de la République et sous l'Empire.

*) Plutarque emploie ici les mêmes expressions (νίκην καὶ κράτος) que Polybe à propos de l'ambassade des Galles auprès de Cn. Manlius (22, 20). Mais il n'est pas nécessaire de croire, comme le fait Goehler, \emph{De Matris M. apud Rom. cultu}, 1886, p. 10, que Plutarque a été victime d'une confusion.

*) Plut., \emph{Marius}, 18 ; cf. 46, le consul Octavius s'entourant de prêtres et devins orientaux ; \emph{Sylla}, 34 ; Sylla, en Apulie, faisant annoncer la victoire par un esclave transporté de la fureur de Bellone.

*) Plutarque emploie le mot ἀγύρτης, Métragyrte.

*) Le septième jour, d'après Plutarque, le troisième d'après Diodore.

*) Diodore et Plutarque déclarent tous deux qu'il y avait eu, de la part du tribun, ὕβρις.

Dans la multitude qui se pressait autour de lui, nombreux sans doute étaient les Orientaux, esclaves, affranchis ou marchands ; car, dès ce temps-là, surtout depuis la formation d'une province d'Asie, les Levantins commençaient à envahir Rome. Il s'y trouvait beaucoup de femmes, c'est Diodore qui nous l'apprend. Les femmes romaines ont profondément subi l'attrait des cultes d'Orient ; ceux-ci leur procuraient ce qu'elles auraient en vain demandé à la religion nationale, l'intensité de l'émotion religieuse. Parmi les citoyens dominait l'élément plébéien. De tout temps la plèbe de Rome, moins conservatrice et plus superstitieuse que la classe aristocratique, fut la première à se laisser entraîner vers les religions nouvelles. Mais ce qui paraît le plus étrange en cette circonstance, c'est l'attitude des optimates. Ni le Sénat, ni les premiers magistrats, par indifférence, manque d'énergie ou complicité, n'ont rien fait pour réprimer ces manifestations populaires. Ils n'avaient ni prévu ni prévenu l'ascendant que prit sur la population le prêtre de Pessinonte. Combien avait été clairvoyante la politique des ancêtres, soucieux d'éviter par de rigoureuses mesures la contagion du fanatisme métroaque !

L'année qui suivit la visite du Battakès, un esclave de Servilius Caepio se mutilait en l'honneur de la Grande Mère.* La législation romaine n'avait peut-être pas encore prévu le cas de castration volontaire parmi les dommages injustement causés aux maîtres. D'autre part, observe Tite Live, « chaque fois que la religion sert de prétexte à un crime, on craint, en châtiant le coupable, de commettre une impiété.* » On se contenta donc d'expédier le nouveau Galle dans un pays lointain d'outremer. Vingt-cinq ans après, en 677/77, un affranchi, qui portait le nom d'une des principales familles plébéiennes, un certain Genucius,* est Galle de la Mère des Dieux ; et personne ne l'inquiète. La juridiction prétorienne lui reconnaît même la capacité d'hériter. Il fallut que, sur appel de l'héritier naturel, l'un des consuls, un patricien de la gens Aemilia, signifiât opposition au décret du préteur et prononçât la nullité du testament. Le motif essentiel de cette éviction fut que l'héritier institué « n'était ni homme ni femme » ; et bien des Romains se réjouirent du jugement porté contre l'eunuque, « dont la présence obscène et la voix ignominieuse risquaient de souiller les tribunaux. » Mais ce conflit des deux juridictions n'est-il pas le symbole de celui qui commence entre les traditions nationales et l'esprit nouveau ? Un fait certain, c'est que, dans les dernières années de la République, les affranchis pouvaient entrer dans l'ordre des Galles sans être poursuivis par la loi ; car Genucius est seulement disqualifié par sa tare physique. Rien plus, il est fort vraisemblable que déjà l'autorité compétente recrutait dans la classe libertine le clergé de l'Idéenne. A plus forte raison, esclaves et affranchis pouvaient-ils pratiquer impunément le culte phrygien en qualité de simples fidèles. Ils formaient dans Rome une ou plusieurs églises phrygiennes, qui dissimulaient mal, sous l'apparence d'associations funéraires, leur caractère cultuel.* Tous ces zélateurs d'Attis furent en même temps d'ardents propagandistes de leur foi. Les prosélytes sans doute n'étaient point rares. La longue tourmente des guerres civiles favorisait toutes les superstitions. La foule se détournait de plus en plus de la religion nationale pour demander aux cultes d'Orient et d'Égypte des consolations et des espérances. En vain le Sénat luttait contre les dévots d'Isis. Lui-même, pour que la grande déesse de Cappadoce, Mâ la belliqueuse et l'invincible, devînt une alliée de Rome contre Mithridate, il avait reconnu son culte et accepté son clergé. La Dame, que les soldats avaient appris à redouter et à révérer dans les gorges du Taurus,* fut pour Rome une nouvelle Bellone, la déesse du Courage, Virtus. Or ses rites étaient semblables, dans leur barbarie, à ceux delà Grande Mère phrygiolydienne ; ils avaient conservé un caractère encore plus farouche. Les fanatiques de Mâ, vêtus de robes noires, armés de glaives et de bipennes, accompagnés d'une musique sauvage de tambours et de trompettes, tournoient sur eux-mêmes, se tailladent bras et jambes, répandent leur sang en guise d'offrande, en arrosent leur idole, en aspergent les fidèles, le boivent et, saisis du délire prophétique, vaticinent.* Ces Bellonaires étaient les dignes compagnons des Galles. Ils entrèrent désormais dans le cortège de la Mère des Dieux, qui prit « à sa suite* » et sous sa protection la nouvelle venue. Voici enfin que, du haut des rostres, les meilleurs citoyens glorifient la ville sacrée de Pessinonte, « séjour et domicile de la Mère des Dieux, » le temple de Pessinonte, « universellement vénéré, » la religion de Pessinonte, « sainte et antique entre toutes,* » dont un Clodius avait osé trafiquer !

*) Obsequens, 104 : « Servus Servilii Caepionis Matri Idaeae se praecidit et trans mare exportatus ne unquam Romae reverteretur. » Il s'agit sans doute du Servilius qui fut consul en 648/106 et condamné en 649/105, puis en 651/103, après l'allaire de l'or de Toulouse, ou de son fils, qui lut questeur urbain en 651 ou 654.

*) Cf. Liv. 39, 16.

*) Val. Max. 7, 7, 6. Il s'agit d'un cas de « bonorum possessio secundum tabulas. » Préteur : Cn. Aufidius Orestes.

*) Cf. les \emph{cultores Matris Deum} au 1er siècle de notre ère : Sueton., \emph{Oth.} 8.

*) Sylla, propréteur en Cilicie, avait franchi le Taurus en 662/92 ; cf. les officiers de Lucullus qui, pendant ces campagnes d'Orient, se font initier aux mystères de Samothrace : Plut., Lucullus, 19.

*) V. les textes dans Wissowa, \emph{Rel. u. Kult. d. R.}, pp. 289-292.

*) Bellone est qualifiée de \emph{dea pedisequa}, \emph{CIL.} 6, 3674 a ; cf. Wissowa, \emph{op. l.}, p. 291.

*) Cic., \emph{De harusp. resp.} 13, 28 et 29. Cf. \emph{Pro Sextio} 26: « Fanum sanctissimarum atque antiquissimarum religionum. »

Pendant ce temps, la déesse ne cessait de manifester à Rome même sa toute-puissance. Dans cette période de crise, elle semblait s'intéresser activement à la vie publique et multipliait les miracles. Elle protestait contre les fauteurs d'anarchie. Après le scandale provoqué par Clodius aux Mégalésies (698/56), elle fit entendre ses protestations irritées.* Après la mort de César, elle ne veut plus avoir devant les yeux ces pays qui vont servir d'asile aux meurtriers. Une statue de l'Idéenne, dans son temple du Palatin, regardait l'Orient ; on la trouva, un matin, tournée vers l'Occident.* Durant la guerre d'Octave contre Sextus Pompée, il y eut abondance de prodiges.* Sur le Palatin, la cabane de Romulus brûla. Des « possédés de la Mère des Dieux* » annoncèrent que la Dame était mécontente de Rome et exigeait des expiations. On consulta les Livres Sibyllins et, sur leur avis, on alla purifier l'idole dans les eaux de la mer, à Ostie. La Dame s'avança d'elle-même au milieu des vagues et demeura longtemps sans vouloir revenir au rivage. Rome fut saisie de frayeur. Mais des palmes, symbole de paix, poussèrent subitement autour du temple. Rome reprit courage. Ce fut le dernier miracle de l'Idéenne avant l'Empire.

*) Cf. p. 83.

*) Dio Cass. 46, 33. Ce prodige est rapporté parmi ceux qui précédèrent la bataille de Modène. C'était, ce semble, un genre de miracle assez commun. Cf. Minerve, à Troie, détournant les yeux pour ne pas être témoin du viol de Cassandre, et un miracle analogue en Grande Grèce, dans Strab. 6, 1, 14.

*) Dio Cass. 48, 43. C'était en l'année 716/38.

*) Κάτοχοί τέ τινες ἐκ  τῆς τῶν Θεῶν μητρὸς γενόμενοι.

\subsection{4.}

Parmi les esprits cultivés, quelques-uns, par dilettantisme d'artistes, s'intéressaient aux Galles et à leurs orgiasmes. Le plus grand nombre, par patriotisme de vrais Romains, s'indignaient du progrès des superstitions orientales, qui faisaient oublier la religion des ancêtres. D'autres, libres penseurs, s'insurgeaient au nom de la raison.

Les poètes de la jeune école, grands artisans de rythmes, grands admirateurs et imitateurs de la poésie alexandrine, avaient été séduits par la musique étrange et tourmentée des galliambes. Le vers galliambique ou métroaque,* à l'aspect efféminé, à l'allure sautillante, était une création des Alexandrins,* qui l'avaient exclusivement consacré à la Mère des Dieux. Callimaque* avait donné le modèle du genre dans un poème qui fut célèbre : « ô Galles, prêtresses* de la Mère des montagnes, qui dans vos courses vous réjouissez d'agiter le thyrse et de faire retentir vos crotales d'airain.... » A Rome, le galliambe devint vite à la mode dans la littérature précieuse, parce qu'il était un tour d'adresse, un de ces « riens difficiles » et laborieux auxquels se plaisent les poètes qui ont plus d'ingéniosité que de génie. Caecilius de Côme,* Catulle,* Varron,* plus tard Mécène,* s'y essaient avec succès. Vers la fin du Ier siècle de l'Empire,* « le bel Attis dicte » encore aux contemporains de Martial, qui s'en moque, « des galliambes débiles et mous. » Nous avons conservé l'Attis de Catulle.* C'est un bel adolescent de race grecque. Mais il doit être né dans quelque île de l'archipel.* Il a le type délicat, les formes indécises, « les mains de neige et les lèvres de rose* » d'un jeune Levantin. L'éphèbe est possédé de la fureur sacrée des orgiastes. Son enthousiasme l'a entraîné par-delà les flots, vers la sainte Phrygie, dans le mystère religieux des bois de pins. Il s'y mutile en l'honneur de Cybèle. Pour la Dame il prend le tympanon et danse. A l'appel de ses ululements rythmés a répondu le chœur des Galles ; et tous, dans une course folle que stimule la musique des flûtes, des cymbales et des timbales, s'élancent vers les neiges de l'Ida. Mais voici la nuit. Les sommets sont loin. La troupe épuisée de fatigue s'endort. Au réveil, le nouvel Attis, à qui le repos a rendu la raison, comprend l'irréparable, redescend mélancolique vers la grève et pleure, en regardant du côté de la mer et de sa patrie. Il pleure sa jeunesse heureuse et fêtée, la palestre dont il était l'ornement, ses amis qui chaque jour suspendaient à sa porte des guirlandes, toutes les douceurs du pays hellénique, et la barbarie de la vie qui pour lui commence au milieu des bêtes de la montagne. « Et maintenant je ne suis qu'une femme, une prêtresse des dieux, la servante de Cybèle, un débris de moi-même, un eunuque.* » Il voudrait avoir la force de fuir. Mais la terrible déesse des hauts lieux a entendu sa plainte et surpris ses regrets. Jalouse, elle envoie, pour l'épouvanter, un des lions de son char. Le lion rugit ; et l'Attis éperdu rentre dans la forêt sacrée, qui ne le rendra plus à la Grèce.

*) Hephaestion, 12, p. 39 (Westphal).

*) C'est à proprement parler une transformation de l'ancien ionique mineur tétramètre catalectique, jadis employé par les auteurs dramatiques d'Athènes. En devenant le galliambe, le vers classique subit une modification importante, qui en change le caractère. « A la place du petit ionien qui, à l'origine, entrait seul dans la composition de chaque pied, on admit les substitutions les plus diverses, et en outre on toléra l'anaclase, c'est-à-dire qu'une syllabe longue put être, par convention, comptée pour deux brèves, qu'on partageait entre la fin d'un pied et le commencement du suivant. » Lafaye, \emph{Catulle et ses modèles}, p. 83 ; cf. Thompson, Dunn et Hardie, \emph{On the gall. metre}, dans \emph{Classical Rev.}, 1893, pp. 145 et 280 (Thomas, dans \emph{Rev. critique}, 1893, 1, p. 284).

*) Hephaestion, qui cite les deux premiers vers du célèbre poème, n'en dit pas l'auteur ; mais l'attribution à Callimaque ne paraît pas douteuse : Wilamowitz, \emph{Die Galliamben des Kallimachos und Catullus}, dans \emph{Hermes}, 14, 1879, p. 194 ; Susemihl, 1, p. 364-365 ; Lafaye, \emph{op. l.}, p. 84.

*) Γαλλαὶ Μητρὸς Ὀρείης φιλόθυρσοι δρομάδες, \emph{etc.} Catulle emploie le même féminin, avec la même intention méprisante, pour désigner les Galles eunuques : \emph{Gallae}.

*) Catull., 35, 18.

*) Catull., 43 : \emph{Attis}.

*) \emph{Cycnus, Marcipor, Eumenides} ; cf. \emph{Saturarum Menipp. reliq.}, éd. Riese, pp. 114, 131-133, 164 ; éd. Buecheler, p. 173 ss.

*) Diomedes dans les \emph{Grammat. latini}, 1, 514 ; Caes. Bass., \emph{ibid.} 6, 262 ; cf. L. Mueller, \emph{De re metrica rom.}, p. 108 : Baehrens, \emph{Poet. Lat. Min.} 6, p. 339.

*) Martial. 2, 86. C'est lui qui traite les galliambes de « \emph{difficiles nugae} ».

*) Voir Wilamowitz, \emph{op. l.}, et Grant Allen, édition de l'Attis avec commentaire et excursus, Londres, 1892 ; Lafaye, \emph{op. l.}, p. 82-90.

*) Ce n'est point un Grec de la côte levantine. Il a traversé la mer : « super alta vectus Attis celeri rate maria » (vers 1).

*) « Niveis manibus » (8), « roseis labellis » (74).

*) Vers 68-69.

Il est malaisé de distinguer ici la part personnelle et les éléments d'emprunt. Le thème n'appartient pas plus à l'auteur que ce type grécisé de l'Attis. La plainte de l'éphèbe est d'une telle couleur locale qu'à travers le texte latin on entrevoit l'original grec. Quant à la rencontre du Galle et du lion, elle avait déjà fourni aux petits poètes de l'anthologie la matière de nombreuses épigrammes. On a cru reconnaître dans les derniers vers de Catulle un sincère accent d'effroi :

déesse grande, déesse Cybèle, déesse dame du Dindyme,  
écarte tes fureurs, toutes tes fureurs, de ma maison, ô Maîtresse* !

*) Vers 90-91.

Mais cette supplique est-elle de Catulle ou de Callimaque ? Nous dit-elle l'épouvante d'un Romain ou celle d'un Grec ? Il ne faudrait pas être dupe d'un pastiche littéraire. Aussi bien, si Catulle est vraiment effrayé de la frénésie des Galles, de la sauvagerie de leurs rites orgiastiques, du danger de leur fanatisme contagieux, n'oublions pas qu'il a pu les voir ailleurs qu'au Palatin et dans les rues de Rome. Il connaît l'Orient. Il a été de la suite d'un propréteur de Bithynie. Il a visité les rivages où aborde son Attis et que dominent, à l'horizon, les cimes de l'Ida boisé. Quoi qu'il en soit, il n'a qu'un but, qui est de faire œuvre d'artiste. Mais par la peinture raffinée de cet exotisme à demi barbare, par les éclatantes descriptions de cette nature solitaire et sacrée, avec l'ombre de ses forêts de pins, la neige de ses montagnes, les antres de ses fauves et la présence invisible de sa Dame, par cette glorification de la puissance mystérieuse et fatale de la déesse aux lions, ne fournissait-il pas aux imaginations malades et déjà troublées un poison nouveau ? Il prête au culte phrygien le secours et la séduction de son talent. A la légende d'Attis il ajoute un élément de sentimentalisme qui en atténue le caractère farouche et qui pouvait avoir prise sur les âmes tendres. Il est vrai que ses poésies s'adressaient à une élite si restreinte !

Nous retrouvons Galles et galliambes dans les fragments des Ménippées de Varron. Dans le « Cygne, » c'est sans doute un Attis, comme chez Catulle, qui gagne à pas précipités les hauteurs où s'élève le temple. Dans les « Euménides, » l'inspiration paraît être plus personnelle. C'est à Rome que se passe l'action, puisque l'édile y joue un rôle. Un sage racontait, avec une verve à la fois amusée et méprisante, une cérémonie phrygienne, entrevue par hasard, un jour qu'il longeait le temple de la Mère des Dieux.* Au bruit des cymbales, curieux, il s'était approché. Une bande de Galles était là, qui chantaient d'une voix puérile, frappaient sur des tympanons et secouaient la tête avec frénésie, excités par la double flûte.* En galliambes, naturellement, il pastiche leur cantique à la gloire d'Attis.* Il décrit l'envolement de leurs longues chevelures frisées.* Il insiste avec ironie sur leur air de tendres adolescents, leur grâce efféminée, leurs robes et leur chasteté.* Sa raillerie n'oublie point l'Archigalle qui, revêtu d'un surtout pourpre, couronné d'or et de pierreries, rayonne « comme une aurore.* »

*) Ed. Buecheler, 149 (33) : « Iens domum praeter Matris Deum aedem exaudio cymbalorum sonitum. » Essais de reconstitution du sujet : Vahlen, \emph{Conjectanea}, pp. 171-177 ; Ribbeck, \emph{Hist. de la poésie lat.}, p. 311 ss. ; Buecheler, \emph{Mus. phil.}, 14, pp. 419-452.

*) Fr. 150 (34), 131 (15) et 132 (16).

*) Fr. 132 (16).

*) \emph{Ibid.} : « teretem comam volantem iactant tibi famuli. »

*) Fr. 119 (3) et 120 (4) : « nam quae venustas hic adest \emph{gallantibus}, etc. »

*) Fr. 121 (5) : « aurorat ostrinum hic indutus supparum, coronam ex auro et gemmis fulgentem gerit luce locum afficiens. »

Ce qui attriste et indigne Varron, c'est que ce spectacle en impose à la multitude superstitieuse ; c'est que ces clergés exotiques ravagent les âmes. Des hauteurs de sa philosophie, il lui semble voir des Furies déchaînées contre le peuple romain. Fervent défenseur des traditions qui ont fait la grandeur de Rome, il est encore plus patriote que philosophe, plus Romain que disciple des Grecs, dans son aversion pour Attis. Cet Attis, comme Serapis,* est un étranger, un barbare, c'est-à-dire un ennemi ; et la popularité de ses prêtres devient un péril d'état. En revanche, la Grande Mère est une divinité nationale. Aussi Varron lui donne-t-il place dans ses « Antiquités divines, » essai de réaction contre la décadence de la religion officielle. S'il n'y citait même pas le nom d'Attis,* il y exposait en détail sa conception théologique de la Mère des Dieux. On connaît les idées fondamentales de la théologie varronienne, toute pénétrée de stoïcisme. Le premier principe, c'est l'âme du monde ; le monde animé est lui-même dieu. Ceux que nous appelons les dieux sont les éléments du monde ; et les éléments primordiaux du monde sont le ciel et la terre, Caelus et Tellus. A Tellus se rattachent les divinités féminines, qui en constituent les manifestations particulières. La terre en effet ne peut être conçue sous un aspect unique. La Grande Mère est l'un des aspects de la Terre. Par conséquent elle est, dans son essence, la même divinité que Gérés, Ops, Junon, Vesta et Proserpine.* C'est ainsi que s'opéra théologiquement la romanisation de la déesse phrygienne. Varron, qui emprunte aux Grecs leur méthode explicative, se laisse entraîner ici à des interprétations bizarres. Mais encore celles-ci nous intéressent-elles par leur tendance générale à nous présenter chaque déesse dans sa fonction propre. Le vocable de Grande Mère signifie que « la Terre est nourricière des hommes. » Il est, par suite, le symbole de la fécondité du sol. La Mère des Dieux est donc bien restée, pour les contemporains de Varron, la protectrice des moissons. Les attributs, l'iconographie, les rites, tout précise ce caractère de divinité tellurique et agraire. Les uns témoignent que la Grande Mère est la Terre. Le tympanon, par exemple, figure l'orbe terrestre. La couronne murale est l'emblème des villes que porte la Terre. La Dame est assise, parce que la Terre demeure immobile pendant que tout se meut autour d'elle. D'autres particularités spécifient son rôle agricole. Elle s'entoure de lions apprivoisés pour enseigner aux hommes que la culture dompte le sol le plus rebelle, de même que le dressage dompte les fauves. Elle aime le son des cymbales, parce qu'il lui rappelle le bruit des instruments aratoires. Les cymbales sont d'airain pour perpétuer le souvenir de la primitive agriculture, celle de l'âge du bronze, qui précéda l'âge du fer.* Les Galles s'agitent autour de Cybèle parce que les cultivateurs n'ont jamais de repos. Ainsi Varron rattache les rites orgiastiques des Galles à cette fonction capitale de la divinité. Comment s'étonner, alors, que plus d'un paysan romain ait recours à ces prêtres et à ces rites pour obtenir les grâces de \emph{Mater Deum Agraria} et, en même temps, de ce dieu que les Galles ne séparent point d'elle, « l'épi vert fauché » de leurs hymnes mystiques ?

*) Le nom de Serapis revient par deux fois dans les fragments de l'\emph{Euménides}.

*) Augustin., \emph{Civ. Dei} 7, 25 : « Et Attis ille non est commemoratus » ; cf. 26.

*) Cf. \emph{ibid.}, 7, 16-28, surtout 24. De même Nigidius, contemporain de Varron, met à l'origine du monde Saturne et la Grande Mère = Caelus et Tellus ; Arnob. 3. 32.

*) Augustin., \emph{l. c.}, 24 : « aere, quod eam (la terre) antiqui colebant aere antequam ferrum esset inventum. »

On retrouve dans Lucrèce* la même identification de la Grande Mère et de la Terre. Il en tire d'autres conséquences. Il demande à la théologie des armes contre la religion. Mère des Dieux, mère et souveraine des fauves, mère et nourrice des hommes,* autant de poétiques allégories, qu'il faut laisser aux poètes, à ceux qui disent Neptune pour désigner la mer et Bacchus pour désigner le vin. Car la terre n'est pas une divinité ; c'est une combinaison d'atomes, « un amas de matière insensible. » Lucrèce sait comment les théologiens expliquent les attributs de la Mère et les rites de son culte. Il connaît même des interprétations morales dont ne parle point Varron et dont certaines avaient cours dans les mystères.* Mais lions, chars, couronne de tours, processions de l'idole, danses sacrées, tonnerre des tympanons et des cymbales, son menaçant de la corne, musique énervante de la flûte, tout n'a d'autre raison d'être, à ses yeux, que l'exploitation des foules par un clergé fanatique. « Pièces de bronze, pièces d'argent jonchent les rues que parcourt la déesse ; copieuse est l'aumône dont elle s'enrichit. On jette aussi des roses ; et c'est comme une neige de fleurs qui enveloppe la Grande Mère et son escorte.* » Il y a, dans les pompes de la religion phrygienne, une part de poésie gracieuse et toute féminine, qui contraste avec la sauvagerie des scènes orgiastiques et qui n'est pas l'un des moindres attraits du culte. Mais Lucrèce ne veut pas nous arrêter sur cette aimable et reposante vision. Voici, derrière la statue processionnelle, les Corybantes de Cybèle, qui sont des compagnons danseurs. Coiffés d'un casque à longue aigrette, armés de glaives et de boucliers, ils exécutent des pas guerriers, agitent violemment la tête, se frappent, se blessent et s'excitent à la vue du sang.* Assistait-on à ce spectacle dans les rues de Rome, vers la fin de la République ? Lucrèce, à vrai dire, semble plutôt n'en parler que d'après ses lectures. Il ignore le nom des Corybantes ; il les appelle Curètes phrygiens, « comme disent les Grecs.* » Il a lu les philosophes de la Grèce qui ont disserté sur la Mère des Dieux et les poètes qui l'ont chantée.* Par eux il a connu la Meter Oreia, la Meter Sôteira, qu'il ne nomme point, mais dont le souvenir transparaît à travers ses formules et ses périphrases.* Il évite d'autre part toute allusion à Rome et au temps présent,* comme s'il voulait conserver à son œuvre un caractère universel. Mais aurait-il, à propos d'atomes, tellement insisté sur le culte phrygien, s'il n'avait vu se propager autour de lui le fanatisme asiatique ? C'est de tous les cultes de la Terre divinisée le seul qu'il décrive, parce qu'il y trouve la matière d'un brillant tableau, parce qu'il admire en poète la beauté de fictions dont il n'est pas dupe. C'est aussi le seul qu'il combatte, parce qu'il en a constaté les néfastes effets sur les âmes crédules.

*) Lucret. 2, 598-658.

*) Lucrèce commence par énumérer les titres de la déesse : « Magna deum mater, materque ferarum (= πότνια θηρῶν) | et nostri genitrix... corporis. »

*) Les Galles sont eunuques parce que l'homme ingrat envers ses parents n'est pas digne de revivre dans une postérité (614-617) : allusion aux amours d'Attis et de la nymphe et au châtiment d'Attis, puni de son ingratitude envers Cybèle ; cf. le parti que Julien tire de ce mythe dans son Traité sur la Mère des Dieux.

*) Vers 626-628.

*) « Sanguine laeti » (631) ; cf. les libations de sang des Galles.

*) Strab. 10, 3, 12 : « C'est encore par l'appellation de Curètes que les Grecs désignent les ministres de la M. d. D. ; ils se gardent bien de les confondre avec les Curètes de Zeus et ne les considèrent à vrai dire que comme des desservants subalternes (diacres) ; ils leur donnent quelquefois aussi le nom de Corybantes. » Cf. 3, 21 : « D'après Demetrios de Skepsis (début du 2e s. avant notre ère), il est probable que les noms de Curètes et Corybantes sont équivalents ; ils s'appliquent tous deux aux jeunes garçons chargés, dans les fêtes de la M. d. D., d'exécuter la danse des armes. »

*) Il le dit lui-même : « Hanc veteres Graium docti cecinere poetae » (600).

*) C'est la Meter Oreia qui est la souveraine des fauves (v. 598). C'est à la Meter Sôteira qu'il semble faire allusion au vers 625 : « munificat tacita mortales muta salute. » Je signale, sans l'adopter, une autre interprétation de ce vers. D'après Duvau. dans \emph{Rev. de Philologie}, 16, 1892, p. 109 s, \emph{salus} aurait le sens précis de salutation. La déesse salue du geste, donne une sorte de bénédiction Les prêtres porteraient une idole aux bras articulés.

*) Il nomme cependant l'Idéenne (611), mais non point pour rappeler que nous sommes à Rome et qu'il s'adresse à des Romains. Il déclare que ce vocable est très ancien dans le culte et se retrouve chez divers peuples.

Quae bene et eximie quamvis disposta ferantur,  
Longe sunt tamen a vera ratione repuisa.*

*) Vers 643-644.

Il se révolte au nom de la vérité et de la saine raison, dans le même temps que Varron protestait au nom de la tradition nationale. Cette coïncidence n'est point fortuite. Ils vivaient à une époque où Rome se laissait aller à la séduction des religions orientales.
\clearpage
\section{Chapitre 3}
\begin{center}
Le Culte Public sous L'Empire. Fêtes Phrygiennes du Printemps à Rome.
\end{center}
\paragraph{}
\emph{Sacra Matris Deum Romani phrygio more coluerunt}. Servius ad \emph{Aen.}, 12, 836.

1. Cybèle protectrice du nouveau régime. Grande vénération d'Auguste pour l'Idéenne. Témoignages littéraires: Virgile, rôle de la Mère des Dieux dans l'Énéide ; Ovide. Témoignage des monuments figurés : camées de Vienne, base de Sorrente, reliefs de l'Ara Pacis. Caractère national du culte officiel sous Auguste, mais augmentation du clergé phrygien. Recrutement de ce clergé dans la maison du prince. Nombreux esclaves et affranchis d'origine phrygienne dans la maison des premiers Césars. Ce sont eux qui, sous le principat de Claude, imposent à Rome le culte d'Attis. --- 2. Introduction des fêtes phrygiennes dans le calendrier romain. Le 15 mars, « Canna Intrat. » Neuvaine de pénitence. Le 22, « Arbor Intrat. » Le 24, « Sanguis. » Le 25, « Hilaria. » Le 27, « Lavatio. » Huit jours après, Mégalésies romaines d'avril ; les Jeux. --- 3. Rôle des 15virs dans l'organisation du nouveau culte et dans la constitution du clergé. Situation nouvelle de la confrérie des dendrophores. Rome remplace Pessinonte comme centre de la religion phrygienne. --- 4. La réforme de Claude et les dévots d'Attis. Elle ne bouleverse pas les habitudes religieuses de Rome. Le culte phrygien reste un culte d'affranchis et de femmes. Rôle des femmes dans l'expansion du culte. Dévotion des impératrices à Cybèle. Le culte phrygien et les successeurs de Claude jusqu'aux Antonins.

\subsection{1.}

Sur le bouclier fabuleux d'Énée, Virgile nous montre Octave escorté, le jour d'Actium, par les Pénates et les Grands Dieux de Rome. Magna Mater figurait au premier rang du cortège. Elle n'était point, il est vrai, du groupe des vingt grands dieux de la République selon la théologie varronienne. Mais elle était une des divinités protectrices de la gens Julia. Elle avait, à ce titre, favorisé l'établissement du nouveau régime. L'avènement d'Auguste lui conférait une place éminente dans le panthéon romain. Elle devenait l'une des divinités suprêmes de la Rome impériale. Nous possédons une \emph{ara augusta} qui fut consacrée l'an 1 de notre ère, «pour la prospérité de l'empereur, de son empire, du sénat et du peuple romain, et des nations* » ; la Grande Mère y est invoquée aussitôt après la triade Capitoline, le couple d'Apollon et Diane, spécialement cher à Auguste, et la Fortune.

*) \emph{Not. Scavi}, 1890, p. 388 ; Cagnat, \emph{Ann. épiqr.}, 1891, n° 109 ; cf. \emph{Arch. ep. Mitt.}, 1892, p. 77.

La maison des Julii vénérait d'un culte tout particulier la Grande Mère, parce que celle-ci était la dame de l'Ida troyen et qu'ils prétendaient descendre du troyen Énée. Un rôle s'imposait donc h la déesse dans la légende du héros, dont Auguste devient le successeur politique et religieux. A vrai dire, le rôle était préparé par de lointaines traditions. On le trouvait indiqué déjà dans les épopées grecques du cycle de Troie.* Virgile, sous l'inspiration du prince, se chargea de le développer et de le compléter.

*) Gruppe, \emph{Gr. Mythol. u. Religions Gesch.}, p. 311, n. 2, à propos de l'\emph{Iliou Persis} ; Farnell, \emph{Cults of the greek States}, 2, p. 641 ; cf. supra, p. 43.

La déesse apparaît souvent dans l'Énéide. Elle s'y révèle très puissante ; car elle est vraiment la mère de tous les dieux, hère de sa postérité d'Olympiens, qui l'entourent d'un filial respect 2. Mais Virgile a soin de lui conserver le caractère pittoresque de sa physionomie orientale. Elle reste la Mère Idéenne, la Mère Phrygienne, la Bérécynthienne,* celle qui habite les hauteurs du Dindyme et le mont Cybèle, la Cybébè, la dame des lions* ; nous la retrouvons avec sa couronne murale, son char « qui la promène par les villes de Phrygie, » ses bois sacrés, les danses de ses prêtres aux longues robes jaunes et rouges, le bruyant orchestre des flûtes de buis, des cymbales d'airain et des tambourins. Elle était une des principales divinités du peuple troyen, qu'elle avait suivi de Crète en Asie Mineure.* Le premier qui célébra ses orgies ne fut-il pas le troyen Idaios, fils de Dardanos* ? Elle avait donc son clergé, dans la Troie que chante Virgile. Le dernier de ses prêtres est devenu le compagnon d'Énée.* Nous le voyons caracoler comme un guerrier barbare, vêtu d'une chlamyde jaune safran, d'une tunique pourpre et or et de larges pantalons asiatiques ; il est casqué d'or et tient un arc de Lycie. Virgile n'a pas osé présenter un véritable Galle de Cybèle parmi les conquérants du Latium. Chloreus, en quittant Ilion, a cessé d'être prêtre (\emph{sacerdos olim}). D'autre part sa consécration à la déesse (\emph{sacer Cybelae}) le prive à tout jamais de postérité ; les Romains peuvent être sûrs qu'il n'est pas de leurs ancêtres. Virgile était plus à son aise, lorsqu'il s'agissait simplement d'intéresser la Mère des Dieux aux aventures d'Énée. C'est elle qui sauve Créuse des mains des Grecs et l'arrache à la servitude honteuse.* Elle défend son peuple dans les conseils de l'Olympe. Lors-qu'Énée veut construire une flotte pour accomplir ses destinées, après le désastre de Troie, elle lui offre les arbres de sa forêt sacrée, les pins, les sapins et les érables de l'Ida. Plus tard, lorsque Turnus incendie les vaisseaux, elle accourt bien vite auprès de son fils Jupiter, elle intercède en leur faveur, elle veut un miracle ; elle obtient enfin de les métamorphoser en nymphes marines. « Et du côté de l'Orient on vit une lumière surnaturelle et une nuée qui s'avança dans le ciel » ; et l'on entendit dans cette nuée les chœurs de l'Ida et la voix de l'invisible divinité qui disait aux navires : « Allez, je vous délie ; allez, déesses de la mer.* » Celles-ci deviendront les messagères de Cybèle auprès du héros. Énée apprendra par elles que les Rutules assiègent son camp ; et c'est à la Mère Idéenne qu'Énée, comme les généraux de la République, demande la victoire.* Aussi bien la birème sur laquelle il a remonté le Tibre, pour se rendre auprès d'Evandre, est-elle consacrée à la Mère des Dieux. Elle* porte à la poupe, sous l'éperon, les deux lions, qui semblent attelés au vaisseau comme au char de Cybèle ; et au-dessus se dresse l'image du mont Ida, « dont l'aspect mettait de la joie au cœur des exilés. » La déesse connaissait donc les rives du Tibre bien avant le jour où, portée au Palatin par les matrones romaines, elle suivit la voie d'Ostie, qui longe le fleuve. C'était Énée qui le premier l'avait invoquée dans le Latium. Virgile voulut même qu'elle fût l'une des premières divinités dont Énée prononçât le nom sur la terre promise. A dessein il réunit dans les actions de grâces de son héros les dieux topiques, Jupiter, Tellus et la Grande Mère Phrygienne.* C'est que Virgile, en s'ingéniant à rattacher le culte de la déesse aux traditions troyennes, n'accomplissait pas seulement œuvre de poète. Il travaillait pour la gloire d'Auguste, en qui la déesse aimait et favorisait la descendance d'Énée. Il fixait une légende de la Rome nouvelle. Désormais on ne pouvait admettre que l'Idéenne eût jamais été pour Rome une étrangère. Ovide, qui se fait l'écho des mêmes croyances, essaie de concilier à sa façon la légende et l'histoire.* Sur les navires « auxquels étaient confiés les cultes de la patrie, » Cybèle aurait voulu suivre Énée. Mais les Destins, plus forts que les Dieux, ne réclamaient pas encore sa présence dans la nouvelle Troie. Elle comprit et se résigna. De loin elle chérissait Rome en souvenir d'Ilion. Pour y élire domicile, elle attendit que le peuple romain songeât à vaincre la Grèce, dont la défaite fut la revanche de Troie. Mais, dès lors, « elle oublia près de l'Almo les fleuves Idéens.* »

*) \emph{Aen.} 6, 786 ; cf. 2, 788 ; 9, 84.

*) \emph{Aen.} 6, 780 ; 7, 139 ; 9, 82 ; 10, 252 ; cf. Carter, \emph{Epitheta deorum quae apud poet. lat. leguntur}, 1902, (suppl. à Roscher, \emph{Myth. Lex.}), s. v. Cybèle.

*) \emph{Aen.} 3, 111 (mater cultrix Cybeli) ; 10, 220 (alma Cybebe) : 10, 252 (cui Dindyma cordi) : cf. 9, 616 : « ite per alta Dindyma » ; 3, 113.

*) C'est ce que dit Anchise dans Virgile, qui connaît cette tradition : 3, 104-105. Aussi Énée, trompé par la fausse interprétation d'un oracle obscur, commence-t-il par fonder un établissement en Crète.

*) Or Dardanos lui-même est originaire du Latium : 3, 167. Ainsi se ferme le circuit des légendes. Mais Virgile ne dit pas que Dardanos ait amené d'Italie en Crète le culte de la Mère des Dieux.

*) \emph{Aen.} 11, 768-778. Cf. le type du Troyen sur les vases peints.

*) \emph{Aen.} 2, 78S.

*) Voir tout le passage 9, 77-122.

*) Il l'invoque comme Idéenne et Dindymène, 10, 252-255 : « tu mihi nunc pugnae princeps, tu rite propinques | augurium, Phrygibusque adsis pede, diva, secundo. »

*) \emph{Aen.} 10, 156-158.

*) \emph{Ibid.}, 7, 139 : « Idaeumque Iovem, Phrygiamque ex ordine Malrem. »

*) \emph{Fasl.} 4, 250-255, et 272 (v. supra, p. 43) ; cf. Tertull., \emph{Apol.} 25 et Prudent., \emph{Contra Symmachum}, 1, 628 : « Iliacae Malris Megalesia. »

*) Stat., \emph{Silv.} 5, 1, 222.

Le culte de la famille impériale pour l'Idéenne et le caractère politique de cette dévotion nous sont attestés aussi par les monuments figurés. Sur le grand camée de Vienne,* Cybèle occupe une place d'honneur ; elle pose une couronne de chêne sur la tête d'Auguste héroïsé, qui trône aux côtés de la déesse Roma. Un autre camée de la même collection représente Livie avec les attributs de Cybèle.* L'impératrice porte le diadème, la couronne murale et le voile. Comme la déesse, elle est assise. Elle appuie le bras gauche sur le tympanon, que décore l'image d'un lion couché ; et dans la main droite elle tient, avec un petit buste de l'empereur,* un bouquet d'épis et de pavots. De même, sur un troisième camée dé Vienne, où l'on voit les bustes des deux fils et des deux belles-filles de Drusus Germanicus, fils adoptif d'Auguste, Agrippine la jeune est couronnée de murs, comme Cybèle, et d'épis, comme Cérès ou Mater Deum Agraria.* Sur la base dite de Sorrente, qui supportait la statue de l'empereur ou d'un membre de sa famille, sont groupés les dieux de la maison Julienne.* Voici Mars Ultor et Vénus Genitrix, qui entourent le génie du prince, devant la « domus Augustana. » Voici Apollon Palatin, entre Diane et Latone. A leurs pieds est assise la Sibylle ; caries Livres Sibyllins sont maintenant sous le piédestal de la statue d'Apollon. Voici Vesta, qui est assise dans son sanctuaire, auprès du Palladium, et que viennent adorer des jeunes filles. Voici enfin, sur un trône magnifique, la Mère des Dieux, entourée de ses deux lions accroupis. Elle est accompagnée d'une matrone, peut-être Claudia ; derrière elle, un corybante casqué lève son bouclier et brandit l'épée. Enfin, le temple de l'Idéenne figure sur les reliefs de l'\emph{Ara Pacis},* que fît ériger le Sénat, interprète de la reconnaissance nationale envers le prince, et qui fut dédiée en 745/9. La frise de l'enceinte sacrée représentait les stations d'une majestueuse procession, entre la demeure d'Auguste, où se formait le cortège, et le Champ de Mars, où l'autel se dressait en bordure de la voie Flaminienne.* Ces stations étaient réservées aux sanctuaires des divinités chères à la famille impériale. Avant de quitter le Palatin, on contournait celui de Mater Magna, de même qu'en traversant le Forum d'Auguste on s'arrêtait devant Mars Ultor. Le temple de la Dame est facile à reconnaître, grâce aux attributs métroaques de son fronton. L'édifice que nous voyons ici date de la restauration qui suivit l'incendie de 634/111. Il fut brûlé au temps même d'Auguste, en 756, l'an 3 de notre ère. Le prince le fit reconstruire en lui conservant, à dessein, son caractère archaïque.

*) Sardoine à deux couches, dite \emph{Gemma augustea}, autrefois au trésor de Saint-Sernin de Toulouse (citée dans un inventaire de 1246, achetée vers la fin du 16e siècle par l'empereur Rodolphe 2). Voir la description, la reproduction et la bibliographie dans Bernouilli, \emph{Roem. Ikonogr.}, 2, 1, p. 262 ss et pl. 29 ; Furtwaengler, \emph{Ant. Gemmen}, 2, p. 257 et pl. 56. Ajouter : F. de Mély, \emph{Le « camayeul » de S.-Sernin et le grand camée de Vienne}, dans \emph{Mém. Soc. archéol. du Midi}, 15, 1894-1896, p. 67-98 (avec belle héliogravure) ; Gardthausen, \emph{Augustus und s. Zeit}, 1904, 1, p. 1231.

*) Bernouilli, \emph{l. c.}, p. 94, n° 3 B, avec bibliogr. ; pl. 27, 2.

*) Livie considérée comme prêtresse d'Auguste ? Cf. Beurlier, \emph{Culte rendu aux emp. rom.}, p. 29.

*) Bernouilli, \emph{l. c.}, p. 370 et pl. 31.

*) Base, au musée municipal de Sorrente : long, de la face, 1 m. 36 ; des côtés, 0 m. 82 ; haut., 1 m. 18. --- Gerhard, \emph{Ant. Bildw.}, pl. 22 ; Mueller-Wieseler, \emph{Denkm.}, 2, 63, 810 ; Heydemann dans \emph{Roem. Mitt.}, 1889, p. 310 ss et pl. 10 ; Samter, \emph{ibid.}, 1894, p. 130 ; Huelsen, \emph{ibid.}, p. 238 ss ; Amelung, \emph{ibid.}, 1900, p. 198.

*) Le relief où le temple est figuré se trouve à la Villa Médicis, sur la façade intérieure. Descr. et bibliogr. : Matz et Duhn, \emph{Ant. Bildw.} 3, p. 31, n° 3512 ; Petersen, \emph{Ara Pacis}, pp. 65-78 ; Courbaud, \emph{Bas-reliefs rom.}, p. 77. Reprod. : Brunn, qui l'attribue au 2e s., dans \emph{Annali}, 1852, pl. S : Petersen, \emph{op. l.}, p. 63 et pl. 3.

*) Hypothèse de Petersen ; v. le plan de reconstitution des fragments sur la frise, p. 36. Le temple de Mater Magna aurait figuré sur la face postérieure du monument. Dissel, \emph{Der Opfzug der Ara Pacis Aug.}, Hamburg, 1907, croit que le motif principal serait une procession vers la chapelle des Lares d'Auguste, \emph{in summa sacra via}.

Auguste conserva de même au culte de la déesse son caractère purement national. Restaurateur de la religion des ancêtres, il remit en vigueur les anciens règlements de police à l'égard des rites étrangers. C'est précisément la Rome d'Auguste que Denys d'Halicarnasse admire pour son nationalisme religieux, pour sa piété calme et grave, qui contraste, observe-t-il, avec la corruption des mœurs publiques.* Denys manifeste sa franche surprise de n'y point voir pulluler, comme dans les villes grecques, les prophètes, les fanatiques, les prêtres mendiants, de n'y point retrouver le mysticisme de l'Orient hellénique et barbare, d'y constater l'existence d'un culte de la Grande Déesse Phrygienne sans aucune des pratiques orgiastiques du culte phrygien. La déposition del historien grec est confirmée par le témoignage d'Ovide qui, dans les Fastes, décrit les Mégalésies romaines. Ovide ne cite que les sacrifices, les jeux, les mutitations des nobles et le bain de la Dame dans l'Almo. C'était le culte d'autrefois, célébré sans doute avec un nouvel éclat, mais maintenu dans toute son intégrité. Par contre, une transformation importante s'était opérée dans le recrutement du clergé phrygien et préparait le triomphe de la religion phrygienne. Avant l'Empire, au moins dans les derniers temps de la République, on choisissait déjà parmi les affranchis le couple sacré qu'exigeaient les rites. Auguste, qui, en sa qualité de grand maître du collège quindécemviral, est le chef direct du personnel métroaque, le recrute dans sa propre domesticité. Le clergé de l'Idéenne se compose d'affranchis d'Auguste et de Livie.* Mais des affranchis, et surtout des affranchis impériaux, ne peuvent se contenter de l'humble titre de \emph{Famuli}. Les ministres de la Dame ont monté en grade. Ils ont obtenu officiellement la dignité sacerdotale. Peut-être même Auguste, pour rehausser la splendeur du culte rendu à sa divine protectrice, accrut-il le nombre des prêtres et des prêtresses du temple palatin.* Les emplois subalternes sont confiés à des esclaves impériaux. Quand ils se sont montrés bons serviteurs de la déesse et du prince, celui-ci leur octroie en récompense la manumission. L'\emph{aedituus} Cosmus, qui occupait depuis douze ans son poste de concierge, est fier de l'avoir reçue « gratis.* »

*) Dion. Hal. 2, 19.

*) \emph{CIL.} 6, 496 : Onesimus Olympias Livia Briseis, Aug(usti) lib(erti), sac(erdotes) M(atris) D(eum) M(agnae) I(daeae).

*) On lit dans Suétone, \emph{Aug.} 93 : « peregrinarum caerimoniarum sicut veteres ac praeceptas reverentissime coluit, ita caeteras contemptui habuit. » Le culte officiel de la Grande Mère, \emph{more romano}, ne peut être qualifié de \emph{caerimonia peregrina} ; mais le culte phrygien, représenté dès l'origine par le couple sacerdotal, rentre bien dans la première catégorie. Ce texte contiendrait donc, dans sa première partie, une allusion directe à certains cultes orientaux comme celui de Pessinonte et de Comana. Les autres, qu'Auguste méprise, seraient en particulier les cultes d'Alexandrie. Ne pas oublier, d'autre part, qu'Auguste est très superstitieux.

*) \emph{CIL.} 6, 2211. Mommsen complète ainsi la ligne 1 : « Cosmus aedituus Matris D[eum transtiberinae]. » Mais aucun texte ne parle d'une M. D. Transtévérine à Rome. Mommsen confond avec un texte relatif à un sanctuaire d'Ostie. D'autre part il est peu vraisemblable que le chiffre « annis 12 » s'applique au prêtre du temple dont Cosmus était le gardien.

Ce sont les affranchis des Césars qui vont imposer le culte d'Attis à Rome. Ils n'attendront pas au-delà du principat de Claude. Sous cet empereur, en effet, ils deviennent les véritables souverains de la cour et de l'Empire.* Or beaucoup d'entre eux, sinon la plupart, étaient d'origine phrygienne. Ceux-ci provenaient, en général, des domaines impériaux d'Asie ; et le vrai centre de ces biens-fonds était précisément la Phrygie.* On trouve un peu partout, en Phrygie, des propriétés impériales. Au sud, nous en connaissons sur la frontière de Lycie, dans la région de Kibyra ; vers l'extrême nord, il y en a dans la vallée du Tembrogios, affluent du Sangarios. Il en existe de très considérables dans la région des lacs. Les riches carrières de marbre phrygien appartiennent aux Césars. Aussi bien Auguste et ses successeurs, continuant une tradition des Attalides, s'approprièrent la plupart des territoires sacrés des temples. Ainsi firent-ils en Lycaonie, par exemple, pour les terres de la Métêr Ziziménè, nominalement exploitées par son Archigalle.* Sans doute ils agirent de même à Pessinonte.* Les procurateurs chargés de régir ces domaines envoyaient au maître l'élite des esclaves. Le Palatin se peuplait de dévots d'Attis. Parmi ces esclaves et ces affranchis de la cour, on comptait tout un personnel d'eunuques ; tel Halotus, officier de la table impériale* ; tel ce Posidès, que Claude estimait presque à l'égal d'un Narcisse ou d'un Pallas.* Désormais l'eunuque Attis pouvait paraître au grand jour. L'empereur, en qui survit quand même l'esprit conservateur de sa famille, déplore bien devant le Sénat « la prépondérance des superstitions étrangères.* » Mais, dominé par ses courtisans, il a la faiblesse d'accueillir celles qui leur sont familières. Suétone déclare formellement qu'il « institua des cérémonies religieuses d'un caractère nouveau.* » Lydus spécifie que la fête de l'\emph{Arbor Intrat} au Palatin est une fondation de Claude.* On dut faire valoir auprès du César les liens puissants dé reconnaissance qui attachaient à la Grande Mère la maison Claudienne. Pouvait-il mieux manifester sa gratitude envers la Dame qu'en adoptant, avec Attis, les rites dont elle ne peut se passer ? On dut lui persuader qu'Attis n'était vraiment plus un étranger pour Rome depuis que Pessinonte était romaine.* Les Claudes n'avaient-ils pas pour ancêtre Attus Clausus, qui portait le nom même du dieu phrygien ? Enfin, puisqu'il fallait à la religiosité du siècle les satisfactions que seuls procurent les cultes mystiques, puisque Claude introduisait à Rome les mystères de Demeter Eleusinienne, puisque Caligula, son prédécesseur, avait autorisé ceux d'Isis, pourquoi exclure les mystères de la Mère des Dieux ? Le privilège qu'avait obtenu l'Égyptienne, pouvait-on le refuser à la déesse phrygienne, dont Rome, depuis deux siècles et demi, tolérait les rites secrets dans un enclos du Palatin ? Claude céda peut-être moins à ces considérations d'ordre sentimental et religieux qu'à la nécessité politique de discipliner une force grandissante, contre laquelle tout effort de réaction demeurait impuissant. R établit donc un cycle nouveau de fêtes métroaques dans la religion romaine. Aux Mégalésies d'avril, fêtes particulières à Rome, il ajouta les fêtes universelles que célébraient en mars toutes les églises phrygiennes.

*) Sueton., \emph{Claud.} 25 : « Haec (les affaires religieuses) et caetera... non tam suo quam uxorum libertorumque arbitrio administravit » ; 29 : « his addictus, non principem se, sed ministrum egit. »

*) V. surtout Ramsay, \emph{Historical Geogr. of Asia Minor}, 1890, p. 172 ss ; \emph{Cities and Bishoprics of Phrygia}, 1, 1895, p. 272 ss ; \emph{Studies in eastern Provinces}, 1906, pp. 305-377 ; Sterret, \emph{Epigr. Journey in Asia M.}, 38-72 ; Schullen dans \emph{Roem. Mitt.}, 1898, p. 222 ; Hirschfeld, \emph{Grundbesitz der roem. Kaiser.} 2, 1902, p. 299 ss ; Chapot, \emph{Province rom. d'Asie}, 1904, pp. 373-381. Dans la vaste agglomération de propriétés impériales qui entoure le lac Ascania, inscr. du 1er siècle. L'impératrice Livie avait un domaine aux environs de Thyatira : \emph{CIG.} 3484, 3497.

*) Ramsay, \emph{Classical Rev.}, 1905, p. 368. Zizima (Sizma) est à 5 heures au nord d'Iconium.

*) L'organisation nouvelle du sacerdoce de P., qui nous est connue au temps des Flaviens, est sans doute le résultat de cette usurpation. 

*) Il était « praegustator » : Sueton., \emph{l. c.}, 44.

*) Sueton., \emph{l. c.} 28.

*) Tacit.. \emph{Ann.} 11, 15.

*) \emph{L. c.} 22 : « Quaedam circa caerimonias... aut correxit... aut etiam nova instituit. »

*) Lyd., \emph{De mens.} 4, 59 (41), éd. Wünsch, 1898 : δένδρον πίτυς παρὰ τῶν δενδροφόρων ἐφέρετο ἐν τῷ Παλατίῳ τὴν δ᾽έοστὴν Kλαύδιος ὁ Bασιλεὺς κατεστήσατο. Wissowa, \emph{Rel. u. Kultus d. R.}, p. 266, n. 8, suppose à tort que Lydus a peut-être confondu avec Claude 2 le Gothique. Suivi par Bloch, dans \emph{Berl. Phil. Wochenschrift}, 1902, p. 722, il ne croit pas que l'adoption du culte phrygien soit antérieure aux Antonins. Mais nous connaissons des dendrophores à Reggio en 79 (\emph{CIL.} 10, 7), et à Rome peut-être en 97 (\emph{CIL.} 6, 642), sûrement en 107 (Orelli, 4412), à Sassina peu après Trajan (\emph{CIL.} 11, 6520).

*) La Galatie était province romaine depuis 729/25 av. J.-C.

\subsection{2.}

Les cérémonies de rite phrygien duraient du 15 au 27 mars. Elles coïncidaient avec le retour du printemps. Car Attis, dont elles reproduisent le drame sacré, c'est la végétation qui meurt et ressuscite.* Pour les populations agricoles, le printemps est la période sainte de l'année. C'est avec lui que s'ouvre l'année religieuse, identifiée à l'année solaire. Sans doute le calendrier phrygien commençait-il à l'équinoxe vernal. Il se conformait ainsi à la tradition chaldéenne, qui plaçait le premier degré du zodiaque dans le signe du bélier.* Cette tradition paraît s'être en effet généralisée en Asie Mineure. Les fêtes en l'honneur d'Attis auraient donc sanctifié à la fois la fin d'une année et le début du nouvel an.

*) On trouvera une étude détaillée sur le mythe d'Attis, avec le recueil des textes, dans Hepding, \emph{Attis, seine Mythen und sein Kult}, 1903.

*) L'ancienne année romaine commençait également au printemps. Dans la doctrine astrologique, le Bélier est sous la tutelle de Minerve (Bérécynthienne = Cybèle). Primitivement, l'équinoxe du printemps était dans Taureau ; cf. Bouché-Leclercq, \emph{Astrologie grecque}, pp. 57 et 61.

Le 15, jour des Ides, \emph{Canna Intrat}, ou Entrée du Roseau.* C'est une fête de confrérie. Les Cannophores ou porteurs de roseaux, hommes et femmes, petits garçons et petites filles, sous la conduite de leurs Pères et Mères spirituels, se rendent processionnellement au temple avec leurs insignes. Tel est le prélude du drame. A quel épisode mythique des livres sacrés correspondait-il* ? On a cru y reconnaître une commémoration symbolique de l'enfance d'Attis, sauvé des eaux.* Sur les ordres de son grand-père Sangarios, Attis fut exposé dès sa naissance au bord du Gallos. C'est là qu'il grandit, nourri par des bêtes. C'est là que Cybèle le découvrit parmi les roseaux des berges.* Le jour où l'enfant se manifesta pour la première fois à la déesse, il naquit vraiment à une vie nouvelle. On fêtait donc son Épiphanie, avant de célébrer son Aphanie. L'hypothèse d'une Épiphanie est séduisante. Toutefois, l'intervention des roseaux s'expliquerait aussi bien s'il s'agissait simplement d'une fête de la Naissance, puisqu'Attis est le fils d'une nymphe, le petit-fils du fleuve. Mais pourquoi cette fête, qui devrait apporter la joie aux dévots, est-elle immédiatement suivie d'une longue période d'expiations et de pénitences ? Voici une autre hypothèse, qui semble mieux concorder avec le reste du drame liturgique. La Cannophorie aurait commémoré les amours d'Attis avec une nymphe du Sangarios ou du Gallos, bordé de cannaies, une Dea Canna, comme celle qui fut aimée de Pan.* Ces amours ne furent-elles pas la cause de la mutilation du dieu et de sa lamentable mort ? Le prélude de cette Passion serait la disparition d'Attis, qui se cache auprès de la nymphe et que Cybèle cherche en vain.* Quoi qu'il en soit, rite et légendes confirment ce que nous savions par ailleurs sur les rapports étroits du culte phrygien avec la religion des eaux. A l'origine, cette cérémonie des roseaux n'était sans doute qu'un des nombreux rites magiques dont on usait pour évoquer la pluie fécondante.* De même, dans certains cultes lydiens ; et doriens d'Artémis, autre déesse de la pluie et des sources, des jeunes filles se couronnent de roseaux pour exécuter leurs danses sacrées. Un autre rite du 13 mars se rattache à la même idée agraire. Les Cannophores promenaient un taureau de six ans.* Après la procession, l'archiprêtre le sacrifiait pour la fertilité des champs qui sont sur les montagnes. Il y avait longtemps que « Mater Deum Agraria* » était en grande vénération dans la campagne romaine ; on n'y avait point oublié les belles récoltes de l'année 204, qui furent l'un de ses premiers miracles de joyeux avènement. Sur le Palatin, d'où la vue s'étend vers des horizons de coteaux et de monts, une telle cérémonie ne pouvait paraître déplacée. Aussi bien Rome célébrait-elle déjà, ce même jour, une fête de même ordre, très ancienne et très populaire : joyeux pèlerinage à la source et au bois d'Anna Perenna, sur la Voie Flaminienne, qui contrastait avec la procession morose des confrères orientaux, tristes de la mort prochaine de leur dieu.

*) Les noms et la succession de ces fêtes nous sont donnés par le calendrier philocalien, en l'année 354 ; \emph{CIL.} 1, 2, p. 312 :  
Id. Mart. Canna intrat. 8 K. Apr. Hilaria.  
11 K. Apr. Arbor intrat. 7 K. Apr. Requetio.  
9 K. Apr. Sanguem. 6 K. Apr. Lavatio.  

*) Sur les rapports de ce rite avec le mythe d'Attis, cf. les monuments figurés: 1° une ciste votive dédiée par un archigalle dans le Metrôon d'Ostie, et étudiée par Visconti dans \emph{Annali}, 1869, p. 240 ss, déposée au Musée du Lateran ; la tête d'Attis y est représentée entre des roseaux ; 2° une statuette de bronze, conservée à Toulouse, Musée Saint-Raymond ; Attis tient d'une main la syrinx, de l'autre un bouquet de roseaux.

*) Decharme dans \emph{Rev. archéol.} 1886, 1, p. 288 ss. Je ne signale que pour mémoire l'hypothèse de Visconti, \emph{l. c.} ; Attis, après son émasculation, se serait réfugié dans les roseaux, où Cybèle et ses lions l'auraient retrouvé. Mais sa mutilation, selon le mythe, a lieu dans un bois de pins et est immédiatement suivie de sa mort, laquelle n'est commémorée que le 22 mars.

*) Arnob. 5, 6 : « enititur parvulum ; sed exponi Sangarius praecipit. » Julian., \emph{Or.} 5, p. 165 B : ὃν δή φησιν ὁ μῦθος ἀνθῆσαι μὲν ἐκτεθέντα παρὰ Γάλλου ποταμοῦ τᾶις δίναις, εἶτα καλὸν φανέντα καὶ μέγαν ἀγαπηθῆναι παρὰ τῆς Μητρὸς τῶν Θεῶν. \emph{Ibid.}, p. 180 A : τὸν Ἄττιν ἐκτεθέντα περισωσαμένη. Sallust. phil., \emph{De diis et mundo} 4 : παρὰ τῷ Γάλλῳ λέγεται εὐρεθῆναι ποταμῷ.

*) Ovid., \emph{Fast.} 4, 229 : Sagaritis, considérée à la loi comme naïade et hamadryade ; sur les dryades « filles du fleuve, » cf. Mannhardt, \emph{Ant. Wald und Feld Kulte}, p. 14. D'après une autre légende (texte corrompu d'Arnobe, 5, 7), la nymphe aimée est fille du Gallos. Pour l'identification possible de la nymphe avec les roseaux de la rivière, ajouter à la Dea Canna, dont parle Apulée, la nymphe Syrinx (la syrinx joue un rôle dans la légende d'Arnobe). D'autre part, cf. le rôle des roseaux dans le mythe de Midas, le roi qui voulait marier sa fille à Attis (Arnob. \emph{l. c.}). D'après l'empereur Julien, c'est dans une grotte qu'Attis s'unit à la nymphe ; mais cf. des roseaux figurés à l'entrée îles grottes mithriaques : Cumont, \emph{Mithra}, 1, p. 195, et 2, n°s 107 et 193.

*) Cf. les φυγαὶ θρηνούμεναι καὶ ἀφανισμοι dont parle Julien, \emph{l. c.} 168 C.

*) Cf. le rôle magique de la flûte, invention de Midas ou du phrygien Hyagnis (génie de la pluie ? ) : Gruppe, \emph{Gr. Mythol. u. Relig.}, p. 1521, n. 3. Rhéa-Cybèle est « la cause des pluies, » ὄμβρων αἰτία, dit Cornutus. \emph{Theologiae graecae compend.} 6, éd. Lang.

*) Ioh. Lyd. \emph{De mensibus} 4, 49 (éd. Wünsch, p. 105, 15 et praefat. p. 67) : ἱεράτενον δὲ καὶ ταῦοον ἐξέτη ὑπὲρ τῶν ἐν τοῖς ὄρεσιν ἀγρῶν, ἡγουμένου τοῦ ἀοχιερέως καὶ τῶν κανηφόρων (lire sans doute καννοφόρων) τῆς Mητρύς. L'\emph{archiereus} est l'\emph{archigalle}, et non, comme le croit Mommsen, le \emph{pontifex maximus}. C'est peut-être à cette procession du taureau qu'a succédé celle du Bœuf Gras.

*) Sur une inscr. de l'Ager Labicanus ; \emph{Bull. arch. comunale di Roma}, 20, 1892, p. 358 ; cf. supra, p. 93.

A partir du 16, les fidèles entrent dans le saint temps de pénitence, qui dure neuf jours.* Nous retrouvons semblable neuvaine dans les fêtes de Déméter Eleusinienne, de Cérès, de Bacchus, d'Isis. D'autre part, ces purifications n'étaient point rares à la fin et au début de l'année religieuse.* Dans les mystères de la Mère des Dieux, les fidèles croyaient s'associer ainsi à la douleur de la déesse, privée de son bien-aimé.* Durant cette période, désignée sous le nom de \emph{Castus Matris Deum},* on pratique la continence sexuelle* et certaines abstinences. La principale est le jeûne du pain.* Tout aliment dont les céréales constituent la matière première est rigoureusement interdit. On s'abstient également de racines, de certains fruits, tels que grenades,* coings,* dattes,* de viande de porc,* de poisson,* sans doute aussi de vin.*

*) Ce temps de pénitence, ἁγνείας καιρός (Julian., \emph{Or.} 5, p. 177 A), ne commençait pas seulement le 22 mars, comme le dit Hepding, \emph{Attis}, p. 155. Julien, p. 178, déclare que les fêtes de Bacchus et de Minerve ont lieu pendant ces jours d'abstinence. Or les \emph{Liberalia} se fêtent le 17 mars et les \emph{Quinquatria} le 19 ; il y avait aussi le 21 une fête du \emph{Natalis Minervae}. Les abstinences métroaques devaient donc débuter aussitôt après les cannophories (peut-être y avait-il là une influence astrologique ; c'est en effet le 16 que le soleil entre dans le Bélier). On les désignait en général sous le nom d'ἁγιστείαι = purifications, cf. Hesychius, s. v.

*) L'année romaine se terminait par des purifications en février.

*) Arnob. 5, 16 : « Quid temperatus ab alimonio panis, cui rei dedistis nomen \emph{castus} ? Nonne illius temporis imitatio est, quo se numen ab Cereris fruge violentia mocroris abstinuit ? »

*) Cf. Tertull. \emph{De jejunio}, 2 (éd. Reifferscheid et Wissowa, 1, p. 275) : « xerophagias vero novum adfectati officii nomen et proximum ethnicae superstitioni, quales \emph{castimoniae} Apim, Isidem et Magnam Matrem certorum eduliorum exceptione purificant » ; \emph{ibid.}, 16 (p. 296) : « sed bene quod in nostris xerophagiis blasphemias ingerens \emph{casto} Isidis et Cybeles eas adaequas. » Hieronym., \emph{Ep. 107 ad Laetam} : « gulosa abstinentia phasides aves ac fumantes turtures vorant, ne scilicet cerealia dona contaminent » ; \emph{Adv. Jovinian.} 2, 5 : « \emph{castum} Matris Deum et Isidis » ; 17 : « \emph{castum} Isidis et Cybeles. » Marin., \emph{Vita Procli}, 19 : τὰς μητρῳακὰς παρὰ Ῥωμαίοις ἢ καὶ πρότερόν ποτε παρὰ Φρυξὶ σπουδασθείσας καστείας. Marinus fait allusion ici aux jeûnes mensuels des mystes. Les \emph{Religiosi} pratiquaient aussi certaines abstinences perpétuelles, « quorumdam ciborum in aeternum abstinentia, » Hieronym. \emph{Adv. Jovinian.}, 2, 17 ; entre autres, abstinence de porc, de poisson (Julian., \emph{l. c.}, 176 C), de vin (Arnob. 5, 6 : accès du lieu saint interdit « vino poilu tis » ), d'ail (Athaen., 10, 422 D). Julien, pp. 173 D-l77 C, donne la liste de tous les aliments défendus par le θεῖος θεσμός, en y ajoutant des commentaires parfois puérils. Il y avait discussion au sujet de certains aliments, les légumineuses, par exemple. Du reste, à l'époque de Julien, sans doute pour ne point décourager les âmes tièdes, la loi religieuse savait atténuer ses rigueurs : « elle ne prescrit pas tout à tous, mais seulement le possible. »

*) Elle se retrouve à Rome dans le \emph{Castus} de Cérès et dans celui de Juno Lucina (\emph{CIL.} 1, 813). D'autre part le mysticisme phrygien 'a toujours manifesté, sous sa forme chrétienne comme sous sa forme païenne, une violente réprobation de la sexualité.

*) Hieronym., \emph{l. c.} : « Jejunium panis. » Il en est question dans Catulle, \emph{Attis}, 36 : \emph{sine Cerere}. Dans le texte d'Arnobe, 5, 16, cité plus haut, il s'agit bien d'une abstinence et non de la \emph{casta mola}, comme le suppose C. Pascal, \emph{De Cereris castu}, dans Hermès 1895, p. 550. L'abstinence de pain se retrouve dans certains cultes syriens ; à Hepding, \emph{Attis}, p. 156, ajouter Clermont-Ganneau, \emph{Recueil d'archèol. orient.} 2, 1898, p. 134, inscr. de Nîha, près Zahlé : « Hocmaea virgo dei Hadranis (vierge consacrée au dieu Hadran), quia annis 20 panem non edidit, iussu ipsius dei, v. s. l. m. »

*) Rôle de la grenade dans le mythe pessinontien, Arnob. 5, 6 : (lors delà mutilation d'Agdistis) « cum discidio partium sanguis fluit immensus, rapiuntur et combibuntur haec terra, \emph{malum} repente cum his \emph{punicum} nascitur ; cuius Nana speciem contemplata, regis Sangari vel fluminis filia, carpit mirans atque in sinu reponit ; fit ex eo praegnans. » Ainsi naquit Attis. Sur le caractère sacré de la grenade dans les cultes d'Aphrodite, Dionysos. Harpocratès, liera et Perséphone, cf. Gruppe, \emph{op. l.} index, p. 1905, s. v. \emph{Granale} ; Hepding, \emph{Attis}, pp. 106 et 156. La grenade en Asie Mineure : Hehn, \emph{Kulturpflanzen u. Haustiere}, 6e éd., 1894, p. 233 ss ; dans la symbolique religieuse et dans l'art de l'époque mycénienne, Furtwaengler dans \emph{Münch. Akad. Sitzungsber.}, 1887, 2, p. 111.

*) Ce sont les pommes d'or dont parle Julien ; cf. Hehn, \emph{op. l.}, p. 241 ; Kannenberg, \emph{Klein Asiens Naturschätze}, 1897, p. 87. Elles ont pour « archétypes, » dit Julien, les pommes d'or du jardin des Hespérides ; celles-ci étaient, dans les mystères orphiques, le symbole de la délivrance de lame qui échappe aux terreurs d'Hadès ; cf. Gruppe, \emph{op. l.}, p. 385.

*) Parce que le palmier est un arbre consacré au soleil, suppose Julien. A vrai dire, il est déjà un arbre sacré dans le culte crétois de la déesse Mère et du dieu Fils ; cf. supra, p. 4. De même, dans le culte de l'Idéenne, cf. le miracle de l'an 716 au Palatin ; supra, p. 100.

*) Cf. supra, p. 12.

*) Sur le caractère sacré du poisson dans certains cultes antiques, et particulièrement en Asie, : Hepding, \emph{Attis}, p. 157, n. 2 ; Gruppe, \emph{op. l.} index, p. 1903, s. v. \emph{Fische}. Dans son chapitre sur les mystères d'Attis, Hepding, p. 189, tire grand parti de l'épitaphe d'Aberkios, où il est question du poisson mystique, v. 13 s, ἰχθὺν ἀπὸ πηγῆς πανμεγέθη καθαρόν, nourriture du fidèle. Mais il faudrait d'abord prouver qu'Aberkios était un prêtre d'Attis et non pas un chrétien.

*) Cf. supra, p. 119, n. 4. La vigne paraît avoir été consacrée à Cybèle comme à Dionysos : idole de la Dindymène, à Cyzique, taillée dans un cep de vigne, Apoll. Rh., \emph{Argon.} 1, 1117 ; procession de la Bérécynthienne, à Autun (Augustodunum), « pro salvatione agrorum ac vinearum, » Greg. Tur., \emph{In gloriam confessorum}, 76, éd. Krusch.

Aux approches de l'équinoxe, commençait la série des fêtes majeures, c'est-à-dire le drame de la Mort et de la Résurrection.* Le 22 mars, \emph{Arbor Intrat}, fête de la dendrophorie.* L'arbre qui entre dans le temple est le pin. C'était sous un pin qu'Attis avait sacrifié sa virilité et qu'il était mort de sa blessure. De son sang répandu sur le sol étaient nées les violettes, qui avaient entouré l'arbre d'une ceinture fleurie. Cybèle les avait tressées en couronne sur le cadavre de l'adolescent. Puis elle avait emporté son Attis au fond de sa caverne, où elle avait donné libre cours à sa douleur inconsolée. On disait aussi qu'après avoir enseveli Attis elle orna de violettes, fleurs de sang, le pin sous lequel il avait péri, qu'ensuite elle le transporta dans son antre et le consacra pour toujours à son culte.* Mais c'était là une interprétation littérale et grossière du rite phrygien. Ceux qui croyaient à la métamorphose d'Attis en pin* comprenaient mieux la signification profonde du mythe. La procession du pin représente le convoi funèbre d'Attis, esprit de l'arbre. Le pin est identique au Dieu ; en lui réside « une puissance divine, présente et très auguste.* » Comme dans le culte corinthien de Bakkeios,* les prêtres pouvaient chaque année répéter aux fidèles : « Vous honorerez l'arbre à l'égal du dieu. » Cette fête du 22 comprenait trois parties : l'\emph{Ectomè} ou Coupe du pin sacré, la Procession (\emph{Pompè}) et l'Exposition (\emph{Prothesis}). Elle était organisée par une confrérie spéciale, la confrérie des Porteurs de l'Arbre ou Dendrophores.* Le rituel phrygien voulait d'abord que l'arbre fût pris dans les bois consacrés à la Mère des Dieux. IL faut donc admettre que, dans les dépendances des temples métroaques, il y avait au moins une petite pinède. « Irai-je au bois sacré de Cybèle, au bois de pins ? » dit un personnage qui habite Rome.* Il semble d'autre part que toutes les pinèdes aient été consacrées à la déesse ; le pin est sous la tutelle de la Mère des Dieux.* L'arbre choisi doit être coupé, non arraché. Il doit être coupé avant le lever du soleil.* Ce sont les Dendrophores eux-mêmes, généralement bûcherons, charpentiers ou marchands de bois, qui accomplissent cette tâche. Ils conservaient au pin une partie de ses branches ou du moins quelques petits rameaux.* Sur les racines, on immolait un bélier, sans doute primitivement pour apaiser l'esprit de l'arbre.* On enveloppait alors le tronc d'arbre de bandelettes de laine.* Ainsi la fiancée d'Attis ou Cybèle même, accourue vers le pin fatal, avait recouvert d'étoffes de laine le corps déjà glacé de l'éphèbe, dans l'espoir inutile de le réchauffer et de le ranimer. Mais sous le mythe se dissimule mal un rite purement funéraire. Ces bandelettes étaient toutes de couleur pourpre, comme les tissus dont on parait les morts. Quant aux branches, on les couronnait de violettes,* et l'on y suspendait les attributs du dieu, son bâton et sa syrinx, son tambourin, ses cymbales et sa double flûte. Au milieu de l'arbre, on attachait une petite figurine d'Attis. Peut-être cette concession à l'anthropomorphisme est-elle assez récente ; elle paraît être toutefois antérieure au 1er siècle.* Nous retrouvons le même usage dans la dendrophorie d'Osiris. Comme pour Osiris, et afin de mieux affirmer l'identité de l'arbre et du dieu, l'idole était probablement taillée dans un fragment du même pin.

*) Peut-être n'y avait-il à l'origine qu'une idée d'assoupissement du dieu pendant les mois d'hiver ; cf. ce que dit le Ps. Plutarque, \emph{De Is. et Osir.} 69, sur les fêtes phrygiennes des Kατευνασμοί et des ᾿Λνεγέρσεις. Mais, sur la haute antiquité des fêtes célébrant, en Orient, le départ de certains dieux pour l'autre monde et leur mort annuelle, cf. un texte du mythe babylonien d'Adapa, trouvé en Égypte, à El-Amarna, avec la correspondance assyrienne des rois Amenophis 3 et 4 (15e s.) : « ils demanderont à Adapa la raison de son deuil ; Adapa répondra qu'il pleuie deux dieux qui ont quitté la terre, Tammuz et Gishzida. » Loisy, \emph{Mythes babylon.} dans \emph{Rev. d'hist. et littér. relig.}, 6, 1901.

*) Cf. Lyd., supra, p. 115, n. 7.

*) Arnob. 5, 7 et 14, 16, 17 ; cf. Lactant. Placid., ad Stat. \emph{Theb.} 10, 175. Sur les différentes formes du mythe de l'éviration d'Attis, cf. Gruppe, \emph{op. l.}, p. 1542-43, et Hepding, \emph{Attis}, chapitre 2.

*) Ovid., \emph{Metam.} 10, 104 s ; cf. \emph{Ibis} 505 ; cf. Anonym., \emph{Aegritudo Perdicae} 28-30, dans Baehrens, \emph{Poet. Lat. Min.} 5, p. 112. D'après un autre mythe, rapporté par Nonn., \emph{Dionys.} 12, 56, l'homme serait né du pin.

*) Arnob. 5, 17 : « praesens atque augustissimum numen. »

*) Cf. Gruppe, \emph{op. l.}, p. 781 (\emph{Baumfetische}), et Hepding, \emph{Attis}, p. 152.

*) Cf. le chapitre consacré au clergé et aux confréries.

*) Prudent., \emph{Peristeph.} 10, 196 ; cf. Virg., \emph{Aen.} 9, 85 : « pinea silva mihi » : Claudian., \emph{De Raptu Proserp.} 1, 201 ss ; un « lucus Matris Deum, » Serv. ad \emph{Aen.}, 3, 113.

*) Phaedr., \emph{Fab.} 3, 17 (arbores in deorum tutela), 4 : « pinus Cybelae » ; Serv. ad \emph{Aen.} 9, 85 : « pinus in tutela est Matris Deum » ; 115 : « Mater Magna pinum... tutelae suae adscripsit » ; cf. Ovid., \emph{Metam.} 10, 104 : (pinus) « grata Deum Matri » ; Martial., 13, 25 (nuces pineae) : « poma sumus Cybeles. » Pins sur les monuments figurés du culte métroaque : Zoega, \emph{Bassirilievi}, p. 54 ; Boetticher, \emph{Baumkultus}, fig. 5 et 11. Il s'agit du pin-pignon (pinus pinea) ; cf. Hehn, \emph{op. l.}, p. 290 ss. Pin et pomme de pin comme symboles de la M. d. D. sur des monnaies de Scamandreia, Skepsis, Antandros ; bois sacrés de la déesse qui paraissent avoir donné leur nom aux villes de Pitya (Mysie, Carie), Pityus (Mysie), à l'île de Pityodes (Propontide) : cf. Gruppe. \emph{op. l.}, p. 1530, n. 2. Le pin dans d'autres cultes helléniques : \emph{ibid.}, index, p. 1903, s. v. \emph{Fichte}.

*) Firm. Mat., \emph{De err. prof. rel.} 27, 1 : « in sacris Frygiis, quae Matris Deum dicunt, per annos singulos arbor pinea caeditur, et in media arbore simulacrum iuvenis subligatur » ; 4: « arborem suam diabolus consecrans intempesta nocte arietem in caesae arboris facit radicibus immolari. » Julian., \emph{Or.} 5, p. 168 C : τέμνεσθαι τὸ ἱερὸν δένδρο ; cf. 169 : ἐκτομὴ τοῦ ὀένδρου. Sallust. philos., \emph{De diis el mundo} 4 : δένδρον τομαὶ καὶ νηστεία. Anonyme de 394, \emph{Carmen contra paganos}, dans Baehrens, \emph{Poet. Lat. Min.} 3, p. 286 ss, v. 108 : « (vidimus) arboris excisae truncum portare per urbem. » Dans d'autres cultes, au contraire, il fallait déraciner l'arbre ; cf. l'Apollon de la grotte d'Hylae, près de Magnésie du Méandre, Pausan. 10, 32, 6.

*) Arnob. 5, 16 : « quid lanarum vellera, quibus arboris conligatis et circumvolvitis \emph{stipitem} ?... quid compti violaceis coronis et redimiti arboris \emph{ramuli} ?

*) Firm. Mat., \emph{l. c.}

*) Pour tous ces détails, Arnob., \emph{ll. cc.} Bite analogue dans le culte d'Adonis-Osiris à Byblos. Sur le caractère funéraire de la pourpre, dont la couleur rappelle le sang des sacrifices aux morts, cf. Hepding \emph{Attis}, p. 150, n. 3.

*) Rôle des fleurs dans certaines fêtes babyloniennes des morts : Gruppe, \emph{op. l.}, p. 1530, n. 6 ; cf. les mystères d'Adonis.

*) Firm. Mat., \emph{l. c.} C'est peut-être, en effet, à cette coutume que fait allusion Diodore de Sicile, 3, 59 : εἴδωλον κατασκευάσαι τοῦ μειρακίου, πρὸς ὦ θρηνοῦντας, \emph{etc.} Firmicus Maternus ajoute, à propos des \emph{Sacra Isiaca} : « De pinea arbore caeditur truncus ; huius trunci media pars subtiliter excavatur ; illic de segminibus factum idolum Osiridis sepelitur. » De même, dans les mystères de Proserpine « caesa arbor in effigiem virginis formamque componitur. »

Parti du bois sacré ou de la « schola » de la confrérie, sous la conduite des archidendrophores et du clergé, le cortège parcourt la ville avant de se rendre au sanctuaire. Des confrères, sans doute à tour de rôle, portent l'arbre. Les autres tiennent à la main des branches ou des torches de pin. On chante des cantiques funèbres en langue grecque.* Des Galles aux longs cheveux épars, chamarrés de saintes images, font gronder leurs tambourins ; puis ils se frappent la poitrine, en signe de deuil.* Arrivé au temple, l'arbre est exposé à l'adoration des fidèles. Cette exposition se fait, non pas à l'intérieur de l'édifice, mais dans l'\emph{Ager} ou \emph{Campus Matris Deum}, enclos sacré qui précède ou longe le monument.* Durait-elle plusieurs journées ? La coutume de ne donner la sépulture aux morts que le troisième jour était très ancien et très généralement répandu. Hérodote la signale en Thrace ; et les Perses croyaient que l'âme reste trois jours auprès du corps sans l'abandonner.* Il est donc fort vraisemblable que la scène de la mise au sépulcre n'avait lieu que le 24 mars, journée lugubre et néfaste entre toutes. Pendant ce triduum de deuil, le sanctuaire retentit sans cesse de lamentations rythmées, de clameurs aiguës, d'appels plaintifs, qui alternent lugubrement avec les versets des cantiques.* Cymbales et tympanons résonnent.* Car leur son est agréable aux morts et purifie les vivants* ; il éloigne les mauvais esprits. « Aiai pour Attis ! Frappez pour Attis ! » crie par intervalles un prêtre.* Et les fidèles ululent Attis.* Et ils se frappent la poitrine avec la paume des mains, selon le geste familier à la douleur antique.* Les plus zélés tiennent des pommes de pin* et se meurtrissent ainsi jusqu'au sang. Toutes les nuits, de nombreux dévots restent au Palatin pour la funèbre veillée.

*) Il s'agit bien en effet d'une procession funéraire : « funeris pompa, » dit Firmicus Maternus, \emph{op. l.}, 3, 1 ; cf. Min. Fel., \emph{Oct.} 22, 1 : « funera et luctus. » Sur l'emploi du grec comme langue liturgique, Serv. ad \emph{Georg.} 2, 394 : « hymni Matris Deum ubique propriam, id est graecam linguam requirunt. » De même, dans le culte de Mithra, cf. Cumont, \emph{Mithra}, 1, p. 238.

*) Arnob. 5, 16 : « pectoribus adplodentes palmas passis cum crinibus Galli. »

*) Stat., \emph{Theb.} 10, 172-175, décrivant la fête du « Sanguis » (24 mars) : « quatit ille sacras in pectora pinus | sanguineosque rotat crines et vulnera cursu | exanimat ; \emph{pavet omnis ager respersaque cultris} | \emph{arbor}. » Cet \emph{Ager} paraît être identique au \emph{Campus Matris Deum} dont il est question sur une dédicace d'Ostie, \emph{CIL.} 14, 324, faite précisément un 24 mars. Arnobe. 5, 16 : « ilia pinus quam statutis diebus in Deum Matris intromittitis \emph{sanctuarium}. » Mais ce dernier mot qualifie tout ce qui est compris dans l'enceinte sacrée (cf. le sens analogue de \emph{sacrarium Matris} dans Trebellius Pollio, \emph{Claud.} 4, 2) ; ou bien l'auteur fait allusion à la descente du pin dans la crypte, le jour du « Sanguis. » Un peu plus loin, Arnobe semble dire que le pin était dressé, non couché, ce qui n'est possible qu'en dehors de l'édifice ; 5, 17 : « cur Deum Matris constituatur in sedibus. » De même, pendant les fêtes syriennes d'Atargatis, des arbres sont dressés dans l'\emph{aula} du temple ; on y suspend des chèvres, des brebis, des oiseaux, des étoffes, des ex-voto d'or et d'argent ; Lucian., \emph{Dea Syr.} 49.

*) Cumont, \emph{Mithra} 1, p. 37.

*) Sueton., \emph{Oth.} 8 : « die quo cultores Deum Matris lamentari et plangere incipiunt » ; cf. Firm. Mat., \emph{op. l.} 22, 1 : « \emph{per numeros plangitur}. » Maternus désigne ici les thrènes liturgiques dont il est question dans Diod. Sic. 3, 59 ; Procl., \emph{In Platonis Rem puhlicam} (éd. Kroll 1, p. 125) : τῆς μεγίστης θεᾶς ἱεροὺς θρήνους, cf. Marin. \emph{Vita Procli}, 19 ; Julian., \emph{l. c.} 168 C ; Schol. ad Nicandri \emph{Alexipharm.} 8. Démosthène, \emph{Pro Cor.} 260, nous a conservé un de ces appels qui dut se transmettre à travers les âges et retentir au Palatin : ὕης Ἄττης. Cf. l'idée égyptienne, qu'à l'appel de son nom le Dieu sort de son sommeil, de môme qu'il est ranimé par le sacrifice.

*) Mart. 14, 204 ; « aera lugentia Matris amores » ; Claudian. 20 (\emph{in Eutrop.} 2), 301 ; « ad tristes convertit (Cybèle) tympana planctus » ; Anonym. \emph{Aegritudo Perdicae}, 30 : « per tympana plangitur Attis. »

*) Gruppe, \emph{Gr. Mythol. u. Religions Gesch.}, p. 897. Pour les cymbales, cf. Schol. ad Theocr. \emph{Idyll.} 2, 36. Sur les sarcophages, entre les mains des Sirènes, elles ont aussi pour destination d'écarter les mauvais démons. Pour les tympanons, cf. le tympanisme magique, au chapitre des Galles.

*) Cf. ᾶιᾶι Ἄδωνιν, κόπτεσθ᾽ Ἄδωνιν et les thrènes d'Adonis ; Aristoph., \emph{Lys.} 389.

*) Firm. Mat., \emph{l. c.} : « annuis ululatibus » ; Claudian., \emph{l. c.} 302 ; « sacris ululatibus. » On désignait généralement ainsi le hurlement lugubre des chiens et des loups ; d'autre part, « ulula» est le nom d'un oiseau de nuit, la hulotte, à cause de son cri. Les Grecs dénommaient ces lamentations « ololygmes. »

*) Cf. p. 123, n. 4.

*) Stat., \emph{Theb.} 10, 172, cf. supra, p. 123, n. 5 ; Lactantius Placidus interprète ici « pinus » par « consecratas faces » ; Claudian. 18 (\emph{in Eutrop.} 1), 279 : « pectus illidere pinu. »

La journée du 23 n'est signalée par aucune cérémonie publique. Elle est toute à la douleur, à la mortification, à la prière. Les dévots la passent dans le recueillement de la maison, ou vont remplir leurs pieux devoirs au temple. Mais les Saliens célébraient sur le Palatin, ce même jour, le Tubilustrium ou bénédiction des trompettes, un des plus anciens rites du culte national de Mars. Le dieu indigète, auquel était consacré tout le mois, fut appelé à participer aux fêtes phrygiennes.* Son rôle consista, ce semble, en une procession des Saliens autour du temple, avec sonneries de trompettes. Ils agitaient peut-être aussi leurs anciles sacrés, en l'honneur de la Grande Mère. Ce cliquetis de boucliers et ces fanfares, comme le choc des cymbales, comme le bruit de tout instrument et de toute arme d'airain, possédaient une vertu purificatrice. En cette saison, les rites bruyants des Saliens favorisaient l'annuelle rénovation de la nature : telles les pratiques analogues des Curètes crétois de Rhéa, des Corybantes anatoliens de Cybèle.* C'est ainsi que peut s'expliquer le rapprochement des deux divinités. Car la simple coïncidence des fêtes et le voisinage des dieux n'y suffiraient point. Le Mars latin était, comme Cybèle et Attis, un maître du printemps, qui faisait croître les moissons et les vignes, les arbres et les herbages, qui sauvegardait les troupeaux et les pâtres.* Un vieux mythe l'associait à Rhéa Silvia, peut-être identique, certainement identifiée à Rhéa Idaea.* L'une le mit en relations avec l'autre. Toutefois, cette intervention de Mars dans les fêtes métroaques n'est signalée que très tard. Seul, l'empereur Julien en fait mention.* Il est donc possible qu'elle soit de beaucoup postérieure à l'organisation du culte phrygio-romain. En ce cas, les spéculations astrologiques, dont on abuse à partir du 2e siècle, n'y furent sans doute pas étrangères. Dans la chorographie astrologique de Ptolémée, contemporain des Antonins, il est montré comment l'Asie Mineure honore Vénus sous le nom de Grande Mère, et Mars sous le nom d'Adonis j, c'est-à-dire aussi d'Attis. Certain mythe n'attribuait-il pas à un Arès solaire la paternité de cet Attis* ?

*) Julian., \emph{Or.} 5, 168 CD (περισαλπισμός) et 169 D.

*) Cf. supra, p. 6.

*) Roscher, \emph{Myth. Lex.} 2, p. 2399 ss : \emph{Mars als Frühlingsgott}. C'est également aux fêtes du dieu que se rattache celle d'Anna Perenna, supra, p. 119.

*) V. supra, p. 37.

*) Noter cependant qu'une statuette de Mars ornait la « schola » des dendrophores d'Ostie dès l'an 143 : \emph{CIL.} 14, 33. Lucien, \emph{Tragodopod.} v. 30 ss, rapproche aussi les fêtes printanières de Cybèle, Dionysos, Zeus et Mars, la salpinx de Mars et les cymbales des Corybantes. Le mois du printemps est le mois d'Arès (Areios) en Thessalie.

*) Bouché-Leclercq, \emph{Astrologie grecque}, p. 343.

*) Cf. Tomaschek, dans \emph{Wien. Sitzungsber.}, 130, 1894, p. 55.

Le 24, on célèbre les suprêmes funérailles.* C'est un jour d'orgiasme et de sacrifice. La fête porte le nom caractéristique de \emph{Sanguis}, fête du Sang.* Ce jour-là, les sacrifices sanglants constituent donc l'élément essentiel de la liturgie. Mais ce sont des hosties humaines qu'exige la divinité. Chacun des célébrants fait l'oblation de son propre sang. A l'accomplissement du sacrifice participent de leur personne les prêtres, les Galles, les fanatiques. Toutes les manifestations antérieures du deuil, jeûnes, veillées, coups, hurlements, complaintes des flûtes, sanglots des cymbales, gémissements sourds des tympanons, danses tourbillonnantes des Galles, ne sont qu'une préparation à cette effusion de sang, qui est la forme dernière de l'orgiasme. La saignée s'opère à l'aide de deux instruments rituels : le fouet, composé de lanières de cuir et garni d'osselets, qui déchire le dos ; le couteau à double tranchant et à pointe acérée, qui taillade les bras et les épaules.* Cette cérémonie se passe dans l'enclos sacré, autour des autels et de l'arbre divin, en présence de la Mère douloureuse. Elle est présidée par l'Archigalle, qui le premier brandit sur lui-même l'arme liturgique et offre la libation. Pendant que l'orchestre phrygien et les ululations font rage, on asperge de sang le pin et les autels.* Un rite semblable subsistait dans les cultes de Mâ-Bellone, de la déesse syrienne et de nombreuses divinités sémitiques. Celui-ci a conservé nettement son caractère primitif de sacrifice pour les morts. C'est un très ancien rite funéraire, qui nous reporte aux temps où les morts, avides de sang, réclamaient des victimes humaines.* En apaisant la divinité, il rachète les hommes. Il est, par excellence, le sacrifice de rédemption. Le prêtre est la victime élue qui se charge de l'expiation collective. Sans doute prononçait-il en même temps certaines formules, qui proclamaient la purification des souillures et des péchés pour tous les fidèles présents. Telle devait être la matière du mandement que le sanctissime Archigalle, avant de commencer l'office et selon les prescriptions du rituel, adressait à son clergé. Mais, depuis que les rites phrygiens sont entrés dans la religion de Rome, oraisons et expiations s'étendent à tout l'Empire romain. Et le bénéfice en est destiné tout d'abord au souverain maître de l'Empire.* Le mandement de l'Archigalle renferme désormais un ordre de prière publique pour le salut de l'empereur et de la maison impériale,* pour le salut aussi du Sénat, des 15virs qui ont la surveillance du culte, de l'ordre équestre, de tout le peuple romain, de l'armée, de la marine, de l'Empire en général et de la ville de Rome en particulier. En Italie et dans les provinces, on ajoutait ainsi une oraison spéciale pour la colonie ou le municipe. Nous retrouverons ces formules de vœux dans les tauroboles, autres sacrifices sanglants qui présentent le même caractère expiatoire.*

*) Au 3e siècle, le Sénat chôme ce jour-là.

*) Avant Claude. --- Maecen. \emph{Frg.} 5 : « latus horreat flagello, comitum chorus ululet. »  
Au temps de Néron. --- Lucan. \emph{Phars.} 1, 565 : « sanguinei populis ulularunt tristia Galli. » --- Senec. \emph{Agam.} 686 ss : « non si molles comitata viros | tristis laceret bracchia tecum | quae turritae turba parenti | pectora, rauco concita buxo, | ferit ut Phrygium lugeat Attin. »  
Sous les Flaviens. --- Val. Flacc. \emph{Argon}. 8, 241 : (aux jours des Hilaries et du Bain) « quis modo tam saevos adytis fluxisse cruores | cogitet ? » --- Stat. \emph{Theb.} 10, 170 à 175 (cf. supra, p. 123, n. 5) ; 12, 224 à 227. --- Martial. 11, 84, 3.  
Aux 2e et 3e s. --- Arrian. \emph{Tactic.} 33, 4 : τὸ πένθος τὸ ἀμφὶ τῷ Ἄττῃ φρύγιον ὃν ἐν Ῥώμῃ πενθεῖται. --- Lucian. \emph{Tragodopod.} v. 31 s : Φρύγες ὀλολυγὴν τελοῦσιν Ἄττῃ, et 113 ss, où il parle des libations de sang avec le fer et le fouet ; cf. \emph{Deorum dialog.} 12. --- Min. Fel. \emph{Octav.} 22, 4 ; 24, 4 : « sanguine suo libat et vulneribus suis supplicat. » --- Tertull. \emph{Apolog.} 25 : « archigallus ille sanctissimus die nono Kalendarum Aprilium, quo sanguinem impurum lacertos quoque castrando libabat, pro salute imperatoris Marci (Aurelii) iam intercepti. » --- \emph{Adv. Marcionem} 1, 13 : « Magnam Matrem lacertis aratam. » --- Arnob. 5, 17 : « evirati isti cur more lugentium caedant cum pectoribus lacertos. » --- Trebell. Poll. \emph{Claud.} 4, 2 : « cum esset nuntiatum 9 Kal. Aprilis ipso in sacrario Matris \emph{Sanguinis die} Claudium imperatorem factum, neque cogi senatus sacrorum celebrandorum causa posset. »  
Au 4e s. --- Lactant. \emph{Div. Institut.} 1, 21, 16 : « publica illa sacra Matris, in quibus homines suis ipsi virilibus litant » ; \emph{Epitom.} 18, 4 : « quae fiunt etiam nunc Matri Magnae atque Bellonae, in quibus antistites non alieno sanguine, sed suo litant, cum amputatis genitalibus a viris migrant,... aut sectis umeris, detestabiles aras proprio cruore respergunt. » --- Julian. \emph{Or.} 5, 168 D : τῇ τρίτῃ (après la dendrophorie) τέμνεται τὸ ἱερὸν καὶ ἀπόρρητον θέρος τοῦ θεοῦ Γάλλου. --- Greg. Nazianz. \emph{Contra Julian.} 1, 70 et 103. --- Ps. Cyprian. \emph{Ad senatorem} dans Hartel, \emph{Corp. Script. Eccles. Lat.} 3, 3, p. 302, v. 19-20 : « mente fremunt lacerant corpus funduntque cruorem ; | quale sacrum est vero quod fertur nomine \emph{Sanguis}. » --- Prudent. \emph{Peristeph.} 10, 1059-1070 : « sunt sacra quando vosmet ipsi exciditis, | votivus et cum membra detruncat dolor ; | cultrum in lacertos exerit fanaticus | sectisque Matrem bracchiis placat deam | etc. --- Claudian. 18 (\emph{in Eutrop.} 1), 277-280. --- Augustin. \emph{Civ. Dei} 2, 7. --- Paulin. Nol. \emph{Carm.} 19, 87 : « nec Phryges exsectis agerent Cybeleia Gallis | impuram foedo solantes vulnere Matrem » ; 181 s : « propriisque litans furialia sacra | vulneribus. » --- Passio \emph{S. Symphoriani mart.} (Ruinart, \emph{Acta Sincera}, 2e éd. 1713), 6 : « in cuius (Berecynthiae) sacris excisas corporum vires castrati adolescentes infaustae imagini exultantes illidunt et exsecrandum facinus pro grandi sacrifîcio ducitis. »

*) V. le chapitre consacré aux Galles. Ce rite sanglant a persisté dans le christianisme de certaines régions. En Calabre, le jeudi saint, des groupes de jeunes gens courent le pays en se tailladant les chairs avec des morceaux de verre ou des lancettes ; c'est la « corsa dei San Girolami. »

*) Stat., \emph{l. c.} ; Lactant., \emph{Epitom. l. c.} Le verbe \emph{litare}, souvent employé pour désigner ce sacrifice, est d'un usage constant quand il s'agit de sacrifices humains, parce que ce sont les plus agréables aux dieux.

*) Cf. aussi dans Hérodote 7, 114, la coutume persique d'ensevelir vivants de jeunes hommes « pour le dieu que l'on dit être sous la terre. »

*) C'est pourquoi, le 24 mars 224, les « hastiferi sive pastores consistantes Kastello Mattiacorum, » sur les bords du Rhin, consacrent un autel « Numini Augusti » ; \emph{CIL.} 13, 7317.

*) Tertull., \emph{l. c.} Pour les formules tauroboliques, cf. \emph{CIL.} 14, 40 : « pro salute Imp. Caesaris M. Aureli [Antonini Aug. et] L. Aureli [Commodi Caes. et] Faustinae [Aug. Matris Castro]rum, Libe[rorumque eorum, Senatus, 15 vir. s. f., Equestr] Ordin., Ex[ercituum,...] Navigan[tium,...] Decurio[num Col. Ost.] » ; 42 : « pro salute et victoria Imp. Caes. C. V[ibi Treboniani Galli etc. et... Volusiani etc., totiusq. Domus Divin. eor(um) et Senatus, 15 vir. s. f., Equestr. Ordin., Ex[ercituum..., Na]valium Navigantium... » Les prières spéciales pour la navigation s'expliquent plus particulièrement à Ostie et dans les ports. Après la procession isiaque du \emph{Navigium Isidis}, un prêtre lisait du haut d'une chaire des vœux « principi magno senatuique et equiti totoque romano populo, nauticis, navibus, quaeque sub imperio mundi nos tratis reguntur, » cf. Apul., \emph{Met.} 11, 17. Voir aussi un type de prière chrétienne pour les empereurs, à la fin du 1er siècle, dans une lettre de Saint Clément de Rome, 1, 61 ; cf. Duchesne, \emph{Orig. du culte chrétien}, 1889, p. 51 ; 1909, p. 52.

*) Erreur de Cumont dans Pauly-Wissowa, \emph{Real-Encycl.} 2, s. v. \emph{Archigallus}, et de Hepding, \emph{Attis}, p. 165, qui supposent que les vœux de l'archigalle, ce jour-là, précèdent un taurobole. Aussi bien, les tauroboles n'étaient-ils pas accomplis au Palatin.

Les libations de sang humain n'étaient souvent que le prélude d'un sacrifice encore plus barbare, mais plus agréable encore à Cybèle et Attis. Il s'agit de la mutilation sexuelle, que pratiquaient sur eux-mêmes certains dévots fanatisés. Ceux-ci ont pris une part active aux scènes d'orgiasme. Progressivement ils se sont exaltés jusqu'au délire. Parvenus au paroxysme de l'excitation mystique, rendus insensibles à la douleur par leur frénésie, ils saisissent un tesson de poterie, un couteau rituel de terre cuite ou de silex et immolent leur virilité sur l'autel d'Attis. Leur offrande les consacre pour toujours à la divinité qui la reçoit. Ce sont désormais des Galles, des Saints, des Purs. L'« autocastration » volontaire est le degré suprême de la cathartique phrygienne. Elle réalise la communion parfaite entre l'homme et le dieu. Mais, à l'origine, le rite ne comportait pas cette idée de perfection et de sainteté. Comme les autres effusions de sang, et sous une forme moins atténuée, il est un sacrifice de rançon. C'est pourquoi il intervient dans les fêtes funéraires du \emph{Sanguis}. « Le troisième jour après avoir coupé l'arbre sacré, » dit expressément l'empereur Julien, « on coupe la moisson sainte et ineffable du dieu Gallos » ; dans le langage des mystères, on désignait ainsi cette scène de sauvagerie. Un autel, dédié à la Grande Mère dans son temple de Lectoure, relate la mutilation sacrificielle d'un certain Eutychès en 239 ; le fait s'était passé le 24 mars.* La répugnante « moisson» fit-elle, en principe, interdite au Palatin et dans les temples romains ? Nous ne connaissons aucun décret impérial contre l'eunuchisme qui soit antérieur aux Flaviens* ; encore n'est-il jamais question de l'eunuchisme sacré. Les Romains, à vrai dire, paraissent avoir envisagé ces pratiques religieuses avec plus d'indifférence que d'aversion. Le temps n'était pas très lointain où, pour détourner les plus terribles menaces de leurs dieux, ils immolaient des victimes humaines. Ils n'ignoraient pas que, dans certaines provinces, subsistait la coutume des sacrifices humains.* Eux-mêmes ne se refusaient pas toujours à y participer de leur présence ; et Trajan, à Antioche, donna l'exemple.*

*) \emph{CIL.} 13, 510.

*) Cf. supra, p. 75, n. 2.

*) Si Claude prétend abolir les sacrifices humains des Druides, cf. dans les cultes syriens : Lucian., \emph{Dea Syr.} 58 (sacrifices d'enfants, du haut des propylées du temple d'Atargatis) ; Clermont-Ganneau, \emph{Recueil d'archéol. or.} 2, 1898, p. 74 (sacr. d'enfants dans le Harrân) : cf. Dussaud, \emph{Sacrifices humains chez les Cananéens}, dans \emph{Conférences du Musée Guimet}, 1910, p. 77 ss : Gauekler, \emph{Le sanctuaire syrien du Janicule} dans \emph{C. r. Acad. Inscr.}, 1910, p. 389 (calotte crânienne sectionnée, relique sous la statue du dieu). Sur la découverte de crânes d'enfants dans les « adyta » païens d'Alexandrie, lors de leur destruction par les chrétiens au 4e s., cf. les textes réunis \emph{ibid.}, p. 392.

*) Il y renouvela le rite de la création du génie poliade en immolant une cune fille appelée Calliope, à qui l'on éleva, dans le théâtre, une statue avec les attributs de la Tychè de la ville : J. Malala, \emph{Chron.} 11. N'eût-il l'ait qu'assister à l'immolation, le fait n'en serait pas moins à signaler.

Après les sacrifices en l'honneur d'Attis, avait lieu le dernier acte de ses funérailles, qui était la sépulture. Une légende rapportée par Servius fait allusion à ce rite : Attis fut trouvé sous le pin, non par Cybèle, mais par ses prêtres, qui le transportèrent dans le temple et l'ensevelirent.* On dénommait cette cérémonie \emph{Catabase},* ce qui ne peut signifier que la descente au tombeau. Le pin-Attis était donc descendu dans le caveau du temple : tel Osiris au sépulcre, dans les « adyta » d'Isis. Il y demeurait, au dire de Firmicus Maternus,* jusqu'à l'année suivante. On le transformait alors en un bûcher, que l'on brûlait. Mais nous ne savons pas quel jour était choisi pour cette crémation.

*) Ad \emph{Aen.} 9, 115.

*) Macrob., \emph{Saturn.} 1, 21 ; cf. Julian, \emph{l. c.}, p. 171 A. Il y avait à Éleusis un Catabasion, où descendaient le hiérophante et la prêtresse.

*) \emph{Op. l.} 27, 2 ; cf. 8, 3 : « favillas. » Dans le culte de Proserpine, l'arbre était brûlé la 40e nuit.

A la tombée de la nuit, commençait la grande veillée* ou \emph{Pannychis}, qui précédait immédiatement le retour d'Attis à la vie. Elle parachevait la sanctification des fidèles et disposait leur âme aux mystiques joies du lendemain. Ils s'y étaient préparés déjà, non seulement par leurs exercices de piété au lieu saint, mais aussi par un jeûne plus strict. Ce dernier jour, toute nourriture « lourde et grossière » était proscrite.* La loi religieuse n'autorisait plus que le lait, sans doute aussi le miel, aliments des enfants, aliments des dieux, nourriture à la fois symbolique et sacramentelle des néophytes, qui renaissent à une vie nouvelle et divine. A quel moment le prêtre annonce-t-il qu'Attis, revenu du royaume des morts, « est présent » ? A minuit ? Le \emph{Mesonyctium} joue en effet un rôle dans certains rites métroaques.* Mais cet office de la Divine Présence, \emph{Parousia}, correspondait plutôt, ce semble, aux premières lueurs de l'aurore, puisqu'Attis s'est identifié au soleil. Sur un lit de parade, cette fois peut-être à l'intérieur du temple, aux pieds de la Mère, gisait le dieu. La première partie de la nuit se passait en oraisons, en cantiques de lamentation, en processions aux flambeaux, pendant lesquelles on appelait Attis. Après un temps de ténèbres, une lumière brillait au fond du sanctuaire. C'était le signal de la Résurrection attendue. Le prêtre exorcisait alors les assistants par un sacrement d'onction, qu'ils recevaient sur la gorge. Puis il révélait le grand mystère. Il confirmait la bonne nouvelle du triomphe sur la mort, promesse des béatitudes réservées à ceux qui croient en Attis. A voix basse, lentement, il murmurait la formule grecque des livres liturgiques : « Ayez foi, mystes, le dieu est sauvé ! Pour nous aussi, de nos épreuves viendra le salut.* » Et la foule répondait en grec : « Nous sommes tous en joie.* » Ainsi débutait la fête des \emph{Hilaria}.

*) Cf. dans les mystères de Dionysos : Rhode, \emph{Psyche} 2, 2, p. 9. C'est probablement cette Pannychis qui est signalée dans Hérodote, 4, 76, comme l'une des grandes fêtes métroaques de Cyzique au 6e siècle.

*) Sallust. phil., \emph{l. c.}

*) \emph{CIL.} 13, 1751 (dédicace taurobolique).

*) Firm. Mat., \emph{op. l.} 22, 1 : « Nocte simulacrum in lectica supinum ponitur et per numeros plangitur ; deinde... lumen infertur ; tune a sacerdote omnium qui flebant fauces unguuntur, quibus perunctis sacerdos hoc lento murmure susurrat :  
Θαρρεῖτε μύσται τοῦ Θεοῦ σεσωσμένου  
῎εσται γὰρ ἡμῖν ἐκ πόνων σωτηρία. »

*) Συγχαίρομεν, dans le culte d'Isis ; cf. Hepding, \emph{Attis}, p. 167.

Le 25 mars est, pour les anciens, le premier jour que le soleil rend plus long que la nuit. Les Hilaries de la Mère des Dieux* sont la fête du soleil et du printemps. C'est pourquoi elles devinrent si vite populaires à Rome et dans tout l'Empire. La liturgie du jour comportait, dans 1 enceinte du temple, des sacrifices d'actions de grâces à la Dame de Salut et au Dieu sauvé. Mais le principal attrait de la fête est dans la procession des saintes images. Plus tard, dès le 2e siècle, cette pompe religieuse sera la plus belle de Rome. Le préfet de la ville et les hauts fonctionnaires se rendront au Palatin en carrosses de gala, pour prendre rang au cortège. On y verra figurer l'armée, le Sénat, la Cour, l'Empereur. Devant l'Idéenne on portera des œuvres d'art, de magnifiques torchères, des vases de prix, des statues. Les Césars prêteront les pièces les plus rares des collections impériales ; et les familles riches ne laisseront point échapper cette occasion d'étaler leur opulence.* Ce sera vraiment le cortège fastueux d'un triomphe. Au 1er siècle, la solennité reste purement phrygienne. On n'y voit guère participer que les mystes et le clergé. Mais ils y manifestent toute l'exubérance de leur foi. Aussi bruyante est leur gaîté que l'avait été leur douleur. Aux tristes ululements ont succédé les acclamations joyeuses et les hymnes d'allégresse. « Attis est ressuscité* ! Attis évohé* ! » On salue en Attis la splendeur d'une nouvelle lumière* qui va briller sur le monde et revivifier la nature.

*) Val. Flacc. \emph{Argon.}, 8, 240 : « Laetaque iam Cybele festaeque per oppida taedae. » --- Lucian., \emph{Amores}, 42 : τὸν δυσέρωτα κῶμον ἐπὶ τῷ ποιμένι. --- \emph{Tragodopod.}, v. 35 : (après les fêtes de deuil) κῶμον βοῶσι Λυδοί. | Παραπλῆγες δ᾽ ἀμφὶ ῥόπτροις | κελαδοῦσι... | νόμον Kορυβάντες εὐάν. --- \emph{De sacrif.}, 7 : (Rhéa) τὸν Ἄττιν ἐπὶ τῶν λεόντων περιφέρουσα. --- Herodian. 1, 10. 5. : ἦρος ἀρχῇ, Μητρὶ Θεῶν πομπὴν τελοῦσι Ῥωμαῖοι (suit la description delà procession, à propos du complot de Maternus contre Commode). --- Arnob. 7, 34 : « Arbitrantur et numina ex rebus hilarioribus gaudere, » cf. 44. --- Vopisc., \emph{Aurelian.} 1, 1 : « Hilaribus quibus omnia festa et fieri debere scimus et dici, impletis solemnibus vehiculo suo me et iudiciali carpento praef. Urbis, etc. » --- Lamprid., \emph{Alex. Sev.} 37, 6 : « Adhibebatur... Hilariis Matris Deum fasianus. » --- Firm. Mat., \emph{op. l.} 3, 4 : « Ut gaudeas plangis » ; cf. 22, 3. --- Julian., \emph{op. l.} 168 D : Ἱλάρια καὶ ἑορταί ; cf. 169 D et 175 A. --- Sallust. phil., \emph{l. c.} Ἱλαρεῖαι καὶ στέφανοι. --- Macrob., \emph{l. c.} : « Catabasi finita simulationeque luctus peracta, celebratur laetitiae exordium a. d. octavum Kalendas Aprilis, quem diem Hilaria appellant, quo primum tempore sol diem longiorem nocte protendit. » --- Dionys. Areop., \emph{Epist.} 8, 6, et S. Maximi \emph{schol.} (Migne, \emph{Patrol. gr.}, 4, p. 320) : Ἱλάρια, ἑορτὴ ἰδικὴ Ῥωμαίων εἰς τιμὴν τῆς Mητρὸς τῶν Θεῶν. --- Damasc. \emph{Vita Isidori}, dans Photius, \emph{Biblioth.} cod. 242, éd. Migne, p. 1281 ; éd. Bekker, p. 345 A : étant à Hiérapolis, dit-il, je rêvai que j'étais devenu Attis, et que pour moi la Mère des Dieux célébrait (ou faisait célébrer) la fête dite des Hilaries ; ce songe prouvait que nous étions sauvés des enfers, ἐξ ἅδου σωτηρίαν.

*) Herodian., \emph{l. c.}

*) Firm. Mat., \emph{op. l.} 3, 1 : « Revixisse iactarunt. »

*) Lucian., \emph{Tragodopod.}, \emph{l. c.} ; Procl., \emph{Hymn. in Solem}, 23 : εὔιον Ἄττην. Peut-être aussi Socr., \emph{Hist. eccl.} 3, 23 ; cf. Hepding, \emph{Attis}, p. 72.

*) Firm. Mat, \emph{op. l.} 19 (\emph{Poet. Lyr. Gr.} 3, 4, p. 658, éd. Bergk) : χαΐρε, νύμφιε, χαὶρε νέον φῶς.

En l'honneur du maître des astres, on porte des lampes, des torches, des cierges.* Mais ce sont aussi des flambeaux d'hyménée que tiennent les Lampadophores. Car les fêtes de sa Résurrection sont en même temps les fêtes de sa Hiérogamie.* Et les ovations s'adressent au jeune fiancé. « Salut, Nymphios ! » Sur le passage du couple divin, des enfants et des jeunes filles en robes blanches répandent des fleurs.* Les Compagnons Danseurs exécutent des pas rythmés. Hommes et femmes ont le front ceint de couronnes.* Tout vêtement de deuil est interdit.* Beaucoup de fidèles, par vœu ou par plaisir, sont travestis et masqués.*

*) « Le berger des astres blancs, » dit un hymne ; Hippolyt., \emph{Refut. omn. haeres.} 5, 9. De même, à la procession du \emph{Navigium Isidis}, dans Apul., \emph{Met.} 11, 9 : « Magnus sexus utriusque numerus lucernis taedis cereis et alio genere facium, lumine siderum cælestium stirpem propitiantes. »

*) Eutecnii Sophistae \emph{Comment}, in Nicandri \emph{Alexipharm.} (cf. Hepding, \emph{Attis}, p. 9) : τὰ τῆς Ῥέας ὄργια, ὅ τε τοῦ Ἄττεω γάμος καὶ τὰ ἐπὶ τούτοις, ὅσα τελεῖται. Cf. Arnob. 4, 29 : « Matrimonium Magna cuius tenuerit Mater. » D'autre part, Denys l'Aréopagite, \emph{l. c.}, dit que l'on qualifiait aussi d'Hilaries les jours de mariage. Sur le caractère hiérogamique des Hilaries métroaques, cf. Réville, dans \emph{Rev. Hist. Religions}, 1901, 1, p. 191.

*) Cf. Lucret. 2, 626 ss (supra, p. 106) et la procession d'Isis dans Apulée.

*) Sallust. philos., \emph{l. c.} ; cf. Apul., \emph{l. c.} : « Verno florentes coronamine. »

*) Dionys. Areop., \emph{l. c.}

*) Herodian., \emph{l. c.} Caractère votif des travestissements : « votivis studiis, » dans Apul., \emph{l. c.}, 8. Sur le rite des mascarades sacrées, les pages de Lobeck, \emph{Aglaophamus}, 1, 173 ss, sont encore intéressantes ; mais il faut les compléter par celles de Robertson Smith, \emph{Religion of the Semites}, 2e éd., p. 436 ss.

Ces mascarades, qui constituent l'élément orgiastique des Hilaries, mêlent aux pieuses réjouissances l'aimable folie d'un carnaval. Elles sont la survivance d'un rite religieux, qui fut à l'origine un rite magique. Peut-être n'en comprenait-on déjà plus la véritable signification. Chacun avait, en effet, l'entière liberté de se composer un personnage à sa fantaisie.* C'est là, semble-t-il, une dégénérescence de l'orgiasme primordial. Dans un culte où persistent tant de souvenirs d'une préhistorique zoolâtrie, certains déguisements zoomorphes représentent probablement la plus ancienne tradition.* D'autres dévots se métamorphosent en compagnons mythiques de Cybèle. Ceux-ci se conforment, en réalité, aux mêmes croyances. Pour peu qu'ils soient instruits des choses divines, ils pensent entrer en communion plus intime avec leurs dieux. Nous retrouvons une idée similaire dans les cérémonies d'initiation, où le néophyte prend les attributs divins, nouvel Attis qui va s'unir à la Dame dans de mystiques hyménées. Aux Hilaries de 187, Maternus et ses complices, voulant profiter des mascarades pour assassiner l'empereur Commode, s'étaient travestis en doryphores.* Ces doryphores ne seraient-ils pas des Corybantes de Cybèle, armés de la lance et du bouclier, coiffés du casque à longue aigrette ? Chaque année, la Grande Mère avait sans doute auprès d'elle sa garde guerrière de Corybantes.* D'autre part, tout travestissement est une préservation contre les mauvais démons Ce fut cette idée, sans doute, qui finit par autoriser les déguisements les plus divers. En ce jour de liesse, et après les purifications des journées antérieures, il importe d'éloigner les esprits malfaisants et impurs. Pendant cette fête hiérogamique, le myste a pour devoir de les écarter non seulement de sa personne, mais surtout du couple dont il célèbre les noces. De même aux Anthestéries d'Athènes, la hiérogamie de Dionysos et de la Basilissa s'accompagnait de mascarades.* Enfin il ne paraît pas douteux que, dans ces fêtes du renouveau, la procession des masques se combine avec un vieux rite de purification agraire.*

*) De même dans la procession d'Isis, dont Apulée nous a laissé une pittoresque description. Sur les déguisements des Saturnales, cf. \emph{Rev. de philolog.}, 1897, p. 447 s.

*) Cf. dans le culte mithriaque, Cumont, \emph{Mithra}, 1, p. 316.

*) Herodian., \emph{l. c.} On pourrait supposer aussi, tout simplement, qu'ils s'étaient déguisés en soldats. Dans la langue grecque de l'époque impériale, le mot « doryphores » désigne souvent les prétoriens.

*) Dans Lucien, \emph{l. c.}, les Corybantes qui crient Evohé le jour du « Cômos » de Cybèle et d'Attis ; Greg. Nazianz., \emph{Orat.} 39, 4, dans Migne, \emph{Patr. Gr.} 36, p. 337 ; cf. la procession de Notre-Dame du Bérécynthe à Autun, dans \emph{Passio S. Symphoriani}, \emph{l. c.} : « Ubi cymbala pulsate Corybantes. » D'autre part, sur la danse des armes dans la démonologie primitive, v. Gruppe, \emph{op. l.}, p. 898 s ; cf. supra, p. 6.

*) Cf. dans le culte d'une Aphrodite attique, dans une fête hiérogamique de Samos et dans certains rites de mariage, Gruppe, \emph{op. l.}, p. 903.

*) Cf. le carnaval maghrébien : Doutté. \emph{Magie et Religion dans l'Afrique du Nord}, 1909, chap. 11.

Mais voici les musiciens, joueurs de double flûte, cymbalières et joueuses de tambourin ; et voici les chantres, sous la direction du premier hymnologue. Ils précèdent le clergé de la Grande Mère. Des diaconesses portent les vases sacrés.* Prêtres et prêtresses, vêtus de blanc, couronnés d'or, le bras paré de l'occabos sacerdotal, tiennent des rameaux verts* et les attributs divins.* L'Archigalle, qui généralement les domine tous de sa taille,* est en pallium de pourpre. Sous la couronne d'or, où brillent en des médaillons les images de ses Tout-Puissants, son visage s'encadre du voile blanc et des bandelettes perlées de consécration. Dans ses mains il tient peut-être la ciste mystique, enguirlandée de verdure, tabernacle « qui renferme les secrets de la grande religion.* » Des Galles lui font escorte. Leur longue chevelure, laissée inculte aux jours de deuil, est aujourd'hui calamistrée et re-blondie. Sur leurs robes de couleur, brillent des bijoux, des médailles saintes, des pendentifs en forme de petites chapelles, où s'enchâssent les effigies des Dieux. De leurs mains molles ils frappent les peaux des tambourins ; et leur voix glapissante clame les louanges d'Attis. Devant Cybèle on promène sans doute les statues d'autres divinités, qui font partie du cycle métroaque et sont adorées dans les mystères : Zeus Idéen, le dieu père, à la barbe chenue ; Hermès, qui guide vers la Grande Mère les âmes des morts ; Bellone, que l'on appelle la « Suivante » de la Mère ; Minerve, confondue parfois avec la Dame du Bérécynthe ; Dionysos, disciple de Cybèle et qui joue un rôle dans le mythe pessinontien ; Silvain, que vénèrent les Dendrophores. Voici enfin, traîné par des hommes,* le quadrige de la Mère des Dieux,* la droite de la déesse consolée, a pris place le dieu ressuscité. Cybèle, couronnée de tours, revêtue sans doute de robes somptueuses,* le sceptre royal en main, regarde son Attis. Celui-ci, coiffé du bonnet phrygien, portant le bâton pastoral à crosse, tient les rênes.* Le char triomphal est tout plaqué d'argent et garni de somptueuses draperies ; il est attelé de lions d'argent. Au galop des fauves, dans un rayonnement de lumière, il emporte le couple divin vers l'empyrée.

*) Cernophores, Phialéphores (\emph{CIA.} 2, 1, 624), peut-être aussi Canistrariae, comme dans le culte de la Caelestis.

*) Sur le port du thyrse, cf. Anonyme de 394, vers 72 : « Sumere thyrsos... imbuerat... Berecynthia mater. » et les Galles « amis du thyrse » dans Callimaque (v. supra, p. 101). Prêtres tenant des rameaux : archigalle, à Rome, Musée du Capitole, cf. Helbig, \emph{Guide}, 1, n° 425 ; cistophore de Bellone. \emph{CIL.} 6, 2233. Cybèle était parfois aussi figurée avec un rameau en main, cf. Froehner, Médaillons rom., p. 97 (Lucilla), 108 (Faustina Jun.) : Cohen, 2e éd., 4, pp. 115-117 (Domna), 153 (Caracalla), 388 (Soemias).

*) Cf. dans Apulée : « Antistites sacrorum potentissimorum deum proferebant insignes exuvias. » Une fresque de Pompéi, restée en place, représente la procession de la Vénus Pompeiana ; deux hommes portent sur les épaules une sorte de civière (ferculum), sur laquelle est posée un bonnet phrygien ; deux autres tiennent un trône, orné de tiges feuillues et recouvert d'un coussin bleu, où est posée la couronne de la déesse.

*) Cf. Juven. 6, 511 : « ingens. »

*) L'archigalle capitolin a près de lui une ciste, entourée d'une couronne de verdure ; cf. aussi la ciste votive, en marbre, d'un archigalle d'Ostie, au musée du Lateran. Prêtres cistophores de Bellone. Dans la procession isiaque, Apul., \emph{l. c.} 2 : « Ferebatur ab alio (antistite) cista secretorum capax, penitus celans operta magnificae religionis. »

*) Sur ce rite, cf. une intaille du Musée de Berlin, publiée par Furtwaengler, \emph{Beschreib. d. geschnitt. Steine im Antiquarium Berl.}, p. 283, n° 7656 ; elle représente quatre hommes traînant un char à quatre roues, sur lequel est placé un édicule avec idole.

*) Anonyme de 394 (cf. supra, p. 122, n. 3), 103-107 : « Vidimus argente faclo(s ? ) iuga ferre leones, | lignea cum traherent iuncti stridentia plaustra, | dextram issam ( ? deam ? ) laevaque argentea frena tenere, | egregios proceres currum servare Cybellae, | quem traheret conducta manus. » Cf. en Espagne, les chars plaqués d'argent, auxquels sont attelés le lion et le bœuf des évangélistes, et que l'on promène à la procession du Corpus Dei.

*) Cf. Fulgent., \emph{Myth.} 3, 5: « multiplici veste, » à propos de Berecynthia. A Cyzique, des femmes étaient spécialement chargées de la parure de la Mêter Plakianè, \emph{CIG.} 3657.

*) Lucian., \emph{De sacrif.} 7 ; Rhéa promenant Attis sur son char de lions. Nonnius, \emph{Dionys.} 25, 311, montre Attis conduisant le char de lions. Julien, \emph{op. l.}, pp. 166 C et 169 D, dit que Cybèle oblige Attis à se tourner vers elle et à la regarder ; allusion à leur attitude iconographique. --- Le char apparaît sur une série de médaillons contorniates. Cybèle et Attis sur un qnadrige de lions au galop. 1. Attis, le pedum dans la main droite, assis à droite de la déesse qui le regarde et tient les rênes : en haut, deux signes du zodiaque : Cohen, 2e éd., 8, p. 287, n°s 103-105. --- 2. Cybèle tient une haste dans la dr., et Attis le pedum dans la g. ; tous deux ramènent l'autre main sur la poitrine : \emph{ibid.}, p. 281, n° 63, et p. 297, n° 197 ; cf. Daremberg et Saglio, \emph{Dict. des antiq.} 1, 2, s. v. \emph{Cybélé}, p. 1689, fig. 2251. --- 3. Cybèle tient le sceptre et regarde Attis. qui tient les rênes ; en haut, deux signes du zodiaque, dont le bélier : Cohen, p. 318, n° 362 ; Dar. et Saglio, fig. 2252. --- Avec la même signification hiérogamique, Cybèle debout, posant la main g. sur l'épaule d'Attis debout, et tenant la main dr. du dieu dans la sienne : Cohen, p. 300, n° 214. Reproduits aussi par Ch. Robert, \emph{Les phases du mythe de C. et A. rappelées par les méd. cont.}, dans \emph{Rev. numism.} 1885, pl. 3, 7 ; 4, 1 ; 5, 1-6.

Le long du parcours, la Dame s'arrête peut-être à des reposoirs, comme le faisait Isis, pour permettre d'offrir à la vénération publique les objets sacrés. Ou bien elle visite ses autres sanctuaires, comme le faisait Dionysos aux Dionysies d'Athènes. Quand elle est remontée au Palatin avec son cortège et que les statues ont été déposées dans le temple selon les rites, un dernier office groupe autour d'elle toutes ses ouailles. C'est probablement un Salut, analogue à celui qui terminait la procession isiaque du 5 mars. L'Archigalle ou l'un des prêtres y lisait une formule de prière pour l'empereur et sa maison, le Sénat et le peuple romain, l'armée et la marine, tout ce qui compose l'empire, avec mention spéciale des 15virs.* Après avoir défilé devant les saintes images, chacun regagne ses pénates ; et cette journée s'achève en joyeux banquets, où l'on fait bonne chère.*

*) Cf. supra, p. 128. Apulée ajoute que les fidèles, en défilant devant Isis, baisaient les pieds de la statue d'argent dressée sur les degrés du temple.

*) Lamprid., \emph{l. c.} ; cf. Dionys. Areop., \emph{l. c.}, disant que l'on passait en longs festins les jours d'Hilaries. Le « cômos » dont parle Lucien, \emph{ll. cc.}, est précisément un festin accompagné de chants et de danses ; ce mot implique la même idée de liesse que celui d'Hilaries.

Un jour de repos férié, \emph{Requietio}, sépare les Hilaries de la fête du Bain, \emph{Lavatio} (27 mars). La dernière des fêtes phrygiennes, comme la première, nous ramène au culte primitif des sources et des eaux vives. En reprenant le chemin qui mène au ruisseau, à l'étang, à la fontaine, la déesse retourne vers ses origines lointaines. D'autre part, l'immersion des idoles se retrouve souvent comme rite de magie par action sympathique ; c'est un rite agraire, qui a pour vertu d'attirer la pluie.* Mais des siècles d'anthropomorphisme en ont généralement modifié le sens. La Lavation de Cybèle est devenue le complément nécessaire de sa hiérogamie.* Après ses noces mystiques, la Dame se purifie par le bain. De même, après son union avec Zeus, la Héra de Syrie était baignée dans la source Burras, et celle d'Argos dans la source Canathos ; après son union avec Poséidon, la Demeter Lousia des Thelpusiens, en Arcadie, était baignée dans le Ladon. Aphrodite aussi et Adonis étaient annuellement conduits au bain lustral. Dans les cultes métroaques d'Asie Mineure, le rite purificateur avait lieu tantôt dans un cours d'eau, comme à Pessinonte,* tantôt dans le bassin d'une source ou dans un étang, comme à Ancyre,* parfois même au bord de la mer, comme à Cyzique.* A Rome, l'humble ruisseau de l'Almo remplaça le Gallos pessinontien.* Les Romains connaissaient depuis longtemps cette solennité. Elle était aussi ancienne chez eux que le culte de l'Idéenne. Mais le gouvernement de la République et celui d'Auguste, tout en se faisant un scrupule religieux de la tolérer, ne l'avaient pas inscrite au calendrier officiel. Et c'était sans éclat ni faste, malgré la présence d'une délégation des Quindécemvirs, que jadis la Dame Noire se rendait au lieu fixé.

*) Sur ce rite, qui paraît être l'un des plus anciens, des plus répandus et des plus tenaces, cf. Gruppe, \emph{op. l.}, p. 821. Il se retrouvait en Germanie dans le culte de Nerthus, identifiée par les Romains à la Terre : Tacit., \emph{Germ.} 40. Il avait passé dans la religion chrétienne. Les humanistes de la Renaissance y voyaient une survivance du rite phrygien ; Andreas Fulvius, 1, \emph{De Ostia Tiber}. (1545) : « qui lavandi mos servatur hodie Romae in lavandis pedibus imaginis Salvatoris, dum gestatur per Urbem mense Augusti » ; cf. Martinelli, \emph{Roma ex ethnica sacra}, p. 157. Jean Bodin, \emph{Démonomanie}, 2, 8 : « la coustume de traîner les crucifix et images en la rivière, pour avoir la pluye, se pratique encore en Gascogne et l'ai veu faire à Thoulouse » ; cf. Dumège, \emph{Archéol. pyrénéenne}, 2, 1860, p. 322 : immersion de la croix dans la Garonne, le 3e jour des Rogations. De Belleforest, \emph{Histoires prodigieuses} 6, p. 1187 (cité par Dumège, \emph{ibid.}, p. 321) : « j'ai apprins de plusieurs personnes dignes de foy lesquelles estant allées aux pieds des monts Pyrénées, au lieu où sont les Fontaines d'où prend sa source le fleuve de Garonne, que si dans ces fontaines on trempe certaines images de saincts, faictes de bois, attachées près desdites fontaines, soubdainement s'élèvent de grands orages suyvis de grands ravages de pluies. » Autres exemples dans Frazer, \emph{Rameau d'or}, tr. Stiébel, 1, 1903, p. 119 ss ; cf. le bain des prêtres, attirant la pluie, p. 101 ss, et les pierres de pluie, p. 116.

*) Idée mise en valeur par Hepding, \emph{Attis}, p. 216, qui se refuse à reconnaître ici un rite agraire. Il est plus vraisemblable qu'il y a eu juxtaposition des deux idées. D'autre part, le bain qui suit la hiérogamie n'a-t-il que des propriétés lustrales ? Chez certaines populations d'Afrique, le bain de mer ou de rivière passe pour avoir des vertus fécondantes ; cf. \emph{Rev. Hist. Religions}, 1904, 2, p. 77.

*) Herodian. 1, 11, 2 : ὠργίαζον ἐπὶ τῷ ποταμῷ Γάλλῳ παραορέοντι.

*) \emph{Passio S. Theodoti et septem virginum} (Ruinart, \emph{Acta sincera}, 2e éd. p. 342) : « iussit eas (virgines) fieri Dianae atque Minervae sacerdotes, ut quotannis iuxta morem lavarent earum simulacra in vicino lacu. » Cette Artémis et cette Minerve sont d'autres aspects de Cybèle (cf. la Minerve Bérécynthienne de Bénévent).

*) Des prêtresses spéciales, comme à Ancyre, comme les Loutrophores d'Aphrodite à Sicyone, y sont chargées de ce rite : on les appelait ici les Maritimes, Thalassiai : \emph{CIG.} 3657.

*) Avant le principat de Claude. --- \emph{CIL.} 6, 2305, 2306 (Menologia Rusticana) ; Ovid., \emph{Fast.} 4, 337-346 ; cf. supra, pp. 76 et 92.  
Sous Néron. --- Lucan., \emph{Phars.} 1, 600 : (15 viri) « lotam parvo revocant Almone Cybelen. »  
Sous les Flaviens. --- Val. Flacc., \emph{Argon.} 8, 239 s : « Mygdonios planctus sacer abluit Almo | laetaque iam Cybele festaeque per oppida taedae. » --- Sil. Ital., \emph{Pun.} 8, 363 : « tepidoque fovent Almone Cybeben. » --- Stat., \emph{Silv.} 5, 1, 222 ss : « est locus ante Urbem, qua primum nascitur ingens | Appia quaque Italo gemitus Almone Cybebe | ponit et Idaeos iam non reminiscitur amnes. » --- Martial. 3, 47, 2 : (Capena porta) « Phrygium... Matris Almo qua lavat ferrum. »  
Au 2e s. --- Arrian., \emph{Tactic.} 33, 4 : (à Rome) τὸ λουτρὸν ἢ Ῥέα, ἀφ᾽ οὐ τοῦ πένθους λήγει, Φρυγῶν νόμῳ λοῦται.  
Au 3e s. --- Tertull., \emph{Adv. Marcionem} 1, 13 : « Magnam Matrem lavacris rigatam. »  
Au 4e s. : Arnob. 7, 32 : « Lavatio, inquit, Deum Matris est hodie. Sordescunt enim divi et ad sordes eluendas laventibus aquis opus atque adiuncta aliqua cineris frictione » ; cf. 34 : « Superis ducunt lavationum esse munditias gratas. » --- Ambros., \emph{Epist. primae classis} 18, 31 : « currus suos simulato Almonis influmine lavat Cybele » (lettre écrite en 384). --- Vibius Sequester, \emph{De flumin.} (\emph{Geogr. Lat. Min.}, éd. Riese, p. 146) : « Almon Romae ubi Mater Deum 6 Kalendas Apriles lavatur. » --- Amm. Marc. 23, 3, 7 : « diem sextum Kal. Apriles, quo Romae Matri Deorum pompae celebrantur annales, et carpentum, quo vehitur simulacrum, Almonis undis ablui perhibetur. » --- Prudent., \emph{Peristeph.} 10, 154 à 160 : « nudare plantas ante carpentum scio | proceres togatos Matris Ideae sacris. | Lapis nigellus evehendus essedo | muliebris oris clausus argento sedet ; | quem dum ad lavacrum praeeundo ducitis, pedes remotis atterentes calceis | Almonis usque pervenitis rivulum. » --- Claudian., 15 (\emph{de bello Gildonico}, 1), 119 : « praelatoque lavas Phrygios Almone leones. » --- Augustin., \emph{Civ. Dei} 2, 4 : (à Carthage) « Berecynthiae Matri omnium, ante cuiuslecticam die sollemni lavationis eius talia per publicum cantitabantur a nequissimis scaenicis, qualia... nec matrem ipsorum scaenicorum deberet audire. Illam proinde turpitudinem obscenorum dictorum atque factorum, quam per publicum agebant coram Deum Matre, spectante atque audiente utriusque sexus frequentissima multitu line... ; et haec fercula appellabantur. » --- C'est probablement là fête de la Lavation, à Autun, qui est signalée dans \emph{Passio S. Symphoriani} (Ruinart, \emph{l. c.}), 2 : « statuam Berecynthiae quae carpento portabatur agminibus stipata populorum, » et dans Greg. Tur., \emph{Ingloriam confessorum}, 76 : « Berecynthiam cum in carpento pro salvatione agrorum ac vinearum deferrent, adfuit Simplicius episcopus, haud procul adspiciens cantantes atque saltantes ante hoc simulachrum » ; il est dit plus loin que le « plaustrum » est traîné par des bœufs : « iubeat boves procedere. » --- A Ancyre, bain d'Artémis et de Minerve \emph{Passio S. Theodoti}, cf. supra, p. 137, n. 3 : « vehebantur et idola ; ...inter haec audire erat et videre tibiarum ac cymbalorum sonum choreasque mulierum solutis crinibus maenadum instar bacchantium ; multus autem excitabatur strepitus pedum terram plaudentium et musicorum instrumentorum concrepatio. »

Dans la matinée du 27, un nouveau cortège se prépare devant le sanctuaire de la déesse. Les prêtres vont chercher la statue cultuelle, icône d'argent dont la pierre sacrée forme le visage. Ils la déposent sur un char, que tirent deux génisses fleuries de guirlandes. Dans la Rome d'Auguste, ce véhicule était un simple chariot, \emph{plaustrum}. Plus tard, selon l'usage adopté pour les processions religieuses et les pompes funèbres, ce fut un \emph{carpentum}, carrosse couvert en berceau.* Celui de la Mère des Dieux est richement orné de pompons, qui ont la forme symbolique de pommes de pin.* A Sétif, où il était remisé dans une dépendance des thermes, Dendrophores et Religieux se chargeaient de son entretien. A Carthage, on plaçait la « Berecynthia Mater » sur une \emph{lectica}, qui est un siège à porteurs.

*) Il est possible que le « plaustrum » dont parle Ovide soit un « carpentum » ; cf. cette confusion dans Grégoire de Tours et la confusion du « carpentum » et de l'« essedum » dans Prudence. Servius, ad Virg. \emph{Georg.} 1, 163 : (Eleusinae matris plaustra), « id est, qualibus Mater Deum colitur. » A Sétif \emph{CIL.} 8, 8457, (année 288) : « carpentum. » Sur ces types de véhicules, cf. Saglio dans Daremberg et Saglio, \emph{Dict. des antiq.}, s. v. \emph{carpentum} ; Lafaye \emph{ibid.}, s. vv. \emph{essedum et plaustrum}.

*) « Ad therm[as depositum ? ]... carpenti capisteliis et strobilis vellereis exornatum dono dediderunt » : cf. Graillot, dans \emph{Rev. archéol.}, 1904, 1, p. 351 s.

Fidèles, confréries et clergé précèdent le char. Les dévots qui veulent se conformera la loi rituelle marchent pieds nus.* Le collège quindécemviral participe officiellement à la fête romaine. Descendue du Palatin, la Dame Noire prend le chemin qui longe le Caelius ; puis elle franchit la porte Capena, d'où part la voie Appienne. A travers la campagne suburbaine qui commence à fleurir, sous le soleil printanier qui égaie le marbre des innombrables tombeaux, elle se dirige d'une marche rapide* vers l'Almo. Elle y possédait sans doute un petit sanctuaire. On fait entrer dans l'eau le carrosse. L'Archigalle, revêtu de son manteau pourpre, baigne l'idole et la frotte avec de la cendre, pour enlever toute souillure.* On lave ensuite et l'on purifie de même sorte les lions de la déesse, le carpentum, le char d'argent des Hilaries,* les couteaux qui ont servi aux libations sanglantes des Galles,* les vases sacrés et divers objets du culte. Pendant ce temps, il y a des chants et de la musique. Lorsque ces rites sont accomplis, on célèbre probablement un sacrifice ; et les Quindécemvirs prononcent une formule d'oraison, sollicitant la Grande Mère de vouloir bien rentrer dans la ville.* Au retour, on jette des fleurs à la déesse.* On chante et l'on danse. Les chants de la Lavation, ainsi que les pantomimes dont ils s'accompagnent, manifestent un caractère licencieux. De pareilles coutumes existaient dans d'autres cultes. Il s'agit évidemment de rites apotropaïques.*

*) Prudent., \emph{l. c.} ; sur ce rite dans les cultes de Demeter, d'une Athéna anatolienne, d'Isis, \emph{etc.}, v. Gruppe, \emph{op. l.}, p. 912, n. 7 ; cf. Hepding, \emph{Attis}, p. 174, n. 3. Défense d'entrer avec des chaussures dans le Metrôon d'Eresos.

*) Cf. la course à la mer, pendant les Eleusinies, et Arnob. 5, 16 : « sacra quae per cursus annuos factitatis. »

*) Arnob., \emph{l. c.} ; ne serait-ce pas avec la cendre du pin sacré, qui, nous l'avons vu, était brûlé ?

*) Ambros., \emph{l. c.}, dit « currus suos. »

*) Martial., \emph{l. c.} De même à Tarse, on l'avait tous les ans dans les eaux du Cydnus le couteau sacré d'Apollon.

*) Lucan., \emph{l. c.}, interprété par le Comm. Bernens. : « cum ibi lavisset Mater Deum dicitur reverti nisi rogata noluisse. » Un commentateur de Prudence (ms. pal. 1715, au Vatican, 11e s.), à \emph{Peristeph.} 10, 156, dit que l'on célébrait au bord de l'Almo « ludos et sacrificia » ; \emph{Americ. Journ. of Archaeol.}, 1900, p. 296. A rapprocher du texte de S. Augustin, signalant la présence de « scaenici. »

*) Ovid., \emph{l. c.}, 346. A prendre à la lettre ce texte, c'est seulement au retour que l'on pare de Heurs les génisses et le char.

*) Cf. Hepding, \emph{Attis}, p. 175 ; Foucart, \emph{Gr. Mystères d'Eleusis}, p. 105

La Lavation terminait les fêtes de mars. Huit jours plus tard, les vieilles Mégalésies, propres à la ville de Rome, ramenaient les fidèles aux pieds de l'Idéenne. Ils reviennent cette fois avec des plats de \emph{moretum}, qu'ils déposent en offrande sur les tables sacrées, suivant l'antique usage. A vrai dire, ces fériés d'avril, où sacrifices et mutilations n'ont point perdu leur caractère aristocratique, n'offrent guère au public que battrait des Jeux. Mais on attendait ces Jeux avec impatience. Car ils inauguraient la saison théâtrale. Depuis la mi-novembre, en effet, on était privé des longues suites de spectacles.* Ils comptent parmi les plus importants, puisqu'ils se prolongent pendant sept jours.* Ils sont parmi les plus beaux. Le préteur, qui en a depuis Auguste la présidence et la charge, n'y dépense pas moins de cent mille sesterces* ; le trésor impérial contribue également à la somptuosité de ces réjouissances populaires.* Au théâtre, le mythe de Cybèle fournit la matière indispensable de drames lyriques. On joue la Passion d'Attis. Des acteurs revêtus de longues robes, la cithare en main, « chantent les grands mystères sans y rien comprendre.* » Tous les ans, la Grande Mère vient assister, selon le rite, à l'exhibition de ses amours passionnées pour le jeune berger. Et la foule accourt, non pour rendre hommage aux dieux, mais pour applaudir dans les rôles divins ses artistes favoris.* Le 10 avril, jour anniversaire de la dédicace du temple, toute Rome se porte au Cirque Maxime.* Ainsi, pendant près d'un mois, la Grande Mère occupe le calendrier romain. Aucune autre divinité n'y tient une telle place. On réserva pour les fêtes d'avril la dénomination de \emph{Megalesia}, consacrée par la coutume. Il ne semble pas que l'on ait désigné celles de mars, comme en Orient et en Grèce, sous le nom de fêtes d'Attis, \emph{Attideia}.* On les qualifie volontiers de \emph{Dindyma}.* Car la déesse de Pessinonte est la Dindymène ; et c'est sur le mont Dindyme, voisin de la ville sainte, que l'on montrait aux pèlerins la sépulture d'Attis.

*) Juven. 6, 67-69 : \emph{a plebeis longe Megalesia}. Les Jeux Plébéiens, les derniers de l'année, finissaient le 17 novembre.

*) Peut-être six seulement, quand on y eut intercalé des jeux du Cirque le 8, pour le \emph{Natalis} de Castor et Pollux ; cf. Wissowa, \emph{op. l.}, pp. 391 et 499.

*) Martial. 10, 41, 5-6 (année 98).

*) C'est à quoi semble faire allusion, à partir d'Hadrien, toute une série de monnaies impériales. On y voit Cybèle assise sur un lion ; c'est le type constant des statues de la déesse qui ornaient la spina des cirques (cf. Rémy, \emph{La statue équestre de Cyb. dans les Cirques rom.}, dans le \emph{Musée Belge}, 11, 1907, pp. 245-265). Cf. Cohen, 2e éd., 2, Sabine, n° 88 ; Faustine l'aînée, n°s 267, 304 (= Grueber et Stuart Poole, \emph{Roman medallions in the Br. M.}, p. 12, 1 et pl. 17, 1), 305 (= G. et P., p. 12, 5 et pl. 17, 2 ; Froehner, \emph{Médaillons rom.}, p. 79). Il est possible aussi que certaines médailles, où Cybèle figure sans Attis, sur un quadrige de lions, fasse allusion à la « pompa circensis» : cf. Cohen, \emph{ibid.}, Hadrien. n° 284 (= Froehner, p. 34) ; Antonin, n° 1139 (= Froehner, p. 73).

*) Hippolyt., \emph{op. l.} 5, 9. Il cite ensuite deux fragments d'hymnes sur le « polymorphe » Attis. Néron ne dédaigna pas démonter en scène pour chanter, en s'accompagnant de la cithare, un « Attis » de sa composition ; Dio Cass., \emph{Hist. Rom.} 61, 20. Perse y fait allusion, au dire d'un scholiaste, \emph{Sat.} 1, 93.

*) Arnob. 7, 33 : « lenior Mater Magna efficitur, si Attidis conspexerit priscam refricari ab histrionibus fabulam » ; cf. 4, 35 : « saltatur et Magna sacris compta cum infulis Mater et in bubulci unius amplexum flagitiosa fingitur adpetitione gestire » ; 5, 42 : « Attis, quem in spectaculis ludicris theatra universa noverunt. » --- Tertull. \emph{Apol.}, 15 : « Cybele pastorem suspirat fastidiosum, non erubescentibus vobis ; vidimus aliquando castratum Attin » ; et \emph{Ad nationes}, 1, 10, où il dit que ces spectacles succédaient souvent à des combats de gladiateurs : « super sanguinem humanum sallant dei vestri. » --- Essai de reconstitution d'un scénario : Ch. Robert, \emph{Les phases du mythe}, \emph{l. c.} --- En dehors de Rome, les jeux de la Grande Mère étaient sans doute célébrés aux fêtes de mars, ou encore au jour anniversaire de la fondation d'un temple. Sur la présence de l'image cultuelle à ces spectacles, cf. supra, p 86.

*) Juven. 11, 197 : « totam hodie Romain circus capit. »

*) \emph{CIA.} 2, 622 (2e s. avant J. C.) ; cf. Foucart, \emph{Assoc. rel.}, pp. 52 et 196.

*) Columell. 10, 220 ; \emph{CIL.} 6, 10098 = \emph{Carm. Lat. Epigr.}, éd. Buecheler, 1110 ; Min. Fel., \emph{Octav.} 22, 4 : « Cybelae Dindyma » : Auson., \emph{Ep.} 25 \emph{ad Paulin.} 16 ; cf. Martial. 8, 81, 1 : « mystica sacra Dindymenes » ; Arnob. 4, 35 : « ilia Pessinuntia Dindymene » ; \emph{CIL.} 6, 1779 = \emph{Carm. L. Ep.} 111, 1. 26 : « Dindymenes Atteosque antistitem. » On désigne également ces fêtes sous le nom de Cybeleia : Paul. Nol., \emph{Carm.} 19, 87.

\subsection{3.}

Adopté par le Sénat sur l'initiative de Claude, le culte nouveau fut organisé par les Quindécemvirs, seuls compétents en matière de cultes étrangers. Ils ne purent que se conformer, pour la détermination des observances festales, à la tradition anatolienne. La signification symbolique de chaque cérémonie ne permettait d'en modifier ni l'ordre ni la date ; l'ancienne Lavation avait déjà lieu le 27 mars. Mais ils eurent à réviser et à compléter l'eucologe phrygien. Désormais, en effet, chaque office public doit comporter des prières pour le salut de l'empereur et de l'Empire.

Ils avaient déjà la haute main sur le clergé. Dès le temps d'Auguste, le nombre des prêtres suppose une hiérarchie acceptée par l'État. Si le Phrygien et la Phrygienne d'autrefois n'étaient que de simples desservants, sans droit légal au titre de prêtres, leurs successeurs sont maintenant un archiprêtre et une archiprêtresse. Le grand-prêtre, \emph{Sacerdos Phryx Maximus}, conserva son titre étranger d'Archigalle, que le Sénat reconnut et qui devint officiel. Il fut l'Archigalle romain. Cette épithète spécifiait que sa suprématie diocésaine était limitée au territoire de la ville de Rome. Il s'intitula parfois Archigalle du peuple romain ; ce qui n'avait pas en réalité d'autre sens, mais pouvait prêter à confusion. On lui laissa prendre aussi, en sa qualité de métropolitain et par privilège spécial, le titre théophore d'Attis. Selon la tradition établie par Auguste, tout ce clergé romain de la Grande Mère se compose de créatures de l'empereur. Il appartient à l'aristocratie de la classe libertine. Une prêtresse, affranchie de Claude et mariée à un affranchi impérial, paraît occuper une importante situation de fortune.* Ce sont les 15virs qui nomment les titulaires des sacerdoces vacants et accomplissent la formalité de l'investiture.* Enfin le fonctionnement normal du culte exigeait, nous l'avons vu, J.e concours de certaines confréries. Par sénatus-consulte, elles furent reconnues d'utilité publique et admises au nombre des collèges licites.* Rome posséda désormais, à titre officiel, ses Cannophores et ses Dendrophores de la Mère des Dieux, Grande, Idéenne, et d'Attis.* Les Dendrophores romains fêtent l'anniversaire de leur fondation au 1er août,* qui est en même temps le jour anniversaire de la naissance de Claude. Ils se recrutent, comme le clergé, dans le monde des affranchis ; et l'on voit parmi eux, surtout parmi leurs dignitaires, beaucoup d'affranchis impériaux.* Mais les confréries orientalo-grecques de l'Arbre et du Roseau, pour entrer dans la cité, durent se conformer au type des autres associations romaines. De même elles furent soumises, et plus étroitement peut-être que les autres, à la surveillance de l'État. Elles demeurèrent sous la protection et dépendance immédiate des 15virs, qui paraissent avoir exercé sur leurs actes et leurs finances une véritable curatelle.

*) \emph{CIL.} 6, 2260.

*) V. le chapitre consacré au clergé.

*) \emph{Ibid.} 6, 29691 ; cf. 1925.

*) Cagnat, \emph{Ann. épigr.} 1891, n° 59. Époque des Antonins.

*) \emph{CIL.} 6, 29691 (année 206) : « Kal(endis) Aug(ustis) N(atali) Collegi(i). » Le personnage qui donna de l'argent aux confrères pour fêter cette date porte précisément le nom de Claude et parait descendre d'un affranchi de la maison Claudienne.

*) \emph{Ibid.} : « Ti(berius) Claudius Chresymus. » Orelli, 4412 = Waltzing, 1377 : « Eutycheti Caes(aris) n(ostri) liberto » ; sous Trajan.

En gardant ainsi la haute direction du nouveau culte, le Quindécemvirat restait fidèle à son rôle politique et religieux. Mais son domaine s'est agrandi, comme la cité même. Si, à cette date, un culte du peuple romain n'est pas nécessairement un culte de l'empire romain, il est cependant plus qu'un culte de Rome. Car si l'Empire ne forme pas encore un peuple unique et une cité universelle, le peuple romain n'est pas seulement dans Rome ; il est partout où l'on trouve une colonie romaine et partout où la population a reçu le droit de cité. Quand donc, à l'instar de la métropole, une ville de citoyens romains adopte le culte phrygien, il y jouit des mêmes privilèges, mais il y est astreint aux mêmes sujétions. Tout décret municipal qui porte nomination d'un prêtre ou d'une prêtresse de la Mère des Dieux doit être contresigné par la chancellerie quindécemvirale. Celle-ci exerce également son contrôle sur les confréries métroaques. En Italie, en Gaule, où Lyon et presque toutes les villes de la Narbonnaise possèdent le titre de colonies, nous voyons appliquer ce régime. Nous ignorons s'il fut étendu à tout l'Empire.

Rome remplaçait Pessinonte comme capitale de la religion phrygienne. A vrai dire, le rôle historique de Pessinonte est terminé. La dignité des Attis n'est plus qu'un bénéfice ecclésiastique, réparti entre les membres d'un sacré collège. Les 15virs, par une conséquence imprévue, mais logique, de leurs attributions, se sont substitués aux Attis comme chefs suprêmes de cette religion. Mais, à Pessinonte, ces chefs étaient prêtres encore plus que rois et siégeaient dans un temple. A Rome, ils sont fonctionnaires encore plus que prêtres et siègent dans les bureaux d'une administration impériale.

\subsection{4.}

On ne peut dire que la réforme de Claude passa inaperçue, bien qu'aucun écrivain du haut Empire ne la signale comme un des événements notoires de ce principat. Mais elle ne bouleversa point tout d'un coup les habitudes religieuses de Rome. D'abord elle semblait n'introduire aucune divinité nouvelle. Ce n'est pas au nom d'Attis, c'est sur l'ordre de la Mère des Dieux, \emph{ex imperio Matris Deum}, que vaticinent les Archigalles. Les dévots n'osent pas encore se proclamer les mystes d'Attis ; ils se déclarent simplement les adorateurs de la Mère des Dieux, \emph{cultores Matris Deum}.* Attis, prince consort, se dissimulé dans la grande ombre de la dame vénérée. D'autre part, cette réforme se bornait en apparence à régulariser une situation acquise. Déjà le culte phrygien était dans la ville un culte secret d'affranchis et de femmes. Il reste, après son adoption, un culte d'initiés. Il est en majeure partie constitué par des offices qui n'intéressent que des initiés. A voir Cybèle et Attis s'enfermer avec leurs élus dans le mystère du temple, sur le Palatin, croirait-on que leur geste de bénédiction va s'étendre sur les sept collines et qu'ils sont devenus des dieux de Rome ? Les vrais citoyens ne parlent encore de ce qui se passe dans les \emph{secreta Palatia Matris}* qu'avec un mépris ironique ou attristé. Pour eux, comme pour les purs Hellènes dont Plutarque se fait l'interprète, il eût mieux valu laisser aux Barbares toutes ces superstitions ; « Attis usurpe des honneurs qu'il ne méritait pas.* » Aussi, pendant longtemps, garde-t-il la même clientèle qu'autrefois. Ces gens que l'on voit se frapper la poitrine avec des pommes de pin et marcher pieds nus devant les idoles, une torche en main, ce sont toujours des affranchis et des esclaves.* Ils continuent à se grouper en confréries de piétistes. Frères en Attis, ils forment une sainte famille, unie dans la vie et par-delà la mort. Car le myste défunt renaît, comme Attis, à une vie nouvelle et meilleure ; et les morts ne sont pas oubliés des vivants, qui espèrent les retrouver au séjour de félicité. Les confrères ont leur fête des trépassés, fixée au jour où meurt le dieu. Quand ils ont pleuré Attis dans le temple, ils se répandent dans la campagne et vont visiter les tombes qui leur sont chères. Ils y suspendent, comme à l'arbre-dieu, des guirlandes et des couronnes de violettes. La nuit, dans le silence des nécropoles, on les entend gémir et prier.* Leur fanatisme n'admet pas volontiers que Rome reste indifférente à leur douleur. Ils prétendent que le temps de leur deuil soit pour tous néfaste. Funeste présage pour Othon, d'avoir entrepris sa campagne contre Vitellius un 22 mars* ! Mais bien-tôt Rome, si fidèle au culte de ses morts, adoptera la pieuse et touchante coutume qui attire, ce jour-là, les dévots d'Attis vers les tombeaux. Bien des Romains, qui n'ont pas l'habitude d'assister à l'\emph{Arbor Intrat} et de se lamenter autour du pin, iront célébrer dans l'intimité des enclos funéraires la fête de la Violette.*

*) Sueton., \emph{Oth.} 8.

*) Juven. 9, 23.

*) Plut., \emph{Amator.} 13, p. 756 C ; cf. Lucien qualifiant Attis de métèque, \emph{Icaromen}. 27, et faisant protester Momus et Mercure contre l'invasion de tous ces barbares qui s'appellent Attis, Corybas, Sabazios, Mithra, Anubis : \emph{Deorum Concil.} 9, \emph{Iuppiter trag.} 8.

*) Encore au temps d'Antonin, Artemidoros déclare que la dendrophorie est une fête d'esclaves : \emph{Oneirocrit.} 2, 42.

*) \emph{CIL.} 6, 10098 = \emph{Carm. Lat. Epigr.} 1110. Epitaphe d'un certain Hector de Phrygie, appartenant à la domesticité de Domitilla. S'agit-il de la Domitilla qui fut femme de Flavius Clemens, consul en 95, et qui fut exilée par Domitien à cause de ses mœurs judaïques, c'est-à-dire probablement chrétiennes ? Une autre Flavia Domitilla fut aussi exilée. Voir les textes dans Marucchi, \emph{Guide des catacombes rom.}, 1900, p. 104. En tout cas, il s'agit ici d'une proche parente de Domitien. Elle avait donné un terrain funéraire aux gens de sa maison (cf. vers 9 et 10). A la même époque, il est fait allusion au mythe d'Attis sur une autre épitaphe, \emph{CIL.} 6, 21521 = \emph{Carm. L. Ep.} 1109.

*) En l'an 69 ; Sueton., \emph{l. c.}

*) On appelait ce jour \emph{Dies Violae ou Dies Violationis ou Dies Violaris} ; \emph{CIL.} 6, 10234 (année 153), 10239, 10248. Sur la consécration de la violette à Cybèle, cf. 1° le rôle de la violette dans la dendrophorie ; --- 2° le mythe de Pessinonte dans Arnob. 5, 7 : du sang d'Attis naît la violette ; la fiancée d'Attis se nomme la (violette) ; quand elle se tue, son sang se métamorphose en violettes ; --- 3° le nom de Kυβέλιον donné à la violette, Dioscorid. 4, 122 ; --- 4° \emph{CIL.} 6, 9118 = \emph{Carm. L. Epigr.} 467, v. 5 : « ia tibi Cybeles sint » (épitaphe d'affranchie) ; cf. \emph{Anthol. Gr.} 7, 222.

Ce sont les femmes, sans doute, qui ont introduit cet usage dans les familles romaines. Elles avaient toujours été de ferventes adoratrices de la Grande Mère, déesse féconde et nourricière.* Elles furent vite acquises à la religion de Cybèle et Attis, dieux consolateurs, dieux sauveurs, dieux omnipotents. Pour les séduire, le culte phrygien ne leur offre pas seulement le spectacle pathétique de ses fêtes annuelles, qui éveille dans leur âme une mysticité à la fois attendrie et passionnée. Il les invite à des offices probablement quotidiens, dont la fréquentation stimule le zèle pieux, à des pratiques souvent renouvelées, qui apaisent l'âme inquiète, la fortifient contre les épreuves de la vie présente et entretiennent la douceur des espoirs infinis. Il les admet dans la sainte milice de ses confréries. Bien plus, il les appelle aux dignités sacerdotales. Il les fait monter à tous les degrés de la hiérarchie ecclésiastique, de pair avec les hommes. Elles ne retrouvaient pas dans tous les cultes mystiques pareil privilège.* Aussi, dès la fin du Ier siècle, les mystères métroaques ont-ils pris, avec ceux d'Isis, une place prépondérante dans la vie religieuse des femmes. Elles ne jurent plus que par Isis et Cybèle.* La Bonne Déesse elle-même, restée si chère aux Romaines, n'a pu survivre à l'invasion orientale qu'en accentuant son caractère orgiastique.* Les clientes du couple phrygien, comme celles d'Isis, n'étaient tout d'abord que des esclaves, des affranchies, des pérégrines, des plébéiennes. Mais, dans le gynécée, les matrones sont en contact incessant avec les femmes de leur domesticité, microcosme d'Orient, où l'élément phrygien est nombreux.* Elles assistent à des prédications enflammées. Elles entendent parler de miracles. Plus d'un fut convertie à la religion d'Attis, comme à celle du Christ, par la parole d'une servante.* On les vit plus souvent gravir la montée du Palatin qui mène au temple de l'Idéenne. On en rencontrait sans cesse, et non des moindres, qui rôdaient autour du sanctuaire, au risque de compromettre leur réputation.* On les vit aussi passer assidûment le Tibre et, par-delà les jardins d'Agrippine, gagner le Phrygianum du Vatican ; c'est là que s'accomplissaient les cérémonies d'initiation. Enfin elles reçoivent volontiers chez elles le Très Saint Archigalle, vénérable vieillard à cheveux blancs, et son cortège de jeunes Galles, dont la face peinte s'encadre d'une longue chevelure, artistement calamistrée. Au son des tambourins et des cymbales, qui chasse les mauvais esprits, elles font purifier la maison, les gens de la maison, leurs propres enfants. Elles se font imposer des pénitences pour leurs péchés et donner des remèdes contre la fièvre. Elles écoutent avec une religieuse terreur le grand-prêtre, dont la parole est prophétique, et comblent de cadeaux, en argent ou en nature, ses acolytes.* C'est ainsi qu'Attis, après avoir conquis le Palatin, achève la conquête de Rome. La dévotion féminine lui fut une puissante auxiliaire, surtout au 11e siècle, quand il eut gagné à sa cause les impératrices.*

*) Encore au 2e s., Plutarque, \emph{l. c.}, et Lucien, \emph{Amores}, 42, déclarent que les femmes constituent la principale clientèle d'Attis ; cf. Jamblic. \emph{De myster.} 3, 10. ; \emph{Orac. Sibyll.} 3, 143, où Cybèle est dite « déesse des femmes. »

*) Elles étaient exclues, non seulement du clergé mithriaque, mais encore, ce semble, de toutes les cérémonies secrètes de la liturgie persique.

*) Martial. 8, 81, 1s : « per mystica sacra Dindymenes... iurat Gellia » ; cf. Lucian., \emph{Pseudologist.} 2 : « dis-moi par la Pandémos, les Génétyllides et Kybébè. »

*) Juven. 6, 314 ss ; cf. 2, 86 ss. Sur les relations de Mater Deum et de Bona Dea, cf. supra p. 71. Bona Dea est aussi la mère féconde qui fait mûrir l'épi, et la Dame de Santé, dont les prêtresses exercent la médecine.

*) Les indigènes vendaient leurs enfants aux marchands galates qui faisaient la traite.

*) Cf. p. 145, n. 3.

*) Juven. 9, 23 ss.

*) Juven. 6, 510-521 ; cf. 2, 112 ss.

*) Agrippine, la femme de Claude, était-elle déjà une dévote de Cybèle ? Cf. supra, p. 114, n. 3, le passage de Suétone où il est question de son influence sur l'empereur en matière religieuse. Une inscr. de Cos la qualifie, après sa mort, de Sebastè Rhéa : Paton et Hicks, \emph{Inscr. of Cos}, 1891, n° 119.

Les successeurs immédiats de Claude ne manifestèrent pas, ce semble, un zèle particulier à l'égard des dieux phrygiens. Toutefois Néron glorifiait Attis dans ses galliambes, qu'il déclamait, la cithare en main, devant le peuple et les soldats. Et peut-être se passionna-t-il un temps pour le culte de Pessinonte, ce prince superstitieux, épris de l'Orient, qui se fit initier aux mystères de la déesse Syrienne,* qui fut affilié à ceux de Mithra,* qui favorisa ceux d'Isis.* Les Flaviens, parvenus à l'Empire grâce à l'appui de l'Orient, se montrèrent volontiers favorables aux idées et aux religions orientales. D'autre part, Vespasien pouvait-il oublier qu'il était né sous la protection de Rhéa-Cybèle* ? Aussi le voyons-nous prendre à sa charge, en 76, la restauration du temple métroaque d'Herculanum, ruiné par un tremblement de terre.* Julie, fille de Titus, fut sans doute une dévote de Cybèle et d'Isis ; car, après son apothéose, on la représentait avec les attributs de ces deux « Reines » célestes.* Hadrien, qui s'intéressa si généreusement à la prospérité de l'Asie Mineure, surtout de la Bithynie, domaine de la Mère des Dieux, qui rebâtit Cyzique, ville consacrée à la Dindymène, qui s'empressa de faire construire un Metrôon dans sa colonie mysienne d'Hadrianoi,* ne devint pas seulement en Orient, et par politique, un zélateur de la grande déesse anatolienne. Dès le début de son principat, il donne à Cybèle des témoignages publics de sa piété.* Peut-être même était-il depuis longtemps un fervent adepte des mystères phrygiens. Sous une apparence de scepticisme il cachait, en effet, une superstition immodérée, qu'Ammien Marcellin* compare à celle de l'empereur Julien. Quant à l'impératrice Sabine, elle réunissait dans une même vénération la Dame de Pessinonte et celle d'Hiérapolis,* qu elle avait appris à connaître en Syrie même, au temps où son mari était gouverneur de la province. Un fait prouve la situation florissante de l'église métroaque de Rome sous Hadrien. Les Dendrophores ont désormais pour siège la Basilique Hilarienne, qu'un de leurs présidents, M' Poblicius Hilarus, riche marchand de perles, vient de faire construire sur le Cælius.* En 137, année qui précéda la mort du prince, le Deuil autour d'Attis et le Bain de Cybèle sont déjà considérés comme de grandes fêtes romaines.*

*) Sueton., \emph{Nero}, 56 ; mais il n'osa pas, dit-on, se faire initier à Éleusis, d'où les criminels étaient exclus par la voix du héraut, \emph{ibid.} 34.

*) Cf. Cumont, \emph{Mithra}, pp. 24 et 479.

*) Textes dans Lafaye, \emph{Culte des div. d'Alexandrie}, 1884, p. 59.

*) A Reate (Rieti) en Sabine, considérée comme la ville de Rhéa, et qui s'était vouée à la Mère des Dieux ; Sil. Ital., \emph{Punic.} 8, 415.

*) \emph{CIL.} 10. 1406.

*) Ainsi peut s'expliquer qu'une petite ville du Samnium ait, par mesure d'économie, confié le triple culte de Iulia Pia Augusta, de Mater Deum Magna Idaea et d'Isis Regina, à la même prêtresse : \emph{CIL.} 9, 1153. Sur Isis et les Flaviens, Lafaye, \emph{op. l.}, p. 60 ss.

*) Monnaie de cette ville, au type d'Hadrien et avec l'image de Cybèle au revers : Babelon, \emph{Coll. Waddington}, 828.

*) Cf. des monnaies de son 3e consulat (en 119) : Cohen, 2e éd., 2, p. 129, Hadrien 283 : Cybèle trônant ; 284 = Froehner, \emph{Méd. rom.}, p. 34 et Stevenson, \emph{Diction. of roman coins}, p. 300 : Cybèle sur un quadrige de lions, allusion aux processions de mars ou à la « pompa circensis » des Megalesia.

*) Amm. 25, 4. Julien est précisément un dévot de Cybèle.

*) Dans la numismatique de Sabine, Cybèle est assise tantôt, selon le type ordinaire, sur un trône (Cohen, \emph{l. c.}, p. 250. n° 35), tantôt, comme Atargatis et Cælestis, sur un lion (n° 88).

*) \emph{CIL.} 6, 30973.

*) Arrian., \emph{Tactic.} 33, 4 ; œuvre composée en 137.
\clearpage
\section{Chapitre 4}
\begin{center}
Tauroboles et Mystères. Diffusion du Taurobole au 2e Siècle.
\end{center}
\paragraph{}
\emph{Meus iste sanguis verus est, non bubulus}. (Prudent., \emph{Peristeph.} 10, 1007.)

1. Dévotion d'Antonin et de Faustine à « Mater Deum Salutaris. » Apparition d'Attis dans la numismatique impériale. Les Antonins protecteurs du culte, du clergé, des confréries. Romanisation du taurobole pour le salut des empereurs. --- 2. Taurobole et criobole. Origine magique du rite. Le « bolos. » Sacrifices et sacrements. Affusion de sang, consécration des « vires » et du bucrâne. Idée de purification subordonnée à l'idée de rachat. Le baptême régénérateur. --- 3. Taurobole public pour le salut de l'empereur et de l'empire. Caractère oriental de cette institution. La Mère des Dieux et la Fortune des Augustes. La Mère des Dieux et la Fortune des villes. Taurobole et loyalisme à l'égard du prince et de Rome. --- 4. Taurobole d'ordre privé. Sacrifice d'expiation « ex imperio, » « ex voto. » Sacrement d'initiation majeure. Rôle des femmes dans la diffusion du rite. --- 5. Les mystères. Préoccupation de la vie future. Époque des initiations. Épreuves préparatoires. Sacrements de baptême, de communion, de confirmation. L'intronisation du myste. Les révélations suprêmes.

\subsection{1.}

C'est de l'époque antonine que date, à vrai dire, l'extraordinaire fortune des dieux de Phrygie. Antonin et Faustine s'en firent les principaux artisans. Le culte le plus cher aux affranchis impériaux était devenu l'un des cultes de prédilection de la maison impériale. Après avoir envahi les dépendances du palais, il s'était installé en maître dans les appartements privés des princes. L'empereur et l'impératrice l'aidèrent à conquérir l'Empire.

Sur leur dévotion à l'Idéenne, la numismatique nous fournit d'importants témoignages. Nombreux sont les bronzes de Faustine qui, frappés de son vivant, montrent au revers l'image de Cybèle.* Après sa mort, les médailles de consécration la figurent elle-même parée des attributs métroaques, trouant sur le bige ou le quadrige de lions* : la Diva n'avait pu monter au ciel que sur le char de la Mère des Dieux. Très attachée, comme le prince, aux souvenirs religieux de l'ancienne Rome, elle avait fait reproduire sur plusieurs médaillons la scène miraculeuse d'Ostie.* Mais sa piété ne pouvait plus se contenter de la religion d'autrefois. Deux autres médaillons représentent Attis à côté de Cybèle.* C'est la première apparition du dieu phrygien dans la numismatique des Césars. Attis a gardé son costume barbare, sa syrinx et sa houlette. Il se tient debout, à gauche de la déesse, qui trône entre ses lions. Il se tourne vers celle dont l'amour Ta sauvé et qui, sur son char, Ta conduit au divin séjour. Il la contemple, pour indiquer aux mystes qu'elle doit être l'unique objet de toutes leurs pensées.* Sa petite taille, qui contraste avec la majesté surhumaine de la Mère des Dieux et des hommes, rappelle à ces derniers qu'il vécut, comme eux, d'une vie mortelle et qu'ils peuvent tous, comme lui, participer à l'immortalité bienheureuse. L'une des médailles porte en exergue une dédicace à MATER DEUM SALUTARIS. Cette formule, qui se retrouve sur d'autres grands bronzes de Faustine,* est empruntée à la terminologie religieuse de l'Orient. Elle est familière aux dévots qui demandent à leur déesse le chemin du salut dans ce monde et dans l'autre. L'Augusta certainement fut une initiée. Derrière le groupe cultuel, se dresse la colonnade corinthienne d'un temple. Il y a tout lieu de croire que Faustine consacra dans Rome, sous le vocable de Notre Dame du Salut, un nouveau sanctuaire de la Mère des Dieux. Un médaillon à l'effigie d'Antonin, daté de Tannée 158, révèle chez l'empereur la même dévotion. Il témoigne aussi que cette dévotion a pris un caractère national. Dans le groupe allégorique du revers, deux divinités protègent César et Rome, qui ont la main dans la main ; malgré l'absence des lions, ces dieux paraissent être Cybèle et Attis.* La figuration du couple phrygien persista dans la numismatique de Faustine la jeune et de Lucille* ; car la religion de Pessinonte restait en honneur dans la maison antonine. En 191, Commode fit frapper des médailles en argent et en bronze au type de Cybèle, avec dédicace à MATER DEUM CONSERVATRIX AUGUSTI.* Deux années auparavant, le jour même des Hilaria, la déesse avait en effet sauvé le prince, que les sicaires de Maternus voulaient assassiner pendant la procession.*

*) Cohen, 2e éd. 2, pp. 417-439, Faustina senior 55, 56 (avec dessin), 126, 229, 230, 267, 304-307.

*) \emph{Ibid.} 55, 56, avec la devise \emph{Aeternitas S. C.} ; Cybèle emprunte également les traits de Faustine sur le grand bronze 126, \emph{Augusta S. C.} Cette forme d'adulation était adoptée en Asie Mineure dès le début de l'Empire. A Tiberiopolis de Phrygie, Livie semble avoir été identifiée à la M. d. D. : Ramsay, \emph{Histor. Geogr.}, p. 147 ; cf. à Cos, Sebastè Rhéa, qui paraît être Agrippine ; supra, p. 147, n. 5.

*) Cohen, \emph{l. c.}, 307 ; Froehner, \emph{Méd. rom.}, p. 78 ; Stevenson, \emph{Dict. of rom. coins}, p. 211. La numismatique d'Antonin et de Faustine s'inspire souvent des fastes de l'ancienne religion romaine.

*) Donaldson, \emph{Archit. numism.}, pp. 83-85 et pl. 21 ; Stevenson, \emph{op. l.}, p. 542 ; Cohen, \emph{l. c.}, 306.

*) Cf. supra, p. 135, n. 4.

*) Avec la statue de Cybèle au revers : Cohen, \emph{l. c.}, 229 et 230.

*) Cabinet de France. Cohen, \emph{l. c.}, p. 369, Anlonin 1029, avec dessin. Cybèle ( ? ) est debout, sans les lions ; Atys, plus petit, porte le costume phrygien et tient le pedum ? (corne d'abondance, d'après Cohen). La déesse dans le quadrige de lions ligure sur un magnifique médaillon d'Antonin : Cohen, \emph{l. c.}, p. 382, n° 1139 ; Froehner, \emph{op. l.}, p. 73.

*) Froehner, \emph{op. l.}, pp. 97 et 108 ; Stevenson, \emph{op. l.}, p. 300. Un coin de fer, destiné à la frappe d'une médaille de Faustine et trouvé à Lyon, porte au revers l'image de Cybèle assise et la dédicace \emph{Matri Magnae} : Allmer et Dissard, \emph{Inscr. ant. du musée de Lyon}, 1, p. 25. Autre médaillon de Lucille avec Cybèle sur le lion : Froehner, \emph{op. l.}, p. 97 : Grueber et St. Poole, \emph{Roman Med. in the Brit. Mus.}, 1874, p. 12, pl. 36, 2.

*) Cohen, 3, p. 274, n°s 354 et 355, avec dessin ; cf. Milani, \emph{Studi e mater}. 1, p. 55, fig. 3. Sur le médaillon de Faustine la jeune où figurent Cybèle et Attis (Froehner, p. 97), il est assez vraisemblable que Commode est identifié au jeune dieu, de même que sa mère est identifiée à la déesse ; cf. sur d'autres médaillons et sur des monnaies son assimilation à Jupiter enfant, à Jupiter Juvenis, à Hercule.

*) Herodian. 1, 10, 5-7.

La piété des Augustes se manifestait avec éclat au temps des fêtes.* Elle en accrut la magnificence. Le 25 mars en particulier, jour des Hilaria qui étaient les Pâques phrygiennes, l'empereur, toute la famille impériale, la cour, le Sénat, les fonctionnaires, les troupes prennent part à la procession. Devant le carrosse de la Dame, on promène les plus beaux et les plus rares objets du trésor des Césars. A l'exemple du maître, les riches Romains font défiler dans le cortège triomphal des merveilles d'art et de luxe, orgueil de leurs maisons.* Les Hilaries dépassent ou tout au moins égalent en splendeur les plus belles fêtes de Rome. Les Dindymes rejettent au second plan les Mégalésies. Il semble que la somptuosité des Jeux d'avril ne soit plus que le complément indispensable des réjouissances de mars. Le clergé bénéficia, comme le culte, des pieux sentiments d'Antonin et de Faustine. En attachant un privilège aux vaticinations de l'Archigalle, le prince confirmait le caractère officiel et la haute situation de ce prophète.* Il combla aussi de sa protection les confréries de l'Arbre et du Roseau. Son patronage immédiat s'étendit sur les Dendrophores et les Cannophores d'Ostie, en souvenir du miracle. Dès l'année 139, les Dendrophores de la colonie ont l'occasion de témoigner à l'empereur leur reconnaissance* ; ils sont alors très prospères. Plus tard, quand ils agrandissent ou restaurent leur local, ils le dédient au \emph{Numen} de la Maison Auguste* ; celle-ci leur avait probablement accordé une subvention. Dans d'autres villes, comme Lyon, les confrères furent autorisés à porter le titre d'Augustaux, qui leur conféra certains avantages. Enfin ce fut Antonin, selon toute vraisemblance, qui romanisa définitivement le criobole et le taurobole.

*) Le Sénat et le Peuple Romain expriment leur reconnaissance envers Antonin « ob insignem in caerimonias publicas curam ac religionem. » \emph{CIL.} 6, 1001.

*) Ces détails nous sont donnés dans le récit d'Hérodien et nous reportent au temps de Commode. Il est possible que la somptuosité des fêtes se soit encore accrue sous cet empereur. Commode avait pour favori un Phrygien, ancien esclave, du nom de Cléandre, qui devint son chambellan et dont il fit un général.

*) \emph{Fragm. Iur. Rom. Vatic.} 148 : « is qui in portu pro salute imperatoris sacrum facit ex vaticinatione archigalli a tutelis excusatur. » Le \emph{sacrum pro salute imperatoris} ne peut guère être qu'un taurobole ; cf. la même formule sur un autel criobolique « ex vatic. archigalli, » \emph{CIL.} 8, 8203 = 19981. Il n'est pas certain que le texte, conservé par Ulpien, soit d'Antonin. Mais il paraît être destiné à favoriser le « Portus » de Trajan sur la rive dr. du Tibre, et ne peut être antérieur à ce dernier prince. D'autre part, les sacrifices pour l'empereur « ex vaticinatione archigalli » ne sont pas connus avant la dynastie antonine. Nous voyons aussi qu'Antonin favorise le Metrôon d'Ostie.

*) \emph{CIL.} 14, 97.

*) \emph{CIL.} 14, 45.

\subsection{2.}

Ces sacrifices du bélier et du taureau n'étaient point particuliers aux mystères de la Mère des Dieux. Le taurobole se retrouve en Asie, des deux côtés du Taurus, dans les cultes d'Anahita et d'Artémis Tauropole,* d'Aphrodite Ourania,* de Mên et de Mâ.* C'est un véritable taurobole qu'accomplissent encore sous l'Empire, en l'honneur de Pluton et de Korè, les éphèbes d'Acharaca.* Le criobole existait peut-être dans le culte pergaméen des Cabires.* Mais un détail du mythe pessinontien nous révèle l'antiquité de ces rites en Phrygie. Pour s'emparer du monstre Agdistis, Dionysos-Liber l'enveloppe dans un filet.* Or le \emph{bolos} était précisément la capture au filet, ou au lasso, de la bête que l'on voulait immoler* ; et les noms grecs du criobole et du taurobole perpétuaient le souvenir de cette ancienne coutume, qui précéda longtemps le sacrifice même. La chasse au fauve finit par devenir un jeu de cirque, une corrida. Nous voyons pratiquer ce sport religieux à Pergame au temps d'Attale 3.* A Rome, les courses de taureaux furent importées par Jules César* ; elles passionnaient l'empereur Claude. Mais celles que l'on donnait au cirque n'avaient plus, ce semble, aucun caractère religieux. En tant que rite sacrificiel, l'immolation subsista seule, sans chasse ni course. C'est sous cette forme simplifiée qu'à l'époque antonine s'accomplit le rite, élément essentiel de la télestique phrygienne.

*) Cumont dans \emph{Rev. archéol.} 1888, 2, p. 132 ss ; dans \emph{Rev. de Philol.} 17, 1893, p. 195 : \emph{Mithra}, 1899, 1, p. 334 ; dans \emph{Rev. d'Hist. et Litt. relig.}, 6, 1901, p. 97 ; dans \emph{Rev. Archéol.} 1905, 1, p. 24 ss.

*) \emph{CIL.} 10, 1596 ; taurobole à Venus Caelesta. Sur les rapports de celle déesse avec Anahita et Mâ, cf. l'épithète commune d'Invicta : \emph{CIL.} 6, 80 : avec Cybèle, cf. les sacrifices offerts à Aphrodite dans le Metrôon du Pirée, et une dédicace à Vénus dans le Metrôon de Philippi : Foucart, \emph{Assoc. rel.}, p. 99. Leurs rapports à Chypre : Dieterich dans \emph{Philologus}, 52, 1894, p. 12. Le Zeus Baal de Kyrrhos, en Syrie, avait pour attribut un bélier bondissant ; allusion à un criobole ?

*) Cumont, \emph{Le taurobole et le culte de Bellone} dans \emph{Rev. d'Hist. et Litt. rel.}, \emph{l. c.} ; \emph{Les religions orientales dans le paganisme rom.}, 1907, p. 268 ; Moore dans \emph{American Journ. of archaeol.}, 1905, p. 71. Sur le sacrifice du taureau à Mâ et Rhéa en Lydie : Steph. Byz., s. v. \emph{Mastaura} ; cf. aussi les lions tauroctones de la Meter Oreia. --- Mên est associé en Lydie à Anaïtis : Reinach, \emph{Chron. d'Or.}, 1, pp. 157 et 215 ; cf. Mên posant un pied sur la tète du taureau, motif fréquent en Anatolie : Drexler dans Roscher, 2, 2761 ; Perdrizet, dans \emph{Bull. Corr. Hell.} 20, p. 102, n. 7.

*) Strab. 14, 1, 44.

*) Inscr. de Pergame ; Schroeder dans \emph{Ath. Mitt.}  29, 1904, p. 152 ss ; il est probable qu'il existait un lien entre ces criobolies et l'initiation des éphèbes aux Cabiria ; cf. Cumont dans \emph{Rev. archéol.} 1905, 1, p. 29.

*) Arnob. 5, 6 : « ex setis scientissime complicatis... inicit laqueum. » Le rôle de Liber signifie-t-il que ce genre de chasse fut introduit sur les plateaux d'Anatolie par les conquérants thraces ? Cf. infra, les Thessaliens de l'époque impériale.

*) Pour la capture des taureaux au filet, à l'époque mycénienne, voir les vases de Vaphio. Ce genre de chasse était également très apprécié en Assyrie et dans l'Égypte pharaonique : Maspero, \emph{Hist. anc. des peuples de l'Or.} 1, pp. 53 et 769.

*) Schroeder, \emph{l. c.} ; c'est cette inscr. de Pergame qui a révélé la véritable signification du taurobole et du criobole, déjà entrevue par Meyer dans \emph{Jahrb. d. Arch. Inst.} 7, 1892, p. 77 (explication de l'épithète de tauropolos par le rite de la capture du taureau). Sur le caractère religieux de cette chasse (en Égypte, une partie de la bête brûlée en face de l'idole ; dans la Grèce mycénienne, course de taureaux servant peut-être de prélude à un banquet sacré ; au-delà de l'Euphrate, chasse aux buffles d'Anaïtis), v. S. Reinach, dans l'\emph{Anthropologie}, 15, 1904, p. 272 ; Cumont dans \emph{Rev. archéol.} 1905, 1, pp. 26-29 ; v. aussi le culte du taureau à l'époque crétoise, supra, p. 4.

*) Plin., \emph{H. n.} 8, 70 ; Sueton., \emph{Claud.} 21. Les « toreros » étaient des Thessaliens à cheval ; après avoir fatigué le taureau, ils sautaient sur son dos et le terrassaient en le saisissant par les cornes.

Ce qui importe en effet, c'est l'effusion du sang et le baptême de sang. Le taurobole est à la fois un sacrifice et un sacrement. Le myste descend dans une fosse que recouvre un plancher à claire-voie.* Avec un épieu muni d'un harpon, survivance du rite préhistorique,* on égorge au-dessus de lui l'animal. Sur tout le corps, sur la face, dans la bouche, il reçoit la rouge pluie. « De tout son être il boit le sang noir.* » Boire et répandre sur soi le sang du fauve que l'on vient de tuer, c'est transfuser en soi la vigueur de la bête. Les populations primitives de l'Anatolie centrale pensaient à cet égard comme les sauvages d'aujourd'hui.* De même, si l'on arrache les \emph{vires} de la victime ou de l'ennemi, ce n'est point à seule fin de les consacrer aux dieux, encore que cette offrande leur soit précieuse entre toutes* ; c'est que, par des rites et des formules magiques, on espère s'en approprier les énergies. Les \emph{vires} du bélier et du taureau jouent un rôle considérable dans les cérémonies crioboliques et tauroboliques.* On les recueille dans le « kernos, » on les transporte au temple, on les consacre à la Grande Mère, on les enfouit dans un lieu saint, sous un autel commémoratif. On conserve également, pour le dédier à la divinité, le bucrâne aux cornes dorées, siège de force.* Le taurobolié qui avait accompli ces rites se trouvait régénéré dans ses forces vitales et dans sa santé. Assurément il croyait avoir reculé l'heure fatale du trépas.

*) Prudent., \emph{Peristeph.} 10, 1016-1020.

*) Cette \emph{harpè}, à croc simple ou double, est représentée sur les autels tauroboliques,par exemple \emph{CIL.} 12, 1751, et 13, 573 (reprod. par Jullian, \emph{Inscr. rom. de Bordeaux}, 1, p. 32). C'est la dernière forme du harpon préhistorique. Cette arme s'était surtout perpétuée en Thrace. Les Romains la désignaient sous le nom d'\emph{ensis hamatus}. Prudence, \emph{l. c.}, 1027, précise son caractère d'arme de chasse, \emph{venabulum}.

*) Prudent., \emph{l. c.}, 1031-1040 ; surtout 1037 : « supponit aures, labra, nares obicit » et 1039 : « nec iam palato parcit et linguam rigat. » A rapprocher de l'accusation portée contre les chrétiens, de faire tuer par les néophytes un enfant, dont on léchait avidement le sang ; cf. Renan, \emph{Marc Aurèle}, p. 395.

*) Sur l'initiation chez les non-civilisés, et sur les rites qui communiquent une âme nouvelle, empruntée à quelque totem, cf. Goblet d'Alviella dans \emph{Rev. Hist. Religions}, 1902, 2, p. 196 s ; cf. p. 334, un rite des Chamanes.

*) Cf. le chapitre consacré aux Galles. Les \emph{vires} sont les testicules.

*) Cf. les formules tauroboliques : \emph{vires excipere et transferre}, \emph{CIL.} 13, 510, 1751 ; \emph{vires consecrare}, 522, 525 ; \emph{vires condere}, 12, 1567 et \emph{Ephem. epigr.} 8, 1899, p. 118, n° 455. De même on dédie des autels \emph{Viribus Aeterni}, \emph{CIL.} 5, 6961-62.

*) « Bucranium consacravit, » \emph{CIL.} 13, 1751 ; cf. les cornes de consécration dans la religion crétoise et mycénienne. Sur la dorure des cornes, cf. Prudent., \emph{l. c.}, 1024 et \emph{CIL.} 6, 504. Il semble, d'après Prudence, que l'on dorait aussi les poils de la bête en signe de consécration. Crânes dorés d'enfants, dans les sacrifices humains, v. Rufin., \emph{Hist. eccl.} 2, 21 : « infantum capita desecta inauratis labris. »

Mais une idée domine celle de purification revivifiante ; c'est la notion de sacrifice rédempteur. Ce regain de vie suppose un rachat. Aux dieux, qui sont les maîtres de la vie et de la mort, il faut une victime. Bélier et taureau sont des victimes substituées. Un mythe grossier, issu du rite, ne laisse à cet égard aucun doute. Il nous est transmis par Clément d'Alexandrie, qui fut initié aux mystères phrygiens « avant d'aller à la parole du Sauveur.* » Zeus, ayant arraché les \emph{vires} d'un bélier, vint les jeter dans le sein de Déo, sa mère, « sous le prétexte mensonger de se punir de ses violents embrassements, et comme s'il s'était mutilé lui-même.* » Déo et Zeus représentent ici, sous un autre aspect, Cybèle, la déesse mère, et Attis, le dieu fils qui devient le dieu époux. Aussi bien, Clément prend-il soin d'ajouter : « voilà les mystères que célèbrent les Phrygiens en l'honneur d'Attis, de Cybèle et des Corybantes. » Suivant un autre mythe que rapporte Arnobe, Zeus-Attis se transforma en taureau.* Telles étaient les deux explications mythiques du criobole et du taurobole. La fraude pieuse des mystes s'autorisait ainsi de l'exemple divin. Toutefois elle ne satisfait les dieux que pour un temps. Seuls les Galles, qui sacrifient leur propre virilité, se rachètent pour la vie éternelle. Par le taurobole on ne se rachète qu'à terme, pour vingt ans.* De même, par certains sacrifices d'animaux, les Etrusques pensaient retarder de dix ans la mort.* Isis aussi pouvait prolonger la vie. Le principe du rachat, qui détermine le renouvellement du taurobole, nous fait comprendre aussi le rite de la fosse. C'est par elle que l'on communique avec le monde internai. Quant au choix du bélier et du taureau, comme victimes de substitution, il n'est point particulier aux Phrygiens. Ce sont en général ces bêtes que l'on immole aux dieux infernaux. Rome pratiquait depuis longtemps ce rite. Aux Jeux Tarentins et Séculaires, qui furent introduits par les Livres Sibyllins, elle sacrifiait des béliers et des taureaux noirs à Dis Pater et à Proserpine, au-dessus d'une fosse ; évidemment ces victimes animales avaient remplacé des hosties humaines.

*) Euseb., \emph{Praep. evang.} 2, 2.

*) Clem. Alex., \emph{Protrept.} 2, 15 (éd. Dindorf). Clément est mort vers 215. Sur les rapports de Cybèle et de Deo, cf. \emph{CIL.} 6, 30966.

*) Arnob. 5, 20.

*) \emph{CIL.} 6, 504, 512 (années 376 et 390). Anonyme de 394, dans Baehrens, \emph{Poet. Lat. Min.} 3, p. 286, vers 62. Ce cycle de 20 ans, après lequel une nouvelle purification est nécessaire, se retrouve dans le culte de Liber Pater : \emph{CIL.} 10, 1584 (époque de Septime Sévère). Les \emph{vicennalia}, vingtième anniversaire du règne d'un empereur, paraissent être un rite oriental ; cf. Tigrane fondant Tigranocerte 20 ans après son avènement et là même où il avait reçu la couronne. Dioclétien abdique 20 ans après son avènement, \emph{dies natalis imperii}, pour « obéir aux destins vainqueurs. »

*) Serv. Ad \emph{Aen.} 8, 39 ; cf. Arnob. 2, 62. Isis, dans Apul., \emph{Met.} 11, 6 ; « ultra statuta fato tuo spatia vitam tibi prorogare mihi tantum licere. »

Ce qui paraît n'avoir subsisté que dans les mystères de Phrygie, c'est la descente de l'homme dans la fosse. Elle précise le caractère de l'offrande substituée. Car l'homme est arrivé lui-même jusqu'aux portes de l'Hadès ; et le sang qui coule tout le long de son corps, avant de pénétrer dans le sol, les dieux infernaux peuvent croire que c'est le sien. En même temps, le baptême rouge parachève l'œuvre de rédemption. Or l'idée qui préside à ce rite s'était spiritualisée au cours des siècles. Le sang qui lave toute souillure efface également la faute. Seraient-ce les conquérants thraces qui, en adoptant cette pratique indigène, lui auraient attribué une vertu moins grossière, plus conforme à leurs notions d'immortalité* ? Ces vagues notions, à vrai dire, semblent déjà faire partie du patrimoine religieux de l'Asie Mineure. Une influence du sémitisme est très vraisemblable. Car la purification des fautes par le sang est un rite familier aux religions sémitiques. Pour purifier de ses péchés le peuple hébreu, le souverain sacrificateur l'aspergeait avec le sang des veaux et des boucs.* Toutefois l'influence la plus efficace dut être celle des mages perses. Elle ne cessa de s'exercer en Phrygie depuis le temps des Achéménides ; et les mages avaient apporté avec eux une doctrine. Attis profita des enseignements de Mithra le tauroctone.* La vertu cathartique et palingénésique du taurobole passa du corps à l'âme. De même que, dans la bible mazdéenne, la mort du taureau mythique détermine une seconde et plus belle création, de même le taurobole communique à l'âme une naissance nouvelle pour une meilleure vie. La fosse où s'ensevelit le myste est vraiment le sépulcre où va rester la dépouille du vieil homme. C'est un office des morts que l'on chante autour de lui.* Les rites se prolongent pendant trois jours,* comme pour les funérailles d'Attis. Quand l'homme nouveau se dresse hors de la tombe, sanguinolente apparition qui saisit les fidèles d'une horreur sacrée, il est pur entre les purs, saint entre les saints. Et les assistants, comme s'ils se trouvaient en présence d'un être divin, prennent l'attitude de l'adoration.*

*) Il y avait peut-être un baptême sanglant dans le culte thrace de Bendis ; cf. Foucart dans \emph{Mélanges Perrot}, 1903, p. 102.

*) Paul., \emph{Hebr.} 9, 19 : Saint Paul ajoute, 22 : « sans effusion de sang il n'y a pas d'absolution » (ἄφεσις). Autres exemples dans Gruppe, \emph{op. l.}, p. 891, n. 3.

*) Cumont, \emph{Mithra}, 1, pp. 9-11, 186-188 et 335 ; \emph{Rel. orient. dans le pagan. rom.}, p. 84. Sur les rapports étroits qui se sont établis entre Mithra tauroctone et l'Attis des tauroboles, cf. les terres cuites représentant Attis-Mithra dompteur du taureau : Id., \emph{Mithra}, monum. figurés 5 et 5 \emph{bis}.

*) Cf. Themistius dans Stob., \emph{Florileg.} 120, 28.

*) \emph{CIL.} 12, 1782 ; 13, 1753 et l754. D'autre part, 13, 1751 : \emph{mesonyctium} ; or, c'est l'heure des sacrifices aux dieux infernaux.

*) Prudent., \emph{l. c.} 1048.

\subsection{3.}

A Rome, on célébrait les tauroboles dans un sanctuaire spécial, le Phrygianum, réservé aux initiations. Ce Phrygianum était situé au Vatican, dans le voisinage immédiat ou dans les dépendances mêmes du Cirque de Caligula. Il fut donc bâti, ce semble, sur un terrain qui faisait partie des jardins d'Agrippine. Par suite, il y a tout lieu d'en attribuer la construction à la dynastie claudienne. L'empereur Claude, qui protégeait les mystères phrygiens au Palatin, ne pouvait leur refuser un emplacement sur les domaines impériaux, en dehors du pomoerium. Ses affranchis durent profiter d'une concession pour en obtenir une autre. Dès cette époque, sans doute, on accomplissait des sacrifices tauroboliques sur la colline vaticane. Ce sont, en principe, des cérémonies secrètes ; et le rite ne comportait pas nécessairement la dédicace d'un autel. Pour, expliquer l'absence de documents épigraphiques, au 1er siècle, il n'est pas besoin d'autre raison. Mais une dédicace datée de l'an 108, à Lisbonne, paraît faire allusion à un taurobole reçu par une femme.* L'œuvre propre d'Antonin aurait été de romaniser, en lui conférant un caractère officiel, le taurobole pour le salut de l'empereur et de l'Empire.*

*) \emph{CIL.} Il, 179. Aux titres de Mère des Dieux, Grande, Idéenne, se joint celui de Phrygienne qui la caractérise comme déesse des mystères. La dédicace est faite par une cernophore ; cf. infra le rôle du kernos dans le taurobole. Le \emph{sacrum} a été accompli «per M. Iu. Cass. Sev., » formule analogue à celle des tauroboles.

*) Liste (A) chronologique des tauroboles « pro salute imperatoris » :  
Antoninus. --- \emph{a.} Lyon, 9 déc. 160 ; un sévir aug., dendrophore, « vires excepit et a Vaticano transtulit aram et bucranium suo inpendio consacravit » : \emph{CIL.} 13, 1751. --- \emph{b.} Antonin ou Caracalla. Fréjus ; « pro sal. Antonin. » ; 12, 251 (suspecte).  
M. Aurelius. --- \emph{c.} Ostie, entre 170 et 174 ; par les cannophores, pour le salut de l'empereur et de sa famille, du Sénat, des 15virs, de l'ordre équestre, des armées, des marins, des décurions de la colonie ; \emph{CIL.} 14, 40 (texte reconstitué à l'aide de l'inscr. 42). --- \emph{d.} Lectoure, peut-être en 176 ( ? ) par la république des Lactorates ; 13, 520 ; sur la date, cf. Allmer dans \emph{Rev. épigraph.} 1, p. 168.  
Commodus. --- \emph{e.} Ostie (sur la face postérieure de \emph{c} ; probablement peu après la mort de M. Aurèle, 180) ; criobole ; 14, 41. --- \emph{f.} Orange, entre 185 et 193 ; « num(ini) Aug(usti), Matri Deum, » par un homme et une femme ; 12, 1222. --- \emph{g.} Tain, 20-23 mai 184 ; « pro sal. imp. etc., colon(iae) Copiae Claud(iae) Aug(ustae) Lug(duni) » ; le pontife perpétuel de Lyon « fecit » ; 12, 1782 et add., p. 827. --- \emph{h.} Lyon, 16 juin 190 ; « [pro sal. imp. etc.] numinib(us) Aug(usti) totiusque domus divinae et situ c(oloniae) C. C. Aug. Lugud. » ; par les dendrophores lyonnais ; 13, 1752.  
Septimius Severus et Clodius Albinus. --- \emph{i.} Lyon, 9-11 mai 194 ; par deux femmes, ex-voto ; 13, 1753.  
Septimius Severus. --- \emph{j.} Lyon, 4-7 mai 197 ; par deux femmes, ex-voto ; 13, 1754.  
Septimius Severus et Caracalla. --- \emph{k.} Lyon, après 199 (à cause du titre de Parthicus Max.) ; 13, 1755. --- \emph{l.} Valence, entre 209 et 211 ? taur. et criobole, par un prêtre ; 12, 1745. --- \emph{m.} Vienne ; « pro salute Augustorum et victoria et reditu, et statu civitatis Viennensium » ; 12, 1827. --- \emph{n.} Die, entre 198 et 209 ; par la république des Voconces ; Cagnat, \emph{Année épigr.}, 1888, n° 81. --- \emph{o.} Narbonne, après 199 ; « tauropolium provinciae Narbonensis, factum per... flaminem Aug. » ; \emph{CIL.} 12, 4323. --- La dédicace 8, 2230 = 17668 a été donnée à tort comme criobolique.  
Caracalla. --- \emph{p.} Aulnay, en 212 ; « Iui. Drutedo et Balorice taur. f. ex v. » ; Cagnat, \emph{Ann. épigr.} 1889, n° 83.  
Severus Alexander. --- \emph{q.} Auxerre, en 228 ; « pro salute dominorum » (Severus et sa mère Mammaea) ; ex-voto ; \emph{CIL.} 13, 2922. --- \emph{r.} Tipasa ; « pro salute et incolumitate etc., sacerdos ex ordine poni criobolium et taurobolium... » ; 8, 4846, texte complété par Gsell, dans \emph{Bull. archéol. du Comité}, 1896, p. 179, n° 60. --- \emph{s.} Mileu ; deux hommes « criobolium fecerunt et ipsi susceperunt, per... sacerdotem » ; 8, 8203 = 19981.  
Maximinus. --- \emph{t.} Chieti : « taurobolium movit... sacerd(os) de suo » ; 9, 3014, cf. 3015. --- \emph{u.} Cordoue, 25 mars 238 : « pro salute imperii tauribolium fecit Publicius etc., suscepit crionis(bolium) Porcia Bassenia, sacerdote Aurelio Stephano » ; 2, 5521.  
Gordianus 3. --- \emph{v.} Lectoure, 8 déc. 241 ; « fecit ordo Lact. » ; 13, 511.  
Philippi --- \emph{w.} Die, 30 sept. 245 : sacrifice de trois taureaux, avec les hosties mineures, fait par le pontife perpétuel de la cité de Valence, sa femme et sa fille ; « praecuntibus » les prêtres de la Grande Mère à Orange, Apt et Die, et un prêtre de Liber Pater, assistés de tous les autres prêtres ; ex-voto ; « loco vires conditae » ; 12, 1567.  
Trebonianus et Volusianus. --- \emph{x.} Ostie, entre 251 et 253 : « [pro salute] imp. etc., sen[atus], 10[15] vir(orum) s. f., equestr(is) ordin(is), ex[ercituum... na]valium navigantium » ; 14, 42.  
Probus. --- \emph{y.} Mactar, entre 276 et 282 ; par un chevalier romain, prêtre « perfectis rite sacris cernorum crioboli et tauroboli, suffragio ordinis col(oniae) comprobatus antistes, sumptibus suis, tradentibus Rannio Salvio, eq. r., pontifice, et Claudio Fausto sacerdotibus, una cum universis dendroforis et sacratis utriusque sexus v. s. l. a. » ; Cagnat, \emph{Année épigr.}, 1892, n° 18.  
Diocletianus et Maximianus. --- \emph{z.} Mactar, entre 286 et 305 ; par un prêtre, « perfectis etc. (cf. \emph{y}), tradente... sacerdote, una cum etc. » : \emph{ibid.}, 1897, n° 121.  
\emph{Sans nom d'empereur ni date}. --- \emph{aa.} Ostie ; « pro salut, et redit. et victor. imp. » ; \emph{CIL.} 14, 43. --- \emph{bb.} Vaison ; inscr. mal copiée et perdue ; taurobole et criobole, par un homme et une femme, « pro sal. domus divinae » ; 12, 1311. --- \emph{cc.} Die ; « pro sal. imp., » par un prêtre ; 1568. --- \emph{dd.} Die ; « pro sal. imper., » par un homme et une femme, ex-voto ; même prêtre ; 1569. --- \emph{ee.} Valence ; les dendrophores, à leurs frais ; 1744. --- \emph{ff.} Narbonne ; « Matri Deum taurobolium indictum iussu ipsius, ex stipe conlata, celebrarunt publice Narbon. » ; 4321. --- \emph{gg.} « Tauropolium provinciae » ; 4329. --- \emph{hh.} et \emph{ii.} Lectoure ; à tauropolium pub(lice) factum » ; 13, 522, 525. --- \emph{jj.} Périgueux : « Numinibus Aug(ustis) et Magnae Matri Deum Aug(ustae) » ; pose et dédicace de l'autel par un citoyen romain, fils du « sacerdos arensis » ; Cagnat, \emph{Année épigr.}, 1907, n° 138.  
\emph{Douteux}. Laurolavinium, en 212, par un prêtre et une prêtresse ; Cagnat, \emph{op. l.} 1895, n° 120. --- Narbonne, en 263, sous Postumus ; \emph{CIL.} 12, 4324 add. --- 4328. --- Lyon ; 13, 1756, où Mater Deum est qualifiée d'Augusta.  
Résumé géographique : Italie 5 (Rome 0, Ostie 4, Chieti 1). --- Gaule Narbonnaise 14 (Narbonne 3, Die 4, Vaison 1, Orange 1, Valence 2, Tain 1, Vienne 1, Fréjus 1). --- Lugdunaises 6 (Lyon 5, Auxerre 1). --- Novempopulanie et Aquitaine 6 (Lectoure 4, Périgueux 1, Aulnay 1). --- Espagne, Bétique 1 (Cordoue). --- Afrique procons., Byzacène et Numidie 4 (Mactar 2, Tipasa 1, Mileu 1). 

[Planche 2. --- 1. \emph{Autel de Périgueux}.](https://cdn.solaranamnesis.com/HenriGraillot/2-1.jpeg)

[Planche 2. --- 2. \emph{Autel d'Athènes}.](https://cdn.solaranamnesis.com/HenriGraillot/2-2.jpeg)

Ne faut-il pas chercher encore en Orient l'origine de cette institution ? Un rite analogue avait dû s'y perpétuer à l'intention des rois et des villes. Tous les monarques orientaux, en effet, ne prétendaient pas être des incarnations de la divinité. On vénérait les rois de Perse, non point comme des dieux, mais comme les élus des dieux. Ils se proclament eux-mêmes, dans leur titulature, « ceux auxquels les dieux ont donné le grand hvarenô et la grande royauté. » Le hvarenô, Lumière de Gloire dont Anahita et Mithra nimbent les souverains, est un signe de la protection divine. Il représente victoire et félicité. Qui s'en rend indigne est voué à la défaite et au malheur. Avant d'être les Heureux et les Invaincus, les rois ont donc l'obligation d'être les Pieux. Cette conception religieuse de la monarchie fut adoptée par les Séleucides, les rois de Bactriane, de Commagène, d'Arménie. Elle fut très répandue en Asie Mineure. La Tychè Nicéphore, Épiphane, des rois de Cappadoce et de Pont n'est autre que la Gloire mazdéenne. C'est la Fortune qui leur donne la puissance dominatrice ; mais c'est par la grâce des dieux suprêmes que la Fortune s'attache à leur personne. Mên et Mâ possèdent le hvarenô. Mên Tyrannos lui doit, dans le royaume pontique, son titre de Pharnakou, et Mâ son épithète d'Anikêtos, Invincible.* D'autre part, l'influence du sémitisme, dont il faut toujours tenir grand compte en Anatolie, confirmait la dépendance absolue du monarque à l'égard du dieu, roi des rois et maître des couronnes. C'est donc le devoir des rois de sauvegarder leur propre couronne par des rites d'expiation et de rachat. Le taurobole constituait le plus efficace de ces rites. Peut-être même la diffusion des doctrines astrologiques avait-elle encore accru son importance. Car la lune, qui s'exalte dans le taureau zodiacal, est précisément identifiée à Tychè ; et le bélier du zodiaque est par excellence un signe royal.* Nous ignorons si criobole et taurobole, qui confèrent l'état de grâce et renouvellent l'homme, s'imposaient aux nouveaux rois d'Orient comme rites de consécration. Nous voyons toutefois que le sacre des rois de Perse était célébré dans un temple d'Anahita, à Pasargadès* ; et le culte d'Anahita comportait la pratique du taurobole. Une tradition, qui se perpétuait dans le clergé romain de la Mère des Dieux, nous reporte à la même idée théologique et venait certainement aussi d'Orient : les prêtres fêtaient volontiers par un crio-taurobole leur avènement au sacerdoce.*

*) Sur le rôle du hvarenô dans les monarchies orientales, v. Cumont, \emph{Mithra}, 1, pp. 229-233 et 284-287 ; Textes p. 12 (Arménie) et 89 s (Commagène). Les rois de Pont jurent par la Tychè royale et Mên Pharnakou ; Strab. 12, 3, 31. L'explication de Pharnakou a été donnée par Darmesteter, \emph{Zend Avesta}, 2, pp. 13, 308 et 409. Sur Mâ Aniketos, cf. \emph{Rev. Et. Gr.} 1899, p. 169 ss ; \emph{Ath. Mitt.} 1904, p. 170.

*) Bouché-Leclercq. \emph{Astrol. gr.}, p. 307 ; cf. index, s. vv. \emph{Bélier, Taureau}. De plus, le Taureau est la maison astrologique de Vénus identifiée à la M. d. D. ; et le Bélier est le domicile du Soleil = Attis.

*) Plut., \emph{Artaxerxès}, 2.

*) \emph{CIL.} 9, 1540 : « sacerdos Matri Deum M(agnae) I(daeae) in primordio suo taurobolium a se factum » ; cf. supra, p. 160, liste A, \emph{y}, \emph{z}.

Mais le taurobole, comme tout sacrifice, peut s'accomplir au profit d'autrui. Le bénéfice intégral en passe au destinataire. Puisque de la protection divine dépendent le bonheur des rois et la prospérité des empires, tous les sujets doivent contribuer au maintien de la céleste grâce. C'est une fonction du clergé et l'intérêt des peuples de multiplier les pratiques expiatoires. Offert pour le salut d'un souverain, d'un pays, d'une ville, le taurobole est une forme atténuée de l'antique \emph{devotio}. Il conserve les puissantes vertus du sacrifice humain, qu'il a remplacé.

Or, sous l'influence des idées orientales, les empereurs romains avaient adopté pour patronne la Tychè des rois. Elle est devenue la Fortune des Augustes, \emph{Fortuna Augusta} ; mais, en souvenir de son origine, ils continuent encore à l'invoquer sous le nom exotique de \emph{Fortuna Regia}.* Elle apparaît sur leurs monnaies à partir de Galba et de Vespasien. Au 2e siècle, les Césars manifestent pour elle un culte superstitieux. Ils ont toujours auprès de leur personne, à Rome, en voyage, dans les camps, jour et nuit, une figurine dorée de la déesse. Lorsque Antonin se sentit mourir, il fit transporter de suite sa Fortune chez Marc Aurèle, son fils adoptif. Septime Sévère voulait faire exécuter une copie de la sienne, pour que chacun de ses fils possédât « l'image très sacrée. » La protection de cette Fortune, en même temps et par le fait même qu'elle assure le bonheur et la victoire des princes, légitime leur autorité. Elle garantit leur caractère auguste et saint. Elle les rend dignes de l'apothéose. Mais la Fortune n'est elle-même que l'envoyée des dieux tout-puissants. La félicité des Césars est la récompense de leur piété envers les seigneurs du monde, « régulateurs de la Fortune.* » Le premier titre dont ils s'honorent, eux aussi, est celui de Pieux. Antonin n'en désire pas d'autre.* Et il n'admet pas que, de son vivant, on lui rende les honneurs divins. En général, ces empereurs se considèrent si peu comme des dieux qu'ils ne se croient même pas sauvés. S'ils interdisent les consultations oraculaires « \emph{de salute principis}, » ils demandent des prières et des sacrifices pour leur salut.*

*) Capitolin., \emph{Anton. P.} 2, et \emph{M. Aur.} 7 ; \emph{Acta Apollonii}, 3, dans \emph{Anal. Bolland.}, 17, p. 287 (Tychè de Commode) ; Spartian., \emph{Sever.} 23 ; Amm. Marc. 30, 5, 18 (Tychè de Valentinien) ; cf. Dio Cass. 75, 14, 7 ; Epict., \emph{Dissert.} 4, 1, 14 ; Orig., \emph{Contra Celsum} 8, 65 ; Euseb., \emph{Hist. eccl.} 4, 15, 18.

*) Cf. « Belus, Fortunae rector, » \emph{CIL.} 12, 1277.

*) Capitolin., \emph{Anton. P.} 5. C'est seulement à partir de Commode que le protocole comporte les titres de \emph{Pius, Felix, Invictus}.

*) A cette intention on dressait aussi des autels, on élevait des chapelles sous le vocable de « Mater Deum. » Cf. \emph{CIL.} 3, 764, 1100 ; 8, 2230 = 17668 ; 14, 34 ; Cagnat, \emph{Ann. épigr.}, 1888, n° 78. C'est une coutume importée de l'Orient. De même, à Aera (Syrie), dédicace d'un Tychaion « pour le salut et la victoire de Commode » : \emph{CIG.} 4554. Sur ces mêmes offrandes \emph{pro se et suis}, cf. \emph{CIL.} 3, 1101, 1102 et la dédicace d'un spelaeum mithriaque, 5, 810.

Acceptant les croyances de l'Asie, ils devaient nécessairement lui emprunter ses pratiques. Déjà dans certaines villes d'Anatolie le taurobole était un élément du culte impérial.* Il est vraisemblable qu'à Rome, dans cette église phrygienne où dominaient les affranchis des Césars, on tauroboliait \emph{pro salute principis} avant l'institution officielle du rite. Ce mode de sacrifice, d'un caractère très barbare, avait pour lui l'avantage de se présenter sous les auspices d'une divinité depuis longtemps romaine. Cybèle trône sur le Palatin. L'Idéenne protège la maison des Augustes, comme Anahita, Mâ et l'Ourania protégeaient les dynasties royales de l'Asie. Elle est la Dame des Victoires ; de même Anahita et Mâ donnaient aux rois le succès dans les combats. Jadis les généraux de la république l'invoquaient dans les guerres difficiles. On lui offrira désormais des tauroboles « pour la victoire et le retour de l'empereur.* » Elle est, pour le corps et pour l'âme, la Dame de Salut. Par le taurobole elle renouvellera la vitalité physique du prince, indispensable au bonheur de l'État, et sa pureté spirituelle, qui maintient sur sa personne la Fortune céleste. Il n'est donc pas surprenant de rencontrer, sur un autel taurobolique de l'année 190, cette singulière formule : « pour le salut de l'empereur auguste et pour ses \emph{Numina}, » c'est-à-dire pour le maintien de sa puissance divine.* Car la Mère des Dieux est vraiment, comme les omnipotentes déesses de Syrie, de Perse et de Cappadoce, la maîtresse de la Fortune. C'est pourquoi les Césars l'appelleront leur Conservatrice. « Grande Mère, accorde cette grâce d'effacer la tâche d'impiété, » disait un César dans ses prières* ; « fais que la Fortune bienveillante favorise le gouvernement romain pendant des milliers d'années. » On associera sans doute la déesse, avec son taurobole, aux fêtes purificatrices des « Vicennalia* » ; et l'empereur Philippe lui accordera une place importante aux fêtes du millième anniversaire de la fondation de Rome.* Rome s'identifie avec l'Empire, et l'Empire s'identifie avec l'empereur. Taurobolier pour le salut du prince, c'est contribuer à la prospérité de tout le monde romain. Il faut taurobolier en même temps pour les princes héritiers ; car c'est assurer la perpétuité du régime. Le taurobole de la Mère des Dieux va devenir l'une des sauvegardes de l'Empire.

*) Inscr. relative aux Traianea : \emph{Inschr. v. Pergamon}, 2, 554.

*) Cf. p. 159, liste A, \emph{m} et \emph{aa}. La formule \emph{pro reditu} est empruntée aux dédicaces pro \emph{itu et reditu} des voyageurs ; la grande Mère était précisément une des divinités que l'on invoquait en pareil cas : cf. \emph{CIL.} 8, 11797 ; 9, 5061 ; 12, 1223, où la formule est remplacée par l'image symbolique de pieds.

*) Cf. liste A, \emph{h}.

*) Julien, \emph{Or.} 5, à la fin.

*) V. p. 156, n. 5.

*) \emph{CIL.} 6, 488 (année 247).

Les empereurs ne se soumirent pas eux-mêmes au sacrement du taurobole, si contraire aux traditions romaines. Seul, le syrien Héliogabale se fit taurobolier, peut-être à l'instar des rois d'Asie.* Mais les impératrices du 2e siècle, dévotes de Cybèle et d'Attis, avaient-elles pu résister à la magie du baptême rouge ? Nous savons que Faustine la jeune, avant de concevoir Commode, et sur le conseil de prêtres chaldéens, se baigna dans le sang d'un gladiateur égorgé.* Le sang du taurobole ne pouvait lui répugner. D'autre part, le meurtre rituel du gladiateur avait été toléré, sinon ordonné par Marc Aurèle lui-même. Chez ce prince, la philosophie n'avait pas tué la superstition.* Admettant une pratique aussi barbare, il n'a pu que manifester toute sa bienveillance à l'égard du taurobole romanisé. Il n'eut du reste qu'à continuer la tradition de son père adoptif. Avant la fin du principat d'Antonin, en effet, le taurobole « pour le salut de l'empereur » était en vogue dans les provinces. Déjà le Vatican, où s'élevait le Phrygianum et où se consommaient les tauroboles romains, était considéré dans tout l'Empire comme la colline sainte du Phrygianisme. Les colonies, qui avaient leur Capitole à l'image de Rome, eurent leur Vatican, centre des vaticinations métroaques et des fêtes tauroboliques. Lyon en possédait un en 160 ; et le premier taurobole public de Lyon est certainement antérieur à cette date. Pour favoriser la diffusion du rite, les Antonins y attachèrent certains privilèges, sous forme d'exemptions juridiques. C'est ainsi qu'ils avaient privilégié le baptistère phrygien d'Ostie, lequel s'élevait, ce semble, près du port de Trajan, sur la rive droite du Tibre. Quiconque, à la suite d'une vaticination de l'Archigalle, y sacrifiait pour le salut de l'empereur, était exempté des charges de tutelle.* A Rome même, le César se fait-il déjà représenter aux principales solennités du taurobole public ? Plus tard, tout au moins, « les clarissimes membres de l'amplissime et sanctissime collège quindécemviral, » dont il est grand maître, y enverront une délégation.* Mais dès le début du me siècle, au temps de Sévère, nous voyons des 15virs qui tiennent à honneur d'offrir et de célébrer eux-mêmes le sacrifice.*

*) Lamprid., \emph{Ant. Heliog.} 7.

*) Capitolin., \emph{M. Auret.} 19. C'est M. Aurèle, au dire de l'historien, qui consulte les Chaldéens. Sur la coutume romaine de boire le sang des gladiateurs égorgés, comme remède contre certaines maladies, en particulier contre l'épilepsie, cf. Plin., \emph{H. n.} 28, 1 ; Cels. 3, 23 ; Min. Fel., \emph{Oct.} 30, 5.

*) Cf. Capitolin., \emph{l. c.} 13 : « tantus terror belli Marcomannici fuit, ut undique sacerdotes Antoninus acciverit, peregrinos ritus impleverit, Romam omni genere lustraverit. »

*) V. p. 153, n. 1.

*) \emph{CIL.} 6, 508 (année 319) : « praesentib. et tradentib. cc. vv. ex ampliss. et sanctiss. coll. 15 vir. s. f. » ; en 305, taurobole reçu par un 15 vir., \emph{ibid.}, 487 ; cf. 498, 499, 501, 509.

*) \emph{CIL.} 14, 2790. Pompeius Rusonianus, 15vir et consul suffect, qui « taurobolium movit, » fut magister du sacré collège en 204 ; cf. \emph{Acta ludorum saecul. Septim.} dans \emph{Ephem. epigr.} 8, p. 282 ss ; Dessau et Klebs, \emph{Prosopogr.} 3, p. 70.

De son côté, le clergé métroaque est intéressé à la propagation de cette pratique, puisqu'elle crée de nouveaux liens entre le culte phrygien et la maison impériale. Les Archigalles, interprètes de la volonté divine, imposent le taurobole aux dévots.* Beaucoup de prêtres l'accomplissent à leurs frais.* Dendrophores et Cannophores donnent également l'exemple.* Quand il s'agit de simples fidèles, il y a généralement deux personnages pour faire l'offrande, soit deux hommes, soit deux femmes, soit un homme et une femme. L'un reçoit le criobole et l'autre le taurobole.* Déjà vers la fin de la dynastie antonine, ils n'attendent plus l'ordre de la déesse ni les vaticinations de ses prophètes. Ils mutiplient les tauroboles « ex voto.* »

*) Liste A, p. 159, \emph{g}, \emph{h}, \emph{s} (ex vaticinatione archigalli) ; \emph{a}, \emph{o} (eximperio ; cf. 10, 1529, taurob. à Venus Caelestis en 134) ; \emph{u} (ex iussu Matris Deum). Les deux dernières formules sont équivalentes à la première. Peut-être à l'origine n'autorisa-t-on que les tauroboles de cette catégorie ; la divinité et ses prophètes savent seuls quand cette expiation majeure est nécessaire.

*) \emph{Ibid., l, r, t, y, z}.

*) \emph{Ibid., a, c, h, ee}.

*) Comme exemple, \emph{ibid.}, \emph{u}.

*) \emph{Ibid., i, j, p, q, w, y, z, dd}.

Mais de tels sacrifices n'attirent pas seulement au temple la clientèle ordinaire de la Mère des Dieux. Il en est qui sont offerts par les pontifes des cités, par des magistrats et des sénats municipaux, par les villes mêmes au moyen de souscriptions, enfin par les provinces.* Ce ne sont point seulement les prêtres de la déesse qui officient et président ; c'est aussi le haut clergé municipal et provincial : augures, haruspices, pontifes, flammes.* On réunit dans la même formule précatoire le salut de la Maison Divine et la prospérité de la ville où s'accomplit la cérémonie.* Car les villes pieuses ont, comme les hommes, leur Tychè qui les protège* ; et le baptême sanglant efface sur elles, comme sur les hommes, « la tache de l'impiété. » La ressemblance iconographique de Cybèle et de la Tychè dut singulièrement favoriser ce rôle de l'Idéenne ; couronnée de tours, Cybèle est elle-même la patronne naturelle des cités. Mais la Fortune des cités et des provinces est inséparable de celle des Césars. Le taurobole devient ainsi, dans les municipes et dans les colonies, l'une des manifestations religieuses du loyalisme à l'égard de l'empereur et de l'Empire. Il ne s'adresse plus seulement à la Mère des Dieux pour le salut du prince et le maintien de sa puissance divine. Il s'adresse à la fois à la déesse, qui a pris le titre caractéristique d'Augusta,* et aux \emph{Numina} de l'empereur auguste.* Par suite, il rapproche le personnel du culte augustal et celui du culte phrygien. Les sévirs augustaux entrent volontiers dans les confréries métroaques ; ils font partie du conseil d'administration ou des membres d'honneur. Volontiers ils pratiquent le taurobole.* A Narbonne, quand on célèbre celui de la province, le flamine des Augustes y joue le principal rôle. C'est généralement au temple de \emph{Mater Deum} et à celui des \emph{Divi} que les cérémonies du culte sont les plus pompeuses.* C'est là que se presse la population. Les deux cultes absorbent désormais presque toute la vitalité de la religion municipale.

*) \emph{Ibid.}, \emph{g}, \emph{w} (pontifes), \emph{d}, \emph{n}, \emph{v}, \emph{ff} (villes), \emph{o}, \emph{gg} (province Narbonnaise). Le taurobole de 241 (\emph{v}), à Lectoure, fut peut-être célébré à la suite des tremblements de terre « ob quae sacrificia per totum orbem terrarum ingentia celebrata sunt » ; Capitol., \emph{Gordian. Tertius}, 26.

*) \emph{Ibid.}, \emph{g}, \emph{o}, \emph{w}, \emph{y} ; cf. \emph{CIL.} 9, 1538 : le 9 avril 228, l'augure et deux prêtresses de la Grande Mère \emph{tradunt} ; un prêtre \emph{praeit} ; 1540 : un prêtre de la déesse \emph{facit} seul et \emph{tradit} avec la prêtresse ; l'haruspice public \emph{praeit}. Le sens précis des mots qui désignent chaque rôle n'est pas toujours facile à saisir. \emph{Facere}, c'est offrir à ses frais le taurobole et le recevoir (cf. \emph{suscipere}) ; \emph{praeire}, c'est présider aux différents actes du sacrifice en prononçant les formules liturgiques : coutume très romaine par laquelle se manifeste nettement la romanisation du taurobole ; cf. Boissier, \emph{Religion rom. au temps des Antonins}, 1, p. 396. Le verbe \emph{movere}, pour désigner la célébration d'un sacrifice taurobolique est également emprunté au vocabulaire de la vieille religion romaine ; cf. Serv. ad \emph{Aen.} 11, 324. L'expression \emph{tradere}, qui s'oppose en principe au mot \emph{accipere}, n'est appliquée qu'aux prêtres ; caractérise-t-elle seulement leur rôle d'officiants ? L'inscr. \emph{CIL.} 9, 1540 peut s'interpréter ainsi : le prêtre a fait les frais de la cérémonie et reçu le taurobole, qui fut « livré » par la prêtresse ; celle-ci, à son tour, a reçu le criobole, qui fut « livré » par le prêtre ; cf. liste A, \emph{u}, où l'homme reçoit le taurobole et la femme le criobole.

*) Liste A : \emph{g} « pro salute » ; \emph{h} « pro situ » ; \emph{a}, \emph{i}, \emph{j}, \emph{m}, \emph{v} « pro statu. »

*) Pour les villes grecques, cf. Allègre, \emph{Étude sur la déesse gr. Tyché}, Paris, 1889.

*) Liste A : \emph{y}, \emph{z}, \emph{jj}. \emph{CIL.} 8, 1776, 2230 = 17768, 11797, 16440, 19125 ; 13, 1756.

*) Liste A, \emph{f}, \emph{jj} ; cf. Cagnat, \emph{Ann. épiqr.}, 1905, n° 217 (Moutiers).

*) Liste A, \emph{a} ; \emph{CIL.} 12, 358.

*) Cf. par exemple le taurobole du 30 sept. 245, à Die ; liste, \emph{w}.

\subsection{4.}

A la diffusion du taurobole public pour le salut de l'empereur, de l'Empire et des cités, correspond celle du taurobole d'ordre privé, pour le salut individuel des mystes.* Sur 81 sacrifices de cette seconde catégorie ou paraissant tels, il en est 21 qui remontent certainement au 2e siècle ; 22 sont datés du me ; et 14, dont la date ne peut être déterminée avec exactitude, se répartissent entre les deux siècles. Tous les autres furent célébrés entre 305 et 390, deux à Athènes, un à Ostie, le reste au Vatican de Rome ; 18 sur 24 sont postérieurs à l'année 370. Ceux-ci représentent le suprême effort de réaction que tenta l'aristocratie païenne, avant le triomphe définitif du christianisme.

*) Liste B : tauroboles d'ordre privé.  
Rome. \emph{1}. Le 14 avril 305, Iulius Italicus, vir clarissimus, 15 vir, « taurobolium percepi » ; \emph{CIL.} 6, 497. --- \emph{2}. Le 29 avril 350, un clarissime, pontife, 15 vir, « taurobolio confecto » ; 498. --- \emph{3}. Le 19 juillet 374, Clodius Hermogenianus Caesarius, v. c., proconsul d'Afrique, préfet de la ville, 15 vir ; « Matri Deum Magnae Idaee summae parenti, Hermae et Attidi Menotyranno invicto,... taurobolio criobolioque perfecto,... diis animae suae mentisque custodibus » ; 499. --- \emph{4}. Le 13 mai 377, Caelius Hilarianus, v. c., 12 vir de Dea Roma, « pater sacrorum » et hieroceryx de Mithra, prêtre de Liber, prêtre d'Hécate ; « M. D. M. I. et Attidi Menotyranno conservatoribus suis » ; 500. --- Le 5 avril 383, Q. Clodius Flavianus, v. c., pontife majeur, 15 vir, pontife du Soleil, « taurobolio criobolioque percepto » ; 501. --- \emph{6}. Même jour, renouvellement de taurob. et criob., sans doute après 20 ans, par un clarissime, grande prêtresse de la Mère des D. ; « diis omnipotentibus M. D. et Atti » ; 502. --- \emph{7}. Le 23 mai 390, Lucius Ragonius Venustus, v. c., augure public, pontife majeur ; « diis omnipotentibus,... percepto taurobolio criobolioque » ; 503. --- \emph{8}. Le 13 août 376, Ulpius Egnatius Faventinus, v. c., augure public, père et hieroceryx de Mithra, archibucole de Liber, hiérophante d'Hécate, prêtre d'Isis ; « dis magnis,... percepto etc. » ; vœux pour renouveler dans vingt ans ; 504. --- \emph{9}. Le 26 février 295, L. Cornelius Scipio Orfitus, v. c., augure, « taurobolium sive criobolium fecit » ; 505. --- \emph{10}. Le même, « ex voto taurobolio sive criobolio facto » ; 506. --- \emph{11}. Le 15 avril 313, un clarissime, « taurobolium feci » ; 507. --- \emph{12}. Le 19 avril 319, Serapias, « honesta femina, » prêtresse ( ? ) de la M. d. D. et de Proserpine ; « potentiss(imis) diis [M. D. M. et At]ti Menotyranno,... taurobolium criobol. caerno perceptum » par le ministère du grand-prêtre, en présence et avec l'office d'une délégation des 15virs ; 508. --- \emph{13}. Le 16 juin 370, Petronius Apollodorus, v. c., pontife majeur, 15vir, et sa femme Ruf. Volusiana, c. f., « taurobolio criobolioque percepto, » avec dédicace en distiques grecs à Rhéa Mère de tout et Attis Hypsistos ; 509 = \emph{IGSI.} 1018. --- \emph{14}. Le 13 août 376, Sextilius Agesilaus Aedesius, v. c., l'un des directeurs de la chancellerie impériale, « pater patrum » de Mithra, hiérophante des Hécates, archibucole de Liber, « taurobolio criobolioq. in aeternum renatus » ; \emph{CIL.} 6, 510. --- \emph{15}. Le 12 mars 377, Ruf. Caeonius, fils de Sabinus, v. c., pontife majeur, augure public, hiérophante d'Hécate, « pater sacrorum » de Mithra, « tauroboliatus M. D. M. Id. et Attidis Minoturani » ; 1. 6-7 : « et veneranda movet Cibeles Triodeia signa ; augentur meritis simbola tauroboli » ; 1. 15 s : « Mithrae antistes... templi, tauroboliq. simul magni dux mistice sacri » ; 511 = \emph{Carm. Lat. Epigr.} 1529. --- \emph{16}. Renouvellement après 20 ans, le 23 mai 390 ; Ceionius Rufius Volusianus, v. c., ex-vicaire d'Asie, fils de l'ex-préfet Volusianus et de Caecina Lolliana, prêtresse d'Isis ; « [...dis] tutatoribus suis » ; \emph{CIL.} 6, 512. --- \emph{17}. Alfenius Ceionius Iulius Kamenius, 15vir, plus tard pontife majeur et vicaire d'Afrique, mort en 385 ; « tauroboliatus, » 1675 ; cf. \emph{Ephem. Epigr.} 8, 648. --- \emph{18}. Vettius Agorius Praetextatus 15vir, pontife, augure, plus tard préfet et consul désigné ; « tauroboliatus » : \emph{CIL.} 6, 1778-1779. --- \emph{19}. Sa femme Aeonia Fabia Paulina, fille d'un consul de 349 ; 1779, v. 25 s : « te teste cunctis imbuor mysteriis ; tu Dindymenes Atteosque antistitem | teletis honoras taureis consors pius » ; 1780. --- \emph{20}. Crescens, 15vir, pontife du Soleil, et Leontius, criob. et taur. à Rhéa Pammetôr, dédicace en vers grecs, 30780 = \emph{IGSI.} 1020. --- \emph{21}. Année 377. Sabina, fille de Lampadius ; dédicace en vers grecs à Attis et Rhéa ; \emph{CIL.} 6, 30966 = \emph{IGSI.}, 1019. --- \emph{22}. 15vir ; rien ne prouve qu'il s'agît de Praetextatus ; \emph{Ephem. Epigr.}, 4, 866.    
Italie. \emph{23}. \emph{Ostie}, 15 mai 199, une affranchie ; \emph{CIL.} 14, 39. --- \emph{24}. 4e s. ; fragment de dédicace grecque aux immortels Rhéa ? et Attis Menotyrannus ; \emph{IGSI.} 913. --- \emph{25}. \emph{Gabies} ; Pompeius Rusonianus, 15 vir (magister du collège en 204), consul suffect ; « taurobolium movit » ; \emph{CIL.} 14, 2790. --- \emph{26}. Laurolavinium, en 212 ; prêtre ? et prêtresse (peut-être pour un empereur ? ) ; Cagnat, \emph{Ann. épigr.} 1895, n° 120. --- \emph{27}. Pouzzoles, en 134 ; à Venus Caelestis, renouvellement par une femme « imperio deae » ; \emph{CIL.} 10, 1596. --- \emph{28}. \emph{Forum Popilii}, 20 nov. 186, prêtresse « fecit » ; 4726. --- \emph{29}. \emph{Rufrae}, 2e s., ingénue « sacra tauroboli l. m. f. » ; 4829, cf. 4844. --- \emph{30}. \emph{Formies}, en 241 ; prêtresse ; 6075. --- \emph{31}. \emph{Liternum} ; prêtre « condidit (vires) » ; \emph{Ephem. Epigr.} 8, 455. --- \emph{32}. \emph{Bénévent}. Le 9 avril 228, dédicace d'autel par une cymbalière du temple, pour un criobole ; « haec iussu Matris Deum in ara taurobolica duodena cum vitula crem(ata) » ; un augure et prêtre et deux prêtresses du temple « tradentibus, » un prêtre « praeeunte » ; \emph{CIL.} 9, 1538. --- \emph{33}. Une prêtresse, pour un taurobole ; 1539. --- \emph{34}. Un prêtre, « in primordio suo taurobolium a se factum, » la prêtresse « tradente simul, » l'haruspice public « praeeunte » ; 1540. --- \emph{35}. Dédicace d'autel par une prêtresse en second, « ob taur. traditum a Servilia Varia, sac. prima » (les mêmes qu'en 228) ; 1541. --- \emph{36}. Un 22 juillet, dédicace par une tambourinière du temple, « ob tauribol. traditum a Servilia Varia, sac. prima » ; 1542. --- \emph{37}. \emph{Chieti} (\emph{Teate Marrucinorum}), un 26 nov., vers 235 ; « criobolium et aemobolium movit de suo » un prêtre ; 3015, cf. 3014. --- \emph{38}. \emph{Turin} ; « Viribus Aeterni taurobolio, » une femme ; 6961. --- \emph{39}. Id., un homme ; 6962.  
Gaule Narbonnaise. \emph{40}. \emph{Narbonne} : une ingénue « fecit » ; \emph{CIL.} 12, 4322. --- \emph{41}. En 263, une femme ? 4324. --- \emph{42-43}. Deux autels tauroboliques anépigraphes ; cf. 4324 add. --- \emph{44}. 3e s. (cf. les formules de Lectoure) ; une affranchie « imperio accepit » ; 4325. --- \emph{45}. Une femme « fecit » ; 4326. --- \emph{46}. Autel taurob. ; 4327. --- \emph{47}. Autre, 4328 (peut-être taurobole public ? ). --- \emph{48}. \emph{Vence} ; deux affranchies et un prêtre « suo sumptu celebraverunt » ; 1. --- \emph{49}. \emph{Riez}, 2e s. (à cause de la formule) ; un citoyen romain et sa femme « ob sacrum tauropoli » ; 357. --- \emph{50}. 2e s. ; un sévir augustal « ob sacrum, v. s. » ; 358.  
Novempopulanie. \emph{51}. \emph{Lectoure}. « Matri Deum Pomp. Philumene quae prima Lactorae taurobolium fecit » ; \emph{CIL.} 13, 504. --- \emph{52}. Le 18 oct. 176, Ant. Prima « fec(it) host(iis) suis » ; 505. --- \emph{53}. Même jour, une affranchie, même formule ; 506. --- \emph{54}. Même jour, Iulia Valentina et l'esclave Hygia « fecerunt » ; 507. --- \emph{55}. Date identique ou voisine (même prêtre), une affranchie ; 508. --- \emph{56}. Id. ; Marciana, Marciani f(ilia) ; 509. --- \emph{57}. Le 24 mars, jour du \emph{Sanguis}, année 239 : « S(acrum) M(atri) D(eum). Val(eria) Gemina vires escepit Eutychetis » ; 510. Le taurobole est ici remplacé par le sacrifice de la virilité humaine. --- \emph{58}. Le 8 déc. 241, en même temps qu'un taurobole municipal pour le salut de Gordien et la prospérité de la cité ; G. lul. Secundus « accepit hostis suis » ; 512. --- \emph{59}. Même jour, Iul. Clementiana ; 513. --- \emph{60}. Même jour, Iul. Nice ; 514. --- \emph{61}. Même jour, Iunia Domitia ; 515. --- \emph{62}. Même jour, Pomp. Flora ; 516. --- \emph{63}. Même jour, Servilia Modesta ; 517. --- \emph{64}. Même jour, Valeria Gemina (cf. 56) ; 518. --- \emph{65}. Même jour, Verina Severa ; 519. --- \emph{66}. Aprilis, Repentini filius et Saturnina, Taurini filia, « acceperunt » ; 521. --- \emph{67}. 2e s. (à cause de la formule) ; « Severus Iulli fil(ius), vires tauri quo proprie per tauropolium pub(lice) factum fecerat consacravit » ; 522. --- \emph{68}. 2e s., à cause de la formule « fecit hostiis suis » ; Severa, Quarti filia ; 523. --- \emph{69}. Id. ; Iulia Valentina (cf. 53) et sa fille ; 524. --- \emph{70}. Id. ; Viator, Sabini filius, « vires tauri quo proprie, etc. » (cf. 67) ; 525.  
Aquitaine. \emph{71}. \emph{Bordeaux} ; deux femmes « natalici (i) virib(us) » ; \emph{CIL.} 13, 573. --- \emph{72}. Autel taurob., mutilé ; cf. Jullian, \emph{Inscr. de B.}, 1, p. 37.  
Lugdunaise. \emph{73}. Lyon, 2e s. (caractères épigraphiques de l'époque antonine) ; Billia, T. filia, Veneria ; \emph{CIL.} 13, 1756. --- \emph{74}. \emph{Vieu en Val Romey} (\emph{Venetonimagus}) ; autel taurob. anépigraphe ; 2529.  
Lusitanie. \emph{75}. \emph{Lisbonne}, en 108. « Matri Deum Mag. Ideae Phryg. Fl. Tyche cernophor. per M. Iu. Cass. et Cass. Sev. » ; \emph{CIL.} 2, 179. --- \emph{76}. \emph{Medellin} (\emph{Metellinum}) ; « pro salute et reditu Lupi, » ex voto ; 606. --- \emph{77}. \emph{Merida} (\emph{Emerita}), fin du 2e s., d'après l'épigraphie ; Valeria Avita « aram tauriboli sui natalici redditi d. d. » ; mention du prêtre et de l'archigalle ; 5260.  
Numidie. \emph{78}. \emph{Thibilis}, 2e s. (indication de la tribu du citoyen). « Terrae Matri Aerecurae Matri Deum Magnae Ideae » ; une ingénue « taurobolium aram posuit movit fecit » ; \emph{CIL.} 8, 5524 et p. 1807. --- \emph{79}. « Terrae, etc. » ; un citoyen romain « tauribolium et criobolium movit et fecit aramque po. » ; Cagnat, \emph{Ann. épigr.} 1895, n° 81.  
Grèce. \emph{80}. \emph{Athènes}, 4e s. ; Archelaos, d'une ancienne famille ; 1er taurobole « accompli ici, » dédicace en vers grecs à Attis et Rhéa ; \emph{CIA.} 3, 1, 172. --- \emph{81}. Le 27 mai 386. Taurobole reçu par Mousônios, clarissime ; \emph{ibid.}, 173.  
Résumé chronologique. 2e siècle. Années 108 (\emph{75}), 134 (\emph{27}), avant 176 (\emph{51}), 176 (\emph{52-54}), vers 176 (\emph{55-56}), 186 (\emph{28}), 199 (\emph{23}) ; indéterminées : \emph{29}, \emph{49}, \emph{50}, \emph{67-70}, \emph{73}, \emph{77-79}.  
3e siècle. Années 212 (\emph{26}), 228 (\emph{32}), vers 228 (\emph{35}, \emph{36}), vers 235 (\emph{37}), 239 (\emph{57}), 241 (\emph{30}, \emph{58-65}), 263 (\emph{41}), 295 (\emph{9}, \emph{10} ? ). Début du s. : 25. 1re moitié : \emph{33}, \emph{34}. Après les Sévères : \emph{44}.  
Dates indéterminées des 2-3e s. : \emph{31}, \emph{38-40}, \emph{42}, \emph{43}, \emph{45-48}, \emph{71}, \emph{72}, \emph{74}, \emph{76}.  
4e siècle. Années 305 (\emph{1}), 313 (\emph{11}), 319 (\emph{12}), 350 (\emph{2}), 363 (\emph{6}), 370 (\emph{13}, \emph{16}), 374 (\emph{3}), 376 (\emph{8}, \emph{14}), 377 (\emph{4}, \emph{15}, \emph{21}), 383 (\emph{5}, \emph{6}), 386 (\emph{81}), 390 (\emph{7}, \emph{16}). 2e moitié du s. : \emph{17-20}, \emph{22}, \emph{80}.  
Jours indiqués : 26 février (\emph{9}), 12 mars (\emph{15}), 5 avril (\emph{5}, \emph{6}), 9 avril (\emph{32}), 14 avril, (\emph{1}), 15 avril (\emph{11}), 19 avril (\emph{12}), 29 avril (\emph{2}), 13 mai (\emph{4}), 15 mai (\emph{23}), 23 mai (\emph{7}, \emph{16}), 27 mai (\emph{81}), 16 juin (\emph{13}), 19 juillet (\emph{3}), 22 juillet (\emph{36}), 13 août (\emph{8}, \emph{14}), 18 octobre (\emph{52-54}), 20 novembre (\emph{28}), 26 novembre (\emph{37}), 8 décembre (\emph{58-65}).

Il s'agit de sacrifices extraordinaires, accomplis en dehors de la période annuelle des initiations. 32 autels font mention du jour de leur dédicace, laquelle était le dernier rite des cérémonies tauroboliques. Or aucun de ces tauroboles n'est signalé pendant la seconde quinzaine de mars, qui est consacrée aux fériés et mystères de Cybèle. Par contre, sept d'entre eux nous reportent au mois d'avril, et cinq au mois de mai. Évidemment, les fêtes de mars provoquaient des accès de religiosité et des conversions. Certains jours, consacrés à des divinités qui sont en relations avec la Grande Mère, paraissent être choisis à dessein. En 319, un crio-taurobole a lieu le 19 avril, jour où l'on fêtait Cérès, Liber et Libéra. D'autres dates correspondent à celles des « Ludi Ceriales, » des fêtes de Flore, de Mercure et Maia, de Jupiter. Est-ce avec intention qu'en 377 le clarissime Caelius Hilarianus, prêtre d'Hécate, fait fixer la cérémonie aux jours funéraires des Lemuria ?

On profitait aussi de la célébration des tauroboles publics. Un peu de vanité, sans doute, se mêlait alors à la dévotion. Mais il y avait également, surtout dans les petites villes de provinces, une raison d'économie. Ces sacrifices exigent la présence de tout le clergé, prêtres et prêtresses, et de nombreux acolytes. Le joueur de flûte est un personnage indispensable. Sur des autels de Lyon, de Valence et de Tain, son nom paraît après celui du prêtre officiant ; sur un autel de Narbonne, est sculptée l'image du \emph{tibicen}. Tambourinières et cymbalières font entendre la musique purificatrice de leurs instruments, qui chasse les mauvais génies ; tympanons et cymbales sont même reproduits, comme la double flûte, sur les côtés des autels. On ne peut se passer des chantres. Il faut des victimaires pour saigner les bêtes. Parmi les motifs en relief qui décorent les autels, figurent des patères, des plateaux, des phiales, des préféricules, des aiguières, des aspersoirs, des pommes de pin, des gâteaux, tout un appareil qui suppose, outre le sacrifice proprement dit, des rites multiples et compliqués. Peut-être, comme aux jours normaux d'initiation, faut-il un personnel de jeunes filles et de femmes, pour remplir les fonctions de Cernophores, de Phialéphores, de Dadophores, de Lampadophores. Il y a des \emph{Apparatores}, pour ordonner tous les minutieux détails d'une cérémonie qui, en principe, dure plusieurs jours et au moins une nuit. Sans compter le bélier et le taureau, les crio-tauroboles devaient être fort coûteux. Il y avait avantage, pour les fidèles, à se succéder le même jour dans la fosse baptismale.

Tantôt le sacrifice est ordonné par la déesse, qui manifeste ses volontés directement par un songe, ou indirectement par les vaticinations de ses Archigalles. C'est alors un rite d'expiation majeure, et non point un sacrement d'initiation. La divinité seule, en ce cas, sait quand il est nécessaire. Ce taurobole \emph{ex imperio, ex jussu}, fut sans doute le premier à se répandre, sous l'impulsion du clergé phrygien. Mais nous le voyons se perpétuer aux 3e et 4e siècles. Tantôt le sacrifice présente un caractère votif. Tels sont, en général, les tauroboles accomplis dans l'intérêt d'autrui. C'est ainsi qu'à Metellinum, en Lusitanie, « pour le salut et le retour» d'un certain Lupus, divers membres de sa famille taurobolient \emph{ex voto}. Enfin, à mesure que s'accentue la lutte avec le christianisme, le baptême phrygien par le sang s'oppose au baptême chrétien par l'eau, comme rite de régénération spirituelle.*

*) De Rossi dans \emph{Bull. d'arch. crist.} 7, 1869, p. 27, et Leblant, \emph{Inscr. chr. de la Gaule}, pp. 85 et 412, ont insisté sur l'identité des formules employées par les chrétiens et par les tauroboliés.

Toutefois les tauroboles extraordinaires, ceux dont nous conservons les autels dédicatoires, ne constituaient pas une première initiation. Ils parachevaient l'œuvre d'initiation. Par eux on monte dans la hiérarchie des mystères. Le « taurobolié de la Mère des Dieux » est un dignitaire de haut rang, au même titre que le Père des Pères ou le Père de la Religion chez les Mithriastes, que l'Archiboucole de Liber ou l'Hiérophante d'Hécate. Telles sont les dignités, en effet, que cumulent certains Clarissimes du 4e siècle, quand ils sont parvenus aux grades suprêmes des hiérarchies mystiques. Sur l'autel de son taurobole, en 377, Rufius Cæonius se proclame chef des mystes et directeur des mystères. Le titre de Taurobolié, comme celui de Père de la Religion Mithriaque, équivaut à celui d'\emph{Antistes}, Supérieur. De même que Cæonius s'intitule « Antistes » du temple de Mithra, la Tauroboliée Paulina, femme du préfet Praetextatus, est « Antistes » de la Dindymène et d'Attis. Elle est la Supérieure de la Sainte Congrégation des Initiés. Mais, dans la langue religieuse de Rome, le mot implique généralement des fonctions sacerdotales. Ainsi donc, le taurobole conférait au myste une sorte de sacerdoce. C'est pourquoi les prêtres municipaux de la Grande Mère, en prenant possession de leur siège, célèbrent volontiers ce sacrifice. A la consécration officielle, émanant d'une chancellerie, s'ajoute la consécration religieuse, qui octroie une supériorité de droit divin.

Le jour du taurobole est considéré comme un jour de naissance, \emph{Natalicium}. Une femme de Bordeaux dédie l'autel aux symboliques « Vires, » dont l'offrande l'a rachetée de la mort et lui a permis de renaître à une meilleure vie : \emph{Natalicii Viribus}. Cette renaissance étant vicennale, il faut à l'échéance renouveler le sacrement. Un nouvel autel commémore la réitération du taurobole : \emph{ara tauroboli natalicii redditi}. Cette tradition persiste jusqu'à la fin du paganisme. Le 5 avril 383, nous voyons une grande prêtresse de l'Idéenne à Rome recommencer taurobole et criobole. Le dernier autel qui porte une date (23 mai 390), l'un des derniers sans doute qui furent consacrés dans le Phrygianum du Vatican, est celui d'un renouvellement vicennal. En 394, dans un poème anonyme contre les païens,* il est fait allusion à ce rite. Cependant, en 376, le Clarissime Sextilius Agesilaus Aedesius, membre du consistoire impérial, déclare que son taurobole et son criobole l'ont fait renaître à une vie éternelle. A Turin, peut-être dès le me siècle, deux autels tauroboliques sont dédiés aux « Vires » du dieu Éternel, qui donne l'éternité : \emph{Viribus Aeterni}. Les mystes qui élevèrent ces autels partageaient donc la même croyance qu'Aedesius. Faut-il croire à une période d'indécision dans la doctrine, sous l'influence d'autres mystères et, en particulier, des idées chrétiennes ? On peut supposer aussi qu'il y avait une gradation dans la durée d'efficacité des rites. Le renouvellement des vœux et actes de consécration, pour les simples mystes, devait avoir lieu plus souvent ; dans certains mystères il se reproduisait tous les trois ans, aux fêtes dites triétériques. Les grâces du premier taurobole se maintenaient durant vingt ans ; mais il est possible que le second régénérât l'homme pour l'éternité. Ainsi se concilieraient les témoignages contradictoires qui nous sont parvenus.

*) Vers 62 (cf. supra, p. 156, n. 5) : « vivere cum speras viginti mundus in annos. »

Le taurobole s'est surtout propagé en Italie, dans les Gaules, en Espagne, dans les provinces de l'Afrique du Nord. Il paraît continuer, sur le sol africain, une tradition du culte de la Reine Céleste, culte importé de l'Asie sémitique. Mais, en général, c'est à ses propres vertus qu'il doit tout son succès. Aux âmes mystiques il apporte d'ineffables satisfactions. Malgré sa grossièreté, il correspond à un progrès moral. Sa barbarie même est un attrait et augmente la foi superstitieuse en son efficacité.* Les femmes, qui avaient tant aidé au triomphe de la religion phrygienne, ne contribuèrent pas moins à la vogue du baptême rouge. Ce sont presque toujours des noms de femmes que portent les dédicaces d'autels tauroboliques. En Italie, sur six dévots qui n'appartiennent pas au personnel des temples, il y a quatre femmes. En Narbonnaise, sur dix personnages, on compte sept dévotes. A Lyon, à Bordeaux, il n'est fait mention que de femmes. L'exemple de Lectoure est des plus instructifs. Sur vingt-deux autels, seize ont été consacrés par des femmes, affranchies ou ingénues. Un autre fut dédié par une femme et son mari. Trois seulement ne portent que des noms d'hommes. Les deux autres sont l'œuvre de la cité Lactorate. Une femme, Pompeia Philuménè, se vante d'avoir la première « fait » un taurobole à Lectoure. C'est une affranchie. Il est regrettable qu'elle n'ait pas transmis à la postérité la date de cet événement. On peut supposer qu'il n'est pas antérieur au principat d'Antonin. Car, en 176, les deux prêtres du temple sont encore des esclaves. Le 18 octobre de cette année-là, nous assistons successivement à six tauroboles ; et il y en eut probablement davantage.* C'était une habitude, au Phrygianum de Lectoure, comme en beaucoup d'autres, de faire coïncider le même jour taurobole public et tauroboles privés. Le premier fut accompli pour le salut de l'empereur Marc Aurèle, qui venait de combattre en Syrie la révolte d'Avidius Cassius, et pour la conservation de l'Empire, que cette révolte avait failli démembrer.* Ensuite, le baptême est reçu par des femmes. Elles-mêmes ont fourni les victimes. Deux des néophytes sont de condition libertine. Mais les trois autres paraissent être de condition libre. L'une d'elles, Julia Valentina, est accompagnée de l'esclave Hygia, qui sans doute l'avait convertie. Est-ce la même Valentina, fille de Valens, qui reparaît sur une autre dédicace ? Elle aurait à la fois renouvelé le sacrement pour elle et consacré sa fille Valeria. Le 8 décembre 241, on célébrait un taurobole public pour le salut de l'empereur Gordien et la prospérité de la « civitas. » Il fut suivi d'au moins huit tauroboles privés, où deux affranchies, cinq ingénues et un citoyen romain reçurent le baptême de sang. L'une des femmes, Valeria Gemina, fait montre d'un véritable fanatisme. Deux années auparavant, le 24 mars, à la fête solennelle du \emph{Sanguis}, elle avait recueilli les \emph{vires} de l'esclave Eutychès, qui sacrifiait à Cybèle sa virilité pour entrer dans l'ordre des Galles.

*) C'est sur cette barbarie qu'insistent les polémistes chrétiens ; Firm. Mat., \emph{De err.} 27, 8 et 28, 1 : « Polluit sanguis iste, non redimit ; miseri sunt qui profusione sacrilegi sanguinis cruentantur ; tauribolium quid vel criobolium scelerata te sanguinis labe perfundit ? Quaere fontes ingenuos » ; cf. Prudent., \emph{Peristeph.} 10, 1006-1010 ; \emph{Contra Symm.} 1, 395. Allusion au taurobole dans les actes d'un synode des évêques de Gaule, en 374, à Valence : « Qui se vel profanis sacrificiis daemonum, vel incesta lavatione polluerint » ; Mansi, 3, p. 493.

*) Espérandieu, \emph{Inscr. des Lactorates} dans \emph{Rev. de Gascogne}, 1892, p. 80 s, rapporte onze autels à cette date.

*) La même année, embellissement d'un temple d'Aoste pour le salut du prince : \emph{CIL.} 12, 2391-92.

\subsection{5.}

L'initiation mineure ou \emph{Initium},* celle des simples mystes, avait lieu chaque année à date fixe. Elle suivait immédiatement les fêtes de mars. Celles-ci prenaient fin le 27, avec la Lavation. Le lendemain même, 28 mars, les catéchumènes se rendaient au Phrygianum pour l'admission aux mystères (\emph{acceptio sacrorum}).* On célébrait de même au printemps l'initiation des chrétiens.* C'était dans la nuit de Pâques, en principe, que les nouveaux disciples du Christ recevaient le baptême.

*) Calendrier de 354, dans \emph{CIL.} 1, p. 312 : « 5 K(alendas) Apr. Initium Caiani. » Sur les rapports du Caianuni ou Cirque de Caligula et du Phrygianum, cf. le chapitre consacré aux sanctuaires de Rome.

*) Ce mot est employé par les chrétiens pour désigner le baptême. La \emph{traditio} est le l'ait de conférer grades et sacrements.

*) Cf. Duchesne, \emph{Origines du culte chrétien}, 5e éd., 1909, p. 300.

Le candidat s'est préparé aux sacrements par de rigoureuses pratiques de continence et d'abstinence, par des expiations réitérées et des sacrifices propitiatoires à la Dame de Salut. Assidûment il a fréquenté le temple et les prêtres. Il vient d'assister à la représentation du drame sacré, de pleurer Attis mort, d'acclamer Attis ressuscité et glorieux.* Toutes ces émotions religieuses, frissons d'une sainte terreur au jour du Sang, voluptés de l'amour mystique au jour des Hilaries, ont exalté sa foi. Déjà il se sent plus près de ses Dieux. Avec impatience et inquiétude il attend d'eux les suprêmes révélations.

*) La représentation du drame sacré constitue l'un des éléments communs à tous les mystères. Lorsque l'on rendit officielles les fêtes phrygiennes de mars, une partie de ces fêtes conserva son caractère ésotérique. Julien, \emph{Or.} 5, p. 169 A, déclare nettement qu'il y avait alors deux catégories de cérémonies, celles qui pouvaient être divulguées à tous et qui étaient devenues publiques, et celles dont les rites cachés appartenaient aux mystères proprement dits. Il est vraisemblable que les candidats pratiquaient alors les pénitences exigées des mystes et assistaient au moins à une partie des rites secrets.

Ce qu'étaient ces mystères, les initiés juraient d'en garder le secret au plus profond d'eux-mêmes* ; et ils ont à peu près tenu leur serment. Les polémistes chrétiens des 3e et 4e siècles parlent souvent des mystères de Phrygie.* Mais leur information est généralement superficielle ; ou bien, par prudence, ils se contentent d'obscures allusions. Sous le fatras des spéculations théologiques et philosophiques, la « bible métroaque » du néoplatonicien Proclus nous eût fourni d'utiles renseignements* ; mais nous l'avons perdue. L'empereur Julien, qui fut un myste, écrivit à Pessinonte même un opuscule sur la Mère des Dieux. Son œuvre, à la fois mystique et pédantesque, nous est parvenue. Mais il s'y refuse à lever le voile qui dérobe les « secrets ineffables. » Ces secrets, ce sont des rites et des formules pour atteindre à la félicité suprême.

*) \emph{CIL.} 6, 1779 : « Mentis arcano premis » ; cf. Cumont, \emph{Mithra}, 1, p. 319.

*) Clem. Alex., \emph{Protrept.} 2, 15 ; Hippolyt., \emph{Refut. omn. haeres.} 5, 7, pp. 138 et 140, éd. Duncker et Schneidewin ; 8, p. 162 ; 9, pp. 168-170. Arnobe, 5, 5-7, puise ses renseignements sur le mythe de Pessinonte dans l'Eumolpide Timothée (3e s. avant J.-C.), qui les tenait lui-même « ex reconditis antiquitatum libris et ex intimis mysteriis. » Firmicus Maternus, \emph{De errore profan. relig.}, 18, 22, 27 et 28, est un des plus précis, ainsi que Prudence, \emph{Peristeph.} 10, 1006-1085 ; cf. Augustin., \emph{Civ. Dei} 6, 7 ; \emph{In Johann, evang. tractat.} 7, 1, 6 ; Hieronym., \emph{Adv. Jovinianum} 2, 5 et 17 ; \emph{In Hoseam} 1, 4, 14 ; Paul. Nol., \emph{Carm.} 19, 179 ss ; 32, 80 ss.

*) Marin., \emph{Vita Procli}, 33.

Un dogme, en effet, domine tous les mystères ; c'est celui d'une immortalité bienheureuse, réservée aux seuls initiés. Ce que le myste demande à l'initiation, c'est l'art de survivre à la mort et de réaliser le plus de bonheur possible dans la vie posthume. « Trois fois heureux les initiés lorsqu'ils pénètrent dans l'Hadès, » disait un personnage de Sophocle ; « à eux seuls est donnée la vie éternelle ; pour les autres il n'y a que souffrance.* » La raison d'être des mystères, déclare Cicéron, c'est d'offrir « une consolation pour la vie présente, une meilleure espérance pour la vie future.* » Une tauroboliée du 4e siècle adresse à son mari, qui l'initia, cette parole de reconnaissance : « en me procurant le bonheur de savoir, tu m'as arrachée au destin de la mort.* » Si l'on devient myste d'Attis ou d'Osiris, qui ont subi la mort et qui l'ont vaincue, c'est pour échapper, par leur exemple et leurs préceptes, aux horreurs de la nuit d'outre-tombe. A chacun de ses féaux Cybèle réserve la gloire de nouvelles Hilaries.* Sur des figurines en terre cuite, que l'on déposait dans les sépultures, elle tient parfois une clef à la main.* Cette clef ouvre l'entrée de son domaine, qui est un lumineux paradis. « De Mikkè, c'est le nom seul que possède le tombeau,* » affirme l'épitaphe d'une Cyzicène qui fut sans doute, comme la plupart de ses compatriotes, une zélatrice de la Dindymène. « Les âmes pieuses ont son âme. Arrivée au but, dans les Champs Élyséens, elle a obtenu cette récompense de sa vertu, avec l'ambroisie des immortels. Le temps, qui outrage le corps, n'a pu la fouler aux pieds. Mais elle siège avec les femmes pieuses. » La piété (\emph{eusebeia}), condition de cette immortalité, c'est l'exact accomplissement des rites qui furent révélés et prescrits aux mystes.* Mikkè savait aussi les formules occultes qui permettent seules de traverser les régions infernales et de parvenir au terme souhaité du voyage.

*) Dans Plutarque, \emph{De Aud. poet.} 4.

*) Cic., \emph{De leg.} 2, 14.

*) \emph{CIL.} 6, 1779 : « Disciplinarum bono puram ac pudicam sorte mortis eximens. »

*) Damascius, \emph{Vita Isidori.}, cf. supra, p. 131, n. 2.

*) Frœhner, \emph{T. cultes de la collection Gréau}, pl. 4.

*) \emph{Rev. archéol.}, 1876, 1, p. 354.

*) Sur le titre d'Eusébès = Pius, cf. Foucart, \emph{Assoc. relig.}, p. 147. Dans Apulée, \emph{Met.} 11, 6, Isis promet ses récompenses aux mystes qui les auront méritées « sedulis obsequiis et religiosis ministeriis et tenacibus castimoniis. » Les mystes d'Attis étaient soumis à des purifications et abstinences mensuelles ; cf. supra, p. 119, n. 4.

Si nous ignorons ces formules, du moins connaissons-nous en partie les rites sacramentels, qui constituaient les degrés de l'initiation : rites de purification, rites de consécration, rites de communion avec la divinité.

Dans ce culte d'une déesse des eaux, il y avait un baptême par affusion d'eau.* Le télestarque, ou prêtre initiateur, répandait sur la tête du néo-myste l'eau d'un cratère, qui remplaçait la source sacrée. Ce type de baptême n'est probablement ici qu'une lustration, sans propriété régénératrice. Un autre rite de purification, qui se retrouve dans les mystères de Sabazios, consistait à verser de la farine sur le récipiendaire accroupi, à le frotter de farine ou de son, d'argile ou de plâtre.* En se relevant, on devait prononcer la formule suivante : « j'ai fui le mal, j'ai trouvé le mieux. » De telles cérémonies, à vrai dire, appartiennent plutôt à la période préliminaire du catéchuménat, désignée d'une façon générale sous le nom de Catharsis. C'est alors que s'accumulent exorcismes et lustrations. Car, pour devenir l'élu des dieux, il ne suffit pas d'être pur ; il faut être issu de purs.* Et le catéchumène doit laver en lui non seulement ses propres fautes, mais aussi les péchés de ses ancêtres.

*) Cf. les mystères de Sabazios et de la thrace Cottyto. Au Metrôon d'Athènes, vase pour l'eau lustrale, engagé dans le mur même du temple, comme certains bénitiers d'églises : Foucart, \emph{op. l.}, p. 87. Dans la \emph{schola} des Dendrophores de Rome, fontaine à ablutions ornant le vestibule, et bassin lustral adossé à la porte.

*) Cf. les Galles : Apul., \emph{Met.} 8, 27, et Aug., \emph{Civ. Dei} 7, 26 ; dans les mystères : Démosth., \emph{Pro cor.}, 259 ; Plut., \emph{De superst.} 3 et 7 (πήλωσις) ; Harpocr., s. v. ᾿Απομάττων ; Bekker, \emph{Anecdota}, p. 293, 13, πηλοὶ καὶ πίτυρα ; Nonn., \emph{Dionys.} 27, 228 : γύψῳ μυστιπόλῳ. Sur le caractère magique de ces rites, cf. Lang, \emph{Mythes, Cultes et Religion}, 1896, 1, p. 263 ss ; Gruppe, \emph{Griech. Mythol. u. Relig.}, p. 903.

*) C'est là une formule orphique ; l'âme du défunt se déclare ἐκ καθαρῶν καθαρά, cf. \emph{IGSI.} 641. Mais cette notion paraît s'être répandue dans la plupart des mystères. Platon, \emph{R. publ.} 2, 7, p. 364 C, parle d'agyrtes, prêtres mendiants, qui viennent frapper aux portes des riches Athéniens et prétendent avoir obtenu des dieux le pouvoir de les « guérir, » au moyen de sacrifices et d'incantations, εἴ τι ἀδίκημά του γέγονεν αὐτοῦ ἢ προγόνων. La même idée se retrouve dans l'Atharva-Véda, 7, 86, 5 : « Pardonne-nous les péchés commis par nos pères. » Il faut en rapprocher celle de prières et de sacrifices pour les morts en Égypte ; c'est peut-être par l'intermédiaire des Juifs d'Alexandrie que cette pratique d'intercession avait passé dans l'église de Corinthe, comme nous l'apprend saint Paul, \emph{Ep. 1 Cor.} 15, 29 ; cf. S. Reinach, \emph{Cultes, Mythes et Religions}, 1, 1905, p. 327.

Le véritable baptême de régénération paraît être l'affusion de sang. Il est vraisemblable que les catéchumènes recevaient au moins le baptême criobolique. Le bélier est par excellence l'animal sacré d'Attis.* C'est le rite du criobole qui a produit le mythe répugnant de Zeus et Déo, élément du grand mythe auquel est initiée toute la Phrygie.* Pour les simples mystes, comme pour les tauroboliés, la régénération s'opérait donc par le rachat. Mais il ne peut s'agir ici que d'un rachat collectif. C'était en ce cas l'Archigalle, ou un prêtre, ou un patro-myste,* qui descendait dans la fosse. Il recevait le sang, par substitution, pour l'ensemble des néophytes. En même temps, sans doute, il en faisait bénéficier toute la communauté. Dans l'initiation mineure, on aurait ainsi renouvelé tous les ans le sacrifice de rédemption. Probablement, pour en conserver le caractère et l'efficacité, le prêtre aspergeait-il de sang les assistants, après le sacrifice* ; ou bien il leur conférait une onction sanglante.*

*) Cf. Attis assis sur un bélier, supra, p. 72, n. 2. Attis sur un quadrige de béliers (triomphe symbolique d'Attis à l'équinoxe du printemps, dans le signe du bélier) : lampe trouvée à Rome, décrite par Visconti dans \emph{Annali}, 1869, p. 239.

*) Clem. Alex., \emph{l. c.} ; cf. supra, p. 156.

*) Dans le taurobole décrit par Prudence, \emph{Peristeph.} 10, 1011 ss, c'est le \emph{summus sacerdos} qui descend dans la fosse ; mais il s'agirait plutôt d'un rite de consécration sacerdotale ; 1012 : \emph{consecrandus}, d'après les mss ; \emph{consecrandis} est une conjecture ingénieuse de Dressel. Dans le culte de Mithra, ce sont les Pères qui confèrent l'initiation ; Cumont, \emph{l. c.}, p. 318.

*) Comme dans les rites sémitiques ; cf. supra, p. 157.

*) Cf. un rite des Lupercales ; on touchait le front de deux jeunes gens avec un couteau trempé dans le sang des boucs sacrifiés, puis on essuyait le sang avec de la laine trempée dans du lait. Ce rite s'était substitué à un sacrifice humain.

N'existait-il pas une connexion entre ce sacrifice télestique et la Kernophorie ? L'hypothèse est autorisée par certaines dédicaces tauroboliques. Le \emph{Kernos} est un plateau circulaire qui supporte une série de petits récipients.* Il avait été d'un usage constant dans les cultes égyptiens.* On le retrouve aux mystères d'Éleusis.* La religion punique l'avait de même emprunté soit à l'Égypte, soit à l'Asie.* Dans la liturgie métroaque, c'est sur le \emph{Kernos} que l'on dépose les \emph{vires} du bélier et du taureau, pour les consacrer à Cybèle.* D'autre part, le culte des \emph{vires} intervient dans la célébration des mystères phrygiens. Comme dans les mystères de Dionysos et dans ceux d'Isis, l'office divin comporte une adoration delà ciste, où est enfermée la virilité du Dieu.* Il est donc possible que, tour à tour, les néophytes aient présenté à la déesse les \emph{vires} de l'animal sacrifié. Ce rite aurait parachevé le sacrifice de substitution et de rachat. Mais deux autres hypothèses sont également plausibles. Les godets du \emph{Kernos} servaient aussi de lampes ou de petits réchauds.* Après les avoir allumés à l'autel ou aux lampes, symbole astral, qui sans cesse brûlaient autour d'Attis, l'initiateur aurait placé le vase et sa couronne ignée sur la tête du néophyte. Pour les mystes du dieu solaire Attis, la Kernophorie serait un rite de purification et de consécration par le feu. Le baptême de feu était pratiqué dans les mystères. Certain mythe de Déo, cachant à l'intérieur d'un foyer embrasé le fils de la reine Métanire, paraît bien y faire allusion.* De même les torches sacrées, en bois de pin, si souvent figurées sur les monuments du culte, flambaient pour des fumigations purifiantes. Ces divers baptêmes par l'eau, par l'argile, par le feu, constituaient sans doute, avec les insufflations d'exorcisme,* le mystique « passage par les éléments,* » condition nécessaire du renouvellement des êtres. Enfin, si le \emph{Kernos} peut être un lampadaire, nous lui connaissons encore un autre usage. Dans les cotylisques du plateau l'on mettait les graines, les pâtes, les liquides et le miel offerts à la divinité chthonienne.* En Égypte et à Éleusis, ils servaient ainsi de gobelets d'offrande. Un type de Kernophorie était analogue au rite éleusinien du Van. Sur sa tête, sans doute voilée pour la consécration, l'initié recevait de la main du prêtre le plateau liturgique. Il le portait aux dieux en exécutant des évolutions rythmées. Car la Kernophorie est toujours accompagnée d'une sorte de danse qui lui est propre.* Après l'oblation, le porteur goûtait au contenu des tasses. Ce dernier rite, également accompli par le porte-van des Lichnophories, est une communion mystique avec les dieux.

*) Athen. 11, p. 478 \emph{C} \emph{D} ; Hesychius, s. v. κέρνος ; donne comme synonyme στεφανίς.

*) Maspero dans \emph{C. r. Acad. Inscr.}, 1895, 12 juillet, p. 295.

*) Rubensohn dans \emph{Ath. Mitt.} 23, 1898, p. 273 ss ; Kuruniotis dans Ἐφημ. ἀρχ., 1898, p. 21 ss ; Skias, \emph{ibid.}, 1901, p. 13 ss.

*) Delattre dans \emph{Mém. Soc. Antiquaires}, 1895, p. 297 et fig. 29 ; les godets communiquent ici avec un cylindre creux auquel ils sont soudés ; c'étaient probablement des lampes ; le cylindre contenait l'huile. Dans cette tombe punique, on a trouvé aussi des cymbales en bronze.

*) Cf. supra, listes de tauroboles, p. 160, \emph{y}, \emph{z}, et p. 167, \emph{12}. Galle cernophore : \emph{Anthol. Pal.} 7, 709 ; femmes cernophores, dans le culte métroaque : Nicand., \emph{Alexipharm.} 217 ; \emph{CIL.} 2, 179 ; 10, 1803.

*) Paulin. Nol., \emph{Carm.} 19, 186 : « abscisa colanl » ; 32, 90 s : « Intus et arcanum quiddam quasi maius adorant, etc. » ; cf. l'adoration de la ciste qui contient les « aidoia » d'Osiris ou de Dionysos.

*) Schol. ad Nicand., \emph{l. c.} (à propos des cernophores de Rhéa) ; Pollux, 4, 103 ; cf. Drexler dans Roscher, \emph{Myth. Lex.} 2, 2861 s, qui croit reconnaître ce genre de Kernoi sur des monnaies de Smyrne, 3e s. av. J.-C. Voir aussi les plateaux ronds qui étaient suspendus comme lampadaires dans les temples de la Grande Mère, et qui sont figurés sur les deux autels tauroboliques d'Athènes.

*) Hom., \emph{Hymn. in Cer.} 232 ss. L'expression d'\emph{enfants du foyer}, dans les mystères d'Éleusis, rappelait peut-être un rite analogue ; cf. Goblet d'Alviella dans \emph{Rev. Hist. Religions}, 1902, 2, p. 344 s.

*) Cf. le rôle des exsufflations, avec formule d'exorcisme, dans l'initiation chrétienne : Duchesne, \emph{op. l.}, pp. 302, 324, 336.

*) Apul., \emph{Met.} 11, 23 : « Per omnia vectus elementa remeavi. »

*) Athen., \emph{l. c.} ; il compare le Kernos au Van mystique ; même assimilation dans Pollux, 4, 103 ; dans le scholiaste de Platon, ad \emph{Gorg.} 497 C, et dans celui de Clément d'Alexandrie, ad \emph{Protrept.} 2, 15.

*) Poll., \emph{l. c.} : κερνοφύρον ὄρχημα.

Il y avait en effet un sacrement de communion. Purifié dans son corps, dans son âme, dans ses ancêtres, par de successifs baptêmes qui lui ont conféré une seconde naissance, lié pour toujours à ses dieux par les consécrations rituelles, désormais le myste peut prendre place au divin banquet.* Cybèle et Attis l'admettent à leur table sainte. Il n'y est pas seulement leur hôte ; il devient un membre de la famille divine. Participer à la nourriture des Dieux, c'est se pénétrer de leur essence et devenir un immortel. La patène et la coupe que lui tend le prêtre sont semblables à celles que l'on présente aux dieux. L'une, comme le plateau où l'on apporte à la Mère les prémices des récoltes, prend la forme du tympanon qu'elle tient dans sa main. L'autre rappelle la cymbale d'airain, qui est l'attribut d'Attis.* Dans le tympanon, l'élu mange la nourriture de vie divine ; dans la cymbale, il boit le breuvage d'immortalité.* Ce mets que lui offre la Dame du blé, c'est le pain ou le gâteau de pure farine.* Comme dans les mystères de Samothrace, il était partagé par le prêtre officiant, qui le distribuait ensuite aux fidèles. Cette liqueur paraît être le vin, qui est consacré à Cybèle aussi bien qu'à Dionysos. Un autre breuvage mystique était fait d'un mélange de lait et de miel.* Dans la hiérarchie des sacrements, il semble correspondre à un degré inférieur d'initiation. C'est l'aliment imposé au néophyte aussitôt après le baptême. On nourrit de lait et de miel les petits enfants ; et le nouveau myste vient de naître à une vie nouvelle. Mais les dieux en sont friands ; et par ce breuvage il communie pour la première fois avec la divinité. Dans les cultes anatoliens et helléniques, l'offrande du lait fut toujours chère à la déesse qui protège le bétail* ; et primitivement, chez les populations agricoles, la consommation du lait fut sans doute un rite religieux.* Quant au miel, il est sacré. L'Abeille joue un rôle important dans le culte delà Rhéa crétoise.* Encore sous l'Empire, les prêtresses de la Grande Mère ont conservé le nom mystique d'Abeilles.* Il est probable que le miel était utilisé, dans les mystères de la déesse comme dans ceux de Mithra, pour certaines onctions d'exorcisme. Car on lui attribuait de merveilleuses vertus.*

*) Cf. les repas sacrés dans l'initiation mithriaque : Cumont, \emph{l. c.}, p. 320 ; Dieterich, \emph{Eine Mithras liturgie}, p. 100 ss ; dans les mystères de Samothrace, \emph{Arch. ep. Mitt.}, 6, 1882, p. 8, n° 14 ; dans le culte de Jupiter Dolichenus, où les dédicaces signalent des \emph{cenatoria et des triclinia}, \emph{CIL.} 3, 4789 ; 6, 30931 ; 11, 696 ; dans celui de Venus Caelestis, où le mobilier liturgique comporte des \emph{promulsidaria} (plateaux pour breuvages ou gâteaux au miel) et un \emph{mantelium} (sorte de nappe pour la communion), \emph{CIL.} 10, 1598. Sur le lien créé par ce rite entre le dieu et le myste, et aussi entre les membres de la communauté, cf. Robertson Smith, \emph{Religion der Semiten}, p. 204.

*) Babr., \emph{Fab.} 141, 8 et 9.

*) Clem. Alex., \emph{Protrept.} 2, 15 : ἐκ τυμπάνου ἔφαγον, ἐκ κυμβάλου ἔπιον, ἐκερνοφόρησα, ὑπὸ τὸν παστὸν ὕπεδυν. --- Firm. Mat., \emph{op. l.} 18, 1 et 2 : « In quodam templo, ut in interioribus partibus homo moriturus possit admitti, dicit : \emph{de tympano manducavi, de cymbalo bibi et religionis secreta perdidici}, quod graeco sermone dicitur ἐκ τυμπάνου βέβρωκα, ἐκ κυμβάλου πέπωκα, γέγονα μύστης Ἄττεως. » De même, 8 : « Nihil vobis sit cum tympani cibo ; salutaris cibi gratiam quaerite et immortale poculum bibite. » Il faut interpréter comme une allusion à ce rite le passage de saint Augustin, \emph{Civ. Dei} 7, 24 : « Tympanum, turres, Galli... vitam cuiquam pollicentur æternam » ; cf. Catull., \emph{Attis}, 9 : « tympanum tuum, Cybebe, tua, mater, initia. »

*) Hepding, \emph{Attis}, p. 188 ss, tire parti de l'épitaphe d'Aberkios, où il est question du poisson, de la source, du pain et du vin. Il considère Ab. comme prêtre d'Attis. Mais les « démonstrations » auxquelles il se réfère, p. 84, ne sont pas du tout probantes. La thèse païenne a été soutenue par Ficker, \emph{Der heidnische Charakter der Abercius Inschrift} (le Basileus est Jupiter, la Basilissa est Cybèle, le pasteur est Attis), dans \emph{Sitzungsber. Akad. Berl.}, 1894, 1, p. 87 ss ; Harnack dans \emph{Texte und Untersuchungen}, 12, 1895, fasc. 4 (Jupiter, Junon, Attis) ; Dieterich, \emph{Die Grabschrift des Aberkios erklart}, 1896 (Ab. envoyé à Rome par le clergé d'Attis pour assister à la hiérogamie de l'Elagabal palatin et de l'Ourania, Reine de Carthage). L'inscription paraît beaucoup plutôt avoir un caractère chrétien. La meilleure argumentation en faveur de la thèse chrétienne est celle de Duchesne dans \emph{Mélanges Éc. fr. de Rome}, 15, 1895, p. 157 ; cf. Paton dans \emph{Rev. archéol.}, 1906, 2, p. 93 ss.

*) Sallust. phil., \emph{op. l.} 4 ; cf. Usener, \emph{Milch und Honig} dans \emph{Rhein Mus.} 57, 1902, p. 177 ss. Dans l'initiation chrétienne, un breuvage de lait et de miel était présenté aux néophytes après la première communion ; Duchesne, \emph{Orig. du culte chr.}, 1909, pp. 322, 338, 341.

*) Cf. supra, p. 80.

*) Hahn, \emph{Demeter und Baubo}, 1897.

*) Cf. Cook, \emph{The Bee in greek Mythology} dans \emph{Journ. of Hell. St.} 15, 1895, pp. 1-24.

*) Lactant., \emph{Divin. Inst.} 1, 22 ; cf. Porphyr., \emph{De antro Nymph.} 18, où il est dit que l'on donnait le nom de Melissae aux âmes pures des initiés. La crétoise Melissa avait été, disait le mythe, choisie par son père Melisseus pour être la première prêtresse de Rhéa. Sur les rapports de Meter Adrasteia et Melissa, cf. Gruppe, \emph{op. l.}, p. 301. Prêtresses dites Melissai dans les cultes de Demeter, Artémis d'Éphèse, Apollon Pythien, cf. \emph{ibid.}, pp. 136 et 909 ; Drexler dans Roscher, \emph{Myth. Lex.} 2, 2639 s.

*) Cumont, \emph{l. c.}, p. 320.

Un autre rite est l'imposition des Stigmates.* Les chrétiens la comparaient à leur sacrement de confirmation. Mais il ne s'agit pas ici d'une onction, comme dans la liturgie chrétienne. Les stigmates sont de véritables tatouages, opérés par piqûres, au moyen d'aiguilles rougies au feu. Ils impriment sur la peau des signes mystérieux et indélébiles. Étaient-ils appliqués sur le front, comme le « signe du soldat » chez les Mithriastes, comme le signe de la noblesse chez les anciens Thraces* ? Sur le cou et sur les mains, comme dans les cultes syriens ? D'après Prudence, on en marquait plusieurs parties du corps. C'est le sceau divin de la consécration. En perpétuant le souvenir de la profession de foi, il témoigne que désormais le myste appartient au Dieu.

*) Prudent., \emph{Peristeph.} 10, 1076 ss.

*) Mithriastes : Cumont, \emph{l. c.}, p. 319. Thraces : Herodot. 5, 6 ; cf. Hoernes, \emph{Urgeschichte der bild. Kunst}, p. 210 s. Syriens : Lucian., \emph{Dea Syr.} 59. Dans les mystères de Dionysos, Ménades tatouées : Rapp, \emph{Beziehungen des Dionysoskults zu Thrakien u. Kleinasien}, p. 25. Époque néolithique (idoles tatouées, déesses funéraires ? ) : Déchelette, \emph{La peinture corporelle et le tatouage} dans \emph{Rev. archéol.}, 1907, 1, p. 38 ss ; cf. Bertholon, \emph{Origines néolithique et mycénienne des tatouages des indigènes du nord de l'Afrique}, 1904. Tatouages sacrés chez les chrétiens : cf. Hepding, \emph{Attis}, p. 163, n. 2.

La dernière partie de l'initiation semble avoir comporté, selon la tradition commune à la plupart des mystères, un mariage mystique avec la divinité. Ce rite avait lieu la nuit. Le myste pénétrait dans le « Pastos » de la déesse, qui est la chambre nuptiale.* Mais auparavant il était soumis à de pénibles épreuves, image de celles qui attendent l'homme sur la route des enfers, dans les noirs pays d'outre-tombe. Longtemps il courait dans les ténèbres. Obstacles, bruits étranges, apparitions monstrueuses, toute une mise en scène était habilement disposée pour lui inspirer un effroi sacré. Son imagination affolée peuplait l'ombre de fantômes et de démons. Accablé de fatigue, étreint d'angoisse, suant et frissonnant d'épouvante, il croit que sa dernière heure est venue.* « J'approchai des limites du trépas ; je foulai du pied le seuil de Proserpine, » affirme le myste isiaque d'Apulée. Mais, au moment où il pense succomber, une lueur le guide. Il se souvient des réconfortantes paroles du prêtre, lors de la résurrection d'Attis : « ayez foi, mystes ; pour nous aussi de nos épreuves viendra le salut. » Une ouverture se devine, petite et basse, comme celle d'une crypte ; ou simplement une épaisse tenture ferme l'extrémité d'un couloir sombre.* Il se glisse par l'étroit passage, ou sous le rideau, en clamant avec la mélopée des incantations la formule confessionnelle : « dans le tympanon j'ai mangé ; dans la cymbale j'ai bu ; je suis devenu myste d'Attis.* » Et tout à coup à l'obscurité malfaisante succède une radieuse clarté. Il est dans le « Pastos » de Cybèle. La déesse l'attend, accompagnée de ses suivantes. Seules en effet avec les prêtres, ce semble, les femmes ont accès dans cette chambre de gynécée.* Au fond se dressent deux trônes, somptueusement préparés par les prêtresses. Tout autour sont rangées les suivantes, tenant des objets liturgiques, des vases sacrés et des lampes. L'un des trônes est celui de la Dame. Sur l'autre s'assied l'initié.* Peut-être lui met-on dans les mains les attributs d'Attis, bâton ou flambeau,* après l'avoir revêtu d'une somptueuse chlamyde et coiffé d'un bonnet phrygien. Il est possible aussi qu'il y ait eu, comme symbole d'union mystique, imposition de la couronne tourelée.* Devant l'élu et devant la déesse qui le fait participer à sa gloire, les personnages sacrés dansent et chantent. Ils accomplissent les gestes rituels de la salutation et prennent les attitudes de l'adoration.* Dans l'éblouissement de cette \emph{Thronôsis}, le myste pouvait s'imaginer qu'il était vraiment devenu dieu.*

*) \emph{Le Pastos} paraît être identique au \emph{Thalamos} et au \emph{Cubiculum}. Thalamépoles de la déesse en Asie Mineure : \emph{Anthol. Pal.} 6, 173 (femme), 220 (un Atys qui dédie à Kubélè, sur les bords du Sangarios, une Thalamè, sans doute une grotte) ; ce titre était aussi donné aux eunuques fie harem, cf. Plut., \emph{Alex.} 30. Thalamoi de la déesse sur les monts Kubela : Hesych., cf. supra, p. 16, n. 2. Thalamoi de la Meter Lobrinè (grottes) : schol. ad Nicand. \emph{Alexipharm.} 8. Cubiculum de Mater Deum au promontoire de Circeo : \emph{CIL.} 10, 6423. De même, Cubiculum d'Isis dans Apul., \emph{Met.} 11, 17 ; Thalamos d'Atargatis dans Lucian., \emph{Dea Syr.} 31. Le \emph{Pastophorion} était une chapelle secrète dans les temples sémitiques. Le \emph{Pastos} était aussi un petit tabernacle portatif, renfermant une image de la divinité.

*) Cf. Hesych., s. v. ἁδοφοίτης. Sur le sens du mot \emph{moriturus} dans le passage de Firmicus Maternus cité supra, p. 181, n. 2, voir Hepding, \emph{Attis}, p. 194 s ; cf. \emph{Apul.}, \emph{Met.} 11, 23 : « Accessi confinium mortis et calcato Proserpinae limine, etc. ; nocte media vidi solem candido coruscantem lumine ; deos inferos et deos superos accessi coram et adoravi de proxumo. »

*) Le mot \emph{Pastos} indique un objet brodé ; c'est le rideau brodé du lit nuptial. Par extension il a pris le sens de chambre nuptiale.

*) Sur les deux variantes citées supra, p. 181, cf. Dieterich, \emph{Eine Mithras-liturgie}, p. 217, et Hepding, \emph{Attis}, p. 184 s.

*) \emph{CIA.} 2, 1, 624. Décret des Orgéons du Pirée, vers l'an 180 av. J.-C., cf. Poland, \emph{Gr. Vereinswesen}, 1909, p. 548. Parmi les fonctions de la prêtresse : στρωννύειν θρόνους δύο ὡς καλλίστους, περιτιθέναι δὲ ταῖς φιαληφόροις καὶ ταῖς περὶ τὴν θεὸν οὔσαις ἐν τῷ ἀγερμῶι κόσμον, \emph{etc.}

*) Le texte précédent est expliqué par un passage de Platon, \emph{Euthyd.} 277 D, relatif aux mystères des Corybantes : ὅταν τὴν θρόνωσιν ποιῶσι περὶ τοῦτον ὃν ἂν μέλλωσι τελεῖν καὶ γὰρ ἐκεῖ χορεία τίς ἐστι. Le myste de Sabazios s'asseyait sur le σκίμπους, celui d'Isis se tenait debout sur une estrade « ante deæ simulacrum, » Apul., \emph{Met.} 11, 24.

*) Cf. Apul., \emph{l. c.} : « Manu dextera gerebam flammis adultam facem, » dit le myste d'Isis. On l'avait aussi coiffé d'une couronne de feuillage et revêtu de la \emph{stola Olympiaca}.

*) C'est ce que pourrait signifier le passage de saint Augustin cité supra, p. 181, n. 2, où il est question de la couronne tourelée de Cybèle. La couronne de la déesse est figurée sur un autel dédié par des mystes « de la Religion, » \emph{CIL.} 12, 405. Un rite signalé par Synesius (début du 5e s. après J. C.), \emph{Ep.} 3, p. 639 H, consistait à couronner de tours les fiancées, comme Cybèle ; sur les ressemblances entre rites de mariage et rites d'initiation, cf. S. Reinach, \emph{Cultes}, \emph{etc.}, 1, p. 309.

*) Hypothèse suggérée par Prudence, \emph{l. c.} 1048 : « Omnes salutant atque adorant, » quand le taurobolié sort de la fosse, régénéré par le baptême.

*) Farnell, \emph{Cults of the greek States}, 3, p. 301 : « The solemn θρόνωσις was part of the mesmeric process winch aimed at producing the impression of deification in the mortal. »

Désormais son salut est assuré. C'est alors qu'a lieu la Tradition des derniers Symboles, \emph{Signa et Symbola}. Les autres lui avaient permis de franchir chaque degré de l'initiation.* Ceux-ci sont les instruments et les gages de l'éternel bonheur : signes de reconnaissance, amulettes qui le protégeront, instructions secrètes qui lui permettront d'arriver au but, paroles infaillibles, qui lui ouvriront l'entrée des béatifiques demeures. Pour nous faire une idée de ces instructions, il faut lire celles du rituel orphique au 6e siècle, écrites à la pointe sur des lames d'or et déposées dans les tombeaux des initiés. Aussi bien, Kybélè y figure-t-elle parfois avec Démeter et Gê Pammêtor, la Terre Mère universelle.* « Tu trouveras dans les demeures d'Hadès, à gauche, une source voisine d'un cyprès blanc ; garde-toi de cette source et n'en approche point. Tu en trouveras une autre, dont les eaux fraîches coulent du lac de Mémoire. Et des gardiens se tiennent devant. Dire : De la Terre et du Ciel étoilé je suis l'enfant ; mais moi (l'âme), j'ai une origine céleste* ; je suis desséchée et me meurs de soif ; donnez-moi vite de l'eau fraîche qui coule du lac de Mémoire. Et ils te donneront à boire de la source divine.* » Sur la route que suivaient les mystes de Cybèle, Dame des sources et des eaux vives, probablement aussi jaillissaient de pernicieuses ou bienfaisantes fontaines. « J'arrive pure et issue de pures, reine des Enfers, et vous tous, dieux immortels. Car moi aussi je me glorifie d'être de votre race bienheureuse. Maintenant me voici suppliante devant la sainte Perséphonè, afin que sa grâce m'envoie au séjour des âmes pures. » Ou bien, sûre du triomphe, l'âme disait : « J'ai pris mon vol, j'ai échappé au cercle terrible des pesantes douleurs ; je suis entrée dans la couronne désirée ; je suis descendue dans le sein de la Despoina, reine des enfers.* » Et l'initié sait à l'avance quelles seront les paroles de bienvenue : « Salut et joie, à toi qui as subi la souffrance* ; mortel tu es devenu dieu. Salut et joie ! prends la route de droite, vers les prairies sacrées et les bois de Perséphonè.* » Au me siècle de notre ère, les mystes d'Attis invoquent de même, avec Cybèle, Mère des Dieux, la Terre Mère et Aerecura, qui est l'un des aspects de Proserpine.* Ils invoquent aussi Mercure Evangelos, qui les conduit auprès des Grands Dieux.* Peut-être sont-ils guidés sur la voie des enfers par leur Bon Ange, comme les Sabaziastes.* Nous ne savons s'ils croyaient, comme ceux-ci, à un jugement des âmes. La conscience morale, élargissant de plus en plus son domaine, avait fini par transformer l'eschatologie de tous les mystères.

*) Arnob. 5, 26 : « Symbola quae rogati sacrorum in acceptionibus respondent. »

*) Cf. supra, p. 35, et une inscription provenant de Crète : Joubin, dans \emph{Bull. Corr. Hell.}, 1893, p. 177 ; voir aussi H. Weil, \emph{Études sur l'antiquité grecque}, 1900, p. 37 ss. A l'influence des Thraces, il faut ajouter celle de la mystique babylonienne, qui paraît avoir été considérable sur les cultes anatoliens de la Métèr, précisément vers ce 6e s. ; cf. Gruppe, \emph{l. c.}, p. 1544 ss.

*) Cf. l'écho de ces croyances dans certaines formules gravées sur les tombeaux de mystes, à l'époque impériale ; \emph{CIL.} 3, 6384 : « Corpus habent cineres, animam sacer abstulit aer » ; 5, 2160 ; 6, 9663 : « In hoc tumulo jacet corpus exanimis cuius spiritus inter deos receptus est. » A propos des mystes de Cybèle, Prudence, \emph{l. c.}, 1065, emploie l'expression « caelum meretur. »

*) \emph{IGSI.} 638.

*) \emph{Ibid.} 641.

*) Cf. les paroles du prêtre au moment de la Résurrection d'Attis. Il semble bien qu'il y ait, dans cette notion phrygienne de la souffrance qui purifie, une influence de l'orphisme.

*) \emph{Ibid.} 642 ; cf. S. Reinach, \emph{op. l.} Il, 1906, p. 123 ss.

*) \emph{CIL.} 8, 5524 ; Cagnat, \emph{Ann. épigr.}, 1895, 81. D'autre part, on a retrouvé des figurines de Cybèle dans un temple de Demeter et Persephonè, à Halicarnasse : Newton, \emph{Halic., Cnidus and Branchides}, p. 328.

*) Hermès porte cette épithète dans les mystères de Samothrace ; cf. H. Psychagôgos, Psychopompos : Hippolyt. 5, 7 (à propos des Naasséniens, qui ont beaucoup emprunté aux mystères phrygiens). Rôle d'Hermès dans les mystères de Cybèle : Pausan. 2, 3, 4 ; Julian., \emph{Or.} 5, p. 179 B ; \emph{CIL.} 6, 499 (supra, p. 167, liste B, \emph{3}). --- Sur les monuments figurés de l'époque impériale, Mercure apparaît à côté de Cybèle et d'Attis. Tablettes de bronze en forme de « naiskoi » : Ovid., éd. Burmann (Amsterdam, 1727), 3, p. 247 ; Friederichs, \emph{Kleine Kunst}, 2005 \emph{b} ; \emph{Jahrbuch d. Inst.}, 1892, \emph{Archaeol. Anzeiger}, p. 111, n° 15. Lampe en terre cuite, à Borne, Palais des Conservateurs, salle des terres cuites, vitrine centrale : Cybèle assise entre Attis et Mercure. Pectoral de l'archigalle publié par Montfaucon : Cybèle debout, entre Zeus et Hermès. Stèle de Trajanopolis. Monnaie de Thyateira : Mionnet, 4, p. 166, n° 955 (Caracalla). Peut-être le caducée sur un autel taurobolique, \emph{CIL.} 12, 1568.

*) Cf. le tombeau d'un \emph{antistes Sabazis} : Garrucci, \emph{Tre sepolcri con pitture d. superstiz. pagane}, Naples, 1852, et dans les \emph{Mélanges d'archéol.} de Cahier et Martin, 4, 1854, p. 1 ss ; Maass, \emph{Orpheus}, 1895, p. 205 ss ; Cumont, \emph{Les mystères de Sab. et le judaïsme}, dans \emph{C. r. Acad. Inscr.}, 1906, p. 72 ss).

Pendant de longs siècles, en effet, ceux de Phrygie n'avaient contenu aucun enseignement moral. D'abord ils n'étaient destinés qu'à faire réussir les récoltes d'une communauté, d'une tribu, d'un peuple. Comment s'opéra la substitution des félicités d'outre-tombe aux prospérités delà vie présente ? Toutes les divinités chthoniennes sont en même temps des divinités infernales. Elles reçoivent dans leur sein les défunts comme elles reçoivent les semences ; elles peuvent donc leur préparer une semblable destinée. Dans les mystères d'Éleusis, la présentation de l'épi de blé, rite final de l'époptie, « ne constituait sans doute à l'origine qu'un rite agricole ; il n'y avait rien à y changer pour en faire un symbole de palingénésie humaine.* » Dans les mystères métroaques d'Anatolie, la nature même du mythe facilitait cette substitution. Mais, quel que fût l'objet des rites, ils gardaient une portée magique. Jusqu'aux derniers temps du paganisme, dans les symboles et formules des mystères phrygiens, l'antique magie ne perdit jamais complètement ses droits.* L'idée morale, toutefois, n'avait pas attendu l'empereur Julien ni les clarissimes du 4e siècle pour améliorer et spiritualiser la vieille doctrine. En Orient même, la notion de pureté rituelle, largement développée sous la double influence de l'orphisme et du sémitisme, lui en avait ouvert l'accès. Ce sont les Galles surtout qui, dans le monde occidental, nuisent à la réputation de ces mystères.* D'autre part, ce sont des adversaires, les polémistes chrétiens, qui dénoncent l'immoralité de ces nocturnes cérémonies. A dire vrai, en dehors de tout enseignement moral, par les seules préoccupations qu'elle suppose et les seules disciplines qu'elle impose, l'initiation était capable d'éveiller les consciences, d'exalter la vie intérieure, de réagir heureusement sur les âmes. « Ceux qui ont participé aux mystères, » rapporte Diodore de Sicile, « passent pour devenir plus pieux, plus justes et meilleurs en toutes choses. »

*) Goblet d'Alviella, \emph{l. c.}, p. 351.

*) La tablette de bronze reproduite dans l'\emph{Archaeol. Anzeiger}, \emph{l. c.}, forme diptyque avec une autre plaque où figure un dieu barbu (Sabazios ? ) à bonnet phrygien, entouré d'une quantité considérable de symboles évidemment magiques.

*) Saint Augustin, \emph{Civ. Dei} 7, 26, indique nettement que la consécration du Galle correspond aux mêmes idées que celle du simple myste : « Ut post mortem vivat beate. » Sur la participation des Galles à la tradition des mystères dans les cérémonies d'initiation, cf. Justinus Martyr, \emph{Apolog.} 1, 27, p. 70 E ; Augustin., \emph{op. l.} 6, 7.

[Planche 3. --- 1. Plaquette en bronze. \emph{Cybèle au sceptre}, dans un naiskos (provenant de Salonique). --- Au Musée de Lyon.](https://cdn.solaranamnesis.com/HenriGraillot/3-1.jpeg)

[Planche 3. --- 2. Diptyque en bronze : \emph{Sabazius, Cybèle entre Hermès et Attis} (provenant de Rome). --- Au Musée de Berlin.](https://cdn.solaranamnesis.com/HenriGraillot/3-2.jpeg)
\clearpage
\section{Chapitre 5}
\begin{center}
La Doctrine Métroaque au 3e Siècle. Cybèle et Attis Dieux Tout-Puissants.
\end{center}
\paragraph{}
1. Prépondérance officielle de la Grande Mère sur les autres divinités importées d'Orient. Avantages réciproques du patronage qu'elle exerce sur leurs cultes. L'évolution delà doctrine métroaque s'achève en Italie. --- 2. L'Omnipotente Cybèle, reine des cieux, maîtresse des saisons, maîtresse des éléments, reine de la terre et des enfers. L'Omniparente, mère et nourricière universelle, maîtresse de la fortune, dame de Bon Secours, dame de Salut pour le corps et pour l'âme. --- 3. Attis. Ses rapports de dieu lunaire et solaire avec Mên et Adonis, qu'il supplante. Ses rapports avec Mithra. L'Attis Soleil, roi des cieux, maître de la génération ;son rôle bienfaisant de médiateur. Ses rapports avec les dieux judaïsants, Sabazios et Zeus Hypsistos. Attis Hypsistos, dieu Éternel et Omnipotent. --- 4. La part de Rome dans l'élaboration définitive de la doctrine. L'orientalisme à Rome au ni» siècle. Empereurs et impératrices de race syrienne. Héliogabale myste de Cybèle.

\subsection{1.}

Entre toutes les divinités venues d'Orient, la déesse phrygienne occupait depuis longtemps à Rome une situation privilégiée. L'institution d'un culte phrygio-romain avait assuré sa prépondérance. Le monopole du sacrifice taurobolique, organisé et favorisé par l'État, rendit encore plus efficace cette suprématie. En Occident, Cybèle était devenue la haute protectrice des dieux d'Asie. Au 2e siècle, elle exerce sur eux un véritable patronage.

Déjà Mâ-Bellone, l'Omnipotente et l'invaincue, qui possédait à Comana de Cataonie plus de six mille hiérodules des deux sexes,* avait dû s'effacer ici devant elle. La grande déesse* de la Cappadoce et du Pont se résignait au rôle de dame de compagnie. Les Romains la considéraient comme la suivante et la servante de la Mère des Dieux.* Aussi bien, les Bellonaires font-ils partie du cortège de l'Idéenne. On les voit processionner avec les Galles, se joindre à eux pour mendier dans les maisons,* se taillader les bras, comme eux, pour offrir des libations de sang humain. On ne les distinguerait pas des Galles s'ils ne portaient la robe noire. Rome confond les uns et les autres sous la dénomination commune de fanatiques. Parmi les statues de Cybèle retrouvées à Rome, il en est une qui lui fut consacrée par un fanatique de Bellona Pulvinensis.*

*) Strab. 12, 2, 3.

*) « Dea Magna, » Tibull. 1, 6, 50.

*) « Dea pedisequa, » \emph{CIL.} 6, 3674 \emph{a} ; Cagnat, \emph{Ann. épigr.}, 1898, n° 61. Sur la frise de Pergame, Enyo, autre aspect de Bellone, est rapprochée de Cybèle : Puchstein, \emph{Zur pergam. Gigantom.}, dans \emph{Sitzungsber. Akad. Berl.}, 1889, p. 330.

*) Juven. 6, 512: « Bellonae Matrisque deum chorus. » Le clergé de Bellone était organisé comme celui de la M. d. D. et comportait des prêtres des deux sexes ; cf. une grande prêtresse, Tibull. 1, 6, 43.

*) \emph{CIL.} 6, 490. La statue est au musée des Thermes de Dioclétien.

Une autre déesse mère,* l'égyptienne Isis, admise dans le panthéon romain au temps de Caligula, sut mieux sauvegarder son indépendance. Elle fut même une redoutable rivale de l'Anatolienne. Les prédications de ses prêtres, habiles théologiens, lui gagnaient de nombreux prosélytes ; et les séductions de son culte attiraient vers elle bien des âmes féminines. Cybèle et Isis vécurent cependant en bonne intelligence. Les deux déesses avaient déjà pris contact en Asie Mineure, au temps où les Ptolémées occupaient militairement une partie de la côte.* Elles eurent en Italie des chapelles communes.* Parfois la Grande Mère admet Isis à partager son temple municipal, l'un des plus importants de la ville, et son clergé officiel, où les meilleures familles s'honorent d'entrer.*

*) Les dédicaces s'adressaient en général à \emph{Isis Regina}, parfois aussi à \emph{Isis Mater}. Sur les causes de l'attraction exercée par la religion égyptienne sur le monde romain, cf. Cumont, \emph{Religions orientales dans le paganisme rom.}, 1907, pp. 106-123.

*) En Égypte même, cf. les rapports d'Isis avec la phrygienne Misé (v. supra, p. 13), importée à Alexandrie : Orph., \emph{Hymn.} 52, 9 s.

*) \emph{CIL.} 5, 4007, sur la rive droite du lac de Garde : « Matri Deum et Isidi... fanum et pronaum ex voto. »

*) \emph{CIL.} 9, 1153, à Aeclanum, au temps de Domitien.

A plus forte raison les autres divinités orientales, transportées sur une terre étrangère où elles demeurent hors la loi, recherchaient-elles avec empressement l'appui de l'Idéenne. Leurs prêtres, sans cesse menacés d'expulsion, avaient tout bénéfice à se concilier un puissant clergé. De plus, certains cultes, qui se seraient mal adaptés à la religion gréco-romaine, pouvaient s'assimiler aisément au nouveau culte phrygio-romain. Une affinité naturelle rapprochait de Cybèle et Attis tous ces émigrés. Sans renier leur foi, des Syriens et des Phéniciens pouvaient invoquer la Mère des Dieux, qui est reine du ciel, prier dans ses sanctuaires et lui consacrer des monuments votifs. En pénétrant dans un Metrôon, l'adorateur d'Atargatis croyait retrouver sa déesse syrienne. Assise sur le lion ou trônant entre deux lions, coiffée du calathos, vêtue du chiton et du péplum, tenant d'une main le sceptre royal et de l'autre le tambourin, la dame de Bambyce ressemble, à s'y méprendre, à celle de Pessinonte.* Ce sont deux sœurs, affirme un Galle dans le roman d'Apulée.* Toutes deux ont leurs Galles eunuques ; et l'analogie des rites explique celle des images. C'est la même divinité, déclare un officier-poète, qui vivait au temps des Sévères.* A Héliopolis (Baalbek), la déesse locale a revêtu le double aspect de Cybèle et d'Atargatis.* Il en est de même à Ptolémaïs* et à Gabala,* pour les parèdres du Baal solaire au caducée. A Samarie, la Baalat du mont Garizim, entourée de lions, une patère dans la main droite, le bras gauche appuyé sur le tympanon, ne diffère plus de la Mère Phrygienne.* Il n'est donc pas surprenant qu'un Ascalonite transplanté en Dalmatie, un soldat probablement, ait dédié dans un temple de la Mère des Dieux un autel aux divinités d'Ascalon.* Par reconnaissance, il ajouta sur la dédicace le nom de la Mère hospitalière ; enfin, pour qu'à son Baal et à son Astarté correspondît un couple divin, il réunit à la Grande Mère le Grand Janus, seigneur latin du ciel et Père des Dieux.* Dans la ville cosmopolite d'Ostie, un Levantin de l'Asie sémitique, adorateur d'Aphrodite Astarté, choisit le Metrôon pour y consacrer une figurine de Vénus.* De même jadis, au Pirée, quand les Phéniciens de Citium n'avaient pas encore fondé le temple de leur Aphrodite Ourania, c'était dans le Metrôon qu'ils déposaient leurs ex-voto* ; et d'autres Orientaux y apportaient des offrandes à leur Artémis Nana, qui est une déesse chaldéo-babylonienne.* Sur les bords du Rhône, entre Tarascon et Arles, on a trouvé une patère historiée où figure l'Astarté aux colombes ; elle porte une dédicace à Magna Mater.* La théologie astrologique autorisait toutes ces confusions. Car la planète Vénus, qui est en rapport avec Atargatis et Astarté, comme avec Isis, est également consacrée à la Mère des Dieux.* Ainsi s'explique la divinité complexe de la Mère des Dieux Aphrodite,* que l'on adorait au Pirée dans le premier siècle de notre ère. La coutume sémitique des triades divines, importée dans certaines villes cosmopolites d'Occident, contribua aussi au développement de ce syncrétisme. Brindes possédait un temple de Mater Magna, Dea Suria et Isis.*

*) Monnaies d'Hierapolis à l'effigie de Caracalla, d'Alexandre Sévère, de Mammaea, des deux Philippes et d'Otacilia ; l'identification de la divinité est assurée par l'exergue : θέας συρίας Ἱεροπολίτων. Déesse accostée de deux lions : Mionnet 5, pp. 141 et 142, n°s 51, 52, 54, 55, 58 ; Wroth, \emph{Catal. Galatia}, \emph{etc.}, pl. 17, 14 et 17 ; cf. un groupe de statuettes votives à Rome, \emph{CIL.} 6, 115, 116, 399. Elle est décrite sous cet aspect par Lucien, \emph{Dea Syr.}, 31 (cf. Reinach dans \emph{Rev. archéol.}, 1902, 1, p. 31) et Macrobe, \emph{Saturn.} 1, 23, 18. --- Assise sur le lion : Mionnet, \emph{l. c.}, n°s 50, 53, 56, 57 ; Wroth, \emph{l. c.}, pl. 17, 15, et pp. 144, 145. Ce type existe déjà au 4e siècle avant notre ère, avec la légende « Até » en syriaque : Six dans \emph{Numism. Chronicle}, 1878, p. 103 ; de Vogué, \emph{Mél. d'arch. or.}, p. 47 ; Babelon, \emph{Monnaies gr., Perse et Phénicie}, p. 52, fig. 15 ; cf. Six, \emph{l. c.}, 1884, p. 112, n° 24, monnaie d'un satrape de Cilicie, au temps d'Alexandre. Sur de nombreuses monnaies de Phrygie et de Galatie, à l'époque impériale, Cybèle est aussi assise sur le lion ; v. en particulier Pessinonte (M. Aurèle, L. Verus, Caracalla).

*) \emph{Met.} 9, 10 : « Deum mater sorori suae deae syriae. » Sur les rapports de Cybèle avec les grandes déesses sémitiques : Baudissin, \emph{Stud. z. semit. Relig.}, 2, p. 203 ss.

*) \emph{CIL.} 7, 759. Le type des lettres est du 3e s.

*) \emph{CIL.} 6, 423 (prise à tort pour Rhéa). La Vénus Héliopolitaine trouvée dans le temple rond du 3e s. de notre ère est également assise, mais accostée de sphinx : Frauberger, \emph{Akrop. von Balbek}, 1892, p. 5, fig. 4 ; S. Reinach dans \emph{Rev. archéol.}, 1902, 1, pp. 19-33 et pl. 2-4.

*) Mionnet 5, p. 481, n° 42 ; Cohen, 2e éd. 5, p. 334, Valérien père, n° 369 ; derrière la déesse, prise à tort pour Cybèle, un caducée. Sur les Baals au caducée, cf. Dussaud, \emph{Notes de mythol. syrienne}, dans \emph{Rev. archéol.}, 1903, 1, p. 142 s.

*) Déesses improprement dénommées Cybèles par Mionnet, 5, pp. 234-238, n° s 633 (Trajan), 639 (Commode), 646 (Julia Domna), 653 (Caracalla) ; Wroth, \emph{op. l.}, pl. 38, 7-10, 13 ; cf. Dussaud, \emph{l. c.}, 1903, 1, p. 365, et 1904, 2, p. 247, fig. 26.

*) Monnaie de Philippe père : De Saulcy, \emph{Numism. de la Terre Sainte}, p. 266, n°s 5, 6 ; Cohen, \emph{l. c.}, p. 131, n°s 363, 364. A rapprocher d'un bronze d'Otacilia, Cohen, \emph{l. c.}, p. 157, n° 118 : déesse tourelée, debout, le pied droit sur un lion couché, la main g. sur la haste, la dr. tenant le mont Garizim. L'ancien temple du Garizim avait été détruit par Jean Hyran, au 2e s. av. J.-C. ; il fut remplacé par un temple romain où l'on vénérait le Jupiter et la Vénus d'Héliopolis.

*) \emph{CIL.} 3, 6428: « Dis Asca(lonitanis) et Matri Magn(ae) Iane (\emph{sic}) M(agno). » La déesse représentée sur les monnaies d'Ascalon est une Astarté : De Saulcy, \emph{op. l.}, pl. 10, 1, 5 et 11 (peut-être influencée ici par Isis). Hérodote 1, 105, dit que la divinité locale est Aphrodite Ourania, et Pausanias 1, 14, 7, qu'elle est d'origine phénicienne ; cf. Dussaud dans \emph{Rev. archéol.}, 1904, 2, p. 242.

*) Sur le rapprochement de Cybèle et Janus, \emph{CIL.} 8, 11797, et Augustin., \emph{Civ. Dei} 7, 28 : « Dicitur caput deorum Ianus, caput dearum Tellus, Mater scilicet magna » ; de même, 3 : « Ianus omnium initiorum potens » ; cf. Graillot, dans \emph{Rev. archéol.}, 1904, 1, p. 344 s ; ajouter Lyd., \emph{De mens.} 4, 3, sur Janus, souverain des âmes.

*) Visconti dans \emph{Monumenti} 9, 1869, pl. 8, et \emph{Annali}, 1869, p. 210 ss et 222 s : statuette de Vénus nue, à sa toilette, en bronze ; musée du Latran ; Helbig, \emph{Guide}, 1893, 1, n° 699 ; Reinach, \emph{Répert. de la Stat.} 2, 1, p. 359.

*) Foucart, \emph{Assoc. relig.}, pp. 99 et 198. On voit aussi une prêtresse d'Aphrodite Syrienne offrir des sacrifices à sa déesse pour les Orgéons de la Mère des Dieux. Sur les rapports d'Aphrodite et de Cybèle à Chypre, cf. Dieterich dans \emph{Philologus} 52, 1894, p. 12.

*) \emph{CIAtt.} 3, 131 (sous l'empire) ; sur cette déesse, cf. supra, p. 12.

*) \emph{CIL.} 12, 5697, 3.

*) Plin., \emph{H. n.} 2, 37 : « Alii enim Iunonis, alii Isidis, alii Matris Deum appellavere. » Bouché-Leclercq, \emph{Astrol. gr.}, p. 343. Noter que le Taureau, consacré à Cybèle, est pour les astrologues le véhicule d'une divinité lunaire, dont ses cornes symbolisent le croissant et qui répond au type d'Ishtar-Astarté-Aphrodite ; il est la maison astrologique de Vénus.

*) \emph{CIAtt.} 3, 136 ; Sybel, 436, et Arndt-Amelung, \emph{Einzelaufn.} 3, n° 724. Déjà Charon de Lampsaque dit que Kybébè est l'Aphrodite des Phrygiens et des Lydiens : \emph{Fr. Hist. Gr.} 4. p. 627 ; cf. Hesychius, s. v. Cybebè : « C'est la Mère des Dieux, dit-il, et aussi Aphrodite. »

*) \emph{CIL.} 9, 6099. Sur une intaille, triade Sérapis, Cérès, Cybèle : Tassie et Raspe, \emph{Descript. catal. of gems}, 1496 ; Lafaye, \emph{Divin. d'Alexandrie}, p. 315, n° 174. --- Sur une lampe, triade Sérapis, Isis, Cybèle : Lafaye, \emph{op. l.}, p. 304, n° 134. --- Triade syrienne sur une intaille de l'époque impériale : Furtwaengler, \emph{Geschnitt. Sleine im Antiq. Berl.}, 7158, et pl. 54 ; triade syro-phénicienne dans un temple à triple cella, sur une monnaie d'Orthosia : Mionnet, 5, p. 366, n° 187 (Eliogabal) ; cf. De Vogüé, \emph{Inscr. sémit.}, p. 77, n° 126. C'était une conception chaldéenne, que les dieux se groupaient par triades : Maspero, \emph{Hist. anc. des peuples de l'Orient}, 1, p. 650.

Quant au dieu perse Mithra, il avait, dès le temps des Achéménides, au 6e siècle, suivi en Asie Mineure ses mages émigrés. Des colonies de Maguséens, c'est-à-dire de mages, s'étaient en effet implantées en Cilicie, en Cappadoce, en Phrygie, en Galatie et dans le Pont, « conservant partout les lois de leurs ancêtres et les rites de leurs mystères.* » Pendant de longs siècles, le dieu avait paisiblement vécu au milieu des tribus phrygiennes. Entre les conceptions religieuses des deux cultes un rapprochement et des échanges s'étaient produits. La doctrine persique avait épuré les croyances indigènes. Mithra sut profiter, en Occident, de ces antiques relations avec la Grande Mère. Le plus ancien Mithraeum que nous connaissons en Italie, antérieur à l'année 142, est contigu à un Metrôon municipal.* A Milan, nous constatons des rapports entre les mystes du dieu iranien et les Dendrophores de la Mère Phrygienne.* Dans presque toutes les localités où nous rencontrons la déesse, nous retrouvons Mithra.* Il semble que, sur toute l'étendue de l'Empire, les deux cultes aient vécu en intime communion. Ils se complétaient l'un l'autre. Seuls, en effet, les hommes sont admis aux cérémonies secrètes du mithriacisme. L'austérité de ses dogmes religieux proscrit l'élément féminin. La religion métroaque, tout au contraire, sentimentale et sensuelle, faite pour les âmes tendres et passionnées, s'adressait de préférence aux femmes. Non seulement elle les initiait à tous ses mystères ; mais elle leur réservait aussi, dans la liturgie mystique comme dans le culte public, un rôle de protagonistes. Cybèle « accueillit les épouses et les filles des mithriastes. »

*) Bardesane dans Euseb., \emph{Praep. evang.} 6, 10, 37 ; v. les autres textes réunis dans Cumont, \emph{Mithra}, 1, pp. 9 et 10.

*) A Ostie ; cf. Cumont, \emph{op. l.}, 1, pp. 245, 265, 333 ; 2, pp. 418 et 523.

*) \emph{CIL.} 5, 5465.

*) A l'aide des textes épigraphiques et des monuments figurés, j'ai pu faire la constatation pour ces villes d'Italie : Angera (Milan), Aquilée, Bénévent, Bergame, Bolsène ( ? ), Brescia, Lanuvium, Ostie, Padoue, Pola, Praeneste, Sentinum, Syracuse, Tibur, Velletri, Vérone ; --- de Gaule : Lyon, Vienne, Vieux-en-Val Romey, Trêves et, sur les bords du Rhin, Bonn, Cologne, Mayence, Rheinzabern ; --- d'Afrique : Philippeville et Sétif.

Le patronage de l'Idéenne fut certainement précieux pour toutes les divinités de l'Asie dont l'existence était illicite. Mais la déesse en retira pour elle-même un bénéfice considérable. D'abord son clergé tenait ainsi la haute main sur les cultes d'importation orientale. Il pouvait exercer sur ces autres cultes une véritable police, encouragée sans doute par l'administration municipale, qui le nommait, et par l'État, dont il dépendait en dernier ressort. C'était un surcroît d'autorité morale pour ce clergé qui, par ailleurs, prenait déjà tant d'importance dans la vie religieuse du monde romain. Il avait, de plus, un avantage matériel à détourner vers ses propres temples la dévotion des Asiatiques. Cette foule d'émigrés comprenait des gens de toute classe, surtout des esclaves, il est vrai, mais beaucoup d'affranchis parvenus, de riches marchands, de fonctionnaires impériaux, dont la piété pouvait être généreuse. Il s'efforça donc d'élargir sa religion, pour satisfaire au plus grand nombre possible de dévotions. Aussi bien, au contact incessant de cultes similaires, cette transformation était-elle devenue fatale. La Grande Mère, qui laisse Isis et l'Ourania trôner à ses côtés, qui accepte dans son sanctuaire les ex-voto de leurs adorateurs, peu à peu s'approprie certains de leurs attributs et une part de leur puissance. Le couple phrygien finit même par s'assimiler complètement plusieurs divinités anatoliennes. Alliances, compromis, assimilations partielles ou totales, tels furent, pour le culte métroaque, les éléments d'une vie nouvelle et supérieure. L'évolution, commencée en Asie Mineure, s'achevait en Italie.

\subsection{2.}

Cybèle n'est pas seulement déesse « de grand pouvoir* » ; elle est la Dame de toute puissance, parèdre de Zeus Pantocrator. Ainsi se l'imaginait-on, invisible et présente, sur la cime de l'Ida* et sur celle du Sipyle* ; ainsi l'invoquait-on à Délos dans le temple des Dieux Étrangers, où les Syriens venaient prier Hadad et Atargatis, couple divin de leur ville sainte.* Elle tient le sceptre* ; car elle partage avec Zeus la domination du monde, et « Zeus ne fait rien sans elle.* » L'aigle du dieu va se poser à côté de la déesse, dont il manifeste ainsi la souveraineté.* « Assise sur le trône du roi Zeus,* » elle est Reine : \emph{Basileia, Anassa}.* En elle s'absorbe la personnalité de toutes les Junons de Syrie, de Grèce et de Rome. Elle est la reine de tout, \emph{Pambasileia} ; elle est celle qui soumet tout à son empire, \emph{Pamdamator}.* Elle est la Dame suprême, \emph{Despoina},* \emph{Domina}, comme Zeus est le suprême Seigneur ; et seule elle a le droit de porter ce titre.* Car elle est à l'origine des choses ; on l'appelle l'\emph{Archegonos}.* Elle est « celle qui existe par elle-même » et qui n'a pas eu de mère.* Sœur de Cronos, disait-on, elle était devenue son épouse,* et elle avait engendré Zeus, qui devint aussi son époux. Mais, à vrai dire, le Zeus Idéen et Phrygien s'identifie lui-même à Cronos, à Saturne, à Janus, à ce dieu du temps et de la durée que d'autres cosmogonies déclaraient antérieur au dieu du ciel ; il est Cælus Aeternus Jupiter Optimus Maximus.*

*) Πολυπότνια : Apoll. Rh. 1, 1151 ; cf. l'épithète δυνατή, Ramsay, \emph{Cities and Bish.}, p. 154. Kρα[τ]ο(υ)ς μεγάλου, conjecture de Domaszewski dans \emph{Arch. ep. Mitt.} 7, 1883, p. 176, à propos d'une inscr. de Phrygie ; mais il s'agirait delà Mère des Dieux du bourg de Cranos Mégalos : Mordtmann dans \emph{Ath. Mitt.} 10, 1885, p. 14 ; Radet dans \emph{Archives des Missions}, 1895, p. 272, n° 21 ; cf. \emph{Ath. Mitt.} 22, 1897, p. 352.

*) Ils y avaient tous deux un autel : Ps. Plut., \emph{De flum.} 13 ; en Arcadie, sur le mont Azanion, on les adore tous deux \emph{more Idaeo} : Lact. Plac., ad Stat., \emph{Theb.} 4, 292.

*) Zeus Acraios et le lion de la Mère Sipylène sont réunis sur des monnaies de Smyrne : Mionnet 3, p. 208, n°s 1139-1141 ; \emph{Suppl.} 6, p. 320, n° 1570 ; Macdonald, \emph{Hunterian Coll., greek coins}, p. 370, n° 132 = \emph{British Mus. coins}, pl. 27, 5. Ajouter : Cybèle et Zeus Hypsistos sur des reliefs mysiens, \emph{Bull. Corr. Hell.}, 1899, p. 598 ; Zeus et Meter Dindyménè à Cyzique, \emph{ibid.} 1888, p. 187 ; Zeus Sôter et la M. d. D. Agdistis à Eumeneia, Phrygie : \emph{CIG.} 3886 ; Zeus et Cybèle sur des monnaies de Maeonia, 1.3'die : Babelon, \emph{Coll. Waddington}, 5058. De même, sur les monnaies d'Hierapolis de Syrie, on voit le couple de Baal au sceptre et d'Atargatis au lion.

*) Hauvette dans \emph{Bull. Corr. Hell.}, 1882, p. 502, n° 25. Sur l'union de Zeus et de Cybèle dans le culte phrygio-romain, monuments figurés : l'Attis d'Ostie appuyé sur le buste de Zeus Idéen ; Zeus avec le sceptre et la foudre, sur un pectoral que porte l'archigalle reproduit par Montfaucon ; buste de Zeus sur la poitrine d'une « sacerdos maxima, » au musée du Vatican, \emph{CIL.} 6, 2257 ; id. sur la ciste votive de l'archigalle Modius Maximus à Ostie, \emph{CIL.}, 14, 385 ; id. sur le manche du fouet de l'archigalle, au musée Capitolin.

*) Σκηπτοῦχος, Orph., \emph{Hymn.} 27, 4 ; Fulgent., \emph{Myth.} 3, 5 : « sceptrum fert. » Attribut fréquent sur les monnaies d'Asie-Mineure ; cf. Pessinonte (M. Aurèle : Imhoof Blumer, \emph{Monn. gr.}, p. 415, n° 172 ; Mionnet 4, p. 394, n° 122), Ancyre (Sévère : Mionnet 4, p. 379, n° 29), Eucarpia en Phrygie (Gordien le pieux : p. 291, n° 556), Nacoleia (Trajan : p. 346, n° 870), Daldis-Flaviopolis, Lydie (Commode : Macdonald, \emph{op. l.}, 2, p. 450, n° 2), Hermocapelia (Héliogabale : Mionnet 4, p. 46, n° 240), Magnésie du Sipyle (Otacilia, Philippe j. : p. 480 s, n°s 438, 441), Nicée (Antonin, Maxime, Trébonien : Mionnet 2, p. 452, n° 219 ; \emph{Suppl.} 5, p. 144, n° 834, et p. 154, n° 898), Abydos (Commode : Mionnet 2, p. 636, n° 50). Sur les reliefs mysiens où Cybèle figure à eôté de Zeus, elle porte aussi le sceptre. Intailles : Furtwaengler, \emph{l. c.}, 2838, 2841, 2842, 8626 (= Mueller-Wieseler, \emph{Denkm. d. alten Kunst}, 2e éd., 2, taf. 63, 809) ; int. inédite (vue en 1895) de la coll. Marsili, à Cortone, avec l'épigraphe \emph{Mater Deum}. Le sceptre est l'attribut de la déesse syrienne sur le lion.

*) Julian., \emph{Or.} 5, pp. 166 et 180. L'hymne orphique 27, 5, dit que son trône occupe le centre du Cosmos. Ce trône est la Terre, considérée comme centre de l'univers. D'autre part, Serv. ad. \emph{Aen.} 12, 118 : « Mater Deum, cuius potestas in omnibus zonis est. »

*) Intaille de l'époque impériale : Furtwaengler, \emph{op. l.}, p. 267, n° 7158. Aigle décorant l'autel sur lequel sacrifie une grande prêtresse : \emph{CIL.} 6, 2257 ; éployé sur une sphère, symbole de Caelus (Cumont, \emph{Mithra}, 1, p. 88), sur un autel taurobolique : \emph{CIL.}, 12, 1827. Sur les rapports de l'aigle et du dieu Soleil, cf. Cumont, \emph{l. c.}, p. 120.

*) Julian., \emph{l. c.}, pp. 170 et 180. Pindare exprime la même idée, lorsqu'il dit qu'elle occupe « le trône le plus élevé » : \emph{Ol.}, 2, 140.

*) Meter Basileia, cf. supra, p. 40. Anassa : Orph., \emph{Arg.}, 604, 611 et \emph{Hymn.}, 41, 1 et 9. Regina : Firm. Mat., \emph{De err. prof. rel.} 3, 1. Le titre de reine lui est commun avec Demeter, Persephone, Aphrodite, Nemesis, Artémis (cf. à Pergé), Isis, Astartè (Plut., \emph{De Iside et Os.}, 15). Mais, de même que Milkat, c'est-à-dire la Reine par excellence, est Astartè, de même la déesse adorée en pays de langue grecque sous le nom unique de Basileia paraît bien être Cybèle, d'après Diodore, 3, 57 ; cf. Loeschke, \emph{Dorpat. Programm}, 1884 et Roscher, \emph{Über Selene}, 1890, p. 96.

*) Orph., \emph{Hymn.} 14, 7. Sur toutes ces épithètes, ref. Bruchmann, \emph{Epitheta deorum quae apud poetas graecos leguntur}, 1893, s. vv. \emph{Kubelè} et \emph{Rhéa}.

*) Orph., \emph{Hymn.} 27, 12 ; Nonn., \emph{Dionys.} 25, 322 ; cf. Pasikrateia, qui paraît être un vocable mystique de Persephonè en Sicile : \emph{IGSI.} 268 (5e s. av. J.-C.).

*) Aristoph., \emph{Av.} 877 ; cf. \emph{Kuria}, à Iconium : \emph{J. of Hell. St.} 22, 1902, p. 341, n° 64. C'est l'épithète ordinaire des déesses sémitiques. A la même idée se rattache l'épithète de Grande ; cf. Megalè Anaeitis, \emph{J. of Hell. St.}, 10, 1889. p. 226 : Megalè Artémis : Ramsay, \emph{Studies in the Hist. and Art of the East. Prov.}, 1906, p. 319 ; Magna Virgo Caelestis : \emph{CIL.} 8, 9796.

*) Serv. ad \emph{Aen.} 3, 113 : « sane dominam proprie Matrem Deum dici Varro et ceteri adfirmant. » Sur ce titre, cf. Roscher, \emph{Myth. Lex.}, 1, 1, 1197, s. v. \emph{Domina} et 2, 1, 1767, s. v. \emph{Kyria}.

*) Orph. \emph{Hymn.}, 14, 8 (mais elle est dite aussi fille du Prôtogonos polymorphe, \emph{ibid.}, 1) ; Nonn., \emph{Dionys.} 13, 291 ; Ps. Eudoc., \emph{Floril.}, éd. Flach, p. 370 : ἀρχηγὸς τῆς πρώτης κἀὶ ἀρχετύπου οὐσίας τοῦ κόσμου. Dans Orph., \emph{Argon.} 604, elle est nommée πρεσβυγενής. En Égypte, cf. la Nit de Saïs, mère de Râ, « la première qui naquit au temps où il n'y avait pas encore eu de naissance. »

*) Julian., \emph{l. c.}, p. 166. Certains mythes lui attribuent comme mère Gê, Basileia ou Dindymè, qui sont d'autres aspects d'elle-même.

*) Nigidius dans Arnob. 3, 32 : « \emph{Matrimonium tenuisse Saturni} » ; doctrine des théologiens de Rome à la fin de la République.

*) \emph{CIL.} 6, 81 ; cf. 404 : « Iovi Optimo Maximo Caelestino » par un « collegium sanctissimum » qui devait être composé d'Orientaux, adorateurs d'un Baal Shamin, « maître des cieux. » Le Baal de Baitocaicé devient Zeus Ouranios : Lebas-Wadd. 2720 \emph{a}. De même, Jupiter Frugifer dans Apul., \emph{De mundo} 37, est analogue au Saturne Frugifer de l'Afrique impériale, qui est l'ancien Baal des peuples syro-puniques ; cf. à Lambèse, \emph{CIL.} 8, 18219 un \emph{Jupiter O. M. deorum princeps, coeli terrarumque rector}.

L'\emph{Optima Maxima}* est reine du ciel. Elle est l'\emph{Ourania}. Comme Astartè, Atargatis et Isis, « elle gouverne les voûtes lumineuses.* » Sans douté, suivant la méthode astronomique d'exégèse qui était à la mode sous l'Empire, des théologiens prétendaient-ils que le nom même de Cybèle rappelait la coupole du firmament.* Quand elle chevauche le lion, principe igné, symbole astral,* pour parcourir triomphalement l'espace, son péplum qui flotte au vent se gonfle en demi-cercle et ressemble à la voûte céleste.* Les cymbales, qui lui sont consacrées, prenaient la même signification symbolique.* Elle préside à la révolution des astres.* Voilà pourquoi elle a pu gratifier Attis du bonnet étoilé, image du ciel visible.* Elle a pour doryphores les Corybantes, qui sont des dieux héliastes.* On lui associe les Dioscures, autres divinités astronomiques, dieux grands, que l'on confondait avec les Cabires syriens, et que nous retrouvons aux côtés d'Héra Ourania, d'Astartè, de Baal-Saturne.* Parfois le flambeau, emblème de la lumière astrale, remplace dans sa main le sceptre* ; ou bien elle tient la clef qui ouvre les célestes portes,* comme le Samas babylonien, « qui tire les verrous du ciel pour que le monde entier devienne resplendissant.* » Devant la couronne tourelée, elle porte le diadème, « dont se pare généralement seul, » affirme Lucien, « le front d'Ourania » ; on y voit même luire le croissant, surmonté du disque lunaire, insigne d'Isis.* Ailleurs son visage s'encadre des bustes d'Hélios et de Sélénè,* ou de deux étoiles, celle du matin et celle du soir,* ou des signes du zodiaque,* ou d'un nimbe lumineux.* La tête radiée du soleil se dresse au sommet de son trône,* ou au fronton des petites chapelles qu'on lui dédie en ex-voto.* La planète Vénus brille sur son tambourin.* Elle-même est la Brillante.* Et parce que le cours des astres et la marche des saisons s'accomplissent avec une immuable régularité, on l'invoque aussi sous le nom d'Isodromos.* Dans un temple de Galatie, elle est la Mère aux quatre visages, Mêter Tetraprosôpos,* c'est-à-dire la déesse des quatre saisons. Son quadrige en est le symbole.

*) \emph{CIL.} 10, 4829 : \emph{Matri Deum Optimae Maxim}. ; taurobole du 2e s.

*) Apul., \emph{Met.} 11, 5. Atargatis a pour symbole le croissant lunaire associé au disque solaire ; cf. Dussaud dans \emph{Rev. archéol.}, 1904, 2, p. 228. De même l'égyptienne Nouit, personnification du ciel étoilé, est aussi Mère des Dieux.

*) Cf. supra, p. 16, n. 2.

*) Julian., \emph{l. c.} 167 B et 168 B: Macrob., \emph{Saturn.} 1, 21, 8 et 17 ; Mythogr. Vat. 3, 8, tome 2, p. 53. Le babylonien Nergal, dieu du soleil dévorant, est un lion ailé à tête humaine. Il y a là d'antiques croyances d'origine astrologique ; le soleil traverse le signe du lion au moment de la canicule et y possède même son domicile.

*) Mionnet, \emph{Suppl.} 6, p. 144, n° 424 (Éphèse, sous Antonin), p. 537, n° 485 (Stratonicée) ; cf. à Clazomène, Mionnet, 3, p. 74, n° 101, la déesse tenant dans chaque main les extrémités de son voile.

*) Elles figuraient les deux hémisphères du ciel qui enveloppent la terre, six des signes zodiacaux se trouvant toujours au-dessus de la terre et six au-dessous (Manil. 3, 240). Cymbales, bonnet phrygien et lion groupés comme symboles uraniens sur une intaille : Furtwaengler, \emph{op. l.} 7032.

*) Orph., \emph{Hymn.} 27, 4.

*) Julian., \emph{l. c.}, p. 165 B ; Sallust. phil., \emph{De diis et mundo} 4.

*) Julian., \emph{l. c.}, p. 168 B (sur la signification astrologique du mot \emph{doryphorie},v. Bouché-Leclercq, \emph{Astrol. gr.}, p. 253) ; Clem. Al., \emph{Cohort.} 2 ; cf. Fraenkel, \emph{Inschr. v. Pergamon}, 68 ; supra, pp. 8 et 133. Pour les monuments figurés, v. infra, à propos des acrotères du temple de l'Idéenne.

*) V. les textes et monuments que j'ai réunis dans la \emph{Revue archéol.}, 1904, 1, pp. 345-347.

*) Monnaies de Kibyra, à l'effigie d'Héliogabale : Mionnet, 4, p. 261, n° 394 ; Babelon, \emph{Coll. Wadd.} 5837, repr. pl. 15, 20 ; de Philadelphie, Babelon, \emph{op. l.} 5128 et pi. 14, 16.

*) Froehner, \emph{T. cultes Gréau}, pl. 4 ; Serv. ad \emph{Aen.} 10, 252.

*) Fragment d'hymne à Samas, cité dans Chantepie de la Saussaye, \emph{Manuel d'hist. des Religions}, 1904, p. 138.

*) Tête en marbre blanc sur buste en marbre jaspé, à Rome, au musée du Capitole, salle des Colombes, n° 25. C'est un portrait de femme, avec la couronne murale de Cybèle, influencée par la coiffure d'Isis (les symboles lunaires sont accompagnés de serpents) ; antérieur à la seconde moitié du ne siècle, d'après la technique des yeux.

*) Intailles antérieures à l'empire : Furtwaengler, op. I., p. 85, n° 1438 et pl. xvi (= Mueller-Wieseler, 2° éd, 2, pl. i.xm, 808 ; Dar. et Saglio, s. v. Cybelé, fig. 2246) ; --- de basse époque : Furtw., p. 316, n° 8626. Cybèle, mère d'Hélios et de Sélénè, cf. supra, p. 40, n. 4.

*) Intaille au Cabinet de France : Chabouiliet, \emph{Cat.} 1408.

*) Monnaie d'Adana, en Cilicie : Mionnet, 3, p. 564, n° 133 (Gordien 3). Médaillon contorniate : deux signes du zodiaque, bélier (équinoxe du printemps) et cancer (début de l'été), au-dessus du quadrige de lions qui emporte Cybèle et Attis ; au-dessous du char, femme à demi-couchée, peut-être la Vierge (équinoxe d'automne), ainsi représentée sur des reliefs mithriaques (Cumont, \emph{Mithra}, 2, mon. 246 \emph{b} et 251 \emph{d}) : Robert dans \emph{Rev. numism.} 1885, p. 47, pl. 5, 6 (cf. p. 15, et pl. 3, 7) ; Dar. et Saglio, \emph{l. c.}, fig. 2252 (la femme indiquée à tort comme symbole de la Phrygie) ; Cohen, 2e éd. 8, contorn. 362 (cf. 105 et 213).

*) Plaque de marbre gravée, au Cabinet de France : De Longpérier, \emph{OEuvres}, 2, pp. 360-362 ; reprod. dans Dar. et Saglio, \emph{l. c.}, fig. 2250 ; cf. Stephani, \emph{Nimbus und Strahlenkranz}, p. 47 s.

*) Même plaque. Reliefs en bronze, v. supra, p. 186, n. 2 ; dans le tympan du naiskos, buste radié d'Hélios dominant le quadrige solaire. --- Monnaie de Briula, Lydie : buste d'Hélios et, au revers, Cybèle ; Mionnet, 4, p. 24, n° 123 ; Macdonald, \emph{op. l.} 2, p. 449, 1 et pl. 55, 14.

*) Relief provenant d'Asie M., sans doute de Smyrne : \emph{Bullettino}, 1829, p. 80. --- Croissant dans le champ, sur une plaque de bronze en forme de naiskos, reprod. par Burmann dans son Ovide, t. 3, p. 247.

*) Médaillon en bronze, à l'effigie de Lucilla (Cybèle, assise sur le lion, tient le sceptre) : Grueber et St. Poole, \emph{Roman med. in the British M.}, 1874, p. 12 et pl. 26, 2 ; Froehner, \emph{Méd. rom.}, p. 97. --- Plaque gravée, v. supra, n. 1. L'étoile est à huit rayons. Ce type d'étoile est, dans l'écriture chaldéenne, le signe de la divinité : Heuzey, \emph{Orig. or. de l'art} 1, p. 8.

*) \emph{Anthol. Gr.}, 6, 234, 3.

*) Temple, sous ce vocable, dans la vallée du Caïstre : Strab. 9, 5, 19.

*) A Kutchuk Hassan : \emph{J. of Hell. St.} 19, 1899, p. 303, n° 237.

Il est aussi le symbole des quatre éléments.* Maîtresse du monde sidéral, elle dispose en effet de toutes les forces de la nature. Elle possède le feu qui illumine le ciel, qui anime les êtres vivants, qui provoque la croissance des plantes, qui fond les métaux,* qui constitue le foyer du monde et de la maison,* qui brûle sur l'autel et dans les lampes perpétuellement allumées de son sanctuaire. De même que la déesse syrienne, elle emprunte au seigneur des cieux le oudre* ; et c'est elle qui lance sur la terre les météorites embrasés. L'air est en son pouvoir.* Le mythe lui donne pour suivante Aura, qui personnifie la brise.* Comme Isis et Caelestis, « elle gouverne les souffles de l'Océan.* » Elle envoie ou retient les tempêtes, comme Astartè. Elle est aussi la déesse des eaux éternelles,* de même que certains dieux sémitiques sont les « seigneurs des fleuves. » Dans l'iconographie métroaque, le dauphin et le cratère représentent l'élément humide,* comme les génies ailés et les oiseaux envolés figurent l'élément aérien,* comme le lion symbolise le feu. Ainsi que la Céleste punique, elle est « celle qui promet la pluie.* » Elle est dispensatrice de la rosée.* Elle fait jaillir les sources qui fertilisent et celles qui guérissent.* Son culte est lié à celui des Nymphes.* Elle protège les étangs, les lacs et les cours d'eau. Elle est la dame de la mer ; on la met en rapport avec l'Aphrodite Pontia,* qui rappelle l'Astartè maritime des Sidoniens, sans doute aussi avec l'Isis Pelagia. Sur les promontoires de la côte mysienne, ses temples se dressent comme des phares de salut. Les marins l'implorent comme leur patronne et déposent à ses pieds leurs ex-voto.* Enfin sa domination s'étend sur l'élément solide. Elle a parmi ses attributs le serpent, qui passait pour naître de la terre.* Au siècle des Antonins, dans ses sanctuaires d'Hiérapolis, en Phrygie, de Myra, en Lycie, d'Aphrodisias, en Carie, elle est encore la dame aux serpents.*

*) Cf. le quadrige cosmique du soleil avec cette double signification : textes dans Cumont, \emph{Mithra}, 1, p. 126.

*) Sur ses rapports avec l'art des forgerons, en Phrygie, v. infra, p. 202.

*) Elle est mère d'Hestia-Vesta ; cf. Gruppe, \emph{op. l.}, p. 1406. Mais elle est aussi confondue avec elle : Orph., \emph{Hymn.} 27, 9 ; Arnob. 3, 32 ; Augustin., \emph{Civ. Dei} 4, 10. Sur l'idée pythagoricienne qui est à l'origine de cette identification : Gruppe, \emph{op. l.}, p. 1406, n. 6 ; Gomperz, \emph{Les penseurs de la Grèce}, 1, 1904, p. 127.

*) Aristophane, dans les Oiseaux, dit que Basileia = Cybèle « dispose de la foudre de Zeus » ; cf. Basileia disparaissant dans un orage Diod. 3, 57. L'orchestre phrygien imite le bruit de la foudre et les lampadophories rituelles symbolisent l'éclair, d'après le Ps. Eudoc., \emph{Floriteg.}, éd. Flach, p. 619. Le nom de Cybèle (Kubile) ne serait-il pas à rapprocher de Gibil, le dieu-éclair des Assyriens ? 

*) Orph., \emph{Hymn.} 14, 11.

*) Nonn., \emph{Dionys.} 48, 238.

*) Apul., \emph{Met.} 11, 5 ; cf. Lafaye, \emph{Div. d'Alexandrie}, p. 255. En Numidie, au 3e s., dédicace à Caelestis : « Tu nimbos ventosque cies, » \emph{CIL.} 8, 4635 = Buecheler, \emph{Carm. Lat. Epigr.}, 254, vers 2.

*) D'après l'une des plus anciennes théogonies, elle est fille d'Okeanos ; cf. Gruppe, \emph{op. l.}, p. 421. C'est de l'élément humide que la déesse syrienne « avait tiré les principes et les semences de tous les êtres » ; Plut., \emph{Crass.} 22.

*) Dauphin : stèle de Trajanopolis (Phrygie) au musée du Louvre : \emph{Bull. Soc. Antiquaires}, 1901, p. 350, n° 54 ; dauphin et poisson : \emph{CIL.} 3, 13903 (Salone). Cratère : fronton du temple de l'Idéenne au Palatin, v. infra. Sur le symbolisme du cratère (= source) dans les mystères mithriaques : Cumont, \emph{Mithra}, 1, p. 101.

*) Génies ailés : stèle de Trajanopolis. Oiseaux: \emph{ibid.} ; patère du musée d'Avignon : \emph{CIL.} 12, 5697, 3.

*) Ps. Eudoc., \emph{op. l.}, p. 618 : ὄμβρων αἰτίαν ; cf. une inscr. de Phrygie dans \emph{Ath. Mitt.}, 1900, p. 421. La pluie joue un rôle dans le mythe ; cf. supra, p. 40, n. 4. En Afrique, Virgo Caelestis est « pluviarum pollicitatrix » : Tertull., \emph{Apot.} 23. Sur une fête des Kabyles en l'honneur de la « fiancée des eaux » : Masqueray dans \emph{Bull. Corr. Afr.} 1, 1882, p. 11.

*) Comme la planète Vénus, qui lui est consacrée en Phrygie. Il n'est pas nécessaire d'admettre que la Lune et, par suite, les seules divinités lunaires dispensent la rosée : P. Lagrange, \emph{Et. Relig. sémit.}, p. 127 s ; Dussaud dans \emph{Rev. archéol.} 1903, 1, p. 125 s.

*) Dans le mythe rapporté par Arnobe, 5, 6, il y a la source, « fons familiaris, » d'Agdistis ; v. aussi supra, p. 9. Sur le lion gardien des sources, en Assyrie et en Égypte, cf. O. Keller, \emph{Die antike Tierwelt} 1, 1909, p. 48.

*) Cf. supra, pp. 11, 35 et 117. A l'époque romaine : Cagnat, \emph{Ann. épigr.} 1905, n° 140, \emph{Matri deum et matronis Salvennis} (déesses des eaux salines de Moutiers).

*) Relief de Cyzique : Mordtmann dans \emph{Ath. Mitt.} 10, 1885, p. 204 ; il croit à tort qu'il n'y a pas de relations entre la dédicace à Aphr. P. et le relief qui représente la M. d. D. Celle-ci est implorée pendant une tempête par l'Amazone Myrina, qui naviguait entre Lesbos et Samothrace : Diod. Sic., 3, 54.

*) Conze dans \emph{Ath. Mitt.}, 16, 1891, pp. 191-193 ; Perdrizet dans \emph{Bull. Corr. Hell.}, 23, 1899, p. 597 et pl. 5, 1 : relief de Bryllium, sur la côte mysienne, avec les images de Cybèle, Hypsistos et Hermès ; derrière un promontoire, vaisseau de guerre.

*) Les devins de Telmissos expliquaient déjà le symbolisme du serpent à Crésus : Herodot. 1, 78 ; cf. Plin., \emph{H. n.} 8, 59.

*) Mionnet, 3, p. 439, n° 51 ; 4, p. 298 s, n°s 588, 597 ; Goblet d'Alviella, \emph{Migrations of symhols}, 1894, p. 142 ; Philpot, \emph{Sacred tree}, 1897, p. 87, fig. 24. En Lydie, cf. le médaillon publié dans les \emph{Mélanges Perrot}, 1903, p. 141 ss.

Son caractère chthonien primait tous les autres.* C'était comme déesse de la terre que les Romains l'avaient d'abord connue et vénérée. Il ne semblait pas douteux aux théologiens de l'école symboliste que la forme de son tambourin reproduisait l'orbe terrestre.* Si elle habite de préférence les hauts lieux, elle aime à parcourir sur son char de lions la terre tout entière. Elle tient sous sa tutelle les forêts de la montagne et les cultures de la plaine. Elle fait pousser la verdure nouvelle des arbres et mûrir les fruits.* Aussi la voyons-nous parfois avec une corbeille de fruits posée sur sa tête ou sur ses genoux* ; et l'extrémité de son sceptre s'épanouit en fleur.* Elle répand ses bénédictions sur les champs.* Comme Ishtar, Isis, Déméter et Cérès, elle a enseigné à ses premiers adorateurs l'usage du blé. Elle est la dame des moissons, la \emph{Cereria}, l'\emph{Agraria}* ; sur plusieurs images cultuelles, en Asie Mineure, elle porte à la main un bouquet d'épis.* C'est elle encore qui fait prospérer les vignobles* ; à Cyzique, sa plus ancienne image était sculptée dans un cep de vigne.* Dionysos, roi des vendanges, romanisé sous le vocable de Liber Pater, est un dieu du cycle métroaque* ; sa statue prend place dans l'enceinte du Metrôon.* Cybèle enfin maîtrise le lion, parce qu'il est le démon des régions infertiles, que dévore le feu du soleil. Souveraine des fauves, elle défend contre leur férocité les troupeaux que la terre nourrit.* Couronnée de tours, comme la Tychè poliade, elle préside à la destinée des villes dont se peuple la terre. A sa divinité sont également soumises les forces souterraines. Dans les profondeurs de la terre, d'après une conception primitive, sommeille l'énergie créatrice qui éveille la nature à la vie. Qui possède ce mystérieux domaine est maître de toutes ces puissances, qu'elle met au service des hommes par l'intermédiaire de ses prophètes. C'est elle, nous l'avons vu, qui des entrailles du sol fait sourdre les eaux bienfaisantes. Les minerais que recèle la terre lui appartiennent aussi. C'est par elle que les Dactyles Idéens ont connu les premiers l'art d'exploiter et de travailler les métaux* ; c'est à elle que les Phrygiens doivent leur habileté de forgerons.* Des cavernes, entrées du monde infernal, lui sont consacrées. Il n'est pas rare, en Anatolie, qu'un Metrôon s'élève dans le voisinage immédiat d'un Plutonium.* La déesse, en Afrique, est associée ou identifiée à Aerecura, qui partage le trône de Dis Pater. Comme à Proserpine, on lui met en main la grenade et le pavot, symboles funéraires.* La reine du ciel et de la terre est aussi reine des enfers.

*) Augustin., \emph{Civ. Dei} 4, 10 : « Terrain matrem deum vocant » ; 6, 8 : « Mater deum terra est » ; cf. 7, 16 et 28. --- Tertull. \emph{Adv. Marcion.} 1, 13 : « Vulgaris superstitio... figurans magnam matrem in terram, etc. » --- Arnob. 3, 32 : « Terram... matrem esse dixerunt magnam. » --- Anonyme de 394, dans Baehrens, \emph{Poet. Lat. Min.} 3, p. 286, v. 94 : « Terra parens, mater formosa deorum » ; Macrob., \emph{Sat.} 1, 21, 7 : « Quis ambigat Matrem Deum terram haberi ? » --- Image de Terra mater, dans la schola des dendrophores d'Ostie : \emph{CIL.} 14, 67 (ann. 142) ; on connaît en Numidie la triade Terra Mater, Aerecura, Mater Deum Magna. Mais la M. d. D. n'est pas plus confondue avec Tellus ou Terra qu'elle ne l'était jadis avec Gê (Rhéa, fille de Gaia : Callim., \emph{Fragm.} 258, éd. Schneider ; Orph., \emph{Argon.} 554).

*) Varr. dans Augustin., \emph{op. l.} 7, 24 ; Serv. ad \emph{Georg.} 4, 64. La comparaison de la forme de la terre avec celle d'un tambourin est déjà dans Anaximandre ; cf. Gomperz, \emph{Penseurs de la Grèce}, 1, 1904, p. 121.

*) Χλοόκαρπος : Orph., \emph{Hymn.} 58, 6.

*) Monnaies de Kibyra, sous Héliogabale : Mionnet 4, p. 261, n° 394 ; Babelon, \emph{Coll. Wadd.} 5837 et pl. 15, 20 ; sous Gallien : Loebbecke, dans \emph{Zeitschrift f. Numism.} 12, 1885, p. 344, n° 3 et pl. 14, 6.

*) Peut-être s'agit-il ici de l'éclair-fleur, symbole de la reine des cieux ; cf. Jacobsthal, \emph{Der Blitz in der orient. u. gr. Kunst}, 1906.

*) Elle les désensorcelé : Augustin., \emph{l. c.}

*) \emph{CIL.} 3, 796, et supra, p. 118. Ishtar est « déesse des céréales, dispensatrice de l'aliment. » Sur l'unification des Mères du blé, cf. Goblet d'Alviella, dans \emph{Rev. Hist. Religions} 1902, 2, p. 190 s.

*) Mionnet, 4, pp. 280 et 291, n°s 495 et 556 ; H. Weber dans \emph{Num. Chronicle}, 12, 1892, p. 208, n° 38 et pl. 16, 18 ; cf. sur des intailles : Furtwaengler, \emph{Geschnitt. Steine}, pp. 267 et 319, n°s 7158 et 8714 (basse époque) ; Tassie et Raspe, \emph{op. l.} 797 ; Visconti, \emph{Opere varie} 3, p. 401 ; sur un camée, supra, p. 112. On voit aussi des épis dans la main d'Atargatis ; et Virgo Caelestis est qualifiée de \emph{spicifera}, \emph{CIL.} 7, 759.

*) Greg. Tur., \emph{In glor. confess.} 77 : « Pro salvatione agrorum ac vinearum. »

*) Apoll. Rh. 1, 1117 ss, et schol. (= Meineke, \emph{Analecta Alexandr.}, p. 150, 146) ; cf. \emph{Roem. Mitt.} 14, 1899, p. 8. Sur un pithos de Béotie, Rhéa entre deux lions ; des pampres lui sortent de la tête : Milani, \emph{Studi e maleriali} 1, p. 8, fig. 8.

*) Cf. supra, pp. 15, 45. Midas, fils de la M. d. D., thiasote de Dionysos ; cf. Dieterich dans \emph{Philologus}, 52, 1894, pp. 5 et 6. Strabon, 10, 3, 13, insiste sur le lien qui unit les deux cultes. Au 2e s., prêtre de la M. d. D. et de Dionysos Kathégémôn : \emph{CIG.} 6206 = \emph{IGSI.} 1449. A la fin de l'Empire, Julien, \emph{l. c.}, pp. 179 B et 180 A, signale la parenté mystique des deux divinités.

*) \emph{CIL.} 8, 8457 et 16440 ; cf. \emph{Rev. archéol.} 1904, 1, p. 350. Participation des prêtres de Liber aux tauroboles : \emph{CIL.} 12, 1567 (ann. 245).

*) Dédicace pour le salut « des hommes et des quadrupèdes » en Asie Mineure : \emph{J. of Hell. St.} 19, 1899, p. 303, n° 237. Incantations « pour la naissance et le salut des fruits et des troupeaux, » Dion. Chrys., \emph{Or.} 1, p. 61. Dans le mythe rapporté par Diodore, 3, 58, Cybèle enfant enseigne des remèdes pour les bestiaux malades.

*) Diod. Sic. 17, 7, 5 : σίδηρον ἐργάσασθαι πρώτους, μαθόντας τὴν ἐργασίαν παρὰ τῆς τῶν θεῶν μητρός. Strabon, 10, 3, 22, déclare qu'ils ont les premiers travaillé le fer dans l'Ida et que, selon la tradition générale, ils se rattachent au cycle de la M. d. D. ; cf. les deux Dactyles Titias et Kyllenos, parèdres de la déesse sur le Dindymos de Cyzique, d'après Apoll. Rh., \emph{Argon.} 1, 1127, et leur culte dans le Metrôon de Milet, d'après le scholiaste, citant Maiandros. Strabon ajoute qu'on les croit quelque peu magiciens. L'art de forger les métaux dut garder longtemps un caractère surnaturel ; aujourd'hui encore, au nord-ouest du Soudan, les forgerons forment une caste qui passe pour posséder des pouvoirs magiques : Réville, \emph{Religion des non-civilisés}, 1, p. 40. Wilamowitz, dans \emph{Nachrichten Gesellsch. zu Goettingen}, 1895, p. 241 s, compare les Dactyles aux nains (Daümlinge) des bois. Voir aussi Rossignol, \emph{Métaux dans l'antiq.} p. 27 s ; Milchhoefer, \emph{Anfange der Kunst}, p. 26 ss. Sur le rôle des forgerons armés de la lance, dans le culte du soleil, v. les forgerons du dieu solaire Horus : Maspero dans \emph{l'Anthropologie} 2, p. 401 ss et dans \emph{Bibliothèque égyptologique}, 2, p. 313 ss ; et les forgerons ou chaudronniers en Irlande et dans le pays de Galles : Déchelette, \emph{Le culte du soleil}, cf. supra, p. 4, n. 3.

*) De très bonne heure les Phrygiens excellaient dans le traitement du minerai. Aussi Théophraste attribue-t-il la découverte du bronze au phrygien Délas : Plin., \emph{H. n.} 7, 57, 6. Mais les Phr. étaient forgerons, et non bronziers. Dans le plus ancien tumulus fouillé à Gordion (environ 700 av. J.-C.), les bronzes paraissent être tous d'importation et ont été réparés avec du fer : G. et A. Koerte, \emph{Gordion}, 1904, p. 91 s.

*) Cf. en Asie Mineure, Hiérapolis de Phrygie, Synnada, la grotte Steunos ; en Béotie, le Metrôon consacré par Echion : Ovid., \emph{Met.} 10, 686 ; en Campanie, supra, p. 34.

*) Grenade ou pavot : grande lampe en terre cuite, à 3 anses, où Cybèle figure entre Attis et Mercure, v. supra, p. 186 n. 2 ; cf. \emph{Catal. du musée Fol, Antiq.}, 2, p. 144, n° 1905. Caractère funéraire de la Grenade : Boetticher, \emph{Baumkultus der Hellenen}, 1856, p. 471 ss. Son rôle dans le mythe et dans le culte : v. supra, p. 120. --- Pavot, en général avec des épis, cf. p. 201 n. 5. Relief du musée de Berlin, dans \emph{Bonn. Jahrh.} 23, pl. 31. Sur une antéfixe en terre cuite, provenant d'un sanctuaire métroaque, un pavot orne la proue du vaisseau de l'idéenne ; Visconti dans \emph{Annali}, 1867, p. 297 et tav. d'agg. \emph{G}. Ce fruit apparaît aussi dans la main des Attis funéraires. --- Enfin, sur le caractère plus particulièrement funéraire du titre de Basileia, cf. Gruppe, \emph{op. l.}, p. 1521.

L'Omnipotente est l'Omniparente. La déesse « sans mère » est par excellence la Mère, Ammas,* la mère universelle, la Pammétôr,* de même que Zeus est le Pappas,* père de tout.* Mais, par un divin mystère, la dispensatrice de fécondité reste la Vierge,* comme l'Ishtar des Babyloniens, comme P Allât nabatéenne, comme la Céleste des Phéniciens et des Carthaginois, C'est pourquoi on l'assimile à Pallas* et Artémis* ; ses dévots de Bénévent, aux 2e et 3e siècles, ne l'appelaient que la Minerve du Bérécynthe. « Première entre les habitants du ciel, » la Mère des Dieux réunit virtuellement en elle les puissances multiples de tous les dieux issus d'elle. Avec Z eus, elle « engendre et organise tous les êtres » ; comme les dieux sémitiques, « seigneurs des naissances, » elle est « maîtresse de toute vie,* cause de toute génération.* » Elle est donc l'Aphrodite éternelle. Hermès Epaphroditos est invoqué par ses mystes.* Eros voltige près de son char* ou se dresse sur un cippe, près de son trône.* A Tarse et sur la côte ionienne, dès l'époque hellénistique, Attis se métamorphosait lui-même en Cupidon.* Dans les temples romains de l'Idéenne, on dédie des Amours* ; et Mars, que la chorographie astrologique unit à Vénus-Cybèle, a son image dans la chapelle des Dendrophores d'Ostie.*

*) Hesych., s. v. ; il attribue aussi ce titre à Demeter, identifiée à Cybèle, cf. Gruppe, \emph{op. l.}, p. 1542, n. 1.

*) Orph., \emph{Argon.} 547 ; Apul., \emph{Met.} 8, 25: « omnipotens et omniparens » ; Firm. Mat., \emph{De err. prof. rel.} 3, 1 : « Phryges hanc volunt omnium esse matrem » ; Augustin., \emph{Civ. Dei} 2, 4 : « Berecynthiae matri omnium » ; \emph{CIL.} 6, 499 : « summa parens » ; 509 et 30780 = \emph{IGSI.} 1018 et 1020 (4e s.) ; Nonn., \emph{Dionys.} 13, 35. Le mythe de l'androgyne Agdistis suppose une croyance analogue ; cf. Gruppe, \emph{op. l.}, p. 1546.

*) V. supra, p. 15 ; Pappas dans l'onomastique phrygienne, \emph{Bull. Corr. Hell.} 1882, p. 515, et dans l'onom. thrace, Tomaschek, \emph{Wien. Sitzungsber.} 130, 1894, p. 42. De même, la M. d. D. est l'Ammas : \emph{Etymol. magn.}, s. v.  Ἀμμά ; c'est aussi le nom delà déesse-mère des Phéniciens ; \emph{CISem.} 1, 177 (Carthage).

*) Julian., \emph{l. c.}, p. 166 A.

*) Parthenos : Julian, \emph{l. c.} ; cf. supra, p. 5. Dans le mythe raconté par Arnobe, Zeus ne parvint pas à la posséder, quand il voulut lui faire violence sur le sommet de l'Agdos. De même la Vierge céleste est identifiée tantôt à Diane (\emph{CIL.} 8, 1424), tantôt à Vénus (5, 8137, 8138), tantôt à Junon (8, 1424) ou à Mater Deum (7, 759) ou à Dea Nutrix (8, 8245-8247 ; cf. Audollent, \emph{Carthage}, p. 377). Il n'est cependant pas sûr qu'elle se confonde avec Tanit, laquelle fait expressément fonction de déesse-mère : Clermont-Ganneau, \emph{El. d'arch. or.} 1, p. 152. Le mythe totémique des vierges-mères se retrouve dans de nombreuses légendes de l'ancien et du nouveau monde ; exemples cités par Frazer dans son commentaire à Pausanias 7, 17, 11 = tome 4, pp. 138-140 ; cf. Lang, \emph{Mythes, cultes et relig.}, 1896, p. 486.

*) Statuette de Cybèle ( ? ) portant l'égide de Minerve et trônant entre deux lions : Matz et Duhn, \emph{Ant. Bildw. in Rom}, 1881, 1, 904. Julien, \emph{l. c.}, p. 179 A, insiste sur l'affinité des deux déesses. L'astrologie les rapprochait et les identifiait, en attribuant à Minerve le bélier zodiacal. D'autre part, si la mère est qualifiée de Vierge, on connaît une Athéna qui porte le vocable de Meter : Pausan. 5, 3, 2 ; Clermont-Ganneau, \emph{L'imagerie phénicienne}, p. 107, n. 3. La Minerve d'Ancyre a des rites identiques à ceux de la Pessinontide ; cf. supra, p. 137 s. La Dame de Gabala, tourelée et assise entre deux lions, tient une chouette : Mionnet 5, p. 234, n° 633 (Trajan).

*) Gruppe, \emph{op. l.}, pp. 1532-1540 : \emph{Kybele der Artémis gleichgesetzt}.

*) Julian., \emph{l. c.}, p. 166 A.

*) Orph., \emph{Hymn.} 27, 13.

*) Paus. 2, 3, 4 ; Julian., \emph{l. c.}, p. 179 BC ; cf. supra, p. 186, n. 2.

*) De Wilde, \emph{Gemmae selectae antiquae}, 1703, p. 4 et pl. 2, 2.

*) Smyrne, monnaie à l'effigie de Julia Titi : Mionnet, 3, p. 225, n° 1261.

*) Heuzey, \emph{Les fragments de Tarse au musée du Louvre}, dans la \emph{Gazette des B.-Arts}, 1876, 2, p. 404 et fig. 4, 12 (= Pottier, \emph{Statuettes de t. c.}, p. 188, fig. 61), 13 (= Cumont, \emph{Mithra.} 2, p. 437, fig. 381) ; Pottier et Reinach, \emph{Fouilles de Myrina}, pp. 398 et 405, pl. 31, 210, 215 ; t. c. de Myrina au musée de Berlin, \emph{Arch. Anzeiger}, 1889, p. 90. Autres types : Panofka, \emph{Terrakott. zu Berlin}, pl. 25, 2 ; de Witte, \emph{Coll. Janzé}, pl. 7, 2. En général, aux attributs d'Eros s'ajoutent des attributs dionysiaques.

*) \emph{CIL.} 13, 2500.

*) \emph{CIL.} 14, 33 (ann, 143) ; cf. supra, p. 125.

Mère des hommes, la déesse est maîtresse de leur fortune, puisqu'elle est reine du ciel et que le sort des humains est réglé par le mouvement quotidien des astres.* Elle distribue à chacun son lot ; elle est la Némésis,* et elle est en même temps la suprême justice.* Elle veille sur tous et elle voit tout. « Pas un homme, bon ou mauvais, n'échappe à sa connaissance* » ; elle est l'inévitable, Adrasteia.* Il faut redouter sa colère quand elle se revêt de terreur ; car elle est la Forte, l'impétueuse, l'Effrayante ; et « ses pieds sont rapides.* » Si elle est la Paix,* elle est aussi la Bellone « qui excite le tumulte de la guerre* » ; les Romains savent bien qu'elle est « reine dans les combats.* » Son pouvoir de destruction égale sa puissance de création. Aucune force ne lui résiste. Elle a pour emblème le lion tauroctone.* Mais elle est, avant tout, bonne et bienveillante. La Pandôra, « qui orne et remplit de ses dons tout le monde visible,* » se plaît à répandre sur les hommes ses bénédictions et ses bienfaits. Elle est la source et le fleuve intarissable de bonheur. Jamais on ne l'invoque en vain. Accessible à tous, toujours prête à donner assistance, elle est la dame de Bon Secours.* Sagesse, Prescience et Providence,* si elle nous envoie des peines, c'est pour notre salut. Car elle est celle qui sauve.* « Elle sauve ce qui naît et ce qui meurt.* » Elle sauve et conserve les hommes, les familles, les villes, les empires. Mère « et nourricière de vie,* » elle a des grâces particulières pour les nouveau-nés ; les mères lui vouent leurs enfants chétifs, vont les poser dans ses bras, et elle exauce leurs supplications.* Ceux qui souffrent peuvent aller vers elle avec foi. Car elle guérit les malades.* Elle leur indique les fontaines miraculeuses, les remèdes efficaces ; et ses prophètes d'Orient pratiquent la médecine, dont elle leur a transmis les secrets.

*) Intaille de basse époque, Furtwaengler, \emph{op. l.}, 8626 = Mueller-Wieseler, taf. 63, 809 : Cybèle avec sceptre et tympanon ; devant elle, Fortune avec corne d'abondance et gouvernail ; pour préciser le caractère de la reine des cieux, buste d'Hélios radié et croissant de lune, entre les deux divinités. La Mère est parfois représentée avec la corne d'abondance qui est l'attribut de Tyché : Mionnet, 4, p. 368, n° 988 (Synnada, sous Hadrien) et p. 379, n° 29 (Ancyre, sous Septime Sévère). De même, \emph{Fortuna deae Caelestis} : \emph{CIL.} 8, 6943 ; Philastr., \emph{De haeres}. 15 : « fortunam Caeli quam et Caelestem vocant in Africa. » Sur la M. d. D. et la Fortune des Augustes ou des villes, cf. supra, pp. 161 s, 166.

*) \emph{CIL.} 14, 34 : \emph{imaginem Matris Deum cum signo Nemesen}, à Ostie, sous Marc Aurèle. A Smyrne, leurs deux cultes étaient associés, ainsi que leurs attributs : lion de la M. d. D. posant une patte sur le tambourin, griffon de N. posant une patte sur la roue: Cybèle assise, tenant deux images de Nemesis : Mionnet, 3, p. 239, n° 1342. Le caractère funéraire de N. (fille de la Nuit, mythe smyrniote, Pausan. 7, 5 ; Nemeseia, fête des morts) contribue à la rapprocher de la M. d. D., surtout à Smyrne, où celle-ci est gardienne des tombeaux. Mais N. est aussi une divinité cosmique : reine, \emph{CIL.} 3, 827, 1438, \emph{etc.} ; « la grande N., souveraine du Cosmos, » \emph{CIL.} 6, 532 = \emph{IGSI.} 1012 ; par suite elle est, eommela Grande Mère, maîtresse de la Fortune : \emph{CIL.} 3, 1125 : \emph{deae Nemesi sive Fortunae} ; Capitol., \emph{Max. et Balb.} 8, 6 ; Amm. Marc. 14, 11, 25 ; Mart. Cap. 1, 88. Aussi la voyons-nous également unie ou identifiée à Isis. : Apul. \emph{Met}, 11, 2 ; et même au soleil : Macrob., \emph{Sat.} 1, 22, 2.

*) Artemidoros (époque antonine), \emph{Oneirocrit.} 2, 39, éd. Hercher, p. 145, déclare qu'elle « cause la perte de ceux qui violent les lois » ; \emph{CIL.} 7, 759 (sous les Sévères) : \emph{Virgo Caelestis} (\emph{eadem Mater divum}), \emph{iusti inventrix,... lance vitam et iura pensitans}. Est-ce bien Cybèle qui tient une balance dans la main droite, sur une monnaie de Prymnessos, en Phrygie, \emph{Ath. Mitt.}, 7, 1882, p. 135 s ?

*) Julian., \emph{l. c.}, p. 160 ; cf. Artemid., \emph{l. c.}

*) V. supra, p. 13. Sur les rapports d'Adrasteia avec Nemesis, cf. Posnansky, \emph{Nem. und Adr.}, 1890 ; avec Cybèle : \emph{ibid.}, pp. 26, 68, 174. La frise de Pergame nous montre A. et la M.d. D. côte à côte. Lewy, \emph{Semit. Fremdw. im Griech.}, p. 248, ne croit pas que son nom soit d'origine grecque et y voit la forme grécisée d'un vocable sémitique ; il lui attribue le sens de « vengeresse » ou de « vigilante. »

*) ᾽Οβριμόθυμος : Orph., \emph{Hymn.} 14, 7 (cf. Demeter Brimô = celle qui gronde de colère) ; φρικτός, \emph{Anth. pal.} 6, 219, 1 ; βλοσυρή, Nonn., \emph{Dionys.} 1, 20 ; ὠκυπέδιλος, \emph{ibid.}, 14, 1 ; ζαμενής, \emph{ibid.}, 21, 33. Chez Agdistis, « robur invictum et ferocitas animi, » Arnob. 5, 6.

*) \emph{CIL.} 7, 759.

*) Πολεμόκλονος, Orph., \emph{Hymn.} 14, 7 ; cf. supra, pp. 32, 36, 93. Furtwaengler, \emph{op. l.}, 2838 : « sur l'épaule g. s'appuie une lance » ; Mionnet, \emph{Suppl.} 5, p. 155, n° 899 : Cybèle avec un « javelot » ; mais dans les deux cas il s'agit sans doute d'un sceptre à longue haste.

*) Cf. supra, p. 5 ; Ishtar présente ce même caractère complexe de « bienfaisante déesse, auteur de l'humanité, » et de déesse guerrière, « fléau des combats. »

*) On le trouve groupé avec l'effigie de la Mêter de Plakia : Babelon, \emph{Coll. Wadd.}, 994, 995. Sur les divers types du lion tauroctone dans l'art funéraire de Phrygie, v. Graillot dans \emph{Mélanges Perrot}, p. 143 s ; sur le caractère astrologique delà lutte du lion et du taureau, cf. Dussaud dans \emph{Rev. arch.}, 1904, 2, p. 235.

*) Julian., \emph{l. c.}, p. 180. Rhéa Pandôra dans Diod. Sic. 3, 57 ; cf. ᾽Ολβοδότις, Orph., \emph{Hymn.} 27, 9 ; μακάρων πηγή τε ῥοή τε, Orph., \emph{Fragm.} 305, 1, éd. Abel.

*) On l'adorait à Pessinonte sous le vocable d'Ilea : \emph{Num. Chronicle}, 1876, p. 79 ; B. Head, \emph{Hist. Num.}, p. 630 ; Babelon, \emph{Coll. Wadd.} 6653. Faut-il rattacher à cette même idée l'épithète de Βοηθηνή, \emph{CIG.} 3993 ? --- Εὐαντής : \emph{CIAtt.} 3, 136 (1er s. de notre ère). --- Eὐάντητος : \emph{ibid.}, 134 (1er s.) ; \emph{Etymol. Magn.}, p. 388, 35 ; sur ce mot considéré au contraire comme un euphémisme, cf. Gruppe, \emph{op. l.}, p. 1539. --- Ἐπήκοος : \emph{Arch. ep. Mitt.} 7, p. 180 ; Letronne, \emph{Inscr. de l'Égypte}, 1, p. 417, n° 36 ; épithète fréquente dans les cultes mystiques et pour les dieux guérisseurs, très usitée à Palmyre, où elle traduit un vocable des religions sémitiques.

*) Σοφή : Diogen. Ath. (5e s. avant J.-C.), \emph{Frg.} 1, 5, dans Nauck, \emph{Trag. Gr. Fragm.}, 2e éd. ; Nonn., \emph{Dionys.} 41, 68. Sophia, Prometheia, Pronoia, dans Julian., \emph{l. c.}, pp. 166 et 179.

*) Σώτειρα : Orph., \emph{Hymn.} 14, 7 et 12 ; Letronne, \emph{l. c.} ; \emph{Salutaris}, v. p. 151. Tous les dieux des mystères, Demeter et Coré, Dionysos, Hécate, Isis et Serapis, sont des dieux sauveurs ; cf. Wobbermin, \emph{Religionsgesch. Stud.}, p. 34. C'est aussi l'épithète d'Apollon, dieu guérisseur, d'Artemis, protectrice des eaux thermales, d'Asclepios médecin, d'Athéna (Paus. 8, 44, 4), des Dioscures, de la Fortune, d'Héraclès Alexicacos, d'Hermès Psychopompe et de Zeus. Il existait en Phrygie un dieu sauveur (Sozôn), qui parait bien n'être qu'une forme altérée du dieu Saoazos = Sabazios : Ramsay, \emph{Cities.} 1, 264 et 293. Intéressante comparaison de cette idée de salut chez les Païens, les Juifs et les Chrétiens, par Wendland : Σωτήρ, dans \emph{Zeitschr. f. d. Neutestam. Wissenschaft}, 1904, p. 335 ss.

*) Julian., \emph{l. c.}, p. 166 C.

*) Orph., \emph{Hymn.} 27, 13.

*) Diod. Sic. 3, 58, fait allusion à ce rite quand il dit que Cybèle se plaisait à bercer les petits enfants dans ses bras.

*) ᾿Ιατρίνη, \emph{CIAtt.} 134, 136, 137. ᾿Ιατρὸς : Diogen. Ath., \emph{l. c.} (elle est qualifiée en même temps d'ὑμνῳδός ; serait-ce une allusion à des chants-médecines ? ) ; cf. Dionysos Hygiatès. Nous retrouverons le culte de la Grande Mère dans beaucoup de stations thermales.

Procréatrice des âmes, elle en veut être la gardienne.* Elle les empêche d'errer dans les ténèbres ; elle est le phare qui luit pendant les tempêtes.* Elle rend l'âme forte contre l'esprit du mal ; elle communique à ses fidèles le courage d'Hercule, qui triompha du dragon à cent têtes et sut conquérir les pommes d'or.* Ceux qui ont commis des fautes doivent espérer en elle ; car elle est la rédemptrice. Elle délie du péché ; elle efface « la tache de l'impiété.* » Elle a enseigné à ses prêtres les rites du pardon, comme ceux de la guérison. A l'âme purifiée selon sa loi, elle promet une félicité sans fin. Avec le pavot, symbole de mort, elle tend vers nous le rameau de pin, dont l'éternelle verdure est emblème d'immortalité.* Celle qui sauva de la mort Attis est la protectrice des trépassés. Sept siècles avant l'Empire, elle trônait déjà sur les nécropoles phrygiennes ; et chaque tombeau, dressé sur le sol ou creusé dans le roc, devenait un sanctuaire inviolable de la Kubilè. On avait coutume de déposer son image dans les sépultures, pour que la « reine des Mânes » n'abandonnât point le défunt. Jadis la divinité chthonienne régnait sur les morts parce qu elle les recevait, à proprement parler, dans son sein.* Mais, au contact de l'orphisme, du mazdéisme, du judaïsme, la conception phrygienne de l'au-delà s'était sublimée. L'influence directe des spéculations astrologiques avait déterminé une nouvelle conception de l'âme et de ses fins dernières. L'âme est d'essence aérienne et divine. « Mon corps est cendre, » lisons-nous sur la tombe d'un zélateur d'Attis ; « mais mon âme fut emportée par l'air sacré.* » Cette âme, qui aspire aux espaces d'en haut, prend son essor vers les régions supérieures d'où elle était descendue. Elle a hâte de présenter à la Mère de toute vie la moisson de vertus* qu'elle a cueillie sur cette terre et qui réjouit la divinité. Par-delà les portes du ciel,* dans les habitacles de l'empyrée, la Grande Mère l'accueille comme un enfant qui revient d'un lointain voyage. « Accorde-nous, » lui demandent ses fidèles à la fin de leurs prières, « accorde-nous la douce espérance de parvenir jusqu'à toi.* »

*) \emph{CIL.} 6, 499 (ann. 374).

*) Julian., \emph{l. c.}, pp. 169 D et 174 C. Mais elle est aussi celle qui envoie les spectres et les revenants, cause d'épouvante ; sur ce sens de l'épithète \emph{Antaia}, cf. Gruppe, \emph{op. l.}, p. 770.

*) \emph{Ibid.}, pp. 167 A et 176 A ; cf. supra, p. 120, n. 3. Hercule comme Dactyle Idéen : supra, p. 8, n. 5. --- Uni à Zeus Papias Sôter en Bithynie, \emph{CIG.} 3817. --- Tenant « le simulacre de Cybèle assise, » monnaie de Cotiaeum, Phrygie : Mionnet 4, p. 274, n° 459 (Caracalla). --- Sur le sarcophage d'une prêtresse de Magna Mater, à Ostie, Hercule ramène Alceste qu'il a sauvée des enfers ; Vatican, musée Chiaramonti ; cf. Helbig, \emph{Guide}, 1, 74. L'image est symbolique.

*) Λυτηριάς, Orph., \emph{Hymn.} 14, 8 ; Julian., \emph{l. c.}, p. 180.

*) Médaillons : v. supra, p. 134. n. 4 ; Mionnet, \emph{Suppl.} 5, p. 331, n° 318. Intaille : Furtwaengler, \emph{op. l.} 2843. Lampe, à Rome, musée Kircher : Cybèle tend à Attis le rameau.

*) Cf. « Dis Manibus et Terrae Matri » ; dans \emph{Bull. comunale di Roma}, 1886, p. 281.

*) Cippe avec l'image d'Attis, \emph{CIL.} 3, 6384 : « corpus habent cineres, animam sacer abstulit aer. » Dans un hymne orphique, 14, 11, Rhéa-Cybèle est dite « semblable à l'air. »

*) Julian., \emph{l. c.}, p. 169 B.

*) Clef du ciel, que tient Cybèle, cf. supra, pp. 176 et l97. Porte du ciel, par où passe Attis ressuscité : Hippol., \emph{Refut. omn. haeres.}, p. 111 s, et les sept portes du ciel mithriaque, selon la conception babylonienne, tradition suivie par les gnostiques.

*) Julian., p. 180, fin.

\subsection{3.}

En même temps que se constituait la souveraineté universelle delà Mère des Dieux, grandissait la personnalité d'Attis. Elle grandissait au détriment d'autres dieux, dont celui-ci s'appropriait les attributs et le pouvoir.

Attis avait à redouter en Orient la concurrence de Mên, qui était l'un des grands dieux d'Anatolie et l'un des plus populaires.* Dieu mâle de la lune, Mên régnait sur le ciel, sur la terre et sur le monde souterrain. Dieu « qui se fait lui-même, » fruit « qui croît de lui-même, » comme tous les dieux lunaires de l'Asie sémitique, comme le Sahar d'Harran et le Sin de Chaldée, il est le père de tous les êtres. Maître de la lune, il est par conséquent le maître de la fortune des hommes et de celle des rois. « Dieu qui n'a pas de juge au-dessus de lui,* » il est le flambeau de toute justice, le suprême justicier des vivants et des morts. Dieu guérisseur, il entretient dans ses temples les plus célèbres de florissantes écoles de médecine.* Seigneur des villes, il est aussi le dieu topique de nombreux villages.* Car il est par excellence le protecteur des campagnes. C'est lui qui fait croître les plantes, mûrir les raisins, se multiplier troupeaux et volailles ; et il est le gardien des propriétés rurales. Or, sous l'Empire, tandis que les dédicaces épigraphiques et les monnaies des villes signalent partout sa présence en Asie Mineure et qu'on le retrouve sur le Danube, en Macédoine, en Béotie, en Attique, il reste inconnu dans l'Occident latin. C'est qu'il s'y confond avec Attis. Leur assimilation avait commencé dans leur pays même d'origine. Déjà les terres cuites de Tarse et de Myrina nous montrent des figurines de types complexes, où se mêlent les attributs des deux divinités.* Les ailes recoquillées d'Attis-Eros ressemblent aux cornes du croissant lunaire qui se profile derrière les épaules de Mên ; elles en sont l'interprétation hellénique. D'autre part, à Juliopolis (Bithynie), à Syllion (Pamphylie), à Ancyre, le dieu nocturne porte un bonnet constellé comme celui d'Attis ; car « il guide le char de la nuit aux nombreuses étoiles.* » A Gordos, à Nysa, il est accompagné de deux lions.* Attis et Mên revêtent le même costume phrygien. L'un et l'autre sont jeunes, étant dieux de la végétation. Mên pose un pied sur la tête du taureau lunaire, comme Attis sur celle du bélier solaire ; et tous deux possèdent les attributs communs aux Seigneurs des morts, pavot, grenade et pomme de pin. Agriculteurs qui priaient pour leurs récoltes, parents qui priaient pour leurs défunts devaient souvent réunir leurs deux noms. A Euméneia, en Phrygie, Mên Ascaénos est rapproché des Agdistis, qui sont Cybèle et Attis.* A Rome, le caractère officiel du culte pessinontien assurait le triomphe d'Attis sur Mên. Dès la première moitié du 2e siècle, celui-ci avait cédé à celui-là son attribut essentiel, le croissant.* Il lui abandonna son propre nom et son titre lydien de Tyran, c'est-à-dire de Maître souverain. L'Attis romanisé prit l'épithète nouvelle de Mên Turannos qui, mal interprétée, passa pour être un vocable du soleil, « Maître des Mois.* »

*) Sur Mên, v. en particulier : Smirnoff dans Στέφανος (Mélanges Sokoloff), Saint-Pétersbourg, 1895, pp. 81-135 ; Drexler dans Roscher, 2, 2, s. v. \emph{Mên} : Perdrizet dans \emph{Bull. Corr. Hell.}, 20, 1896, pp. 55-106 (Mên en Asie M. ; --- diffusion hors de l'Anatolie ; --- surnoms ; --- origine ; --- nature, attributs, fonctions) ; Gruppe, \emph{op. l.}, 1906, p. 1533 ss. L'identité de Mên avec les dieux lunaires sémitiques a été mise en relief par Perdrizet ; Ramsay, dans \emph{J. of Hell. St.} 1889, p. 229 ss, prétendait à tort que Mên était le dieu phrygien du Soleil = Manês.

*) Les formules mises entre guillemets sont empruntées aux hymnes en l'honneur du Sin babylonien ; cf. Chantepie de la Saussaye, \emph{Hist. des Religions}, p. 136 s ; ce dieu est représenté, comme Mên, posant un pied sur une tête de taureau. En principe, les dieux sémitiques de la lune passaient avant ceux du soleil, de même que l'on comptait les jours à partir du coucher du soleil.

*) Strab. 12, 8, 20.

*) Cf. \emph{Papers Amer. School}, 2, n°s 60, 61.

*) V. supra, p. 204, n. 6.

*) Oracle dans Euseb., \emph{Praep. eu.} 3. p. 125.

*) Mionnet 4, p. 41, n° 216 (L. Verus) et \emph{Suppl.} 6, p. 521, n°s 415, 416.

*) \emph{CIG.} 3886 add. Dans un hymne orphique, \emph{Prooem. ad Mus.} 40, on trouve une invocation à la Mère des dieux, Attis et Mên.

*) Statue d'Attis, provenant du temple d'Ostie, Museo Laterano ; Helbig, \emph{Guide}, 1893, 1, p. 522, n° 700 ; \emph{CIL.} 9, 3146, à Corfinium : « Attini... Iunam argenteam... » Attis prend aussi le croissant à Pessinonte : Imhoof-Blumer, \emph{Griech. Münzen} dans \emph{Abhandl. Bayer. Akad., phil.-hist. Kl.}, 18, 1890, p. 750.

*) \emph{CIL.} 6, 499-501, 508, 511, 512 ; \emph{IGSI.} 913. Sur l'épithète de Turannos donnée à Mên, v. Perdrizet, \emph{l. c.}, p. 86 ss ; Zeus porte aussi ce titre en Lydie.

C'est l'action des rayons solaires qui dans la nature est prépondérante. Esprit de l'arbre et de la plante, fils de l'amandier, épi encore vert et pourtant moissonné,* fleur,* sang d'où naît la violette printanière, pin dont l'immortel feuillage proteste contre l'idée d'une mort réelle, Attis s'était transformé depuis longtemps en dieu solaire. Telle est la métamorphose normale des dieux de la végétation. Le bon sens des simples fidèles, comme la science des théologiens, remontait des effets aux causes. De plus, les grandes fêtes de la Passion et de la Résurrection d'Attis se célébraient à l'équinoxe du printemps, qui est aussi la fête du soleil. Le petit dieu était devenu le génie du soleil vernal, qui ramène sur la terre la flore luxuriante, y prépare les belles récoltes, mais doit céder aux forces destructrices du soleil d'été. Le lion totémique, devenu l'adversaire d'Attis,* symbolisa peut-être les torrides étés de Phrygie. Mais Attis avait fini par s'identifier complètement au dieu Soleil,* dans ses énergies multiples et opposées, tour à tour vivifiantes et consumantes. Il a dompté le lion ; maintenant il le chevauche,* ou guide le quadrige des fauves.* Constante est l'assimilation des dieux de la fécondité, maîtres aussi du monde infernal, avec le soleil qui, le jour, fertilise la terre et, la nuit, parcourt les régions souterraines.

*) Hippol., \emph{op. l.} 5, pp. 9, 168, éd. Duncker et Schneidewin.

*) Euseb., \emph{Praep. evang.}, 3, 11, 17 : τῶν ἐαρινῶν ἀνθέων ὁ Ἄττις σύμβολον ; cf. 11, 12 ; Fulgent., \emph{Myth.}, 3, 5, p. 713, éd. Staveren.

*) Julian., \emph{l. c.}, p. 167 B et C ; cf. le rôle du lion dans l'Attis de Catulle, 76-89 et dans l'Anthologie grecque, 6, 217-220, 237, éd. Stadtmueller, 1894, 1, pp. 338-343 et 355.

*) Arnob. 5, 42 : Anonyme de 394, \emph{Carmen contra pag.} 109 ; Firm. Mat., \emph{De err. prof. rel.} 8, 2 ; Macrob., \emph{Saturn.} 1, 21, 7 ss ; Mart. Cap., \emph{De nupt. Phil. et Merc.} 2, 192 ; Procl., \emph{Hymn. in Solem}, 24 ss.

*) Biardot, \emph{T. cultes gr. funèbres}, p. 318 et pl. 16, 2.

*) « Il commande aux lions, » dit Julien, \emph{l. c.}, p. 168 B ; « il conduit le char de la Mère, » p. 171 C. Médaillon contorniate, v. supra, p. 135, n. 4.

Adonis-Tammouz, de l'autre côté du Taurus, avait subi la même évolution. Ses jardins vite flétris représentaient la splendeur brève du printemps ; le sanglier qui tue le dieu signifiait la brusque irruption des rayons dévorants. Mais Adonis est depuis longtemps le Soleil de l'année, dans sa course zodiacale et sa fonction calendaire* ; la bête figure désormais l'hiver meurtrier, « qui est pour le soleil comme une blessure. » Les deux beaux adolescents se ressemblaient comme des frères, dont l'un aurait grandi au milieu des populations sémitiques et l'autre parmi des tribus de race aryenne. Ils avaient eu toutefois des fortunes diverses. Adonis, par l'intermédiaire de Byblos, de Paphos, des marchands phéniciens et des courtisanes syriennes, avait rapidement conquis à son culte tout le bassin oriental de la Méditerranée, jusqu'à l'Égypte, jusqu'à la Sicile. Au 5e siècle déjà, sur les terrasses d'Athènes, les femmes psalmodiaient des cantiques funèbres en l'honneur du dieu étranger et sanglotaient éperdument dans la nuit en criant son nom. En Lydie, il avait imposé son propre mythe aux adorateurs d'Attis ; car l'Attis lydien meurt comme lui sous les coups du sanglier. Au 3e siècle, Théocrite, qui s'intéresse en artiste aux Adonies royales d'Alexandrie et qui compose une idylle sur la mort d'Adonis, ne différencie pas le dieu chasseur de Byblos du dieu berger de Phrygie.* Sous la domination romaine, grâce aux privilèges dont jouissait à Rome la Grande Mère Idéenne, Attis supplante en Occident son rival sémitique.* Il s'empare de l'arc et du carquois d'Adonis le chasseur.* Il se fait appeler parfois Adonis le Pur,* de même qu'il avait accaparé le nom de Mên-Roi, et comme s'il prétendait être l'unique Seigneur.

*) Adonis, dit Macrobe, \emph{l. c.}, est à n'en pas douter le soleil. Mais Amm. Marc. 19, 1 et Lydus savent encore qu'il était le dieu-fruit. Sur les différentes conceptions solaires d'Adonis, v. Vellay, \emph{Culte et fêtes d'Adonis-Thammouz dans l'Orient antique}, 1904, p. 118 ss.

*) \emph{Idyll.} 1, 105 et 109 (98 s, éd. Ahrens) : «va vers l'Ida, où le montagnard Adonis fait paître ses brebis. » Voir aussi les Idylles 15 et 31.

*) Parmi les mystères en vogue à la fin de l'Empire, il n'est pas question de ceux d'Adonis. Il semble qu'ils se soient confondus en Occident avec les mystères métroaques et dionysiaques : « je suis Bacchus chez les vivants, Adonis chez les morts, » Auson., \emph{Epiqr.} 28 ; cf. 29. On trouve réunis Attis, Adonis et Dionysos dans un oracle aux Rhodiens : Socrat., \emph{Hist. eccl.} 3, 23.

*) Attis chasseur, sur le relief du rocher d'Hammamli (époque romaine) : Le Bas, \emph{Itinéraire}, pl. 55 ; Pottier et S. Reinach, \emph{Nécrop. de Myrina}, p. 407. Attis funéraire à l'arc, provenant de Concordia Veteranorum, Musée de Portogruaro. On trouve aussi Mithra muni de l'arc et du carquois ; c'est avec l'arc qu'il combat, dans les hymnes avestiques ; les flèches qu'il lance sont les rayons de la lumière céleste : Cumont, \emph{Mithra}, 1, pp. 165 et 183. Les dadophores mithriaques sont parfois représentés avec l'arc et ont pu influencer le type d'Attis archer.

*) Socrat., \emph{l. c.} ; Hippol., \emph{op. l.} 5, 9.

L'une des raisons de sa supériorité sur le dieu de Byblos, ce fut sa longue intimité avec Mithra.* Quand ils font leur entrée dans le monde latin, c'est en se poussant l'un l'autre, selon l'expression pittoresque de Lucien. Mais si l'un prêtait à l'autre l'appui d'un clergé officiel et puissant, Mithra, par une salutaire influence de plusieurs siècles, avait singulièrement rehaussé la fonction théologique et la valeur morale d'Attis. Les dédicaces romaines à Sol Invictus Mithra prouvent que l'Occident l'a surtout connu sous l'aspect solaire. A l'origine, il était le génie de la lumière et du feu, accouplé à la déesse des eaux fertilisantes et très distinct de l'astre du jour.* Depuis longtemps, toutefois, on l'adorait en Orient comme le maître du Soleil ; et c'est en Phrygie même qu'il avait hâté l'assimilation d'Attis avec Hélios. Les nombreux mages établis dans l'Anatolie purent fournir au culte phrygien les éléments nouveaux d'un symbolisme astrologique. Plus tard, et jusqu'à la fin de l'Empire, l'imagination subtile des théologiens ne cessa de compliquer ce symbolisme. Macrobe explique gravement la signification sidérale du pedum et de la syrinx, attributs constants d'Attis le pâtre. Déjà Lucien savait que Corybantes et Galles reproduisent dans leurs ébats chorégraphiques les mouvements du soleil.* Ce sont là fantaisies de doctes, et non pas articles de foi. Elles devaient naturellement se multiplier en un temps où le monothéisme solaire tendait à se substituer aux religions du paganisme. Mais ce qui importe, c'est que dès ses débuts dans le panthéon romain Attis est couronné de rayons, comme Mithra.* Il est donc, selon l'astrolâtrie chaldéenne, le chef des planètes. Car leurs stations et rétrogradations se trouvent liées au cours du soleil, qui semble les guider comme un maître. « Le soleil est au milieu des planètes comme un bon aurige, conduisant avec sûreté le char du cosmos.* » C'est donc Attis, nouvel Hélios, qui prend en main les rênes du quadrige cosmique des lions, sur lequel trône la Mère Ouranienne. Parfois aussi, pour mieux préciser son caractère héliaque, on lui donne un quadrige de béliers ; car c'est dans le signe du Bélier que le soleil recouvre sa force et reprend sa marche triomphale.* De même que Mithra porte un manteau « brodé d'étoiles, tissé dans le ciel,* » sur le bonnet phrygien d'Attis on sème des astres d'or ; et comme Attis n'est rien sans la Mère des Dieux, on dit qu'il a reçu d'elle le « pilos » constellé, image du ciel visible.* Ses statues sont dorées, parce que l'éclat de l'or symbolise le rayonnement de l'astre du jour.* Enfin, dieu pastoral, ancien protecteur des troupeaux et des bergers, Attis est salué seigneur mystique des vastes pâturages du ciel. Il est « le berger des blanches étoiles.* »

*) Voir supra, p. 192 ; Cumont, \emph{l. c.}, pp. 212 s et 235.

*) Cf. \emph{ibid.}, p. 200.

*) Macrob. 1, 21, 7 ; Lucian., \emph{De salt.} 17.

*) Statue d'Ostie. Sur les monnaies de Pessinonte, au dernier siècle avant notre ère, la tête d'Attis, coiffée du bonnet, est surmontée d'un astre : Mionnet, 4, p. 391, n° 104 ; Barclay Head, \emph{Il. num.}, p. 630.

*) Hermipp. 2, 16, 116 ; cf. Bouché-Leclercq, \emph{Astrol. gr.}, p. 118. Les lampe que l'on suspendait autour des images d'Attis (Julian., p. 179 C) représentaient peut-être les planètes autour du soleil.

*) Cf. supra, p. 178, n. 1.

*) Darmesteter, \emph{Zend-Avesta} 2, p. 507 ; Cumont, \emph{l. c.}, 1, pp. 183 et 201 ; sur certains reliefs, le manteau du dieu porte sept étoiles et le croissant. Le manteau et le bonnet de l'Attis d'Ostie ont encore des traces de dorure, mais on ne peut dire s'ils étaient complètement dorés ou simplement ornés d'étoiles.

*) Julian., p. 165 B C ; cf. 168 C, où il est dit qu'Attis se couvre du ciel comme d'une tiare ; 171 A : « Sa tiare est tachetée d'étoiles. » Le philosophe Salluste, \emph{De diis et mundo}, 4, précise ce symbolisme en déclarant que la M. d. D. a donné à Attis les puissances célestes.

*) Sur ce symbolisme dans les religions syriennes, cf. \emph{C. r. Acad. Inscr.}, 1910, p. 397.

*) Hippol., \emph{l. c.}

Mais le soleil, roi des cieux, est en même temps « l'arbitre suprême de la terre, » ainsi qu'il est dit dans les hymnes antiques de la Chaldée et de l'Égypte.* Par sa chaleur et sa lumière, n'est-il pas « celui qui développe toute vie, » qui féconde et anime la nature ? L'eunuque Attis devient donc, comme Mithra, le maître de la génération.* Il préside à la croissance des êtres, \emph{Deus Almus}.* On lui donne le titre de démiurge.* Un mythe racontait que du sang du dieu mutilé naquit l'amandier, père de tout et de tous.* De même, dans le mazdéisme, l'immolation du taureau mythique fut l'origine de toute végétation terrestre et de toutes les espèces d'animaux terrestres. La naissance du premier homme, selon certaine doctrine iranienne et certaines croyances des Phrygiens, qui prétendaient être le plus ancien peuple de la terre,* paraît se rattacher à cette même idée de sacrifice.* C'est ainsi que la mort salutaire, voulue par les dieux, avait engendré une vie nouvelle, merveilleusement plus riche avec plus de beauté.* Dans les deux religions, les deux drames sanglants représentent le grand drame delà création, au commencement des âges. Mais il est manifeste que, dans sa fonction de démiurge, Attis subit profondément l'influence de la théologie mazdéenne. On avait même vu se répandre en Asie Mineure un type d'Attis tauroctone* ou crioctone,* à l'imitation du prototype mithriaque.

*) Cf. les idées solaires contenues dans les hymnes à Ammon : Maspero, \emph{Hist. anc. des p. de l'Orient}, 2, p. 543.

*) En Asie Mineure, Attis semble avoir porté les titres mystiques de Mêtis, Phanès, Erikepaios, que Malala, 4, p. 74, traduit par Boulé, Phôs, Zôodotêr, Sagesse, Lumière, Donneur de vie ; v. Gruppe, \emph{op. l.}, p. 1544. Le nouvel orphisme aurait emprunté ces vocables aux mystères anatoliens.

*) Publil. Optât. Porfyr., \emph{Carm.} 27, 9.

*) Julian., \emph{l. c.}, pp. 161 C, 165 B, 166 D ; Sallust. phil., \emph{l. c.}

*) Πατὴρ τῶν ὅλων, Hippol., \emph{l. c.}, p. 166. Sur le rôle de l'amandier dans le mythe, cf. Hepding, \emph{Attis}, p. 106. Cet arbre est généralement considéré comme originaire de l'Anatolie centrale et du Pont ; il pousse à l'état sauvage en Afghanistan. Son nom grec, « amugdalos, » n'a rien à voir avec l'hébreu 'êm gedôlâh = Grande Mère, contrairement à ce que croyait Movers, 1, 578, 586, suivi par Hehn, \emph{Kulturpflanzen}, \emph{etc.}, note 81 ; cf. Schrader dans Hehn, 6e éd., 1894, p. 387, citant Muss-Arnolt, \emph{Semitic words in Greek and Latin} dans \emph{Transactions of the American philol. Assoc.}, 23, p. 106 s.

*) Apul., \emph{Met.} 11, 5 : « primigenii Phryges. »

*) La légende mazdéenne de Gayômart rappelle de très près celle d'Attis : quand le héros est dévoré par Ahriman, deux gouttes de sa semence tombent sur la terre et y font pousser deux arbustes, d'où naît le premier couple humain ; cf. Cumont, \emph{l. c.}, p. 190, n. 5 ; Bousset, dans \emph{Goetting. gelehrten Anzeigen}, 1905, pp. 698 et 702. Sur Attis considéré par suite comme l'homme primitif et rapproché d'Adam, v. infra. C'est également en Phrygie, à Pessinonte, que serait née la seconde race d'hommes, après le déluge. Les pierres jetées par Deucalion et Pyrrha auraient été prises sur le rocher Agdos : Arnob. 5, 5.

*) Julian., p. 179 D : Attis a organisé le monde en beauté, dit-il, au point que ni l'art ni l'intelligence des hommes ne sauraient jamais l'imiter.

*) Terres cultes trouvées à Panticapée, mais probablement de fabrication asiatique : Cumont, \emph{op. l.}, 2, mon. fig. 5-6, avec la bibliographie.

*) Peut-être sur un vase du musée de Naples, n° 2843, datant du 4e ou du 3e s. ; cf. Cumont. \emph{op. l.}, 1, p. 180, n. 10 ; de même sur une terre cuite de Myrina, un Eros immole un bélier ; \emph{ibid.}, p. 181, fig. 12.

Les théologies mystiques, d'autre part, ne s'intéressent au problème de la création que pour expliquer la destinée de l'homme. Un lien étroit rattachait les légendes cosmogoniques des mages à leurs idées sur la fin du monde. Ils croyaient que Mithra, en tuant le taureau, avait préparé pour la consommation des siècles une seconde rénovation de l'univers. Les Iraniens, comme les Sémites, admettaient la résurrection finale des corps. Nous ignorons quel devait être le rôle d'Attis dans le dernier acte du drame cosmique. Mais on enseignait aux mystes que l'essence éternelle d'en haut ne comportait plus ni élément masculin, ni élément féminin. Les élus réuniraient les deux sexes,* comme l'Agdistis de la légende pessinontienne, sans doute comme Attis ressuscité, que l'iconographie représente parfois avec tous les attributs de l'hermaphroditisme. Le passage à cet état définitif et parfait, telle était la signification dernière de l'éviration d'Attis.

*) Hippol., \emph{op. l.}, 5, 7, p. 138, à propos des Naasséniens.

C'est ainsi que le vieux mythe naturaliste s'accommodait à l'eschatologie des temps nouveaux. Il se prêtait facilement à cette adaptation. Il s'était combiné depuis longtemps avec des idées de rédemption et d'immortalité. Si les Phrygiens, qui l'empruntèrent aux tribus indigènes ou à leurs voisins de race sémitique, se l'étaient approprié volontiers, c'est qu'ils croyaient eux-mêmes, comme tous les Thraces, à une survivance. Au contact de la religion persique, ces croyances d'abord vagues et confuses se précisèrent. Sous l'Empire romain, les images d'Attis qui décorent les tombeaux de nombreux mystes sont les symboles caractéristiques de cette foi. Mais leur évidente ressemblance avec les dadophores mithriaques prouve une fois de plus les relations intimes des deux cultes et l'assimilation des doctrines. Elle nous aide à mieux comprendre de quelle manière on concevait, dans les mystères métroaques, la fonction protectrice du jeune dieu.

Par ce qu'il est le génie de la lumière, que l'air semble porter, Mithra passait pour habiter la zone mitoyenne entre le ciel et la terre. Aussi le qualifiait-on de Mésitès, c'est-à-dire d'intermédiaire. Il mérita doublement ce titre quand il fut identifié avec le soleil, image visible des dieux invisibles. « Il est le médiateur entre le dieu inaccessible et inconnaissable, qui règne dans les sphères éthérées, et le genre humain qui s'agite et souffre ici-bas.* » Tel est aussi le rôle d'Attis. Chargé par le couple primordial Zeus-Mêter de créer et d'organiser l'univers, il a conséquemment pour attribution d'y maintenir l'ordre et de veiller sur ses créatures. Il sert d'intermédiaire entre les dieux suprêmes et l'humanité. Il est le Ministre de la Grande Mère* et son Messager rapide. On le dénomme l'Ange de Rhéa.* C'est pourquoi l'on disait qu'il fut son premier prêtre et apôtre, et que le premier il enseigna ses mystères. Dieu de toute sagesse,* ainsi qu'il convient au démiurge, il montre le chemin de la sagesse aux hommes de bonne volonté. Il leur apprend qu'il faut vivre selon la loi métroaque, fuir le mal et chercher le mieux, détruire en soi l'impureté et se mutiler des appétits déraisonnables, toujours regarder vers le ciel « et par-delà le ciel.* » Dieu de lumière, dont le flambeau dissipe les ténèbres, il est toujours vigilant ; il voit tout. Par suite, il règne avec justice. Mais il règne aussi avec bonté. Comme Mithra, comme tous les dieux héliaques, « il fait du bien par son regard. » Il protège les dévots de la Mère des Dieux contre les puissances impies et malfaisantes ; et parfois on lui donne pour compagnon, comme à Mithra, le chien qui met en déroute les mauvais génies. Il est le dieu secourable et propice,* le bon pasteur,* de même qu'il est le dieu beau « comme les rayons solaires.* » Éternellement jeune et toujours victorieux,* il est le tuteur et le défenseur,* qui soutient le juste dans les épreuves de la vie, l'assiste à l'heure fatale et guide son âme au divin séjour. Si des Romains, et non point dans la plèbe, peuvent l'invoquer comme le gardien de leur âme et de leurs pensées,* c'est à Mithra surtout qu'il est redevable de ce privilège. En convertissant la religion phrygienne à une morale supérieure, le mazdéisme l'avait rendue digne de figurer avec honneur dans la lutte qui s'engageait pour la domination des âmes.

*) Cumont, \emph{op. l.}, 1, p. 303.

*) Prospolos, dans Harpocration, \emph{Lex.}, s. v. \emph{Attès} ; Bekker, \emph{Anecd. graeca,} 1, p. 461 ; Suidas, s. v. \emph{Attis} ; cf. Julian., p. 171 C.

*) Nonn. \emph{Dionys.} 25, 313 : Ῥείης ταχὺς ἄγγελος.

*) Hippol., \emph{op. l.}, p. 168 ; Julian., p. 179 C. Le texte d'Hippolyte paraît douteux à Wilamowitz-Moellendorf dans \emph{Hermès}, 37, 1902, p. 328, suivi par Hepding, \emph{Attis}, p. 34. Cependant Attis semble avoir porté le titre mystique de Mêtis, dont le sens est voisin de celui de Sophia ; et. supra, p. 213, n. 8.

*) Cette expression est dans Julien, qui développe longuement cette interprétation morale du mythe.

*) \emph{Deus propitius} : Arnob. 1, 41 ; \emph{conservator} : \emph{CIL.} 6, 500.

*) Les épithètes relatives à ses fonctions pastorales sont réunies dans Hepding, \emph{Attis}, p. 206 s.

*) Julian., \emph{l. c.}, p. 165 C.

*) \emph{Invictus.} \emph{CIL.} 6, 499. Épithète des dieux solaires. Les astres divins ne meurent pas ; chaque fois qu'ils semblent s'affaiblir, ils renaissent à une vie nouvelle, toujours invincibles. Le vocable est emprunté, par l'intermédiaire du grec ἀνίκητος, à la terminologie religieuse de l'Orient sémitique ; sur son caractère liturgique à Babylone, cf. King, \emph{Babyl. Magie and Sorcery}, p. 111.

*) \emph{Tutator} : \emph{CIL.} 6, 512.

*) \emph{CIL.} 6, 499.

Toutefois, le titre d'Ange que nous voyons donner au dieu, et qui sans doute était une épithète liturgique, révèle une autre influence. Attis subit aussi l'ascendant des idées sémitiques, et plus particulièrement judaïques. Ce vocable, en effet, tout à fait rare dans les dédicaces païennes de l'âge impérial, n'appartient pas en propre à la terminologie du mithriacisme.* On l'appliquait à certains Baals syriens, considérés comme dieux psychopompes.* Il est possible que l'orphisme ait favorisé, dans le monde orientalo-grec, la croyance aux bons anges et aux anges malfaisants.* Mais là où nous trouvons le culte des Dii Angeli,* l'action du sémitisme est généralement manifeste. Dans les mystères sabaziens, un Angélus Bonus est l'introducteur des morts au festin des bienheureux.* Deux autels de Stratonicée, en Carie, associent le Bon Ange ou Ange Divin à Zeus Hypsistos.* Sur un groupe d'inscriptions de l'île de Rhéneia (Cyclades), qui datent du 2e siècle avant notre ère, Théos Hypsistos est également nommé avec ses Anges.* Or Sabazios et Zeus Hypsistos sont des dieux judaïsants ; et le « Dieu Très Haut » n'est autre qu'un Jéhovah quelque peu paganisé. Son culte s'était répandu en Asie Mineure dès l'époque hellénistique, après que les Séleucides y eurent établi des colonies juives. C'était précisément en pays lydien et phrygien qu'Antiochus le Grand, pour tenir en respect les tribus indigènes et favoriser par-delà le Taurus la politique syrienne, avait implanté deux mille familles juives de Babylone.* Ces colonies, fixées dans les villes fortes, dotées de terres, jouissant des mêmes droits que les Hellènes, formaient un élément considérable et puissant de la population urbaine. Elles se développèrent encore sous l'Empire, quand se produisit la dispersion dernière. Si elles n'avaient pu sauvegarder, au milieu des idolâtres, toute l'intégrité de leur foi nationale, leur esprit de prosélytisme avait conquis de nombreux gentils à l'idée monothéiste. En dehors du judaïsme orthodoxe, il se constitua des communautés mi-juives et mi-païennes d'Hypsistariens, qui rendaient un culte exclusif au Très Haut.* La religion phrygienne se ressentit nécessairement de leur contact et de leur propagande. C'est ainsi que le Sabazios thraco-phrygien, identifié jadis par les Grecs à leur Dionysos, haussé plus tard à la dignité d'un Zeus, s'était confondu avec le Sabaoth d'Israël. Une analogie fortuite de noms avait déterminé cette assimilation,* qui transforma Sabazios en Seigneur Suprême, Très Haut, Tout Puissant, Très Saint. Aussi l'action du Iahvisme sur les mystères du dieu paraît-elle avoir été profonde. Nous en avons d'autres témoignages que cette intervention de l'ange, qui supplée Hermès comme conducteur des âmes. Il y eut adaptation de certaines doctrines, adoption de certaines pratiques. Au 2e siècle de notre ère, dans le monde gréco-romain, les profanes ne différenciaient guère les deux cultes ; et les chrétiens eux-mêmes considéraient moins les Sabaziastes comme des païens que comme des dissidents de la Synagogue.*

*) Sur le rôle des anges dans la théologie des mages, v. les textes réunis par Cumont, \emph{Relig. orientales dans le paganisme romain}, p. 306.

*) \emph{I(ovi) O(ptimo) M(aximo) Angelo Heliop(olitano)} : \emph{CIL.} 14, 24 ; Lévy dans \emph{Rev. des Études Juives}, 43, 1901, p. 5.

*) Maas, \emph{Orpheus}, 222, qui paraît exagérer le rôle des doctrines orphiques. Il est question d'anges catachtoniens, qui sont des anges malfaisants, dans les \emph{Tabellae defixionum} de l'Attique : Audollent, n°s 74, 75.

*) En Mésie : \emph{Jahresh. d. Oesterr. arch. Instit.} 1905, \emph{Beibl.}, p. 6 : à Théra : \emph{IGIns.} 3, 933.

*) Cf. supra, p. 186.

*) Lebas-Waddington, 3, 515 ; \emph{Bull. Corr. Hell.} 5, 1881, p. 182.

*) Wilhelm dans \emph{Jahresh. d. Oesterr. arch. Instit.} 1901, \emph{Beibl.}, p. 13 ; Deissmann dans \emph{Philologus} 61, 1902, p. 252.

*) Sur ces faits, cf. en particulier Ramsay, \emph{Cities} 2, 667 ss, et Isidore Lévy dans \emph{Rev. des Études Juives}, 1900, p. 14 ss ; Cumont, \emph{Les mystères de Sabazius et le judaïsme}, dans \emph{C. r. Acad. Inscr.}, 1906, p. 63 ss.

*) Sur ce culte, Cumont, \emph{Hypsistos} (supplément à la \emph{Rev. de l'Instr. publ. en Belgique}, 40, 1897).

*) Identification constante dès le 2e s. avant notre ère.

*) Cf. la présence du tombeau d'un « antistes Sabazis » dans les catacombes de Prétextât.

Dans un sanctuaire mysien, nous voyons Cybèle associée à Zeus Hypsistos, c'est-à-dire au dieu d'Israël.* Les doctrines judaïques avaient également pénétré la théologie métroaque. Sur bien des points elles y renforcèrent l'apport du mazdéisme. Mais elles l'enrichirent aussi d'éléments nouveaux, qui hâtèrent son évolution. Complexe est en effet l'influence qu'elles ont exercée sur Attis. Tantôt, parce qu'il préside à la génération et qu'il est le père du genre humain, l'Adamas*-Attis se rapproche de l'Adam des Juifs. Entre sa faute et le péché du premier homme biblique, cause de souffrance, de corruption et de mort, le rapprochement s'imposa. Sans doute alors les antiques purifications des mystères prirent une efficacité nouvelle, de même ordre que celle du baptême chrétien.* Tantôt, représenté comme l'émissaire des dieux suprêmes et invisibles auprès des mortels, Attis est semblable à l'ange de Iahveh.* Mais ses fonctions solaires de démiurge le préparaient à de plus hautes destinées. Le dieu créateur et conservateur tend à s'identifier avec le dieu transcendant dont il émane et à la toute-puissance duquel il participe. Dieu héliaque, Attis était déjà roi.* Il était déjà très puissant ; car, d'après les principes de l'astrolâtrie, phénomènes delà nature, dispositions des hommes et événements de ce monde sont déterminés par les énergies planétaires. Ses accointances avec le Roi des Rois* Sabazios, désormais aussi juif que phrygien, facilitèrent son exaltation au souverain empire selon la formule sémitique. Il devient à son tour le Pantocrator, l'Omnipotent,* le Seigneur de pouvoir infini. En lui s'absorbe la personnalité de Zeus Pappas, dont il prend le sceptre et le nom.* Il est donc, non plus le soleil visible, mais le maître du soleil et de toutes choses, celui qui trône par-delà les planètes et les étoiles fixes, au plus haut des cieux, l'Hypsistos,* \emph{Deus Summus}.* Il est donc aussi l'Éternel.* Toutes ces notions se complètent et s'appellent les unes les autres. Nous les trouvons réunies dans cette définition du Très Haut, que donnaient au temps des Antonins les théologiens sémitisants de l'Ionie, inspirateurs des oracles encore célèbres de Claros et de Didymes : « il est celui qui n'a pas eu de commencement et qui n'aura pas de fin, le tout puissant qui s'est engendré lui-même, celui qui a créé l'enveloppe du ciel, étalé la terre et repoussé la mer, celui qui a fait alterner l'hiver et l'été, l'automne et le printemps, celui qui conduit toutes choses à la lumière et qui constitue l'harmonie de l'univers.* »

*) V. supra, p. 200, n. 2.

*) Nom d'Attis, ou du dieu équivalent, dans les mystères de Samothrace : hymne cité par Hippolyte, \emph{l. c.} ; les mss. donnent Ἀδὰμ σεβάσμιον. D'après Hesychius, s. v. ἀδάμνειν, ce serait un nom phrygien, synonyme de φίλος ; cf. Tomaschek dans \emph{Wien. Sitzungsber.}, 130, 1894, p. 43.

*) Cf. un phénomène analogue dans les mystères de Sabazios : Cumont, \emph{op. l.}, p. 9. La croyance relative aux conséquences de la chute d'Adam pour l'humanité est entièrement développée dès le 1er siècle. D'autre part, la doctrine du péché originel était déjà enseignée par l'orphisme : le genre humain, né de la cendre des Titans, qui sont fils du Ciel et de la Terre Mère, est d'origine divine : en lui subsiste un élément céleste ; mais le crime irrémissible des Titans y a mêlé un élément de péché et de corruption ; cf. S. Reinach, \emph{Cultes, Mythes et Religions}, 3, 1908, p. 346 ; sur Cybèle et les Titans, v. Gruppe, \emph{op. l.}, p. 1286.

*) Sur la personnalité de l'ange de Iahveh, cf. Chantepie, \emph{Hist. des Religions}, p. 229. Le patron des Palmyréniens était Malakbel, « le messager du Seigneur » ; cf. Dussaud, \emph{Notes de myth. syrienne}, 1903, p. 24 ss.

*) Julian., pp. 168 C, 169 C ; cf. Buresch, \emph{Aus Lydien}, p. 114.

*) Βασιλεύτατος, dans Orph., \emph{Hymn.} 48, 5.

*) \emph{CIL.} 6, 502, 503 ; 8, 8457, 20246 ; Graillot, dans \emph{Rev. archéol.} 1904, 1, p. 325 s.

*) V. supra, p. 15 ; cf. Tomaschek, \emph{l. c.}, p. 42, et sur le nom de Pappas dans l'onomastique thrace, 131, 1895, p. 18.

*) \emph{IGSI.} 1018 = \emph{CIL.} 6, 509 ; cf. Buresch, \emph{Klaros}, 1889, p. 50.

*) Firm. Mat., \emph{De err.} 3, 3.

*) Deux autels tauroboliques de Turin, \emph{CIL.} 5, 6961, 6962, sont dédiés \emph{Viribus Aeterni}. Il est possible qu'il y ait eu assimilation d'Attis avec le \emph{deus aeternus} d'importation syrienne ; cf. Hepding, \emph{Attis}, p. 209, n. 5.

*) Réponse de l'Apollon de Claros ou de celui de Didymes, insérée dans une \emph{Théosophie} qui fut composée à la fin du 5e siècle par un certain Aristocritos et dont il existe un extrait publié pour la première fois par Buresch, \emph{Klaros}, 42 ; cf. Cumont, \emph{Hypsistos}, p. 10.

Aux idées connexes de domination et d'éternité s'associe l'idée de sainteté.* A vrai dire, la sainteté est par excellence l'attribut des dieux sémitiques. Elle résume tous les autres. Car elle signifie que le dieu est dieu dans l'acception la plus complète du mot. Ce qu'elle exprime avant tout, c'est la sublimité inaccessible et ineffable de l'être divin. Elle implique secondairement l'idée de pureté, qui donne au dieu son caractère moral. L'infiniment Haut est le souverainement Pur. Attis ajoute à ses épithètes rituelles les titres de Pur* et de Très Saint.* Cette conception hébraïque de la divinité modifiait les rapports d'Attis avec ses mystes, qui se sentaient plus loin de lui. Elle exigeait des fidèles une dépendance plus servile à l'égard du maître absolu, une préoccupation plus minutieuse encore de se maintenir en constant état de grâce. Dans la pratique, elle correspondait à une aggravation de la loi morale et du code religieux. Mais Attis, sous l'action du judaïsme, avait pris en réalité les allures d'un Baal. Les Baals syriens, « Seigneurs de l'univers » et « Seigneurs de l'éternité, » restent toujours accouplés à une déesse parèdre, la Baalat, c'est-à-dire la Dame, toute puissante comme eux et très sainte. L'un représente le principe mâle, l'autre le principe féminin, et ils sont les auteurs de toute fécondité. Nous avons vu que la grande déesse phrygienne se confond souvent avec l'Atargatis de Bambyce, qui est devenue sous l'Empire la grande déesse de Syrie ; de même Attis rappelle de très près Hadad, qui fut également assimilé à Hélios et dont les Latins traduisaient la titulature orientale par les qualifications de Summus Maximus Potentissimus.* L'influence directe des cultes syriens n'est pas moins manifeste que l'influence juive. C'est par elle que s'explique la préséance accordée au dieu dans certaines dédicaces métroaques.* Le Baal précède toujours la Baalat.

*) Sur le développement de l'idée de sainteté chez les Sémites, cf. Robertson Smith, \emph{Religion of the Semites}, 2e éd. 1894, p. 446 ss ; Lagrange, \emph{Études sur les rel. sém.}, 2e éd. 1905, p. 141 ss. Sur l'épithète rituelle de \emph{Sanctus} : Baudissin, \emph{Stud. zur sem. Religionsgesch.}, 1876, 2, 33 ; Clermont-Ganneau, \emph{Études d'archéol. or.}, 1, 1896, p. 104 ; Cumont, \emph{Relig. or.}, p. 289.

*)  Ἁγνός : Socrat., \emph{Hist. eccl.} 3, 23.

*) \emph{CIL.} 6, 501 ; 8, 7956 ; Arnob., 1, 41. De même, \emph{Mater Deum Sancta}, 8, 19981.

*) Macrob., \emph{Saturn.} 1, 23, 17 ; cf. \emph{CIL.} 3, 3292, une dédicace aux « Dii Magni Majores et Sanctissima Sanctitas. »

*) A Bénévent, dans la première moitié du 3e s. : \emph{CIL.} 9, 1538-1542. A Rome, en 377 : \emph{IGSI.} 1019. De même, dans Clément d'Alexandrie, \emph{Protrept.} 11, 15. Il est à remarquer que l'initié s'intitule « myste d'Attis » : Firm. Mat., \emph{De err.} 18, 2 ; cf. les Attabocaoi de Pessinonte.

[Planche 4. --- 1. \emph{Buste d'Attis}, en marbre. --- A Rome, Musée des Thermes (Collection Ludovisi).](https://cdn.solaranamnesis.com/HenriGraillot/4-1.jpeg)

[Planche 4. --- 2. \emph{Buste de la statue de Cybèle}, en marbre, trouvée à Formies (Ny-Carlsberg).](https://cdn.solaranamnesis.com/HenriGraillot/4-2.jpeg)

Les épithètes judéo-syriennes de Cybèle et d'Attis n'apparaissent que tardivement dans l'histoire. Elles nous sont connues pour la première fois aux dates suivantes :

Attis Hypsistos, en 370 ;  
Attis Basileus, Megas, Sophos, en 362* ;  
Dii Potentissimi, en 319 ;  
Dii Omnipotentes, vers 230 ;  
Mater Deum Sancta, entre 222 et 235 ;  
Attis Sanctus, au 3e siècle.

*) Date de l'opuscule de Julien sur la Mère des Dieux.

Il ne faut pas conclure qu'avant le 3e siècle ces vocables n'étaient pas entrés dans la terminologie métroaque. Depuis longtemps déjà Cybèle et Attis étaient les Tout-Puissants ; car, vers 230, ce titre seul les désignait clairement, sans que l'on eût besoin de préciser leur identité. Les appellations mystiques prennent place dans les oraisons et dans les chants de la liturgie bien avant de figurer sur les dédicaces. Un scrupule religieux en interdit la divulgation. Elles peuvent être en usage depuis plusieurs siècles quand elles tombent dans le domaine public. De même, l'épithète solaire d'Invictus ne se rencontre qu'en 374, celle de Menotyrannus qu'en 319. Cependant, il y a longtemps qu'Attis est pour ses fidèles le maître du soleil et de la lune. Tel en effet nous le montre, à l'époque d'Hadrien, la statue de marbre qu'un dévot lui consacra dans le Metrôon d'Ostie. Le dieu, couronné de fruits symboliques, est aussi diadémé de rayons. Sur la pointe de son bonnet, jadis constellé d'or, se dresse le croissant lunaire, d'où surgissent des bouquets d'épis. Attributs agraires et attributs astronomiques se superposent et cherchent à se combiner, comme ont fait les deux mythes. Mais c'est à peu près sous le même aspect que, trois siècles auparavant, les Syriens représentaient leur Baal-Schammin, Seigneur des cieux.* A Pessinonte même, au dernier siècle avant notre ère, Attis était en possession déjà de ses attributs sidéraux.*

*) Il a le front surmonté du croissant et porte un soleil à sept rayons ; monnaies d'Antiochus 8 (125-96) ; Babelon, \emph{Rois de Syrie}, 1890, p. 178 ss et p. 159. Ailleurs ce Baal Shamin (= Zeus Ouranios : Clermont-Ganneau, \emph{Études d'archéol. or.}, 1, p. 53 ; Cumont, \emph{Relig. or.}, pp. 154 et 295) est accosté des deux Dioscures, personnifications des deux hémisphères célestes ; de même, sur les monnaies de Pessinonte, les symboles d'Attis sont accompagnés de ceux des Dioscures, confondus avec les Corybantes.

*) Cf. supra, p. 212, n. 5. Dès ce temps-là, il est identifié à Zeus Pappas : Diod. 3, 58.

\subsection{4.}

Rome n'a donc fait qu'adopter des traditions qui s'étaient formées en Orient. Il n'en reste pas moins vrai que sa part fut grande dans l'élaboration définitive de la doctrine. Rome a désormais remplacé Pessinonte comme capitale du culte phrygien ; elle en est à la fois la ville sainte et le centre administratif. Attis est devenu un dieu de la cité et de l'Empire. Ses zélateurs le croient appeler au rôle de dieu universel. Les nouvelles destinées du culte devaient nécessairement influer sur le caractère de la divinité. Les théologiens se préoccupèrent de mettre en valeur, dans la complexité du mythe, ce qui pouvait rehausser la dignité d'un dieu si longtemps méprisé des vrais Romains. Mais il importait aussi de ne point exagérer sa puissance et de réagir contre la tendance des mystes à amplifier leur dieu jusqu'à la souveraine majesté. Ils trouvèrent dans le mythe sidéral d'Attis la raison d'être de son culte. Ils spéculèrent à leur tour sur la notion de l'Attis solaire. Elle leur permettait de concilier des éléments en apparence incompatibles : la barbarie de certains rites et leur noble signification morale, l'éviration répugnante du dieu et sa vertu merveilleuse de fécondité, sa situation inférieure auprès de la Mère des Dieux et son grand pouvoir. Leur Attis est un Médiateur, mais qui n'aspire pas à l'omnipotence. Dieu intermédiaire, il reste un dieu secondaire, un petit dieu, presque un demi-dieu.* Aussi, dans l'iconographie officielle du 2e siècle, Attis conserve-t-il encore la petite taille et l'attitude du Prospolos. Mais cette notion simpliste de l'Attis médiateur, qui pouvait satisfaire aux exigences religieuses de la Rome antonine, devenait insuffisante et caduque dans la Rome des Sévères. D'une part, c'est une loi constante que le concept d'une divinité s'élargit à mesure qu'augmente le nombre de ses fidèles. Attis tend à se métamorphoser en dieu panthée.* Comme Cybèle, il cumule les fonctions et concentre en lui les vertus de toutes les divinités similaires. De l'ensemble des puissances qu'il accapare se constitue sa toute-puissance. Les provinces contribuent à cet enrichissement et apportent leur contingent. Mais c'est à Rome, rendez-vous de tous les dieux et de toutes les races, que se coordonnent les efforts de ce syncrétisme. D'autre part l'action du sémitisme, qui fut si considérable en Anatolie, se continuait sans interruption dans Rome. Car les Syriens ont envahi la capitale de l'Empire. Ils colonisent l'Occident. Progressivement ils lui imposent la civilisation et les croyances de l'Orient. Cette conquête pacifique du monde latin par leurs marchands et leurs esclaves, résultat d'un irrésistible mouvement économique, avait commencé dès le début de notre ère.* Au temps des Flaviens et de Trajan, Juvénal se plaint de ne pouvoir sortir dans les rues de Rome sans entendre parler syriaque.* Il y l'encontre aussi trop de prêtres syriens. Les cultes sémitiques se sont multipliés très vite sur les bords du Tibre. Ils ne se confinent point dans le quartier populaire du Transtévère. Ils ont pris possession du Capitole, où trône une Caelestis. A l'avènement des Sévères, dont la famille et la cour sont plus qu'à demi syriennes, les Baals triomphent. Par reconnaissance, les Césars favorisent le culte de ces dieux, maîtres du destin, qui leur ont donné la couronne. Quatre dévotes et ambitieuses princesses, filles et petites filles d'un prêtre de Phénicie, se font auprès des fonctionnaires et de l'armée les propagandistes de leur religion nationale. Lorsqu'enfin le jeune Héliogabale, élevé du sacerdoce au trône, prétend imposer son Baal solaire comme dieu suprême de l'Empire, il est logique dans sa déraison. Car un Baal est, par essence, souverain du monde. Dieu éternel, il est universel ; son pouvoir est infini dans l'espace comme dans le temps. Il est le plus puissant et l'unique Tout-Puissant. Mais il partage cette toute puissance avec une déesse parèdre, la Dame des cieux. A son Elagabal Invictus d'Emèse, l'empereur associa la Magna* Caelestis de Carthage.

*) La notion d'Attis ἡμιθεός survit dans Julien, 168 A, qui essaie de la concilier avec la notion d'Attis θεὸς τῷ παντί. Sur la différenciation entre les dieux premiers (Mêter) et les dieux seconds (Attis), cf. Sallust. phil., \emph{l. c.}

*) Cf. l'hymne que nous fait connaître Hippolyte.

*) Cumont, \emph{Rel. or.}, p. 128 ss ; bibliographie, p. 282 s.

*) Juv. 3, 62.

*) \emph{Dea magna Virgo Caelestis} : \emph{CIL.} 8, 9796.

Entre ce couple syro-punique et le couple phrygien la différence n'était pas grande. Aussi princes africains et impératrices syriennes avaient-ils pratiqué pieusement et protégé le culte de la Mère des Dieux, si proche parente de leur grande déesse. Julia Domna continue à Rome la tradition des Faustines.* Pour préparer les Romains à sa tentative, Héliogabale mit à profit la situation officielle et la popularité du culte métroaque. Il se fit initier aux mystères de Cybèle. Il voulut recevoir tous les sacrements, même la consécration sanglante du taurobole. Il tint à se rapprocher davantage encore des dieux de Pessinonte. Il devint non seulement leur serviteur, mais leur prêtre, et non point selon la loi romaine, mais selon la coutume orientale. Les rites des Galles, presque semblables dans les deux religions de Phrygie et de Syrie, lui étaient familiers. Il entra dans le clergé des « fanatiques de la Mère des Dieux » ; on prétendait même que ce César, qui dans certains moments d'exaltation mystique s'identifiait à Magna Mater et réclamait un attelage de lions, lui avait sacrifié sa virilité. Il fit transporter ensuite dans le temple de son dieu noir la dame noire du Palatin.* Ce fut la suprême maladresse, qui acheva de révolter la conscience romaine. Mais en rapprochant ainsi les deux fétiches, que l'on disait tombés du ciel, il prétendait rendre plus manifeste l'analogie de ses dieux avec ceux de Rome ; et sans doute espérait-il attirer vers le nouveau sanctuaire la foule des adorateurs qui se pressaient au temple de l'Idéenne. Cette manœuvre, que Rome indignée considéra comme un acte de barbarie, ne porta point bonheur au Baal d'Emèse.* Par contre, Attis semble en avoir tiré quelque bénéfice. Il n'avait certainement pas attendu si tard pour s'égaler aux dieux sémitiques. Depuis longtemps l'instinct de la foi et la science des théologiens l'avaient grandi à leur taille. Mais la première fois que nous le voyons affirmer à côté de Cybèle sa toute-puissance, n'est-il pas intéressant de constater que c'est au lendemain de la tentative de l'empereur syrien ?

*) Les médailles de consécration de Julia Domna la représentent parée des attributs de Cybèle, sur le char de lions. Le type est emprunté à la numismatique des Faustines.

*) Lamprid., \emph{Heliog.} 7, 1, 2. Naturellement les historiens romains, hostiles au César de race étrangère, ont en partie dénaturé la réalité des faits.

*) La réforme tentée par Héliogabale n'était en somme que prématurée et trop violente. Un demi-siècle plus tard, Aurélien s'inspirait de la même pensée en créant un nouveau culte du Soleil Invincible.
\clearpage

\end{document}
